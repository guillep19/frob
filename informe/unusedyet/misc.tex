
%%secciones no incluidas en main.

%%muestra de citas en la bibliografia
Prueba \cite{Arduino:2014:Misc}
Prueba \cite{SimmonsApfelbaum}

%%Probablemente vayan en una seccion Resumen inicial.
\section{Objetivo}

Diseño de un lenguaje de programación para robots con bajas capacidades de cómputo, con fin educativo.

Diseño e implementación de una máquina virtual para dicho lenguaje, que debe ser implementado en alguna de las
arquitecturas presentadas.

El lenguaje debe intentar estandarizar los diferentes kits de robótica.

\section{Relevamiento de kits, placas, procesadores y arquitecturas}


%%requerimientos
\section{Requerimientos}

El lenguaje debe permitir expresar declarativamente o imperativamente el 
comportamiento de un robot.
Se deben proveer abstracciones para el hardware periférico, de tal manera que 
se puedan definir nuevas abstracciones para incluir nuevo hardware.
Se debe contemplar la posibilidad de compilar programas expresados en haskell 
mediante el paradigma de programación funcional reactiva.

A continuación se relevan las arquitecturas consideradas para el proyecto.
Se relevan sus especificaciones y los trabajos relacionados que hay realizados
 en torno a la temática de éste proyecto.

Periféricos estándar:

\begin {enumerate}
  \item Motor
  \item Sensor de distancia
  \item Comunicación (serial, inalambrica, usb, etc)
  \item Acelerometro
  \item Sensor de inclinacion
  \item Boton, Interruptor
  \item Sensor de sonido
  \item Sensor de inclinacion
  \item Gps
  \item Sensor infrarrojo
  \item Sensor de luz
  \item Sensor de voltaje, etc
  \item Brujula / compass
  \item Parlante
  \item Leds
  \item Pantallas
\end {enumerate}



