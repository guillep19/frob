

\section{Programacion funcional Reactiva}

\subsection{Definiciones}

[Programa reactivo] Es aquel que interactúa con el ambiente, 
intercalando entradas y salidas dependientes del tiempo. Por
ejemplo un reproductor de música, video juegos o controladores
robóticos.

Difiere de los programas \emph{transformacionales} los
cuáles toman una entrada al inicio de la ejecución y 
producen una salida completa al final. Por ejemplo un
compilador.

[Señales] Valores que varían en función del tiempo.
  $Signal A \approx Time \rightarrow A$

[Funciones sobre señales] Son funciones que mapean
una señal a otra.
  $SF A B \approx Signal A \rightarrow Signal B$

[Implementacion de un programa funcional reactivo] Se construye
componiendo funciones sobre señales, formando redes de funciones
sobre señales.

[Señales contínuas en el tiempo y discretas en el tiempo]





