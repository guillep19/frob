

Puntos importantes:
 
1 - Los robots deben cumplir varios objetivos al mismo tiempo.
2 - Deben permanecer reactivos al entorno, sensando y actuando continuamente.
3 - Parte de esto es coordinar la concurrencia de esas actividades.
4 - Debe poder recuperarse ante errores.

TDL es un lenguaje embebido en C++:
* Descompone en TAREAS al sistema.
* Sincroniza, ordena las tareas.
* Provee mecanismos de manejo de errores.

Se basa en el concepto de *arboles de tareas*.

Arquitectura en tres capas:
1 - Capa de planes (Alto nivel)
2 - Capa ejecutiva
3 - Capa de comportamiento (Bajo nivel)

La capa 3 interactúa con el mundo real, controla directo sensores y actuadores.
La capa 1 son los planes en alto nivel, abstractos.
La capa 2 ejecuta dichos planes, es el nexo entre el alto y bajo nivel.

La idea es concentrarse en bajar la complejidad de la capa ejecutiva,
naturalmente es de alta complejidad al tener que coordinar y realizar
tareas concurrentes.
También debe manejar los errores y tener capacidad de corregir el flujo
global. Por ejemplo si se descubre que un plan no era factible, detener
la ejecución asociada y tomar otra acción.

Los lenguajes convencionales no proveen este tipo de control, y el codigo
resulta NO lineal y difícil de entender y mantener.

La implementación de TDL
