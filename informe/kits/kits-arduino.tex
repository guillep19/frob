

\section{Kits Arduino}

Arduino es una plataforma abierta de prototipado, basada en software y hardware flexible fácil de usar. Está pensada para ser usada por diseñadores, artistas, como hobby, para crear objetos y ambientes interactivos. Entre sus productos, hay placas y kits de componentes.
Los kits de arduino generalmente tienen interfaz usb con soporte para programarlo usando la propia placa sin necesidad de un programador por hardware. También los pines de entrada/salida del microprocesador están diseñados para poder colocar fácilmente cables y conectar periféricos sin necesidad de soldar. 
También incluyen leds y botones para resetear la placa o utilizarlos como sensor.
Un IDE que utiliza una implementación de gcc para la arquitectura avr puede ser utilizado para programar sobre los kits. Variedad de librerías y abstracciones de sensores, actuadores, protocolos de comunicación, ya están implementados y pueden ser usados en los kits.
Procesadores y arquitectura: Atmel, algunos con arquitectura ARM otros AVR.
Lenguaje estándar: C.
Herramientas de desarrollo: Arduino Ide

\subsection{Arduino Uno}

Kit:
Arduino Uno
Web:
http://arduino.cc/en/Main/ArduinoBoardUno
Características generales:
Es una placa basada en el microcontrolador ATmega328. 
0.5 kb de la memoria flash son utilizados por bootloader.
Existe una placa construida en uruguay llamada Urduino328, la cuál es compatible con la Arduino Uno y tiene un costo aproximado de 50 dólares.
Microcontrolador:
ATmega328


\subsection{Arduino Leonardo}
Web:
http://arduino.cc/en/Main/ArduinoBoardLeonardo
Características generales:
Es una placa basada en el microcontrolador ATmega32u4. Tiene 20 pins de entrada/salida digitales, frecuencia de 16 MHz y conección micro USB. 
La diferencia principal con otras placas es que el microcontrolador permite la comunicación usb sin necesidad de un microcontrolador secundario que la implemente.
Un bootloader es incluído, el cuál se puede utilizar para programar la placa sin un programador por hardware. Éste bootloader ocupa 4 kb de la memoria Flash del microcontrolador, puede ser eliminado pero teniendo en cuenta que luego no se cuenta con su funcionalidad.
Microcontrolador:
ATmega32u4


\subsection{Arduino Due}
Web: 
http://arduino.cc/en/Main/ArduinoBoardDue
Características:
Es la primer placa arduino basada en la arquitectura ARM de 32 bits. Tiene 54 pins de entrada/salida digital, 12 de los cuáles pueden ser usados como salidas PWM. 12 entradas analógicas, un reloj de 84 MHz integrado, conección USB, 2 convertidores digital-analógico. Un botón de reset y un botón de borrado.
A diferencia de otras placas, ésta placa corre con un voltaje de 3.3V.
La mejora sustancial con respecto a otras placas, puede ejecutarhttp://arduino.cc/en/Main/Robot operaciones sobre 4 bytes en un sólo ciclo de reloj, tiene una frecuencia alta de reloj, 96 kbytes de SRAM, 512 kb de memoria flash para código y un controlador DMA para liberar el CPU de tareas basadas en muchos accesos a memoria.
El bootloader que incluye viene de fábrica y está en una ROM dedicada, por lo que no ocupa espacio de la memoria Flash.
Microcontrolador:
Atmel SAM3X8E ARM Cortex-M3

\subsection{Arduino Robot}
Web:
http://arduino.cc/en/Main/Robot
Características:
Es el primer Arduino sobre ruedas oficial. Cuenta con dos procesadores, cada uno sobre una placa. Hay una placa utilizada para controlar los motores, y otra placa de control que maneja los sensores y decide como operar.
Cada placa se puede programar por separado usando el IDE Arduino. Alguons de los pines de la placa ya están mapeados a sensores y actuadores.
El chasis cuenta con una brújula, un parlante, un panel de control de 5 botones, leds, conexiones I2C, dos ruedas y sensores infrarrojos. También tiene zonas de prototipado.
Microcontrolador:
2 microcontroladores ATmega32u4


