

\section{Mbed}

Los kits de Mbed están diseñados para prototipar rápidamente. Mbed desarrolló herramientas web colaborativas como ser un IDE web y interfaz web con control de versiones, donde se pueden publicar proyectos, extender y colaborar con proyectos de otros usuarios.
También hay variedad de bibliotecas desarrolladas para los kits mbed, que implementan funcionalidades básicas como ser protocolos de comunicación y otros.

Procesadores y arquitectura:
  ARM Cortex-M3 (ARM)
  ARM Cortex-M0 (ARM)

Herramientas de desarrollo:
  Mbed Online Compiler. Compilador web de mbed para aplicaciones. 
  Entorno web para crear aplicaciones y herramienta de control de versiones basada en Mercurial integrada. Las aplicaciones pueden ser cargadas en las placas usando el entorno web sin necesitar drivers extra ni instalación del compilador.
  SDK C/C++ especializado para mbed.
  También existe un HDK (Hardware development kit) para diseño de hardware especializado.

\subsection{NXP LPC11U24}

Homepage:
https://mbed.org/handbook/mbed-NXP-LPC11U24
Características:
Diseñada para prototipado rápido, programador USB integrado, aplicaciones de bajo consumo eléctrico.
8KB RAM, 32KB FLASH
USB, 2xSPI, I2C, UART, 6xADC, GPIO 
Procesador:
32-bit ARM Cortex-M0, 48MHz

\subsection{NXP LPC1768}
Homepage:
http://mbed.org/platforms/mbed-LPC1768/
Características generales:
Diseñada para prototipado rápido de aplicaciones de microcontroladores en general, Ethernet, USB.
Tiene flexibilidad para varios periféricos y memoria FLASH.
Ethernet, USB Host y Device, 2xSPI, 2xI2C, 3xUART, CAN, 6xPWM, 6xADC, GPIO.
40 pines de entrada/salida.
Interfaz de programación por USB integrada (drag and drop programmer)
Procesador:
32-bit ARM Cortex-M3, 96MHz
Memoria:
512kb FLASH, 32kb RAM

