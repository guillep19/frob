
  Mbed, al igual que Arduino es una plataforma abierta de prototipado,
su objetivo es que se puedan desarrollar prototipos en un tiempo corto.

  Cuenta con herramientas colaborativas como ser un entorno de
  desarrollo integrado (IDE) web, interfaz web de control de versiones,
donde se pueden publicar proyectos, extender y colaborar con
proyectos de otros usuarios.

  Existe una gran variedad de bibliotecas desarrolladas para los
kits Mbed, que al igual que en Arduino, implementan funcionalidades
básicas como ser protocolos de comunicación e interacción con
componentes externos.

  La arquitectura usada por Mbed es \texttt{ARM}, principalmente
\texttt{ARM Cortex-M3} y \texttt{ARM Cortex-M0}.

  El compilador web es práctico para colaborar con otros
usuarios y no tener que armar un entorno local.
  Las aplicaciones pueden ser cargadas en las placas usando el
entorno web sin necesitar instalación del compilador.

  El lenguaje utilizado es \texttt{C/C++} con bibliotecas especializadas
de Mbed.
  Mbed también cuenta con un HDK (Hardware development kit) para diseño de
hardware especializado, luego de prototipar.

  En la Tabla \ref{table-mbed} se muestra un listado con los modelos
más relevantes y sus características.
  La memoria RAM varía entre 16 kilobytes y 1 megabyte, bastante similar
a Arduino. Sin embargo, el modelo más popular es el \texttt{LPC1768}
con 512 KB de Flash y 64 KB de memoria RAM.
  Éste modelo será utilizado para la implementación, al no ser tan
reducido, se puede crear una implementación modelo para el mismo, y luego
evaluar si es posible reducir el tamaño para trabajar con modelos
con menor capacidad de cómputo.

\begin{table}[htbp]
  \centering
  \scriptsize
  \caption{Modelos Mbed}
  \label{table-mbed}
  \begin{tabular}{|c|c|c|c|c|}
  \hline
    Modelo & Flash & RAM (KB) & Procesador & Frecuencia \\
  \hline
    NXP LPC1768 & 512 KB & 64 (sram) & ARM Cortex-M3 & 96 MHz \\
  \hline
    NXP LPC11U24 & 32 KB & 8 & ARM Cortex-M0 & 48 MHz \\
  \hline
    Freescale FRDM-KL25Z & 128 & 16 KB & ARM Cortex-M0+ & 48 MHz \\
  \hline
    NXP LPC800-MAX & 16 KB & 4 & ARM Cortex-M0+ & 30 MHz \\
  \hline
    NXP EA LPC4088 & 512 KB & 96 (sram) & ARM Cortex-M4 & 120 MHz \\
  \hline
    NXP DipCortex M0 & 32 KB & 8 & ARM Cortex-M0 & 50 MHz \\
  \hline
    NXP DipCortex M3 & 64 KB & 12 & ARM Cortex-M3 & 72 MHz \\
  \hline
    NXP BlueBoard-LPC11U24 & 32 KB & 8 & ARM Cortex-M0 & 48 MHz \\
  \hline
    NXP WiFi DipCortex & 64 KB & 12 & ARM Cortex-M3 & 72 MHz \\
  \hline
    NXP Seeeduino-Arch & 32 KB & 8 & ARM Cortex-M0 & 48 MHz \\
  \hline
    NXP mbed LPC1114FN28 & 32 KB & 4 & ARM Cortex-M0 & 50 MHz \\
  \hline
    Ublox U-blox C027 & 512 KB & 32 & ARM Cortex-M3 & 96 MHz \\
  \hline
    NXP EA LPC11U35 & 64 KB & 10 & ARM Cortex-M0 & 48 MHz \\
  \hline
    ST Nucleo F103RB & 128 KB & 20 (sram) & ARM Cortex-M3 & 72 MHz \\
  \hline
    Freescale FRDM-KL46Z & 256 KB & 32 & ARM Cortex-M0+ & 48 MHz \\
  \hline
    NXP Seeeduino-Arch-Pro & 512 KB & 32 & ARM Cortex-M3 & 96 MHz \\
  \hline
    ST Nucleo F302R8 & 64 KB & 16 (sram) & ARM Cortex-M4 & 72 MHz \\
  \hline
    ST Nucleo L152RE & 512 KB & 80 (sram) & ARM Cortex-M3 & 32 MHz \\
  \hline
    ST Nucleo F401RE & 512 KB & 96 (sram) & ARM Cortex-M4 & 84 MHz \\
  \hline
    ST Nucleo F030R8 & 64 KB & 8 (sram) & ARM Cortex-M0 & 48 MHz \\
  \hline
    Freescale FRDM-K64F & 1 MB & 256 & ARM Cortex-M4 & 120 MHz \\
  \hline
    Nordic nRF51822 & 128 KB & 16 & ARM Cortex-M0 & 16 MHz \\
  \hline
    FRDM-KL05Z & 32 KB & 4 & ARM Cortex-M0+ & 48 MHz \\
  \hline
    LPCXpresso1549 & 256 KB & 36 & ARM Cortex-M3 & 72 MHz \\
  \hline
    LPCXpresso11U68 & 256 KB & 36 & ARM Cortex-M0+ & 50 MHz \\
  \hline
  \end{tabular}
\end{table}


%\subsection{NXP LPC11U24}
%
%Homepage:
%https://mbed.org/handbook/mbed-NXP-LPC11U24
%Características:
%Diseñada para prototipado rápido, programador USB integrado, aplicaciones de bajo consumo eléctrico.
%8KB RAM, 32KB FLASH
%USB, 2xSPI, I2C, UART, 6xADC, GPIO 
%Procesador:
%32-bit ARM Cortex-M0, 48MHz

%\subsection{NXP LPC1768}
%
%Homepage:
%http://mbed.org/platforms/mbed-LPC1768/
%Características generales:
%Diseñada para prototipado rápido de aplicaciones de microcontroladores en general, Ethernet, USB.
%Tiene flexibilidad para varios periféricos y memoria FLASH.
%Ethernet, USB Host y Device, 2xSPI, 2xI2C, 3xUART, CAN, 6xPWM, 6xADC, GPIO.
%40 pines de entrada/salida.
%Interfaz de programación por USB integrada (drag and drop programmer)
%Procesador:
%32-bit ARM Cortex-M3, 96MHz
%Memoria:
%512kb FLASH, 32kb RAM

