
  Luego de estudiar las plataformas, se decide utilizar
el lenguaje \texttt{C} con extensiones de \texttt{C++} para facilitar
la portabilidad entre plataformas.
  Al planificar la implementación, se toma en cuenta el objetivo de
tener un tamaño de programa menor a 64 KB y en lo posible menor a 32 KB.
  Se deberá intentar utilizar memoria estática en lo posible, en lugar
de memoria dinámica, ya que no todas las plataformas cuentan con las
bibliotecas estándar necesarias para su manejo.

  Los proyectos libres mejoran la portabilidad, ya que cuentan con
más y mejor documentación, y una comunidad abierta.
  En el caso de particular de \texttt{Robotis} se evaluó la
posibilidad de desarrollar utilizando
\texttt{C++} y si bien es posible, no cuenta con documentación más
que el propio código fuente.

  En el resto de los proyectos, se cuenta con documentación y una
comunidad abierta que apoya la realización de proyectos que extiendan
sus funcionalidades.
