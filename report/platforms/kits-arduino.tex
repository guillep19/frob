
  Arduino \cite{arduino} es una plataforma abierta de prototipado, basada en
software y hardware flexible fácil de usar.
  Está pensada para ser usada por diseñadores, artistas, como
hobby, para crear objetos y ambientes interactivos.
  Entre sus productos, hay placas y kits de componentes.
  Los kits de arduino generalmente tienen interfaz usb con soporte
para programarlo usando la propia placa sin necesidad de un
programador por hardware.

  También los pines de entrada/salida del microprocesador
están diseñados para poder colocar fácilmente cables y
conectar periféricos sin necesidad de soldar.

  También incluyen leds y botones para resetear la placa o
utilizarlos como sensor.

  Existe un entorno de desarrollo integrado (IDE) que utiliza
una implementación del compilador \texttt{gcc} \cite{gcc} para la arquitectura
\texttt{avr} \cite{avr} de \textit{Atmel} \cite{atmel} y puede ser utilizado
para programar sobre los kits.

  Variedad de bibliotecas y abstracciones de sensores, actuadores y
protocolos de comunicación, ya están implementados y pueden ser
usados en los kits.
  Al ser un proyecto libre las bibliotecas son publicadas y mantenidas
por una comunidad abierta.

  La arquitectura usada por casi todos los kits
es \texttt{avr} \texttt{Atmel} pero existen algunos con
arquitectura \texttt{ARM} \cite{arm}.

  El lenguaje estándar para desarrollar programas se llama \texttt{Arduino},
sin embargo el lenguaje es \texttt{C/C++}, cambiando la forma en que
se invoca el programa principal y con algunas funciones y
formato predefinido.

  La Tabla \ref{table-arduino} muestra un listado de los modelos
de Arduino, cuánta memoria persistente tienen (Flash) en kilobytes, con
cuánta memoria RAM cuentan, cuánta memoria EEPROM tienen en kilobytes,
que procesador tienen y a que frecuencia funcionan.

  Salvo el modelo \texttt{Due}, el resto utilizan la
arquitectura \texttt{avr} de 8 bit. La memoria ram varía entre
16 y 512 kilobytes.
  Los modelos más populares y representativos, son el Arduino \texttt{Uno}
y el Arduino \texttt{Nano 328}, ambos con 32 KB de memoria Flash, 2 KB de memoria
RAM (SRAM) y procesador \texttt{ATmega328} a 16 MHz.

\begin{table}[htbp]
\centering
\scriptsize
\setlength\tabcolsep{2pt}
\caption{Modelos arduino}
\label{table-arduino}
\begin{tabular}{|c|c|c|c|c|c|c|}
  \hline
  Modelo & Flash & SRAM (kb) & EEPROM (kb) & Procesador & Arquitectura & Frecuencia \\
  \hline
  Uno & 32 KB & 2 & 1 & ATmega328 & 8 bit AVR & 16 MHz \\
  \hline
  Leonardo & 32 KB & 2.5 & 1 & ATmega32u4 & 8 bit AVR & 16 MHz \\
  \hline
  Due & 512 KB & 96 & - & AT91SAM3X8E & ARM Cortex-M3 & 84 Mhz \\
  \hline
  Yun & 32 KB & 2.5 & 1 & ATmega32u4 & 8 bit AVR & 16 MHz \\
  \hline
  Tre & 32 KB & 2.5 & 1 & ATmega32u4 & 8 bit AVR & 16 MHz \\
  \hline
  Micro & 32 KB & 2.5 & - & ATmega32u4 & 8 bit AVR & 16 MHz \\
  \hline
  Robot & 32 KB & 2.5 & 1 & ATmega32u4 & 8 bit AVR & 16 MHz \\
  \hline
  Esplora & 32 KB & 2.5 & 1 & ATmega32u4 & 8 bit AVR & 16 MHz \\
  \hline
  Mega ADK & 256 KB & 8 & 4 & ATmega2560 & 8 bit AVR & 16 MHz \\
  \hline
  Ethernet & 32 KB & 2 & 1 & ATmega328 & 8 bit AVR & 16 MHz \\
  \hline
Mega 2560 & 256 KB & 8 & 4 & ATmega2560 & 8 bit AVR & 16 MHz \\
  \hline
  Mini & 32 KB & 2 & 1 & ATmega328 & 8 bit AVR & 16 MHz \\
  \hline
  LilyPad USB & 32 KB & 2.5 & 1 & ATmega32u4 & 8 bit AVR & 8 Mhz \\
  \hline
  LilyPad Simple & 32 KB & 2 & 1 & ATmega328 & 8 bit AVR & 8 Mhz \\
  \hline
  LilyPad (168V) & 16 KB & 1 & 512 Bytes & ATmega168V & 8 bit AVR & 8 Mhz \\
  \hline
  LilyPad (328V) & 16 KB & 1 & 512 Bytes & ATmega328V & 8 bit AVR & 8 Mhz \\
  \hline
  Nano (168) & 16 KB & 1 & 512 Bytes & ATmega168 & 8 bit AVR & 16 MHz \\
  \hline
  Nano (328) & 32 KB & 2 & 1 & ATmega328 & 8 bit AVR & 16 MHz \\
  \hline
  Pro mini (3.3v) & 16 KB & 1 & 512 Bytes & ATmega168 & 8 bit AVR & 8 Mhz \\
  \hline
  Pro mini (5v) & 16 KB & 1 & 512 Bytes & ATmega168 & 8 bit AVR & 16 MHz \\
  \hline
  Pro (168) & 16 KB & 1 & 512 Bytes & ATmega168 & 8 bit AVR & 8 Mhz \\
  \hline
  Pro (328) & 32 KB & 2 & 1 & ATmega328 & 8 bit AVR & 16 MHz \\
  \hline
  Fio & 32 KB & 2 & 1 & ATmega328P & 8 bit AVR & 8 Mhz \\
  \hline 
\end{tabular}
\end{table}

%%\subsection{Arduino Uno}
%%Web: http://arduino.cc/en/Main/ArduinoBoardUno
%%Microcontrolador: ATmega328
%%Características generales:
%%Es una placa basada en el microcontrolador ATmega328.
%%0.5 kb de la memoria flash son utilizados por bootloader.
%%Existe una placa construida en uruguay llamada Urduino328,
%%la cuál es compatible con la Arduino Uno y tiene un costo aproximado de 50 dólares.
%%
%%\subsection{Arduino Leonardo}
%%Web:
%%http://arduino.cc/en/Main/ArduinoBoardLeonardo
%%Características generales:
%%Es una placa basada en el microcontrolador ATmega32u4. Tiene 20 pins de entrada/salida digitales, frecuencia de 16 MHz y conección micro USB.
%%La diferencia principal con otras placas es que el microcontrolador permite la comunicación usb sin necesidad de un microcontrolador secundario que la implemente.
%%Un bootloader es incluído, el cuál se puede utilizar para programar la placa sin un programador por hardware. Éste bootloader ocupa 4 kb de la memoria Flash del microcontrolador, puede ser eliminado pero teniendo en cuenta que luego no se cuenta con su funcionalidad.
%%Microcontrolador:
%%ATmega32u4
%%
%%\subsection{Arduino Due}
%%Web:
%%http://arduino.cc/en/Main/ArduinoBoardDue
%%Características:
%%Es la primer placa arduino basada en la arquitectura ARM de 32 bits.
%%Tiene 54 pins de entrada/salida digital, 12 de los cuáles pueden ser usados como salidas PWM.
%%12 entradas analógicas, un reloj de 84 MHz integrado, conección USB, 2 convertidores digital-analógico.
%%
%%Un botón de reset y un botón de borrado.
%%A diferencia de otras placas, ésta placa corre con un voltaje de 3.3V.
%%La mejora sustancial con respecto a otras placas, puede ejecutar operaciones sobre 4 bytes en un sólo ciclo de reloj,
%%tiene una frecuencia alta de reloj, 96 kbytes de SRAM, 512 kb de memoria flash para código y
%%un controlador DMA para liberar el CPU de tareas basadas en muchos accesos a memoria.
%%
%%El bootloader que incluye viene de fábrica y está en una ROM dedicada, por lo que no ocupa espacio de la memoria Flash.
%%
%%Microcontrolador: Atmel SAM3X8E ARM Cortex-M3
%%
%%\subsection{Arduino Robot}
%%Web:
%%http://arduino.cc/en/Main/Robot
%%Características:
%%Es el primer Arduino sobre ruedas oficial.
%%Cuenta con dos procesadores, cada uno sobre una placa.
%%Hay una placa utilizada para controlar los motores, y otra placa de control que maneja los
%%sensores y decide como operar.
%%Cada placa se puede programar por separado usando el IDE Arduino.
%%Algunos de los pines de la placa ya están mapeados a sensores y actuadores.
%%  El chasis cuenta con una brújula, un parlante, un panel de control de 5 botones, leds, conexiones I2C,
%%dos ruedas y sensores infrarrojos.
%%  También tiene zonas de prototipado.
%%
%%Microcontrolador: 2 microcontroladores ATmega32u4
