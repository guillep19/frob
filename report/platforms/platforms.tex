
  En esta sección se describen las plataformas de hardware relevadas
durante el estado del arte, junto con sus características.
  Las características principales estudiadas 
  son el espacio de almacenamiento,
el lenguaje de programación o herramientas estándar de cada
plataforma, si es un proyecto libre, el procesador y la arquitectura.

  A continuación se presentan las diferentes familias de plataformas
de hardware estudiadas.
  El objetivo es obtener una medida de las capacidades de cómputo de
las plataformas y evaluar las herramientas de desarrollo que se pueden
utilizar.
  Las características estudiadas permiten elegir un lenguaje para
desarrollar común entre las plataformas, y permiten tener una medida
como límite del tamaño de los programas que se podrán implementar.

\section{Arduino}

  Arduino \cite{arduino} es una plataforma abierta de prototipado, basada en
software y hardware flexible fácil de usar.
  Está pensada para ser usada por diseñadores, artistas, como
hobby, para crear objetos y ambientes interactivos.
  Entre sus productos, hay placas y kits de componentes.
  Los kits de arduino generalmente tienen interfaz usb con soporte
para programarlo usando la propia placa sin necesidad de un
programador por hardware.

  También los pines de entrada/salida del microprocesador
están diseñados para poder colocar fácilmente cables y
conectar periféricos sin necesidad de soldar.

  También incluyen leds y botones para resetear la placa o
utilizarlos como sensor.

  Existe un entorno de desarrollo integrado (IDE) que utiliza
una implementación del compilador \texttt{gcc} \cite{gcc} para la arquitectura
\texttt{avr} \cite{avr} de \textit{Atmel} \cite{atmel} y puede ser utilizado
para programar sobre los kits.

  Variedad de bibliotecas y abstracciones de sensores, actuadores y
protocolos de comunicación, ya están implementados y pueden ser
usados en los kits.
  Al ser un proyecto libre las bibliotecas son publicadas y mantenidas
por una comunidad abierta.

  La arquitectura usada por casi todos los kits
es \texttt{avr} \texttt{Atmel} pero existen algunos con
arquitectura \texttt{ARM} \cite{arm}.

  El lenguaje estándar para desarrollar programas se llama \texttt{Arduino},
sin embargo el lenguaje es \texttt{C/C++}, cambiando la forma en que
se invoca el programa principal y con algunas funciones y
formato predefinido.

  La Tabla \ref{table-arduino} muestra un listado de los modelos
de Arduino, cuánta memoria persistente tienen (Flash) en kilobytes, con
cuánta memoria RAM cuentan, cuánta memoria EEPROM tienen en kilobytes,
que procesador tienen y a que frecuencia funcionan.

  Salvo el modelo \texttt{Due}, el resto utilizan la
arquitectura \texttt{avr} de 8 bit. La memoria ram varía entre
16 y 512 kilobytes.
  Los modelos más populares y representativos, son el Arduino \texttt{Uno}
y el Arduino \texttt{Nano 328}, ambos con 32 KB de memoria Flash, 2 KB de memoria
RAM (SRAM) y procesador \texttt{ATmega328} a 16 MHz.

\begin{table}[htbp]
\centering
\scriptsize
\setlength\tabcolsep{2pt}
\caption{Modelos arduino}
\label{table-arduino}
\begin{tabular}{|c|c|c|c|c|c|c|}
  \hline
  Modelo & Flash & SRAM (kb) & EEPROM (kb) & Procesador & Arquitectura & Frecuencia \\
  \hline
  Uno & 32 KB & 2 & 1 & ATmega328 & 8 bit AVR & 16 MHz \\
  \hline
  Leonardo & 32 KB & 2.5 & 1 & ATmega32u4 & 8 bit AVR & 16 MHz \\
  \hline
  Due & 512 KB & 96 & - & AT91SAM3X8E & ARM Cortex-M3 & 84 Mhz \\
  \hline
  Yun & 32 KB & 2.5 & 1 & ATmega32u4 & 8 bit AVR & 16 MHz \\
  \hline
  Tre & 32 KB & 2.5 & 1 & ATmega32u4 & 8 bit AVR & 16 MHz \\
  \hline
  Micro & 32 KB & 2.5 & - & ATmega32u4 & 8 bit AVR & 16 MHz \\
  \hline
  Robot & 32 KB & 2.5 & 1 & ATmega32u4 & 8 bit AVR & 16 MHz \\
  \hline
  Esplora & 32 KB & 2.5 & 1 & ATmega32u4 & 8 bit AVR & 16 MHz \\
  \hline
  Mega ADK & 256 KB & 8 & 4 & ATmega2560 & 8 bit AVR & 16 MHz \\
  \hline
  Ethernet & 32 KB & 2 & 1 & ATmega328 & 8 bit AVR & 16 MHz \\
  \hline
Mega 2560 & 256 KB & 8 & 4 & ATmega2560 & 8 bit AVR & 16 MHz \\
  \hline
  Mini & 32 KB & 2 & 1 & ATmega328 & 8 bit AVR & 16 MHz \\
  \hline
  LilyPad USB & 32 KB & 2.5 & 1 & ATmega32u4 & 8 bit AVR & 8 Mhz \\
  \hline
  LilyPad Simple & 32 KB & 2 & 1 & ATmega328 & 8 bit AVR & 8 Mhz \\
  \hline
  LilyPad (168V) & 16 KB & 1 & 512 Bytes & ATmega168V & 8 bit AVR & 8 Mhz \\
  \hline
  LilyPad (328V) & 16 KB & 1 & 512 Bytes & ATmega328V & 8 bit AVR & 8 Mhz \\
  \hline
  Nano (168) & 16 KB & 1 & 512 Bytes & ATmega168 & 8 bit AVR & 16 MHz \\
  \hline
  Nano (328) & 32 KB & 2 & 1 & ATmega328 & 8 bit AVR & 16 MHz \\
  \hline
  Pro mini (3.3v) & 16 KB & 1 & 512 Bytes & ATmega168 & 8 bit AVR & 8 Mhz \\
  \hline
  Pro mini (5v) & 16 KB & 1 & 512 Bytes & ATmega168 & 8 bit AVR & 16 MHz \\
  \hline
  Pro (168) & 16 KB & 1 & 512 Bytes & ATmega168 & 8 bit AVR & 8 Mhz \\
  \hline
  Pro (328) & 32 KB & 2 & 1 & ATmega328 & 8 bit AVR & 16 MHz \\
  \hline
  Fio & 32 KB & 2 & 1 & ATmega328P & 8 bit AVR & 8 Mhz \\
  \hline 
\end{tabular}
\end{table}

%%\subsection{Arduino Uno}
%%Web: http://arduino.cc/en/Main/ArduinoBoardUno
%%Microcontrolador: ATmega328
%%Características generales:
%%Es una placa basada en el microcontrolador ATmega328.
%%0.5 kb de la memoria flash son utilizados por bootloader.
%%Existe una placa construida en uruguay llamada Urduino328,
%%la cuál es compatible con la Arduino Uno y tiene un costo aproximado de 50 dólares.
%%
%%\subsection{Arduino Leonardo}
%%Web:
%%http://arduino.cc/en/Main/ArduinoBoardLeonardo
%%Características generales:
%%Es una placa basada en el microcontrolador ATmega32u4. Tiene 20 pins de entrada/salida digitales, frecuencia de 16 MHz y conección micro USB.
%%La diferencia principal con otras placas es que el microcontrolador permite la comunicación usb sin necesidad de un microcontrolador secundario que la implemente.
%%Un bootloader es incluído, el cuál se puede utilizar para programar la placa sin un programador por hardware. Éste bootloader ocupa 4 kb de la memoria Flash del microcontrolador, puede ser eliminado pero teniendo en cuenta que luego no se cuenta con su funcionalidad.
%%Microcontrolador:
%%ATmega32u4
%%
%%\subsection{Arduino Due}
%%Web:
%%http://arduino.cc/en/Main/ArduinoBoardDue
%%Características:
%%Es la primer placa arduino basada en la arquitectura ARM de 32 bits.
%%Tiene 54 pins de entrada/salida digital, 12 de los cuáles pueden ser usados como salidas PWM.
%%12 entradas analógicas, un reloj de 84 MHz integrado, conección USB, 2 convertidores digital-analógico.
%%
%%Un botón de reset y un botón de borrado.
%%A diferencia de otras placas, ésta placa corre con un voltaje de 3.3V.
%%La mejora sustancial con respecto a otras placas, puede ejecutar operaciones sobre 4 bytes en un sólo ciclo de reloj,
%%tiene una frecuencia alta de reloj, 96 kbytes de SRAM, 512 kb de memoria flash para código y
%%un controlador DMA para liberar el CPU de tareas basadas en muchos accesos a memoria.
%%
%%El bootloader que incluye viene de fábrica y está en una ROM dedicada, por lo que no ocupa espacio de la memoria Flash.
%%
%%Microcontrolador: Atmel SAM3X8E ARM Cortex-M3
%%
%%\subsection{Arduino Robot}
%%Web:
%%http://arduino.cc/en/Main/Robot
%%Características:
%%Es el primer Arduino sobre ruedas oficial.
%%Cuenta con dos procesadores, cada uno sobre una placa.
%%Hay una placa utilizada para controlar los motores, y otra placa de control que maneja los
%%sensores y decide como operar.
%%Cada placa se puede programar por separado usando el IDE Arduino.
%%Algunos de los pines de la placa ya están mapeados a sensores y actuadores.
%%  El chasis cuenta con una brújula, un parlante, un panel de control de 5 botones, leds, conexiones I2C,
%%dos ruedas y sensores infrarrojos.
%%  También tiene zonas de prototipado.
%%
%%Microcontrolador: 2 microcontroladores ATmega32u4


\section{Mbed}

  Mbed, al igual que Arduino es una plataforma abierta de prototipado,
su objetivo es que se puedan desarrollar prototipos en un tiempo corto.

  Cuenta con herramientas colaborativas como ser un entorno de
  desarrollo integrado (IDE) web, interfaz web de control de versiones,
donde se pueden publicar proyectos, extender y colaborar con
proyectos de otros usuarios.

  Existe una gran variedad de bibliotecas desarrolladas para los
kits Mbed, que al igual que en Arduino, implementan funcionalidades
básicas como ser protocolos de comunicación e interacción con
componentes externos.

  La arquitectura usada por Mbed es \texttt{ARM}, principalmente
\texttt{ARM Cortex-M3} y \texttt{ARM Cortex-M0}.

  El compilador web es práctico para colaborar con otros
usuarios y no tener que armar un entorno local.
  Las aplicaciones pueden ser cargadas en las placas usando el
entorno web sin necesitar instalación del compilador.

  El lenguaje utilizado es \texttt{C/C++} con bibliotecas especializadas
de Mbed.
  Mbed también cuenta con un HDK (Hardware development kit) para diseño de
hardware especializado, luego de prototipar.

  En la Tabla \ref{table-mbed} se muestra un listado con los modelos
más relevantes y sus características.
  La memoria RAM varía entre 16 kilobytes y 1 megabyte, bastante similar
a Arduino. Sin embargo, el modelo más popular es el \texttt{LPC1768}
con 512 KB de Flash y 64 KB de memoria RAM.
  Éste modelo será utilizado para la implementación, al no ser tan
reducido, se puede crear una implementación modelo para el mismo, y luego
evaluar si es posible reducir el tamaño para trabajar con modelos
con menor capacidad de cómputo.

\begin{table}[htbp]
  \centering
  \scriptsize
  \caption{Modelos Mbed}
  \label{table-mbed}
  \begin{tabular}{|c|c|c|c|c|}
  \hline
    Modelo & Flash & RAM (KB) & Procesador & Frecuencia \\
  \hline
    NXP LPC1768 & 512 KB & 64 (sram) & ARM Cortex-M3 & 96 MHz \\
  \hline
    NXP LPC11U24 & 32 KB & 8 & ARM Cortex-M0 & 48 MHz \\
  \hline
    Freescale FRDM-KL25Z & 128 & 16 KB & ARM Cortex-M0+ & 48 MHz \\
  \hline
    NXP LPC800-MAX & 16 KB & 4 & ARM Cortex-M0+ & 30 MHz \\
  \hline
    NXP EA LPC4088 & 512 KB & 96 (sram) & ARM Cortex-M4 & 120 MHz \\
  \hline
    NXP DipCortex M0 & 32 KB & 8 & ARM Cortex-M0 & 50 MHz \\
  \hline
    NXP DipCortex M3 & 64 KB & 12 & ARM Cortex-M3 & 72 MHz \\
  \hline
    NXP BlueBoard-LPC11U24 & 32 KB & 8 & ARM Cortex-M0 & 48 MHz \\
  \hline
    NXP WiFi DipCortex & 64 KB & 12 & ARM Cortex-M3 & 72 MHz \\
  \hline
    NXP Seeeduino-Arch & 32 KB & 8 & ARM Cortex-M0 & 48 MHz \\
  \hline
    NXP mbed LPC1114FN28 & 32 KB & 4 & ARM Cortex-M0 & 50 MHz \\
  \hline
    Ublox U-blox C027 & 512 KB & 32 & ARM Cortex-M3 & 96 MHz \\
  \hline
    NXP EA LPC11U35 & 64 KB & 10 & ARM Cortex-M0 & 48 MHz \\
  \hline
    ST Nucleo F103RB & 128 KB & 20 (sram) & ARM Cortex-M3 & 72 MHz \\
  \hline
    Freescale FRDM-KL46Z & 256 KB & 32 & ARM Cortex-M0+ & 48 MHz \\
  \hline
    NXP Seeeduino-Arch-Pro & 512 KB & 32 & ARM Cortex-M3 & 96 MHz \\
  \hline
    ST Nucleo F302R8 & 64 KB & 16 (sram) & ARM Cortex-M4 & 72 MHz \\
  \hline
    ST Nucleo L152RE & 512 KB & 80 (sram) & ARM Cortex-M3 & 32 MHz \\
  \hline
    ST Nucleo F401RE & 512 KB & 96 (sram) & ARM Cortex-M4 & 84 MHz \\
  \hline
    ST Nucleo F030R8 & 64 KB & 8 (sram) & ARM Cortex-M0 & 48 MHz \\
  \hline
    Freescale FRDM-K64F & 1 MB & 256 & ARM Cortex-M4 & 120 MHz \\
  \hline
    Nordic nRF51822 & 128 KB & 16 & ARM Cortex-M0 & 16 MHz \\
  \hline
    FRDM-KL05Z & 32 KB & 4 & ARM Cortex-M0+ & 48 MHz \\
  \hline
    LPCXpresso1549 & 256 KB & 36 & ARM Cortex-M3 & 72 MHz \\
  \hline
    LPCXpresso11U68 & 256 KB & 36 & ARM Cortex-M0+ & 50 MHz \\
  \hline
  \end{tabular}
\end{table}


%\subsection{NXP LPC11U24}
%
%Homepage:
%https://mbed.org/handbook/mbed-NXP-LPC11U24
%Características:
%Diseñada para prototipado rápido, programador USB integrado, aplicaciones de bajo consumo eléctrico.
%8KB RAM, 32KB FLASH
%USB, 2xSPI, I2C, UART, 6xADC, GPIO 
%Procesador:
%32-bit ARM Cortex-M0, 48MHz

%\subsection{NXP LPC1768}
%
%Homepage:
%http://mbed.org/platforms/mbed-LPC1768/
%Características generales:
%Diseñada para prototipado rápido de aplicaciones de microcontroladores en general, Ethernet, USB.
%Tiene flexibilidad para varios periféricos y memoria FLASH.
%Ethernet, USB Host y Device, 2xSPI, 2xI2C, 3xUART, CAN, 6xPWM, 6xADC, GPIO.
%40 pines de entrada/salida.
%Interfaz de programación por USB integrada (drag and drop programmer)
%Procesador:
%32-bit ARM Cortex-M3, 96MHz
%Memoria:
%512kb FLASH, 32kb RAM



\section{Robotis}

  La empresa \texttt{Robotis} desarrolla robots para uso
educativo, así como una gama de robots para uso competitivo.
  Los kits de Robotis están diseñados para uso final, es decir,
proveen los controladores, así como los componentes para armar
la estructura, sensores y actuadores.

  Para el uso de los kits se deben utilizar las herramientas de desarrollo
de Robotis.

  No es un proyecto abierto, por lo que no puede ser fácilmente
extendido, ni modificado.
  Es posible programar utilizando \texttt{C} embebido, descargando
los archivos fuente para sus plataformas, aunque es un proceso
bastante complejo y no cuenta con buena documentación.

  Robotis cuenta con un lenguaje llamado \texttt{Task} y se necesita
  un entorno de desarrollo integrado propietario llamado 
  \texttt{IDE Roboplus} para utilizarlo.
  En el \texttt{IDE} se pueden generar tareas y programar
  movimientos del robot en base a movimiento de motores y un
  diseño tridimensional gráfico.

  Los microcontroladores utilizados se llaman \texttt{CM-510},
  \texttt{CM-530} y \texttt{CM-100A}. Algunos con
  arquitectura AVR y otros con ARM internamente.

  El \texttt{CM-510} contiene un microprocesador \texttt{ATMega128},
  el \texttt{CM-530} un \texttt{ARM Cortex-M3} y el
  \texttt{CM-100A} un controlador {ATMega8}.

  En la Tabla \ref{table-robotis} se pueden ver las especificaciones
  técnicas de los diferentes kits de \textit{Robotis}.

\begin{table}[htbp]
    \centering
    \scriptsize
    \caption{Modelos Robotis}
    \label{table-robotis}
  \begin{tabular}{|c|c|c|c|c|c|c|}
    \hline
      Modelo & Flash & RAM (kb) & EEPROM (kb) & Procesador & Arquitectura & Frecuencia \\
    \hline
      CM-100A & 8 KB & 1 (sram) & 512 & ATmega8 & 8 bit AVR & 16 MHz \\
    \hline
      CM-5 & 128 KB & 4 (sram) & 4 & ATmega128 & 8 bit AVR & 16 MHz \\
    \hline
      CM-510 & 256 KB & 8 (sram) & 8 & ATmega2561 & 8 bit AVR & 16 MHz \\
    \hline
      CM-530 & 512 KB & 64 & - & STM32F103RE & ARM Cortex-M3 & 72 MHz \\
    \hline
      CM-700 & 256 KB & 8 (sram) & 8 & ATMega2561 & 8 bit AVR & 16 MHz \\
    \hline
      CM-730 & 512 KB & 64 & - & STM32F103RE & ARM Cortex-M3 & 72 MHz \\
    \hline
      CM-900 & 64 KB & 20 (sram) & - & STM32F103C8 & ARM Cortex-M3 & 72 MHz \\
    \hline
    \end{tabular}
  \end{table}

\subsection{Bioloid STEM}
  Creado para uso educativo y competencias robóticas.
  El kit provee el hardware y clases enseñando a construir distintos
  robots para distintos usos, involucrando conceptos de ciencias,
  tecnología, ingeniería y matemáticas.

  Utiliza el controlador \texttt{CM-530} internamente.

  Cuenta con un conjunto de componentes que son:

  \begin{itemize}
  \item Sensor Infrarrojo
  \item Array de 7 sensores infrarrojos (detectan objetos)
  \item Control remoto y receptor
  \item 6 motores dinamixel
  \item Piezas para crear estructura de un robot
  \end{itemize}

\subsection{Bioloid Premium}
  Diseñado para educación, competiciones y entretenimiento.
  Se pueden construir variedad de robots como humanoide y animales.
  El kit contiene 29 ejemplos de robot y programas de ejemplo.
  Utiliza el controlador \texttt{CM-530}.

  Incluye los siguientes componentes:

  \begin{itemize}
  \item 18 motores dinamixel (AX-12A)
  \item Sensor giroscópico
  \item Receptor infrarrojo
  \item Control remoto y receptor
  \item Sensor de distancia
  \item Sensor infrarrojo para detección de objetos.
  \item Piezas para crear estructura de robot.
  \end{itemize}

\subsection{Bioloid GP}
  Humanoide optimizado para competencias robóticas.
  Esqueleto liviano y resistente.
  Instrucciones para jugar al fútbol y hacer tareas de recolección
  pre-programadas.
  Ajuste automático de postura con sensor giroscópico.

  Utiliza el controlador \texttt{CM-530}.

  Cuenta con los siguientes componentes:

  \begin{itemize}
    \item 18 motores dinamixel
    \item Sensor giroscópico
    \item Control remoto y receptor
    \item Sensor de distancia
    \item Piezas de aluminio.
  \end{itemize}



%\section{Lego Mindstorms}
%
\section{Lego}


Kits Lego Mindstorms
Kit:
NXT Intelligent brick
Webpage:
http://shop.lego.com/en-US/NXT-Intelligent-Brick-9841
Procesador:
32-bit ARM7 microprocessor
Interfaces:
Support for Bluetooth, 1 USB 2.0 port, 4 input ports, 3 output ports.
Precio estimado:
US$ 149.99



%\section{Fischertechnik}
%
\subsection{ROBO TX Controller}
Webpage:
http://www.fischertechnik.de/en/Home/info/computing/ROBO-TX-Controller.aspx/usetemplate-1\_column\_no\_pano/
Procesador:
32-bit processor, 200 MHz.


\section{Butiá}

  USB4Butiá es una plataforma que surgió de un proyecto de grado de Facultad de
Ingeniería – UdelaR.
  Como característica principal es un kit económico y con un diseño abierto.

  El diseño de la placa está publicado junto con instrucciones para construirla,
además sus componentes se seleccionaron por ser económicos
y de fácil acceso en el medio local.

  La placa está pensada principalmente para aumentar capacidades sensoriales
  y de actuación del robot Butiá \cite{butia}.

  Utiliza el microprocesador \texttt{PIC 18F4550} \cite{pic18} el cuál se programa
  utilizando el lenguaje \texttt{C}.

  Existen actualmente tres versiones del proyecto Butiá, así como varias
  herramientas de desarrollo.
  Entre ellas se encuentran el entorno de desarrollo integrado \texttt{Tortubots} \cite{Tortubots},
  \texttt{Butialo} \cite{Butialo} que permite programar utilizando el lenguaje \textit{Lua},
  se puede utilizar \textit{Python}, y también otro IDE llamado \texttt{Yatay} \cite{yatay}
  también desarrollado como proyecto de la Facultad de Ingeniería.

  La versión actual del proyecto es \texttt{USB4Butia 3.0}.
  Todas las versiones del proyecto cuentan con todo el software necesario libre, así
como las instrucciones para construir el hardware, que también es libre.


%\section{Microcontroladores}
%
Aquí se presentan las diferentes arquitecturas de los
microcontroladores utilizadas en los ejemplos.

\subsection{ARM}
\subsubsection{ARM Cortex-M0}
Es el procesador más pequeño existente de ARM.
\subsubsection{ARM Cortex-M3}
32 bit
Diseñado para aplicaciónes de bajo costo y alto desempeño.

\subsection{AVR}

Son microcontroladores basados en RISC, la empresa Atmel
produce controladores de éstas características.
Intentan mantener la ejecución a un ritmo de una
instrucción por ciclo de reloj, para lograr velocidades
relativas de 1MIPS/MHz.
Es una línea de bajo consumo y alto desempeño.

\subsubsection{ATmega8}
8 kb de Flash de memoria programable, 1kb de SRAM, 512 k de EEPROM, conversión analógica/digital de 10 bits.
Soporta una frecuencia máxima de 16 Mhz a 1 MIPS/MHz.
Opera entre 2v y 3.5 volts. Se recomiendan 3 volts. Equivale a 2 baterías AA o LR.
Web: http://www.atmel.com/devices/atmega8.aspx

\subsubsection{ATmega128}
Combina 128 kb de Flash, 4kb de SRAM, 4kb de EEPROM, conversión analógica/digital de 10 bits. Soporta hasta 16 MIPS a un máximo de 16 Mhz. (1 MIPS/MHz)
Es un microcontrolador AVR basado en RISC de 8 bit, bajo consumo eléctrico y alto desempeño.
Opera entre 4.5 y 5.5 volts.
Variantes: Atmega128 y ATmega128L.
Web: http://www.atmel.com/devices/atmega128.aspx
ATmega328
Web: http://www.atmel.com/devices/atmega328.aspx
ATmega32u4
Web: www.atmel.com/devices/atmega32u4.aspx
ATmega2560
Web: www.atmel.com/devices/atmega2560.aspx
ATmega168
http://www.atmel.com/devices/atmega168.aspx
Variantes: ATmega168, ATmega168V
ATmega328
http://www.atmel.com/devices/atmega328.aspx
Variantes: ATmega328P, ATmega328V

\subsection{Microchip}

\subsubsection{PIC18F4550}

Producido también por la empresa Digikey.
Frecuencia de 48 MHz (oscilador interno), controlador de
8bit, 32 kbytes de memoria Flash de programa,
2bkytes de memoria RAM, 256 bytes de EEPROM. 



\section{Conclusiones}

  Luego de estudiar las plataformas, se decide utilizar
el lenguaje \texttt{C} con extensiones de \texttt{C++} para facilitar
la portabilidad entre plataformas.
  Al planificar la implementación, se toma en cuenta el objetivo de
tener un tamaño de programa menor a 64 KB y en lo posible menor a 32 KB.
  Se deberá intentar utilizar memoria estática en lo posible, en lugar
de memoria dinámica, ya que no todas las plataformas cuentan con las
bibliotecas estándar necesarias para su manejo.

  Los proyectos libres mejoran la portabilidad, ya que cuentan con
más y mejor documentación, y una comunidad abierta.
  En el caso de particular de \texttt{Robotis} se evaluó la
posibilidad de desarrollar utilizando
\texttt{C++} y si bien es posible, no cuenta con documentación más
que el propio código fuente.

  En el resto de los proyectos, se cuenta con documentación y una
comunidad abierta que apoya la realización de proyectos que extiendan
sus funcionalidades.

