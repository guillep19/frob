
  En esta sección se describen las plataformas de hardware relevadas
durante el estado del arte, junto con sus características.
  Las características principales estudiadas 
  son el espacio de almacenamiento,
el lenguaje de programación o herramientas estándar de cada
plataforma, si es un proyecto libre, el procesador y la arquitectura.

  A continuación se presentan las diferentes familias de plataformas
de hardware estudiadas.
  El objetivo es obtener una medida de las capacidades de cómputo de
las plataformas y evaluar las herramientas de desarrollo que se pueden
utilizar.
  Las características estudiadas permiten elegir un lenguaje para
desarrollar común entre las plataformas, y permiten tener una medida
como límite del tamaño de los programas que se podrán implementar.

\section{Arduino}

  Arduino \cite{arduino} es una plataforma abierta de prototipado, basada en
software y hardware flexible fácil de usar.
  Está pensada para ser usada por diseñadores, artistas, como
hobby, para crear objetos y ambientes interactivos.
  Entre sus productos, hay placas y kits de componentes.
  Los kits de arduino generalmente tienen interfaz usb con soporte
para programarlo usando la propia placa sin necesidad de un
programador por hardware.

  También los pines de entrada/salida del microprocesador
están diseñados para poder colocar fácilmente cables y
conectar periféricos sin necesidad de soldar.

  Tambien incluyen leds y botones para resetear la placa o
utilizarlos como sensor.

  Existe un entorno de desarrollo integrado (IDE) que utiliza
una implementación del compilador \texttt{gcc} \cite{gcc} para la arquitectura
\texttt{avr} \cite{avr} de \textit{Atmel} \cite{atmel} y puede ser utilizado
para programar sobre los kits.

  Variedad de bibliotecas y abstracciones de sensores, actuadores y
protocolos de comunicación, ya están implementados y pueden ser
usados en los kits.
  Al ser un proyecto libre las bibliotecas son publicadas y mantenidas
por una comunidad abierta.

  La arquitectura usada por casi todos los kits
es \texttt{avr} \texttt{Atmel} pero existen algunos con
arquitectura \texttt{ARM} \cite{arm}.

  El lenguaje estándar para desarrollar programas se llama \texttt{Arduino},
sin embargo el lenguaje es \texttt{C/C++}, cambiando la forma en que
se invoca el programa principal y con algunas funciones y
formato predefinido.

  La Tabla \ref{table-arduino} muestra un listado de los modelos
de Arduino, cuánta memoria persistente tienen (Flash) en kilobytes, con
cuánta memoria RAM cuentan, cuánta memoria EEPROM tienen en kilobytes,
que procesador tienen y la frecuencia de funcionamiento.

  Salvo el modelo \texttt{Due}, el resto utilizan la
arquitectura \texttt{avr} de 8 bit. La memoria ram varía entre
16 y 512 kilobytes.
  Los modelos más populares y representativos, son el Arduino \texttt{Uno}
y el Arduino \texttt{Nano 328}, ambos con 32 KB de memoria Flash, 2 KB de memoria
RAM (SRAM) y procesador \texttt{ATmega328} a 16 MHz.

\begin{table}[htbp]
\centering
\scriptsize
\setlength\tabcolsep{2pt}
\caption{Modelos arduino}
\label{table-arduino}
\begin{tabular}{|c|c|c|c|c|c|c|}
  \hline
  Modelo & Flash & SRAM (kb) & EEPROM (kb) & Procesador & Arquitectura & Frecuencia \\
  \hline
  Uno & 32 KB & 2 & 1 & ATmega328 & 8 bit AVR & 16 MHz \\
  \hline
  Leonardo & 32 KB & 2.5 & 1 & ATmega32u4 & 8 bit AVR & 16 MHz \\
  \hline
  Due & 512 KB & 96 & - & AT91SAM3X8E & ARM Cortex-M3 & 84 Mhz \\
  \hline
  Yun & 32 KB & 2.5 & 1 & ATmega32u4 & 8 bit AVR & 16 MHz \\
  \hline
  Tre & 32 KB & 2.5 & 1 & ATmega32u4 & 8 bit AVR & 16 MHz \\
  \hline
  Micro & 32 KB & 2.5 & - & ATmega32u4 & 8 bit AVR & 16 MHz \\
  \hline
  Robot & 32 KB & 2.5 & 1 & ATmega32u4 & 8 bit AVR & 16 MHz \\
  \hline
  Esplora & 32 KB & 2.5 & 1 & ATmega32u4 & 8 bit AVR & 16 MHz \\
  \hline
  Mega ADK & 256 KB & 8 & 4 & ATmega2560 & 8 bit AVR & 16 MHz \\
  \hline
  Ethernet & 32 KB & 2 & 1 & ATmega328 & 8 bit AVR & 16 MHz \\
  \hline
Mega 2560 & 256 KB & 8 & 4 & ATmega2560 & 8 bit AVR & 16 MHz \\
  \hline
  Mini & 32 KB & 2 & 1 & ATmega328 & 8 bit AVR & 16 MHz \\
  \hline
  LilyPad USB & 32 KB & 2.5 & 1 & ATmega32u4 & 8 bit AVR & 8 Mhz \\
  \hline
  LilyPad Simple & 32 KB & 2 & 1 & ATmega328 & 8 bit AVR & 8 Mhz \\
  \hline
  LilyPad (168V) & 16 KB & 1 & 512 Bytes & ATmega168V & 8 bit AVR & 8 Mhz \\
  \hline
  LilyPad (328V) & 16 KB & 1 & 512 Bytes & ATmega328V & 8 bit AVR & 8 Mhz \\
  \hline
  Nano (168) & 16 KB & 1 & 512 Bytes & ATmega168 & 8 bit AVR & 16 MHz \\
  \hline
  Nano (328) & 32 KB & 2 & 1 & ATmega328 & 8 bit AVR & 16 MHz \\
  \hline
  Pro mini (3.3v) & 16 KB & 1 & 512 Bytes & ATmega168 & 8 bit AVR & 8 Mhz \\
  \hline
  Pro mini (5v) & 16 KB & 1 & 512 Bytes & ATmega168 & 8 bit AVR & 16 MHz \\
  \hline
  Pro (168) & 16 KB & 1 & 512 Bytes & ATmega168 & 8 bit AVR & 8 Mhz \\
  \hline
  Pro (328) & 32 KB & 2 & 1 & ATmega328 & 8 bit AVR & 16 MHz \\
  \hline
  Fio & 32 KB & 2 & 1 & ATmega328P & 8 bit AVR & 8 Mhz \\
  \hline 
\end{tabular}
\end{table}

%%\subsection{Arduino Uno}
%%Web: http://arduino.cc/en/Main/ArduinoBoardUno
%%Microcontrolador: ATmega328
%%Características generales:
%%Es una placa basada en el microcontrolador ATmega328.
%%0.5 kb de la memoria flash son utilizados por bootloader.
%%Existe una placa construida en uruguay llamada Urduino328,
%%la cuál es compatible con la Arduino Uno y tiene un costo aproximado de 50 dólares.
%%
%%\subsection{Arduino Leonardo}
%%Web:
%%http://arduino.cc/en/Main/ArduinoBoardLeonardo
%%Características generales:
%%Es una placa basada en el microcontrolador ATmega32u4. Tiene 20 pins de entrada/salida digitales, frecuencia de 16 MHz y conección micro USB.
%%La diferencia principal con otras placas es que el microcontrolador permite la comunicación usb sin necesidad de un microcontrolador secundario que la implemente.
%%Un bootloader es incluído, el cuál se puede utilizar para programar la placa sin un programador por hardware. Éste bootloader ocupa 4 kb de la memoria Flash del microcontrolador, puede ser eliminado pero teniendo en cuenta que luego no se cuenta con su funcionalidad.
%%Microcontrolador:
%%ATmega32u4
%%
%%\subsection{Arduino Due}
%%Web:
%%http://arduino.cc/en/Main/ArduinoBoardDue
%%Características:
%%Es la primer placa arduino basada en la arquitectura ARM de 32 bits.
%%Tiene 54 pins de entrada/salida digital, 12 de los cuáles pueden ser usados como salidas PWM.
%%12 entradas analógicas, un reloj de 84 MHz integrado, conección USB, 2 convertidores digital-analógico.
%%
%%Un botón de reset y un botón de borrado.
%%A diferencia de otras placas, ésta placa corre con un voltaje de 3.3V.
%%La mejora sustancial con respecto a otras placas, puede ejecutar operaciones sobre 4 bytes en un sólo ciclo de reloj,
%%tiene una frecuencia alta de reloj, 96 kbytes de SRAM, 512 kb de memoria flash para código y
%%un controlador DMA para liberar el CPU de tareas basadas en muchos accesos a memoria.
%%
%%El bootloader que incluye viene de fábrica y está en una ROM dedicada, por lo que no ocupa espacio de la memoria Flash.
%%
%%Microcontrolador: Atmel SAM3X8E ARM Cortex-M3
%%
%%\subsection{Arduino Robot}
%%Web:
%%http://arduino.cc/en/Main/Robot
%%Características:
%%Es el primer Arduino sobre ruedas oficial.
%%Cuenta con dos procesadores, cada uno sobre una placa.
%%Hay una placa utilizada para controlar los motores, y otra placa de control que maneja los
%%sensores y decide como operar.
%%Cada placa se puede programar por separado usando el IDE Arduino.
%%Algunos de los pines de la placa ya están mapeados a sensores y actuadores.
%%  El chasis cuenta con una brújula, un parlante, un panel de control de 5 botones, leds, conexiones I2C,
%%dos ruedas y sensores infrarrojos.
%%  También tiene zonas de prototipado.
%%
%%Microcontrolador: 2 microcontroladores ATmega32u4


\section{Mbed}

Los kits de Mbed están diseñados para prototipar rápidamente. Mbed desarrolló herramientas web colaborativas como ser un IDE web y interfaz web con control de versiones, donde se pueden publicar proyectos, extender y colaborar con proyectos de otros usuarios.
También hay variedad de bibliotecas desarrolladas para los kits mbed, que implementan funcionalidades básicas como ser protocolos de comunicación y otros.

Procesadores y arquitectura:
  ARM Cortex-M3 (ARM)
  ARM Cortex-M0 (ARM)

Herramientas de desarrollo:
  Mbed Online Compiler. Compilador web de mbed para aplicaciones. 
  Entorno web para crear aplicaciones y herramienta de control de versiones basada en Mercurial integrada. Las aplicaciones pueden ser cargadas en las placas usando el entorno web sin necesitar drivers extra ni instalación del compilador.
  SDK C/C++ especializado para mbed.
  También existe un HDK (Hardware development kit) para diseño de hardware especializado.

\subsection{NXP LPC11U24}

Homepage:
https://mbed.org/handbook/mbed-NXP-LPC11U24
Características:
Diseñada para prototipado rápido, programador USB integrado, aplicaciones de bajo consumo eléctrico.
8KB RAM, 32KB FLASH
USB, 2xSPI, I2C, UART, 6xADC, GPIO 
Procesador:
32-bit ARM Cortex-M0, 48MHz

\subsection{NXP LPC1768}
Homepage:
http://mbed.org/platforms/mbed-LPC1768/
Características generales:
Diseñada para prototipado rápido de aplicaciones de microcontroladores en general, Ethernet, USB.
Tiene flexibilidad para varios periféricos y memoria FLASH.
Ethernet, USB Host y Device, 2xSPI, 2xI2C, 3xUART, CAN, 6xPWM, 6xADC, GPIO.
40 pines de entrada/salida.
Interfaz de programación por USB integrada (drag and drop programmer)
Procesador:
32-bit ARM Cortex-M3, 96MHz
Memoria:
512kb FLASH, 32kb RAM



\section{Robotis}

\section {Robotis}

La empresa robotis apunta a desarrollar robots para uso educativo, así como una gama de robots para uso competitivo. Los kits de Robotis están diseñados para uso final, es decir, proveen los controladores, así como los componentes para armar la estructura, sensores y actuadores. También la intención es que para el uso de los kits se utilicen sus herramientas de desarrollo.
Procesadores y arquitectura
  CM-510, controlador ATMega128. Arquitectura AVR.
  CM-530, controlador ARM Cortex M3 de 32bit. Arquitectura ARM.
  CM-100A, controlador ATMega 8.
Lenguajes
  C embebido. Archivos .tsk, lenguaje Task.
Herramientas de desarrollo
  IDE RoboPlus, se pueden generar tareas, movimientos programados del robot en base a movimiento de motores y un diseño tridimensional. Se pueden editar las tareas generadas en el entorno de desarrollo.


\subsection{Bioloid STEM}
Creado para uso educativo y competencias robóticas. El kit provee el hardware y clases enseñando a construir distintos robots para distintos usos, involucrando conceptos de ciencias, tecnología, ingeniería y matemáticas.
Controlador:
CM-530
Componentes:
Sensor Infrarrojo, array de 7 sensores infrarrojos (detectan objetos), alimentación con 6 baterías AA o LR6, control remoto y receptor, 6 motores dinamixel (2 AX-12W de alta velocidad para funcionar como ejes de ruedas y 4 AX-12A). Contiene un kit de piezas y partes para dar estructura, compatible con las piezas del kit Robotis OLLO.

\subsection{Bioloid Premium}
Diseñado para educación, competiciones y entretenimiento. Se pueden construir variedad de robots de ejemplo como humanoide, araña, animales. El kit contiene 29 ejemplos de robot y programas de ejemplo.
Controlador:
CM-530
Componentes:
18 motores dinamixel (AX-12A), batería Li-Po, sensor giroscópico, receptor infrarrojo, control remoto y receptor, sensor de distancia, sensor infrarrojo para detección de objetos. Contiene kit de piezas para armar esqueleto compatibles con OLLO.

\subsection{Bioloid GP}
Humanoide optimizado para competencias robóticas. Esqueleto liviano y resistente. Instrucciones para jugar al fútbol y hacer tareas de recolección pre-programadas. Ajuste automático de postura con sensor giroscópico.
Controlador:
CM-530
Componentes:
18 motores dinamixel (8xAX-12A, 10xAX-18A), batería Li-Po, sensor giroscópico, control remoto y receptor, sensor de distancia, marco de piezas de aluminio.

\subsection{Ollo}
Diseñado para aprender y jugar con robots. Incluye variedad de piezas interconectables. 
Los circuitos están ocultos para poder ser utilizados por niños sin peligro. 
Existen varios kits diseñados especialmente para educación, con currículas de ejercicios variados incrementales. Entre ellos se encuentran el kit Ollo Starter, Explorer, Inventor. Hay kits de entretenimiento como Ollo Figure, Ollo Action y Ollo Bug.
Los ejercicios están organizados para comenzar con el diseño de la estructura del robot, luego aprender nociones de física y ciencias, y luego nociones de programación integrando sensores y actuadores.
Cuenta con una batería de larga duración, la cuál dura hasta 10 horas.
Se integra con el IDE RoboPlus de Robotis, y las piezas son reutilizables con los kits de Robotis Bioloid.
Para programarlo hay que utilizar el software RoboPlus, o generar archivos .tsk respetando el formato del lenguaje de las Tasks de RoboPlus.
Controlador:
CM-100A
Componentes:
Varían según el kit, generalmente todos tienen sensores infrarrojos, el CM-100A tiene 3 conectores para sensores infrarrojos. Tiene lugar para agregar un receptor inalámbrico. También hay motores similares a los dinamixel.





%\section{Lego Mindstorms}
%
\section{Lego}


Kits Lego Mindstorms
Kit:
NXT Intelligent brick
Webpage:
http://shop.lego.com/en-US/NXT-Intelligent-Brick-9841
Procesador:
32-bit ARM7 microprocessor
Interfaces:
Support for Bluetooth, 1 USB 2.0 port, 4 input ports, 3 output ports.
Precio estimado:
US$ 149.99



%\section{Fischertechnik}
%
\subsection{ROBO TX Controller}
Webpage:
http://www.fischertechnik.de/en/Home/info/computing/ROBO-TX-Controller.aspx/usetemplate-1\_column\_no\_pano/
Procesador:
32-bit processor, 200 MHz.


\section{Butiá}

USB4Butiá surgió derivada de un proyecto de grado de Facultad de
Ingeniería – UdelaR.
Como característica principal es un kit económico y con un diseño abierto.
El diseño de la placa está publicado junto con instrucciones para construirla,
además sus componentes se seleccionaron por ser económicos
y de fácil acceso en el medio local.

La placa está pensada principalmente para aumentar capacidades sensoriales
y de actuación del robot Butiá.

Procesadores y arquitectura: PIC 18F4550 Microcontroller
Lenguajes: C embebido.
Herramientas de desarrollo: IDE Tortubots.

\subsection{USB4Butia 1.9}
Diseño abierto con elementos disponibles en el medio local (Uruguay), a relativamente bajo costo.
Está diseñada para incluir periféricos de entrada o salida en las interfaces RJ45, y también puede controlar motores Dinamixel mediante una interfaz serial AX12.
Tiene 8 hackpoints, pines de entrada/salida diseñados para ser manipulados directamente.
Procesador:
PIC 18F4550 Microcontroller
Interfaces
1xUsb, 6xRJ45, 1xAX12, 1xLed, 8xHackpoint




%\section{Microcontroladores}
%

Arquitecturas y microcontroladores
Aquí se presentan las diferentes arquitecturas de los microcontroladores utilizadas en los ejemplos 
ARM
ARM Cortex-M0
Es el procesador más pequeño existente de ARM.
ARM Cortex-M3 32 bit
Diseñado para aplicaciónes de bajo costo y alto desempeño.

Atmel
Basados en ARM Cortex
AT91SAM3X8E:  Basado en ARM Cortex-M3
AVR
Son microcontroladores basados en RISC, la empresa Atmel produce controladores de éstas características. Intentan mantener la ejecución a un ritmo de una instrucción por ciclo de reloj, para lograr velocidades relativas de 1MIPS/MHz. Es una línea de bajo consumo y alto desempeño.
ATmega8
8 kb de Flash de memoria programable, 1kb de SRAM, 512 k de EEPROM, conversión analógica/digital de 10 bits.
Soporta una frecuencia máxima de 16 Mhz a 1 MIPS/MHz.
Opera entre 2v y 3.5 volts. Se recomiendan 3 volts. Equivale a 2 baterías AA o LR.
Web: http://www.atmel.com/devices/atmega8.aspx
ATmega128
Combina 128 kb de Flash, 4kb de SRAM, 4kb de EEPROM, conversión analógica/digital de 10 bits. Soporta hasta 16 MIPS a un máximo de 16 Mhz. (1 MIPS/MHz)
Es un microcontrolador AVR basado en RISC de 8 bit, bajo consumo eléctrico y alto desempeño.
Opera entre 4.5 y 5.5 volts.
Variantes: Atmega128 y ATmega128L.
Web: http://www.atmel.com/devices/atmega128.aspx
ATmega328
Web: http://www.atmel.com/devices/atmega328.aspx
ATmega32u4
Web: www.atmel.com/devices/atmega32u4.aspx‎
ATmega2560
Web: www.atmel.com/devices/atmega2560.aspx
ATmega168
http://www.atmel.com/devices/atmega168.aspx
Variantes: ATmega168, ATmega168V
ATmega328
http://www.atmel.com/devices/atmega328.aspx‎ 
Variantes: ATmega328P, ATmega328V
Microchip
PIC18F4550
Producido también por la empresa Digikey. Frecuencia de 48 MHz (oscilador interno), controlador de 8bit, 32 kbytes de memoria Flash de programa, 2bkytes de memoria RAM, 256 bytes de EEPROM. 


\section{Conclusiones}

  Luego de estudiar las plataformas, se decide utilizar
el lenguaje \texttt{C} con extensiones de \texttt{C++} para facilitar
la portabilidad entre plataformas.
  Al planificar la implementación, se toma en cuenta el objetivo de
tener un tamaño de programa menor a 64 KB y en lo posible menor a 32 KB.
  Se deberá intentar utilizar memoria estática en lo posible, en lugar
de memoria dinámica, ya que no todas las plataformas cuentan con las
bibliotecas estándar necesarias para su manejo.

  Los proyectos libres mejoran la portabilidad, ya que cuentan con
más y mejor documentación, y una comunidad abierta.
  En el caso de particular de \texttt{Robotis} se evaluó la
posibilidad de desarrollar utilizando
\texttt{C++} y si bien es posible, no cuenta con documentación más
que el propio código fuente.

  En el resto de los proyectos, se cuenta con documentación y una
comunidad abierta que apoya la realización de proyectos que extiendan
sus funcionalidades.

