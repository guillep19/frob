

Arquitecturas y microcontroladores
Aquí se presentan las diferentes arquitecturas de los microcontroladores utilizadas en los ejemplos 
ARM
ARM Cortex-M0
Es el procesador más pequeño existente de ARM.
ARM Cortex-M3 32 bit
Diseñado para aplicaciónes de bajo costo y alto desempeño.

Atmel
Basados en ARM Cortex
AT91SAM3X8E:  Basado en ARM Cortex-M3
AVR
Son microcontroladores basados en RISC, la empresa Atmel produce controladores de éstas características. Intentan mantener la ejecución a un ritmo de una instrucción por ciclo de reloj, para lograr velocidades relativas de 1MIPS/MHz. Es una línea de bajo consumo y alto desempeño.
ATmega8
8 kb de Flash de memoria programable, 1kb de SRAM, 512 k de EEPROM, conversión analógica/digital de 10 bits.
Soporta una frecuencia máxima de 16 Mhz a 1 MIPS/MHz.
Opera entre 2v y 3.5 volts. Se recomiendan 3 volts. Equivale a 2 baterías AA o LR.
Web: http://www.atmel.com/devices/atmega8.aspx
ATmega128
Combina 128 kb de Flash, 4kb de SRAM, 4kb de EEPROM, conversión analógica/digital de 10 bits. Soporta hasta 16 MIPS a un máximo de 16 Mhz. (1 MIPS/MHz)
Es un microcontrolador AVR basado en RISC de 8 bit, bajo consumo eléctrico y alto desempeño.
Opera entre 4.5 y 5.5 volts.
Variantes: Atmega128 y ATmega128L.
Web: http://www.atmel.com/devices/atmega128.aspx
ATmega328
Web: http://www.atmel.com/devices/atmega328.aspx
ATmega32u4
Web: www.atmel.com/devices/atmega32u4.aspx‎
ATmega2560
Web: www.atmel.com/devices/atmega2560.aspx
ATmega168
http://www.atmel.com/devices/atmega168.aspx
Variantes: ATmega168, ATmega168V
ATmega328
http://www.atmel.com/devices/atmega328.aspx‎ 
Variantes: ATmega328P, ATmega328V
Microchip
PIC18F4550
Producido también por la empresa Digikey. Frecuencia de 48 MHz (oscilador interno), controlador de 8bit, 32 kbytes de memoria Flash de programa, 2bkytes de memoria RAM, 256 bytes de EEPROM. 
