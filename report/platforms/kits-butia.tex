
  USB4Butiá es una plataforma que surgió de un proyecto de grado de Facultad de
Ingeniería – UdelaR.
  Como característica principal es un kit económico y con un diseño abierto.

  El diseño de la placa está publicado junto con instrucciones para construirla,
además sus componentes se seleccionaron por ser económicos
y de fácil acceso en el medio local.

  La placa está pensada principalmente para aumentar capacidades sensoriales
  y de actuación del robot Butiá \cite{butia}.

  Utiliza el microprocesador \texttt{PIC 18F4550} \cite{pic18} el cuál se programa
  utilizando el lenguaje \texttt{C}.

  Además de la placa \texttt{USB4Butia}, también existen tres versiones
del robot Butiá, así como varias herramientas de desarrollo.
  Entre ellas se encuentran el entorno de desarrollo integrado \texttt{Tortubots} \cite{Tortubots},
  \texttt{Butialo} \cite{Butialo} que permite programar utilizando el lenguaje \textit{Lua},
  se puede utilizar \textit{Python}, y también otro IDE llamado \texttt{Yatay} \cite{yatay}
  también desarrollado como proyecto de la Facultad de Ingeniería.

  La versión actual de la placa es \texttt{USB4Butia 1.9}.
  Todas las versiones del proyecto cuentan con todo el software necesario libre, así
como las instrucciones para construir el hardware, que también es libre.

  El robot \texttt{Butia 1.0} utiliza la placa \texttt{USB4Butia 1.9},
la versión \texttt{Butia 2.0} utiliza una placa \textit{Arduino}, y
la versión \texttt{Butia 3.0} utiliza la placa \texttt{Beagleboard Black}
\cite{beagleboard}.

  En la Tabla \ref{table-usbbutia} se muestran las especificaciones del
\texttt{UBS4Butia 1.9}, ya que \textit{Arduino} ya fue relevado y
\texttt{Beagleboard Black} no es una plataforma con bajas capacidades
de cómputo.

\begin{table}[htbp]
\centering
\scriptsize
\setlength\tabcolsep{2pt}
\caption{Modelo USB4Butia}
\label{table-usbbutia}
\begin{tabular}{|c|c|c|c|c|c|c|}
  \hline
  Modelo & Flash & SRAM & EEPROM & Procesador & Arquitectura & Frecuencia \\
  \hline
  USB4Butia 1.9 & 32 KB & 2 KB & 256 Bytes & PIC 18F4550 & 8 bit & 48 MHz \\
  \hline
 \end{tabular}
\end{table}


