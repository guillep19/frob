
USB4Butiá surgió derivada de un proyecto de grado de Facultad de
Ingeniería – UdelaR.
Como característica principal es un kit económico y con un diseño abierto.
El diseño de la placa está publicado junto con instrucciones para construirla,
además sus componentes se seleccionaron por ser económicos
y de fácil acceso en el medio local.

La placa está pensada principalmente para aumentar capacidades sensoriales
y de actuación del robot Butiá.

Procesadores y arquitectura: PIC 18F4550 Microcontroller
Lenguajes: C embebido.
Herramientas de desarrollo: IDE Tortubots.

\subsection{USB4Butia 1.9}
Diseño abierto con elementos disponibles en el medio local (Uruguay), a relativamente bajo costo.
Está diseñada para incluir periféricos de entrada o salida en las interfaces RJ45, y también puede controlar motores Dinamixel mediante una interfaz serial AX12.
Tiene 8 hackpoints, pines de entrada/salida diseñados para ser manipulados directamente.
Procesador:
PIC 18F4550 Microcontroller
Interfaces
1xUsb, 6xRJ45, 1xAX12, 1xLed, 8xHackpoint


