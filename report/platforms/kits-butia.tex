
  USB4Butiá es una plataforma que surgió de un proyecto de grado de Facultad de
Ingeniería – UdelaR.
  Como característica principal es un kit económico y con un diseño abierto.

  El diseño de la placa está publicado junto con instrucciones para construirla,
además sus componentes se seleccionaron por ser económicos
y de fácil acceso en el medio local.

  La placa está pensada principalmente para aumentar capacidades sensoriales
  y de actuación del robot Butiá \cite{butia}.

  Utiliza el microprocesador \texttt{PIC 18F4550} \cite{pic18} el cuál se programa
  utilizando el lenguaje \texttt{C}.

  Existen actualmente tres versiones del proyecto Butiá, así como varias
  herramientas de desarrollo.
  Entre ellas se encuentran el entorno de desarrollo integrado \texttt{Tortubots} \cite{Tortubots},
  \texttt{Butialo} \cite{Butialo} que permite programar utilizando el lenguaje \textit{Lua},
  se puede utilizar \textit{Python}, y también otro IDE llamado \texttt{Yatay} \cite{yatay}
  también desarrollado como proyecto de la Facultad de Ingeniería.

  La versión actual del proyecto es \texttt{USB4Butia 3.0}.
  Todas las versiones del proyecto cuentan con todo el software necesario libre, así
como las instrucciones para construir el hardware, que también es libre.
