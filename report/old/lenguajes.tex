
\section{Lenguajes}

En esta sección se relevan diversos lenguajes existentes pensados para
programacion de sistemas robóticos.
Para cada lenguaje, se evalua si es un lenguaje especifico
de dominio (DSL), o si es un lenguaje de uso general.

\subsection{pblua}

Implementacion de Lua para lego Mindstorms NXT.

\subsection{elua}

Implementación de Lua para diversos microcontroladores.
Elua significa "Lua embebido".

Abstracciones de los periféricos y funciones estándar, elua está pensado para
correr programas lua directamente sobre el controlador deseado.

Las abstracciones las divide en 'Generic modules' y 'Platform modules', 
los primeros aplican a todos los micro, y el resto corresponden a una
plataforma en particular.

\subsection{leJos}

Java for lego mindstorms
TODO

\subsection{mbedlogo}

Implementacion de maquina virtual de logo para mbed.

\subsection{Robotis Task Code}

Es un lenguaje reducido, similar a C, pensado para programar desde una
interfaz gráfica.

\subsection{Task Description Language (TDL)}

TDL es un lenguaje creado para poder especificar tareas en alto nivel.


