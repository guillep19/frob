
  En esta sección veremos un caso de estudio usado para verificar la
implementación.
  El problema fue tomado de la competencia SumoUY \cite{sumouy}, el mismo
fue el desafío planteado a escolares en el año 2013.

\section {Problema}

  Se desea implementar un robot autónomo móvil que sea capaz de
hacer la entrega de un pedido en una casa determinada.
  El mismo debe moverse por un escenario e identificar las casas.
  Para recorrer la ruta de entrega, podrá valerse de una línea negra
que representará la calle de la ciudad.

  Las casas estarán ubicadas a un lado de la calle. En el recorrido
se encuentran varias casas, el robot deberá entregar un pedido
en la quinta casa por la que pase.

  El robot deberá pasar por alto las casas anteriores y
al llegar a la casa objetivo debe detenerse totalmente.

  Para probar la solución, se armará un escenario que consiste de
un piso blanco con una línea negra que puede tener curvas.

  Al lado derecho de la línea se ubicarán cajas a menos de 30
centímetros representando las casas.

\section {Solución}

Se armó un robot móvil (\ref{fig:deliverybot}) que cuenta con 3 sensores:

\begin{itemize}
\item Sensor de grises izquierdo
\item Sensor de grises derecho
\item Sensor de distancia apuntando hacia la derecha
\end{itemize}

  Y 2 actuadores:

\begin{itemize}
\item Motor izquierdo
\item Motor derecho 
\end{itemize}

\begin{figure}[hbtp]
\begin{center}
  \caption{Diagrama del robot móvil
    (realizado utilizando fritzing \cite{fritzing})
  }
\includegraphics[width=0.9\textwidth]{graphs/breaboardbb.png}
\label{fig:deliverybot}
\end{center}
\end{figure}
 

  El robot utilizará los sensores de grises para mantenerse sobre
la línea y el sensor de distancia para saber cuándo está frente
a una casa.

  Con los motores se moverá hacia adelante inicialmente, e irá corrigiendo
  su dirección desacelerando el motor del lado que se salga de la línea.

  Durante el trayecto contará las casas, y se detendrá cuando la cuenta
llegue al valor 5.

  En la figura \ref{fig:delivery} se puede ver gráficamente de que forma
se combinan los eventos para lograr el objetivo.

\begin{figure}[hbtp]
\begin{center}
\caption{Diagrama del caso de estudio}
\includegraphics[width=0.9\textwidth]{graphs/delivery.png}
\label{fig:delivery}
\end{center}
\end{figure}

  Luego se llega a la implementación en el lenguaje \frob :

\begin{verbatim}

INPUT_DISTANCE = 1
INPUT_COLOR_LEFT = 2
INPUT_COLOR_RIGHT = 3
OUTPUT_ENGINE_LEFT = 1
OUTPUT_ENGINE_RIGHT = 2

MIN_DISTANCE = 100
MIN_GREY = 50

hay_casa d = if (d < MIN_DISTANCE) then 1 else 0
distinto a b = if (a /= b) then 1 else 0
velocidad_casa num = if (num >= 5) then 0 else 100

and a b = if (a && b) then 1 else 0
suma a b = (a + b)
multiplicar a b = (a * b)

color_a_vel gris = if (gris > MIN_GREY) 1 else 1/2


do {
    distance <- read INPUT_DISTANCE,
    color_izq <- read INPUT_COLOR_LEFT,
    color_der <- read INPUT_COLOR_RIGHT,

    viendo_casa <- lift hay_casa distance,
    cambio <- folds distinto 0 viendo_casa,
    nueva_casa <- lift2 and viendo_casa cambio,
    cuenta <- folds suma 0 nueva_casa,
    velocidad <- lift velocidad_casa cuenta,

    multip_izq <- lift color_a_vel color_izq,
    multip_der <- lift color_a_vel color_der,

    speed_left <- lift2 multiplicar velocidad multip_izq,
    speed_right <- lift2 multiplicar velocidad multip_der,

    output MOTOR_IZQ speed_left,
    output MOTOR_DER speed_right
}

\end{verbatim}

\section{Solución sin utilizar \frob}

\begin{verbatim}

INPUT_DISTANCE = 1
INPUT_COLOR_LEFT = 2
INPUT_COLOR_RIGHT = 3
OUTPUT_ENGINE_LEFT = 1
OUTPUT_ENGINE_RIGHT = 2

MIN_DISTANCE = 100
MIN_GREY = 50

hay_casa d = if (d < MIN_DISTANCE) then 1 else 0
distinto a b = if (a /= b) then 1 else 0
velocidad_casa num = if (num >= 5) then 0 else 100
color_a_vel gris = if (gris > MIN_GREY) 1 else 1/2

bool viendo = false
int velocidad = 100
int cuenta = 0

while (cuenta < 5) {
  //chequear si hay casa
  distance = read(INPUT_DISTANCE)
  if (hay_casa(distance)) {
    if (!viendo) {
      cuenta = cuenta + 1
      viendo = true
    }
  } else {
    viendo = false
  }
  //chequear que no estoy fuera de la linea
  color_izq = read(INPUT_COLOR_LEFT)
  color_der = read(INPUT_COLOR_RIGHT)
  if (color_izq > MIN_GREY) {
    write(MOTOR_DER, velocidad)
  } else {
    write(MOTOR_DER, 1/2 * velocidad)
  }
  if (color_der > MIN_GREY) {
    write(MOTOR_DER, velocidad)
  } else {
    write(MOTOR_DER, 1/2 * velocidad)
  }
}
//Detener robot
write(MOTOR_IZQ, 0)
write(MOTOR_DER, 0)

\end{verbatim}


\section {Conclusiones del caso}


