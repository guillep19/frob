
% TODO(Marcos): Decir por que se decidio definir un lenguaje propio y no
% utilizar los que se estudiaron anteriormente.
% Tambien decir que comparte con esos lenguajes y en que difiere.

  En este capítulo se describe el diseño del lenguaje \frob{} junto con su semántica.
  Luego se explica de qué manera es traducido al lenguaje \alf{} de
bajo nivel, mas simple de interpretar, el cual podrá ser interpretado
por implementaciones de una misma máquina virtual en diferentes
plataformas de hardware.

  También se describirán las etapas de compilación, desde que se escribe
un programa en alto nivel hasta que es ejecutado en una
plataforma objetivo.

  El diagrama de la Figura \ref{fig:compilacion} resume todas las etapas y
componentes necesarios.

\begin{figure}[h]
\begin{center}
\caption{Etapas y componentes}
\includegraphics[width=0.9\textwidth]{graphs/compilacion.png}
\label{fig:compilacion}
\end{center}
\end{figure}

  En la parte de arriba de la Figura se ve un programa en el
lenguaje \frob{} de alto nivel.
  El desarrollador escribe dicho programa y
ejecuta el compilador \compilador{} y obtiene un archivo \alf{} binario.

  Debajo se muestran diversas plataformas, cada una
con su implementación de la máquina virtual \maquinavirtual{} instalada.

  El desarrollador podrá cargar el mismo código \alf{} en cualquier robot que
esté construido utilizando cualquiera de las plataformas, y la máquina
virtual se encargará de interpretarlo.


\section {Problema}
``{\itshape Se desea implementar un robot autónomo móvil que sea capaz de
hacer la entrega de un pedido en una casa determinada.
  El mismo debe moverse por un escenario e identificar las casas.
  Para recorrer la ruta de entrega, podrá valerse de una línea negra
que representará la calle de la ciudad.

  Las casas estarán ubicadas a un lado de la calle. En el recorrido
se encuentran varias casas, el robot deberá entregar un pedido
en la quinta casa por la que pase.

  El robot deberá pasar por alto las casas anteriores y
al llegar a la casa objetivo debe detenerse totalmente.

  Para probar la solución, se armará un escenario que consiste de
un piso blanco con una línea negra que puede tener curvas.

  Al lado derecho de la línea se ubicarán cajas a menos de 30
centímetros representando las casas.}''




\section {Solución}

  Se armó un robot móvil que cuenta con 3 sensores:

\begin{itemize}
\item Sensor de grises izquierdo
\item Sensor de grises derecho
\item Sensor de distancia apuntando hacia la derecha
\end{itemize}

  Y 2 actuadores:

\begin{itemize}
\item Motor izquierdo
\item Motor derecho 
\end{itemize}


\begin{figure}[hbtp]
\begin{center}
  \caption{Diagrama del robot móvil
    (realizado utilizando fritzing \cite{fritzing})
  }
\includegraphics[width=0.9\textwidth]{graphs/breaboardbb.png}
\label{fig:breadboard}
\end{center}
\end{figure}




  El diagrama de la Figura \ref{fig:breadboard} muestra los
componentes físicos que son montados en el robot
para resolver el problema y cómo se interconectan.
  Arriba se pueden ver los dos motores, que irán uno a cada lado
del robot y sólo se moverán hacia adelante.
  Se utilizan salidas \textit{pwm} \footnote{PWM: Del inglés, pulse width
modulation; Modulación por ancho de pulsos. Se utiliza para crear señales
de voltaje en ciclos periódicos y controlar la cantidad de energía que
se envía.} del MBED para controlar la velocidad de cada motor.

  Los motores necesitan más energía que la que se puede entregar con
los pines de salida del MBED, y para ésto tienen su propia fuente de
voltaje.
  Se utilizan dos transistores para amplificar la señal que
controla cada motor.

  El robot utilizará dos sensores de grises montados al frente
para mantenerse sobre la línea, ambos pueden verse a la derecha abajo
en la Figura \ref{fig:breadboard}.
  Con los motores el robot se moverá hacia adelante inicialmente, e
irá corrigiendo su dirección desacelerando el motor del lado que
se salga de la línea.
  Junto a cada sensor de grises se montará una luz led, que de acuerdo
al color del suelo, se reflejará y se podrá decidir si se está viendo
algo oscuro (la línea) o algo claro (fuera de la línea).

  El sensor de distancia a la izquierda debajo en la figura, se montará
en el robot apuntando hacia la derecha, para saber cuándo el mismo
está pasando frente a una casa.

  Durante el trayecto se mantendrá la cuenta de las casas, y el robot
se detendrá totalmente cuando la cuenta llegue al valor 5.

  En la Figura \ref{fig:robotfisico} se puede ver el
robot físico creado como prototipo para probar el caso de estudio.

\begin{figure}[!htb]
\begin{center}
\caption{Robot físico implementado}
\includegraphics[width=0.9\textwidth]{graphs/alf.jpg}
\label{fig:robotfisico}
\end{center}
\end{figure}




\newpage
\subsection{Implementación usando \frob{}}
  Luego se llega a la implementación en el lenguaje \frob{}:

\begin{verbatim}

INPUT_DISTANCE = 1
INPUT_COLOR_LEFT = 2
INPUT_COLOR_RIGHT = 3
OUTPUT_ENGINE_LEFT = 1
OUTPUT_ENGINE_RIGHT = 2

MIN_DISTANCE = 100
MIN_GREY = 50

hay_casa d = if (d < MIN_DISTANCE) then 1 else 0
distinto a b = if (a /= b) then 1 else 0
velocidad_casa num = if (num >= 5) then 0 else 100

and a b = if (a && b) then 1 else 0
suma a b = (a + b)
multiplicar a b = (a * b)

color_a_vel gris = if (gris > MIN_GREY) 2 else 1


do {
    distance <- read INPUT_DISTANCE,
    color_izq <- read INPUT_COLOR_LEFT,
    color_der <- read INPUT_COLOR_RIGHT,

    viendo_casa <- lift hay_casa distance,
    cambio <- folds distinto 0 viendo_casa,
    nueva_casa <- lift2 and viendo_casa cambio,
    cuenta <- folds suma 0 nueva_casa,
    velocidad <- lift velocidad_casa cuenta,

    multip_izq <- lift color_a_vel color_izq,
    multip_der <- lift color_a_vel color_der,

    speed_left <- lift2 multiplicar velocidad multip_izq,
    speed_right <- lift2 multiplicar velocidad multip_der,

    output MOTOR_IZQ speed_left,
    output MOTOR_DER speed_right
}

\end{verbatim}




\subsection{Diagrama de la solución}
  Utilizando la notación definida en la Sección \ref{section:diseno},
en la Figura \ref{fig:delivery} se puede ver gráficamente de qué forma
se combinan las señales para lograr el objetivo.

\begin{center}
\begin{tikzpicture}
\selectlanguage{english}
  %\draw[step=1cm,gray,very thin,xshift=0cm,yshift=0cm] (0,0) grid (15,16);

\begin{scope}[xshift=0cm,yshift=0cm,very thick,
    node distance=2cm,on grid,>=stealth',
    block/.style={rectangle,draw,fill=cyan!20},
    value/.style={rectangle,draw,fill=red!20},
    func/.style={rectangle,draw,fill=green!20},
    comp/.style={circle,draw,fill=orange!40}]

  \node [block] (i1) [xshift=6cm,yshift=14cm] {$\texttt{I}_{\texttt{INPUT\_DISTANCE}}$};
  \node [block] (i2) [xshift=1cm,yshift=14cm] {$\texttt{I}_{\texttt{INPUT\_COLOR\_LEFT}}$};
  \node [block] (i3) [xshift=13cm,yshift=14cm] {$\texttt{I}_{\texttt{INPUT\_COLOR\_RIGHT}}$};

  \node [comp] (ca1) [xshift=6cm,yshift=12cm] {\tiny{\texttt{distance}}};
  \node [comp] (ca2) [xshift=1cm,yshift=12cm] {\tiny{\texttt{color\_izq}}};
  \node [comp] (ca3) [xshift=13cm,yshift=12cm] {\tiny{\texttt{color\_der}}};

  \node [comp] (ca4) [xshift=6cm,yshift=10cm] {\tiny{\texttt{viendo\_casa}}};
  \node [comp] (ca7) [xshift=8cm,yshift=10cm] {\tiny{\texttt{cambio}}};
  \node [comp] (ca5) [xshift=1cm,yshift=10cm] {\tiny{\texttt{multip\_izq}}};
  \node [comp] (ca6) [xshift=13cm,yshift=10cm] {\tiny{\texttt{multip\_der}}};

  \node [comp] (ca8) [xshift=7cm,yshift=8cm] {\tiny{\texttt{nueva\_casa}}};
  \node [comp] (ca9) [xshift=7cm,yshift=6cm] {\tiny{\texttt{cuenta}}};
  \node [comp] (ca10) [xshift=7cm,yshift=4cm] {\tiny{\texttt{velocidad}}};

  \node [comp] (ca11) [xshift=4cm,yshift=2cm] {\tiny{\texttt{speed\_left}}};
  \node [comp] (ca12) [xshift=10cm,yshift=2cm] {\tiny{\texttt{speed\_right}}};
  
  \node [block] (o1) [xshift=4cm,xshift=0cm] {$\texttt{O}_{\texttt{MOTOR\_IZQ}}$};
  \node [block] (o2) [xshift=10cm,xshift=0cm] {$\texttt{O}_{\texttt{MOTOR\_DER}}$};
  
  \node [func]  (f1) [xshift=8cm, yshift=11.5cm] {\tiny{distinto}};
  \node [value] (v1) [right=of ca7,xshift=0.5cm] {$v_i (v_0 = 0)$};
  \draw[-,line width=1pt] (ca7) -- node[below]{\tiny{state}} (v1);

  \node [func]  (f2) [left=of ca4] {\tiny{hay\_casa}};
  \node [func]  (f3) [left=of ca5] {\tiny{color\_a\_vel}};
  \node [func]  (f4) [right=of ca6] {\tiny{color\_a\_vel}};

  \node [func]  (f5) [right=of ca8] {\tiny{and}};
  \node [func]  (f6) [right=of ca9] {\tiny{suma}};
  \node [value] (v2) [left=of ca9,xshift=-0.5cm] {$v_i (v_0 = 0)$};
  \draw[-,line width=1pt] (ca9) -- node[below]{\tiny{state}} (v2);
  \node [func]  (f7) [right=of ca10] {\tiny{velocidad\_casa}};

  \node [func]  (f8) [left=of ca11] {\tiny{multiplicar}};
  \node [func]  (f9) [right=of ca12] {\tiny{multiplicar}};

  %funciones:
  \draw[-] (f2) -- (ca4);
  \draw[-] (f3) -- (ca5);
  \draw[-] (f4) -- (ca6);
  \draw[-] (f5) -- (ca8);
  \draw[-] (f6) -- (ca9);
  \draw[-] (f7) -- (ca10);
  \draw[-] (f8) -- (ca11);
  \draw[-] (f9) -- (ca12);

  \draw[-] (ca1) -- (i1);
  \draw[-] (ca2) -- (i2);
  \draw[-] (ca3) -- (i3);
  \draw[-] (ca4) -- (ca1);
  \draw[-] (ca5) -- (ca2);
  \draw[-] (ca6) -- (ca3);
  \draw[-] (ca7) -- (f1);
  \draw[-] (ca7) -- (ca4);
  \draw[-] (ca8) -- (ca7);
  \draw[-] (ca8) -- (ca4);
  \draw[-] (ca9) -- (ca8);
  \draw[-] (ca10) -- (ca9);
  \draw[-] (ca11) -- (ca5);
  \draw[-] (ca12) -- (ca6);
  \draw[-] (ca11) -- (ca10);
  \draw[-] (ca12) -- (ca10);
  \draw[-] (ca11) -- (o1);
  \draw[-] (ca12) -- (o2);
 \end{scope} 

\selectlanguage{spanish}
\end{tikzpicture}
\end{center}


% TODO(Marcos): Explicar el diagrama, asociarlo con el codigo


%\section{Solución sin utilizar \frob}

%\begin{verbatim}

%\end{verbatim}

% TODO(Marcos,Jorge): Tenemos alguna version escrita en C y/o tortugarte para
% Explicar el programa, la mecanica de como se cuentan
% las casas. Poner un ejemplo imperativo.

\section {Conclusiones del caso}

Intro .... que se hizo, puntos a favor, etc..

\section{Trabajo futuro}

El principal trabajo futuro sería ...

Sería muy útil contar con una funcionalidad de depuración, la cuál
mostrara dependiendo del tiempo los valores de cada fuente de eventos.

Una opción es comunicar mediante el puerto serial el valor de cada
señal al cambiar, y mostrarlo en una interfaz web como la que provee
RXMarbles (ver \cite{rxmarbles}). 
El lenguaje Elm provee de una herramienta que permite viajar en el 
tiempo, modificar y mostrar la ejecución de un programa, en nuestro
caso no sería posible modificar lo que el robot físico realiza, pero
si sería útil ver en la línea de tiempo que valores tomaron sus
señales. (ver \cite{elmdebug})


