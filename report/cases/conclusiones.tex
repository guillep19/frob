  A diferencia de un programa imperativo tradicional, en el programa
\frob{} dentro del bloque \texttt{do} se puede ver claramente cómo
se procesan las entradas, para generar los valores de las salidas.
  Al razonar utilizando funciones sobre señales, no es necesario
tener en cuenta estado global, ni el orden de ejecución de
diferentes rutinas.

  Al estar obligado a escribir funciones puras, el desarrollador
puede abstraerse mejor al pensar que funciones necesita implementar,
y que entradas tomarán. En un programa imperativo, es normal realizar
operaciones de entrada y salida dentro de cualquier función, lo que
dificulta ver cuándo hay efectos secundarios de invocar cada función.

  Finalmente, no es necesario preocuparse por la concurrencia, sino
por la semántica del programa, la máquina virtual se encargará de
respetar las definiciones del desarrollador.

  Tampoco es necesario preocuparse por las interacciones de entrada
y salida, ya que por decisión de diseño existen abstracciones en
la máquina bien definidas para cada una.

  Si bien \frob{} es muy diferente a los lenguajes de
programación imperativos tradicionales y ésto puede ocasionar una
curva inicial de aprendizaje pronunciada, finalmente puede enseñar al
desarrollador una forma diferente de razonar y pensar la solución
a los problemas de robótica.

  El problema se pudo representar en la plataforma \texttt{MBED LPC1768} antes
mencionada, utilizando el robot construido.
  La memoria Flash total utilizada en la prueba sumó 28 KB, por lo que
aún faltaría mejorar la implementación para llegar a menos de 16 KB y
cubrir mas plataformas. Se debe tener en cuenta que dependiendo de la
plataforma el uso de memoria Flash va a ser diferente.
  Sin embargo el programa \frob{} compilado a \alf{} ocupa tan solo 88 Bytes,
el hecho de que se utilicen abstracciones para entrada y salida, ayuda a mantener
bajo el tamaño de los programas \alf{}. El resto es ocupado por la implementación de
la máquina virtual.
  La memoria RAM utilizada, al usar memoria estática no supera los 512 Bytes, y no
crece con el correr del tiempo.

  Éstos resultados indican que es posible realizar desafíos robóticos y se puede
escalar en tamaño, quedando como necesidad, reducir aún más el tamaño de la
máquina virtual.
