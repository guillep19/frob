Este verificador también implementa operaciones entre sistemas de transiciones, de forma de poder
trabajar con sistemas de transiciones complejos generados a partir de otros más simples.
Las operaciones implementadas son el Intercalado y el \textit{Handshaking}. A continuación
 se muestra la ejecución de cada una de ellas.

\subsection{Intercalado}
Para realizar el intercalado de dos sistemas de transiciones se debe ejecutar el siguiente comando:

\fbox{\parbox{\textwidth}{\scriptsize\verbatiminput{manual/comandos/comandoIC.txt}}}

Donde \texttt{TS1.graphml} y \texttt{TS2.graphml} son archivos previamente generados en los
 cuales se representan sistemas de transiciones en formato \textit{GraphML}, e
 \texttt{intercalado.graphml} indica el archivo en el cual se escribirá el sistema de transiciones
 resultado del intercalado.

\subsection{\textit{Handshaking}}
Para realizar el \textit{Handshaking} de dos sistemas de transiciones se debe ejecutar el
 siguiente comando:

\fbox{\parbox{\textwidth}{\scriptsize\verbatiminput{manual/comandos/comandoHS.txt}}}

Donde \texttt{TS1.graphml} y \texttt{TS2.graphml} son archivos previamente generados en los
 cuales se representan sistemas de transiciones en formato \textit{GraphML}, y
 \texttt{handshaking.graphml} indica el archivo en el cual se escribirá el sistema de transiciones
 resultado del \textit{Handshaking}.
