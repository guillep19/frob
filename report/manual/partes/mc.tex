Una vez que se tiene el sistema de transiciones en formato \textit{GraphML} se puede comenzar
 a verificar propiedades sobre dicho sistema.
Para esto de debe tener en cuenta que hay dos lenguajes implementados para representar propiedades,
 estos son LTL y CTL.
A continuación se muestra la ejecución del verificador para cada uno de estos lenguajes.

\subsection{Propiedades en LTL}
Para verificar las propiedades en LTL se debe ejecutar el verificador de la siguiente forma:

\fbox{\parbox{\textwidth}{\scriptsize\verbatiminput{manual/comandos/comandoLTL.txt}}}

Donde \texttt{archivo.graphml} es el archivo donde se encuentra representado el sistema de
 transiciones y \texttt{formula LTL} es la fórmula LTL expresada según la sección \ref{cap:imp_ltl}.


\subsection{Propiedades en CTL}
Para verificar las propiedades en CTL se debe ejecutar el verificador de la siguiente forma:

\fbox{\parbox{\textwidth}{\scriptsize\verbatiminput{manual/comandos/comandoCTL.txt}}}

Donde \texttt{archivo.graphml} es el archivo donde se encuentra representado el sistema de
 transiciones y \texttt{formula CTL} es la fórmula CTL expresada según la sección \ref{cap:imp_ctl}.

\subsection{Resultados de la ejecución}
Para la verificación de una propiedad existen dos posibles resultados, estas son que el sistema
cumpla la propiedad, o que no la cumpla.
En caso de que el sistema cumpla la propiedad se muestra el siguiente mensaje:
\fbox{\parbox{\textwidth}{\scriptsize\verbatiminput{manual/salidas/cumple.txt}}}

Mientras que si el sistema no cumple la propiedad se muestra el mensaje:
\fbox{\parbox{\textwidth}{\scriptsize\verbatiminput{manual/salidas/no_cumple.txt}}}

\subsubsection{Testigo o contrajemplo}
Además de indicar si el sistema cumple o no la propiedad, el verificador proporciona
 en caso de ser necesario una ejecución testigo de la respuesta dada.

Si se está verificando una propiedad LTL se proporciona una contraejemplo en caso de
 que el sistema no cumpla la propiedad.

Para el caso de CTL si se está verificando una propiedad cuyo cuantificador es universal
 se proporciona un contraejemplo en caso de que la propiedad no se cumpla, mientras que
 si el cuantificador de la propiedad es existencial se proporciona un testigo en caso
 de que la misma se cumpla.
