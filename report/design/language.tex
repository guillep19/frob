%%En \frob{} un programa constará de dos secciones, una sección de
%%declaraciones, donde se declararán funciones y una sección donde se
%%aplicarán combinadores de programación funcional reactiva, especificando
%%cómo son transformadas las señales para especificar el comportamiento de
%%los robots.

  En \frob{} los programas consisten de un conjunto de funciones, y un
conjunto de primitivas del paradigma de programación funcional reactiva.
  
  Es un lenguaje funcional, tipado y con inferencia de tipos.
  En el todos los valores son inmutables, una vez que son declarados
no se pueden modificar.
  Las funciones declaradas son puras, por lo tanto no pueden tener efectos
secundarios.

  Para controlar un robot se debe declarar un conjunto de señales utilizando
las primitivas de FRP.

  La primitiva \texttt{read} crea una señal a partir de un sensor del robot.
  La primitiva \texttt{output} envía a un actuador, el valor de una señal.

  Para simplificar la implementación, solo se permiten valores
naturales ($\mathcal{N}$), y funciones de naturales en naturales. 
($\mathcal{N} \rightarrow \mathcal{N}$).

  El lenguaje evalúa las expresiones, tan pronto como es posible. Las
funciones tienen un conjunto de variables libres, cuando un valor les es
asignado, el resultado es calculado.
  
  En lenguajes como Haskell las expresiones no son evaluadas hasta que es
estrictamente necesario (evaluación a demanda,
del inglés: \emph{lazy evaluation}). En la programación funcional reactiva
esto puede llevar a aumentar el uso de memoria considerablemente, y al no
evaluar incrementalmente, al necesitar un valor puede demorar el cálculo,
enlenteciendo la evaluación de todo el programa y reduciendo la reactividad.

