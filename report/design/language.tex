  En \frob{} los programas consisten de un conjunto de funciones, y un
conjunto de aplicaciones de primitivas del paradigma de programación
funcional reactiva.
 
  Es un lenguaje funcional, tipado y con inferencia de tipos.
  En él todos los valores son inmutables, una vez que son declarados
no se pueden modificar.
  Las funciones declaradas son puras, por lo tanto no pueden tener efectos
secundarios.

  Para controlar un robot se debe declarar un conjunto de señales utilizando
las primitivas de FRP.

  Para simplificar la implementación dentro del alcance del proyecto,
en el lenguaje solo se permitirán valores
naturales ($\mathcal{N}$), y funciones de naturales en naturales.
($\mathcal{N} \rightarrow \mathcal{N}$).

  El lenguaje evalúa las expresiones, tan pronto como es posible. Las
funciones tienen un conjunto de variables libres, cuando un valor les es
asignado, el resultado es calculado.
  Para invocar una función es necesario saturarla, es decir, invocarla con
todos los parámetros que declara.
  
%  En lenguajes como Haskell las expresiones no son evaluadas hasta que es
%estrictamente necesario (evaluación a demanda,
%del inglés: \emph{lazy evaluation}). En la programación funcional reactiva
%esto puede llevar a aumentar el uso de memoria considerablemente, y al no
%evaluar incrementalmente, al necesitar un valor puede demorar el cálculo,
%enlenteciendo la evaluación de todo el programa y reduciendo la reactividad.
\begin{figure}[h!]
  \begin{center}
    \caption{Gramatica de \frob{}}
    \begin{Verbatim}[frame=single]
program := definitions "do {" frps "}";

definitions := definition | definition definitions;

definition := ident arg_list "=" expr;

frps = frp | frp "," frps;

frp := ident "<- read" expr
     | ident "<- lift" ident ident
     | ident "<- lift2" ident ident ident
     | ident "<- folds" ident value ident
     | "output" expr ident

expr := name
      | number
      | expr binop expr
      | "if" expr "then" expr "else" expr;

arg_list := "" | ident arg_list;

ident := [a-z_A-Z]+;

number := [-+]?[0-9]+;

binop := '+' | '-' | '/' | '*' | 'or' | 'and'
       | '==' | '<=' | '>' | '<' | '<>' | '>=';
     \end{Verbatim}
   \label{fig:grammar}
   \end{center}
 \end{figure}


  Un comentario es una línea que comienza con el símbolo `\#'.
  La gramática completa del lenguaje se puede ver en la
Figura \ref{fig:grammar}.

  El sistema de tipos respeta las reglas de inferencia definidas en
la Figura \ref{fig:typesystem}.

\begin{figure}[h!]
  \begin{center}
    \caption{Sistema de tipos en \frob{}}

\[
  \infer[\mathrm{Unidad}]
  {\Gamma \vdash () : unit}{}
\]

\[
  \infer[\mathrm{Numero}]
  {\Gamma \vdash (n : \textit{Number})}{n\ \textit{es un entero}}
\]

\[
  \infer[\mathrm{Variable}]
  {\Gamma \vdash (v : \mu)}{\Gamma(v) = \mu}
\]

\[
  \infer[\mathrm{Aplicacion}]
  {\Gamma \vdash (f x : \nu)}
  {\Gamma \vdash f : \mu \rightarrow \nu & \Gamma \vdash x : \mu}
\]

\[
  \infer[\mathrm{Condicional}]
  {\Gamma \vdash (\texttt{if}\ c\ \texttt{then}\ a\ \texttt{else}\ b : \mu)}
  {\Gamma \vdash c : \nu & \Gamma \vdash a : \mu & \Gamma \vdash b : \mu}
\]

\[
  \infer[\mathrm{Entrada}]
  {\Gamma \vdash \texttt{input} i : \texttt{signal}\ \tau}
  {\Gamma (i) = \tau}
\]

\[
  \infer[\mathrm{Lift_n}]
{\Gamma \vdash \texttt{lift}_n\ f\ i_1\ \dotsb\ i_n : \texttt{signal}\ \nu}
  {\Gamma \vdash f : \mu_1 \rightarrow \dotsb \rightarrow \mu_n \rightarrow \nu & \Gamma \vdash i_1 : \texttt{signal}\ \mu_1 \dotsb \Gamma \vdash i_n : \texttt{signal}\ \mu_n}
\]

\[
  \infer[\mathrm{Folds}]
  {\Gamma \vdash \texttt{folds}\ f\ c\ s : \texttt{signal}\ \nu}
  {\Gamma \vdash f : \mu \rightarrow \nu \rightarrow \nu &
   \Gamma \vdash c : \nu &
   \Gamma \vdash s : \texttt{signal}\ \mu}
\]

\[
  \infer[\mathrm{Salida}]
  {\Gamma \vdash \texttt{output}\ o\ s : unit}
  {\Gamma(o) = \tau & \Gamma \vdash s : \texttt{signal}\ \tau}
\]

   \label{fig:typesystem}
   \end{center}
 \end{figure}


