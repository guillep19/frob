  El lenguaje \frob{} es funcional, tipado y con inferencia de tipos. No
cuenta con evaluación a demanda (\emph{lazy evaluation}).
  Para simplificar la implementación, solo se permiten valores
naturales ($\mathcal{N}$), y funciones de naturales en naturales. 
($\mathcal{N} \rightarrow \mathcal{N}$).

  Un programa constará de dos secciones, una sección de
declaraciones, donde se declararán funciones y una sección donde se
aplicarán combinadores de programación funcional reactiva, especificando
cómo son transformadas las señales para especificar el comportamiento de
los robots.

  En \frob{} todos los valores son inmutables, una vez que son declarados
no se pueden modificar.
  Las funciones declaradas son puras, por lo tanto no pueden tener efectos
secundarios.
  
TODO: Explicar que es funcional, \ref{fig:grammar} sin currying, es tipado, se infieren
los tipos.

\input{design/grammar.tex}

\subsection{Declaraciones}

  Las funciones se declaran de la siguiente manera.

\begin{verbatim}
nombre argumento_1 .. argumento_n = expresion
\end{verbatim}

  Donde una expresión puede ser un valor,
una expresión aritmética (por ejemplo una suma o multiplicación),
una aplicación de una función, o una expresión condicional.\\

%  La sintaxis es muy similar a la del lenguaje \texttt{Haskell}, aunque
% no se permiten funciones anónimas,

  Para declarar un valor constante simplemente se escribe:

\begin{verbatim}
    NOMBRE_CONSTANTE = valor
\end{verbatim}

  Ejemplo de declaración:

\begin{verbatim}
    # fibonacci
    fibo n = if (n < 2) then 1 else fibo(n-1) + fibo(n-2)
\end{verbatim}


