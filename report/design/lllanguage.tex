  Al compilar un programa \frob{}, se obtiene como salida un código
intermedio en lenguaje \alf{}.
  El mismo es independiente de la plataforma en la que va a ser ejecutado.
  Para lograr ésto, se define el lenguaje como un conjunto de instrucciones
con su semántica y una máquina virtual abstracta que las ejecuta.



\subsubsection{Máquina virtual}
  
  
  La máquina virtual será la encargada de recibir el bytecode creado por
el compilador, e interpretarlo en la plataforma que esté ejecutando.
  A diferencia del compilador, es necesario implementar una máquina virtual
para cada arquitectura objetivo.\\

  Por ejemplo, para ejecutar programas en un robot
  con un procesador \emph{arduino}, debe
  existir una implementación de la máquina para ese modelo
  de \emph{arduino}.

  Al momento de implementar la máquina, se tomará en cuenta ésto para
  factorizar partes en común y sólo implementar por arquitectura, las
  partes que realmente sean diferentes como ser la comunicación con
  los periféricos de entrada/salida y las llamadas al sistema.


TODO: Describir el diseño de la máquina virtual.

  La máquina virtual va a ser basada en stack. Ésto significa
que las operaciones van a tomar sus argumentos del stack y
colocar resultados en el mismo.



  Las instrucciones están formadas por un código, un argumento inmediato
  opcional y una lista de argumentos extra opcionales dependiendo del
  código.

  Se usa la siguiente notación para describir las instrucciones:
  \begin{center}
    $\texttt{codigo} [\textit{inmediato}] [, {arg}_1, \dotsb, {arg}_n ]$
  \end{center}

   Se distingue entre los argumentos extra y el inmediato, ya que el
   tamaño de los argumentos extra es el doble del inmediato.
   Para distinguirlos, se utiliza el símbolo `\texttt{,}'.

\subsection{Conjunto de Instrucciones}

\subsubsection{Instrucciones básicas}

\begin{itemize}

\item {
    \texttt{push} ,\textit{value}
    La instrucción \texttt{push} coloca el
    valor \textit{value} como tope del stack.
    En el diagrama a la izquierda se muestra el estado del stack antes
    de la operación y a la derecha el estado luego de su ejecución.

    \begin{tikzpicture}
\selectlanguage{english}
  %%\draw[step=1cm,gray,very thin,xshift=0cm,yshift=0cm] (0,0) grid (12,4);
  \begin{scope}[xshift=0cm,yshift=0cm,very thick,
    node distance=2cm,on grid,>=stealth',
    block/.style={rectangle,draw,fill=cyan!20},
    comp/.style={circle,draw,fill=orange!40},
    stack/.style={rectangle split,rectangle split parts=#1,draw,anchor=center}]
  \node[stack=4,xshift=2cm,yshift=2cm,
        text width=1.1cm,align=center,text height=0.3cm] {
    \nodepart{one} \vphantom{val}
    \nodepart{two} $s_n$
    \nodepart{three} $\dotsb$
    \nodepart{four} $s_o$
  };
 \draw [->,line width=1pt] (0.5,2.2) -- node[below]{\tiny{TOS}} (1.3, 2.2);
 \end{scope} 
 \draw [->,line width=1.7pt] (5,2) -- (7,2);
 \begin{scope}[xshift=8cm,yshift=0cm,very thick,
    node distance=2cm,on grid,>=stealth',
    block/.style={rectangle,draw,fill=cyan!20},
    comp/.style={circle,draw,fill=orange!40},
  stack/.style={rectangle split,rectangle split parts=#1,draw,anchor=center}]
  \node[stack=4,xshift=2cm,yshift=2cm,text width=1.1cm,align=center]  {
    \nodepart{one} \textit{value}
    \nodepart{two} $s_n$
    \nodepart{three} $\dotsb$
    \nodepart{four} $s_o$
   };
 \draw [->,line width=1pt] (0.5,2.8) -- node[below]{\tiny{TOS}} (1.3, 2.8);
   \end{scope} 
\selectlanguage{spanish}
\end{tikzpicture}



}
\item {
    \texttt{add}

      Remueve dos valores del stack, los suma y coloca el resultado en
    el tope.

    \begin{tikzpicture}
\selectlanguage{english}
  %%\draw[step=1cm,gray,very thin,xshift=0cm,yshift=0cm] (0,0) grid (12,4);
  \begin{scope}[xshift=0cm,yshift=0cm,very thick,
    node distance=2cm,on grid,>=stealth',
    block/.style={rectangle,draw,fill=cyan!20},
    comp/.style={circle,draw,fill=orange!40},
    stack/.style={rectangle split,rectangle split parts=#1,draw,anchor=center}]
  \node[stack=4,xshift=2cm,yshift=2cm,
        text width=1.1cm,align=center,text height=0.3cm] {
    \nodepart{one} $s_n$
    \nodepart{two} $s_{n-1}$
    \nodepart{three} $\dotsb$
    \nodepart{four} $s_o$
  };
 \draw [->,line width=1pt] (0.5,2.9) -- node[below]{\tiny{TOS}} (1.3, 2.9);
 \end{scope} 
 \draw [->,line width=1.7pt] (5,2) -- (7,2);
 \begin{scope}[xshift=8cm,yshift=0cm,very thick,
    node distance=2cm,on grid,>=stealth',
    block/.style={rectangle,draw,fill=cyan!20},
    comp/.style={circle,draw,fill=orange!40},
  stack/.style={rectangle split,rectangle split parts=#1,draw,anchor=center}]
  \node[stack=4,xshift=2cm,yshift=2cm,text width=1.1cm,align=center]  {
    \nodepart{one} \vphantom{fruit}
    \nodepart{two} \tiny{$s_n + s_{n-1}$}
    \nodepart{three} $\dotsb$
    \nodepart{four} $s_o$
   };
 \draw [->,line width=1pt] (0.5,2.2) -- node[below]{\tiny{TOS}} (1.3, 2.2);
   \end{scope} 
\selectlanguage{spanish}
\end{tikzpicture}



}
\item {
    \texttt{mul}

      Remueve dos valores del stack, los multiplica y coloca el resultado en
    el tope.

    \begin{tikzpicture}
\selectlanguage{english}
  %%\draw[step=1cm,gray,very thin,xshift=0cm,yshift=0cm] (0,0) grid (12,4);
  \begin{scope}[xshift=0cm,yshift=0cm,very thick,
    node distance=2cm,on grid,>=stealth',
    block/.style={rectangle,draw,fill=cyan!20},
    comp/.style={circle,draw,fill=orange!40},
    stack/.style={rectangle split,rectangle split parts=#1,draw,anchor=center}]
  \node[stack=4,xshift=2cm,yshift=2cm,
        text width=1.1cm,align=center,text height=0.3cm] {
    \nodepart{one} $s_n$
    \nodepart{two} $s_{n-1}$
    \nodepart{three} $\dotsb$
    \nodepart{four} $s_o$
  };
 \draw [->,line width=1pt] (0,2.9) -- node[below]{\small{TOS}} (1.2, 2.9);
 \end{scope} 
 \draw [->,line width=1.7pt] (5,2) -- (7,2);
 \begin{scope}[xshift=8cm,yshift=0cm,very thick,
    node distance=2cm,on grid,>=stealth',
    block/.style={rectangle,draw,fill=cyan!20},
    comp/.style={circle,draw,fill=orange!40},
  stack/.style={rectangle split,rectangle split parts=#1,draw,anchor=center}]
  \node[stack=4,xshift=2cm,yshift=2cm,text width=1.1cm,align=center]  {
    \nodepart{one} \vphantom{fruit}
    \nodepart{two} \tiny{$s_n * s_{n-1}$}
    \nodepart{three} $\dotsb$
    \nodepart{four} $s_o$
   };
 \draw [->,line width=1pt] (0,2.2) -- node[below]{\small{TOS}} (1.2, 2.2);
   \end{scope} 
\selectlanguage{spanish}
\end{tikzpicture}



}
\end{itemize}

\subsection{Instrucciones para manipular señales}
  
  A continuación se presentan las instrucciones utilizadas para manipular
señales.
  \begin{itemize}

\item {
    \texttt{read} \textit{id}

      Toma el tope del stack como identificador de una entrada.
      Crea una señal \textit{id} que contendrá el valor de la entrada
    en el tiempo.
      Como precondición, la señal \textit{id} no debe existir.

    \begin{tikzpicture}
\selectlanguage{english}
  %%\draw[step=1cm,gray,very thin,xshift=0cm,yshift=0cm] (0,0) grid (12,4);
  \begin{scope}[xshift=0cm,yshift=0cm,very thick,
    node distance=2cm,on grid,>=stealth',
    block/.style={rectangle,draw,fill=cyan!20},
    comp/.style={circle,draw,fill=orange!40},
    stack/.style={rectangle split,rectangle split parts=#1,draw,anchor=center}]
  \node[stack=5,yshift=2cm,xshift=4cm,align=center,
        text height=0.3cm,
        text width=1.1cm]  {
    \nodepart{one} \vphantom{fruit}
    \nodepart{two} $s_n$
    \nodepart{three} $s_{n-1}$
    \nodepart{four} $\dotsb$
    \nodepart{five} $s_0$
  };
   \node [block] (s2) [yshift=3cm,xshift=1cm] {$\texttt{I}_{s_n}$};
 \draw [->,line width=1pt] (2.6,2.5) -- node[below]{\tiny{TOS}} (3.3, 2.5);
 \end{scope} 

 \draw [->,line width=1.7pt] (5,2) -- (7,2);

 \begin{scope}[xshift=7cm,yshift=0cm,very thick,
    node distance=2cm,on grid,>=stealth',
    block/.style={rectangle,draw,fill=cyan!20},
    comp/.style={circle,draw,fill=orange!40},
  stack/.style={rectangle split,rectangle split parts=#1,draw,anchor=center}]
  \node[stack=5,yshift=2cm,xshift=4cm,align=center,
        text height=0.3cm,
        text width=1.1cm]  {
    \nodepart{one} \vphantom{fruit}
    \nodepart{two} \vphantom{fruit}
    \nodepart{three} $s_{n-1}$
    \nodepart{four} $\dotsb$
    \nodepart{five} $s_0$
  };
   \node [comp]  (ca2) [xshift=1cm,yshift=1cm] {$\texttt{S}_{id}$};
   \node [block] (s2) [above=of ca2]   {$\texttt{I}_{s_n}$} edge [->] (ca2);
   \draw [->,line width=1pt] (2.6,1.9) -- node[below]{\tiny{TOS}} (3.3, 1.9);
   \end{scope} 
\selectlanguage{spanish}
\end{tikzpicture}



}
\item {
    \texttt{lift} \textit{id}, \textit{src} \textit{f}

      Crea una señal \textit{id} aplicando la función \textit{f}
    a la señal \textit{src}.
      Cada vez que la señal \textit{src} cambie de valor, se le aplica
      la función \textit{f} y la señal \textit{id} cambia de valor.
  
    \begin{center}
\begin{tikzpicture}
\selectlanguage{english}
  %\draw[step=1cm,gray,very thin,xshift=0cm,yshift=0cm] (0,0) grid (12,4);
  \begin{scope}[xshift=0cm,yshift=0cm,very thick,
    node distance=2cm,on grid,>=stealth',
    block/.style={rectangle,draw,fill=cyan!20},
    func/.style={rectangle,draw,fill=green!20},
    comp/.style={circle,draw,fill=orange!40}]
    \node [comp] (ca1) [xshift=2cm, yshift=3cm] {$\texttt{S}_{src}$};
   \end{scope} 
   \draw [->,line width=1.7pt](5,2) -- (7,2);
   \begin{scope}[xshift=10cm,yshift=0cm,very thick,
    node distance=2cm,on grid,>=stealth',
    block/.style={rectangle,draw,fill=cyan!20},
    func/.style={rectangle,draw,fill=green!20},
    comp/.style={circle,draw,fill=orange!40}]
   \node [func]  (f1) [xshift=1cm, yshift=1cm]  {f} ;
   \node [comp]  (cb)  [yshift=1cm]   {$\texttt{S}_{id}$} edge [-] (f1);
   \node [comp]  (ca1)  [above=of cb] {$\texttt{S}_{src}$} edge [->] (cb);
   \end{scope}
\selectlanguage{spanish}
\end{tikzpicture}
\end{center}

}
\item {
  \texttt{lift2} \textit{id}, $\textit{src}_1$ $\textit{src}_2$ \textit{f}

      Crea una señal \textit{id} aplicando el combinador \texttt{lift2}
    usando la función \textit{f}, y las señales $\textit{src}_1$ y
  $\textit{src}_2$.
      Cuando ambas señales cambien de valor, se aplica la función
      y la señal \textit{id} cambia de valor.

    \begin{tikzpicture}
\selectlanguage{english}
  \begin{scope}[xshift=0cm,yshift=-4cm,very thick,
    node distance=2cm,on grid,>=stealth',
    block/.style={rectangle,draw,fill=cyan!20},
    func/.style={rectangle,draw,fill=green!20},
    comp/.style={circle,draw,fill=orange!40}]
   \node [comp] (ca1) [xshift=-1cm] {$S_{src_1}$};
   \node [comp]  (ca2) [right=of ca1] {$S_{src_2}$};
   \end{scope}
  \draw [dashed] (4.5,-3) -- (4.5,-5.5);
 \begin{scope}[xshift=8cm,yshift=-5cm,very thick,
    node distance=2cm,on grid,>=stealth',
    block/.style={rectangle,draw,fill=cyan!20},
    func/.style={rectangle,draw,fill=green!20},
    comp/.style={circle,draw,fill=orange!40}]
   \node [func] (f1) [xshift=1.3cm] {f};
   \node [comp] (cb) {$S_{id}$} edge [-] (f1);
   \node [comp] (ca1) [above=of cb,xshift=-1cm] {$S_{src_1}$} edge [->] (cb);
   \node [comp]  (ca2) [right=of ca1] {$S_{src_2}$} edge [->] (cb);
   \end{scope} 
\selectlanguage{spanish}
\end{tikzpicture}



}
\item {
    \texttt{folds} \textit{id}, \textit{src} \textit{f}

      Crea una señal \textit{id} aplicando el combinador \texttt{folds}.
    El valor inicial de la señal está dado por el tope del stack, luego
    el mismo se actualiza aplicando la función \textit{f} al valor actual
    y a los valores recibidos de la señal \textit{src}.

    \begin{center}
\begin{tikzpicture}
\selectlanguage{english}
  %%\draw[step=1cm,gray,very thin,xshift=0cm,yshift=0cm] (0,0) grid (12,4);
  \begin{scope}[xshift=0cm,yshift=0cm,very thick,
    node distance=2cm,on grid,>=stealth',
    block/.style={rectangle,draw,fill=cyan!20},
    func/.style={rectangle,draw,fill=green!20},
    comp/.style={circle,draw,fill=orange!40},
  stack/.style={rectangle split,rectangle split parts=#1,draw,anchor=center}]
  \node[stack=4,xshift=4cm,yshift=2cm,
        text width=1.1cm,align=center,text height=0.3cm] {
    \nodepart{one} $s_n$
    \nodepart{two} $s_{n-1}$
    \nodepart{three} $\dotsb$
    \nodepart{four} $s_o$
  };
  \node [comp] (ca1) [xshift=1cm, yshift=3cm] {$\texttt{S}_{src}$};
    \draw [->,line width=1pt] (2.7,2.8) -- node[below]{\tiny{TOS}} (3.3, 2.8);
   \end{scope} 
   \draw [->,line width=1.7pt](5,2) -- (7,2);
   \begin{scope}[xshift=7cm,yshift=0cm,very thick,
    node distance=2cm,on grid,>=stealth',
    block/.style={rectangle,draw,fill=cyan!20},
    value/.style={rectangle,draw,fill=red!20},
    func/.style={rectangle,draw,fill=green!20},
    comp/.style={circle,draw,fill=orange!40},
    stack/.style={rectangle split,rectangle split parts=#1,draw,anchor=center}]
  \node[stack=4,xshift=4cm,yshift=2cm,
        text width=1.1cm,align=center,text height=0.3cm] {
    \nodepart{one} \vphantom{$s_n$}
    \nodepart{two} $s_{n-1}$
    \nodepart{three} $\dotsb$
    \nodepart{four} $s_o$
  };
  \draw [->,line width=1pt] (2.7,2.2) -- node[below]{\tiny{TOS}} (3.3, 2.2);
   \node [func]  (f1) [xshift=2.5cm, yshift=1cm]  {f};
   \node [value] (v) [xshift=0cm, yshift=1cm]  {${s_n}$} ;
   \node [comp]  (cb)  [xshift=1.5cm, yshift=1cm]   {$\texttt{S}_{id}$} edge [-] (f1);
   \node [comp]  (ca1)  [above=of cb] {$\texttt{S}_{src}$} edge [->] (cb);
 \end{scope} 
   \draw[-,line width=1pt] (cb) -- node[below]{\tiny{state}} (v);
\selectlanguage{spanish}
\end{tikzpicture}
\end{center}

}
\item {
    \texttt{write} \textit{id}

    Envía los valores de la señal \textit{id} a la salida identificada
    con el valor $s_n$ ($\texttt{O}_{s_n}$) que se encuentra en el
    tope del stack (\texttt{TOS}).
    
    \begin{tikzpicture}
\selectlanguage{english}
  \begin{scope}[xshift=0cm,yshift=-5cm,very thick,
    node distance=2cm,on grid,>=stealth',
    block/.style={rectangle,draw,fill=cyan!20},
    comp/.style={circle,draw,fill=orange!40},
    stack/.style={rectangle split,rectangle split parts=#1,draw,anchor=center}]
  \node[stack=5]  {
    \nodepart{one}
    \nodepart{two}TOS
    \nodepart{three} $s_n$
    \nodepart{four} $\dotsb$
    \nodepart{five} $s_o$
  };
   \node [block] (s2) [yshift=1.5cm,xshift=2cm] {$\textit{Output}_{TOS}$};
 \end{scope} 
 \draw [dashed] (4.5,-3) -- (4.5,-5.5);
 \begin{scope}[xshift=8cm,yshift=-5cm,very thick,
    node distance=2cm,on grid,>=stealth',
    block/.style={rectangle,draw,fill=cyan!20},
    comp/.style={circle,draw,fill=orange!40},
  stack/.style={rectangle split,rectangle split parts=#1,draw,anchor=center}]
  \node[stack=5]  {
    \nodepart{one}
    \nodepart{two}
    \nodepart{three} $s_n$
    \nodepart{four} $\dotsb$
    \nodepart{five} $s_o$
  };

   \node [block] (s2) [xshift=2cm] {$\textit{Output}_{TOS}$};
   \node [comp] (ca2) [above=of s2] {$\textit{S}_{id}$} edge [->] (s2);
   \end{scope} 
\selectlanguage{spanish}
\end{tikzpicture}



}
\end{itemize}

  En el Apéndice \ref{appendix:vmref} se encuentra el listado completo
  de operaciones y su descripción.
