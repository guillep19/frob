  Al compilar un programa \frob{}, se obtiene como salida un código
intermedio en lenguaje \alf{}.
  El mismo es independiente de la plataforma en la que va a ser ejecutado.
  Para lograr ésto, se define el lenguaje como un conjunto de instrucciones
con su semántica y una máquina abstracta que las ejecuta.
  La máquina que interpreta el lenguaje \alf{} es una
\textit{máquina de stack}.\footnote{Stack machine en inglés}.

  En una máquina de stack las instrucciones están en notación
postfija.\footnote{RPN (\textit{Reverse polish notation}) del inglés}
  Para evaluar expresiones se colocan sus argumentos en una pila, y luego
se ejecuta la operación asociada.
  
  Por ejemplo la expresión ``$5 + 19 * 8$'' en RPN se
  escribe ``$5\ 19\ 8 * +$''.
  
  En \alf{} se utilizan 5 instrucciones para ésto:

  \begin{Verbatim}[frame=single]
  push 5
  push 19
  push 8
  mul
  add
  \end{Verbatim}

  La explicación de la semántica de las mismas, se da a continuación.

  Utilizaré la siguiente notación para describir las instrucciones:
  \begin{center}
    $\texttt{codigo} [\textit{inmediato}] [, {arg}_1, \dotsb, {arg}_n ]$
  \end{center}

  La máquina abstracta cuenta con un conjunto \textit{Inputs} de $m$
  entradas fijas. Cada una se identifica con un entero único entre
  $1$ y $m$.

  Cada $\texttt{I}_i, i \in (1 \dotsb m)$ se corresponderá con un sensor
  definido en el robot.
  

  \begin{center}
    $\textit{Inputs} = \{\texttt{I}_1 \dotsb \texttt{I}_m\}$.
  \end{center}

  También cuenta con un conjunto \textit{Outputs} de $k$ salidas,
  identificadas de $1$ a $k$.
  
  Cada $\texttt{O}_i, i \in (1 \dotsb k)$ se corresponderá con un actuador
  del robot.

  \begin{center}
    $\textit{Outputs} = \{\texttt{O}_1 \dotsb \texttt{I}_k\}$.
  \end{center}

  Las señales que se definan se denotarán $\texttt{S}_i$, siendo $i$ un
  índice único que las identifica. El conjunto de las señales se llama
  \textit{Signals}.

  \begin{center}
    $\textit{Signals} = \{\texttt{S}_1 \dotsb \texttt{S}_s\}$.
  \end{center}

  La máquina tendrá una pila global, denotada \textit{Stack}.
  El \textit{Stack} lo representaré graficamente con la notación:

  \begin{center}
\begin{tikzpicture}
\selectlanguage{english}
  %%\draw[step=1cm,gray,very thin,xshift=0cm,yshift=0cm] (0,0) grid (12,3);
  \begin{scope}[xshift=4cm,yshift=0cm,very thick,
    node distance=2cm,on grid,>=stealth',
    block/.style={rectangle,draw,fill=cyan!20},
    comp/.style={circle,draw,fill=orange!40},
    stack/.style={rectangle split,rectangle split parts=#1,draw,anchor=center}]
  \node[stack=4,xshift=2cm,yshift=1.5cm,
        text width=1.1cm,align=center,text height=0.3cm] {
    \nodepart{one} $s_n$
    \nodepart{two} $s_{n-1}$
    \nodepart{three} $\dotsb$
    \nodepart{four} $s_o$
  };
 \draw [->,line width=1pt] (0.5,2.4) -- node[below]{\tiny{TOS}} (1.3, 2.4);
 \end{scope} 
  
\selectlanguage{spanish}
\end{tikzpicture}


\end{center}


  Donde \texttt{TOS}\footnote{Del inglés: Top
    of stack} indica el tope del mismo.


\subsection{Instrucciones básicas}

\begin{itemize}

\item {
    \texttt{push} ,\textit{value}

    Coloca el valor \textit{value} como tope del stack .

    \begin{tikzpicture}
\selectlanguage{english}
  %%\draw[step=1cm,gray,very thin,xshift=0cm,yshift=0cm] (0,0) grid (12,4);
  \begin{scope}[xshift=0cm,yshift=0cm,very thick,
    node distance=2cm,on grid,>=stealth',
    block/.style={rectangle,draw,fill=cyan!20},
    comp/.style={circle,draw,fill=orange!40},
    stack/.style={rectangle split,rectangle split parts=#1,draw,anchor=center}]
  \node[stack=4,xshift=2cm,yshift=2cm,
        text width=1.1cm,align=center,text height=0.3cm] {
    \nodepart{one} \vphantom{val}
    \nodepart{two} $s_n$
    \nodepart{three} $\dotsb$
    \nodepart{four} $s_o$
  };
 \draw [->,line width=1pt] (0.5,2.2) -- node[below]{\tiny{TOS}} (1.3, 2.2);
 \end{scope} 
 \draw [->,line width=1.7pt] (5,2) -- (7,2);
 \begin{scope}[xshift=8cm,yshift=0cm,very thick,
    node distance=2cm,on grid,>=stealth',
    block/.style={rectangle,draw,fill=cyan!20},
    comp/.style={circle,draw,fill=orange!40},
  stack/.style={rectangle split,rectangle split parts=#1,draw,anchor=center}]
  \node[stack=4,xshift=2cm,yshift=2cm,text width=1.1cm,align=center]  {
    \nodepart{one} \textit{value}
    \nodepart{two} $s_n$
    \nodepart{three} $\dotsb$
    \nodepart{four} $s_o$
   };
 \draw [->,line width=1pt] (0.5,2.8) -- node[below]{\tiny{TOS}} (1.3, 2.8);
   \end{scope} 
\selectlanguage{spanish}
\end{tikzpicture}



}
\item {
    \texttt{add}

      Remueve dos valores del stack, los suma y coloca el resultado en
    el tope.

    \begin{tikzpicture}
\selectlanguage{english}
  %%\draw[step=1cm,gray,very thin,xshift=0cm,yshift=0cm] (0,0) grid (12,4);
  \begin{scope}[xshift=0cm,yshift=0cm,very thick,
    node distance=2cm,on grid,>=stealth',
    block/.style={rectangle,draw,fill=cyan!20},
    comp/.style={circle,draw,fill=orange!40},
    stack/.style={rectangle split,rectangle split parts=#1,draw,anchor=center}]
  \node[stack=4,xshift=2cm,yshift=2cm,
        text width=1.1cm,align=center,text height=0.3cm] {
    \nodepart{one} $s_n$
    \nodepart{two} $s_{n-1}$
    \nodepart{three} $\dotsb$
    \nodepart{four} $s_o$
  };
 \draw [->,line width=1pt] (0.5,2.9) -- node[below]{\tiny{TOS}} (1.3, 2.9);
 \end{scope} 
 \draw [->,line width=1.7pt] (5,2) -- (7,2);
 \begin{scope}[xshift=8cm,yshift=0cm,very thick,
    node distance=2cm,on grid,>=stealth',
    block/.style={rectangle,draw,fill=cyan!20},
    comp/.style={circle,draw,fill=orange!40},
  stack/.style={rectangle split,rectangle split parts=#1,draw,anchor=center}]
  \node[stack=4,xshift=2cm,yshift=2cm,text width=1.1cm,align=center]  {
    \nodepart{one} \vphantom{fruit}
    \nodepart{two} \tiny{$s_n + s_{n-1}$}
    \nodepart{three} $\dotsb$
    \nodepart{four} $s_o$
   };
 \draw [->,line width=1pt] (0.5,2.2) -- node[below]{\tiny{TOS}} (1.3, 2.2);
   \end{scope} 
\selectlanguage{spanish}
\end{tikzpicture}



}
\item {
    \texttt{mul}

      Remueve dos valores del stack, los multiplica y coloca el resultado en
    el tope.

    \begin{tikzpicture}
\selectlanguage{english}
  %%\draw[step=1cm,gray,very thin,xshift=0cm,yshift=0cm] (0,0) grid (12,4);
  \begin{scope}[xshift=0cm,yshift=0cm,very thick,
    node distance=2cm,on grid,>=stealth',
    block/.style={rectangle,draw,fill=cyan!20},
    comp/.style={circle,draw,fill=orange!40},
    stack/.style={rectangle split,rectangle split parts=#1,draw,anchor=center}]
  \node[stack=4,xshift=2cm,yshift=2cm,
        text width=1.1cm,align=center,text height=0.3cm] {
    \nodepart{one} $s_n$
    \nodepart{two} $s_{n-1}$
    \nodepart{three} $\dotsb$
    \nodepart{four} $s_o$
  };
 \draw [->,line width=1pt] (0,2.9) -- node[below]{\small{TOS}} (1.2, 2.9);
 \end{scope} 
 \draw [->,line width=1.7pt] (5,2) -- (7,2);
 \begin{scope}[xshift=8cm,yshift=0cm,very thick,
    node distance=2cm,on grid,>=stealth',
    block/.style={rectangle,draw,fill=cyan!20},
    comp/.style={circle,draw,fill=orange!40},
  stack/.style={rectangle split,rectangle split parts=#1,draw,anchor=center}]
  \node[stack=4,xshift=2cm,yshift=2cm,text width=1.1cm,align=center]  {
    \nodepart{one} \vphantom{fruit}
    \nodepart{two} \tiny{$s_n * s_{n-1}$}
    \nodepart{three} $\dotsb$
    \nodepart{four} $s_o$
   };
 \draw [->,line width=1pt] (0,2.2) -- node[below]{\small{TOS}} (1.2, 2.2);
   \end{scope} 
\selectlanguage{spanish}
\end{tikzpicture}



}
\end{itemize}

\subsection{Instrucciones para manipular señales}
  
  A continuación se presentan las instrucciones utilizadas para manipular
señales.
  \begin{itemize}

\item {
    \texttt{read} \textit{id}

      Toma el tope del stack como identificador de una entrada.
      Crea una señal \textit{id} que contendrá el valor de la entrada
    asociada. 
      Como precondición, la señal \textit{id} no debe existir.

    \begin{tikzpicture}
\selectlanguage{english}
  %%\draw[step=1cm,gray,very thin,xshift=0cm,yshift=0cm] (0,0) grid (12,4);
  \begin{scope}[xshift=0cm,yshift=0cm,very thick,
    node distance=2cm,on grid,>=stealth',
    block/.style={rectangle,draw,fill=cyan!20},
    comp/.style={circle,draw,fill=orange!40},
    stack/.style={rectangle split,rectangle split parts=#1,draw,anchor=center}]
  \node[stack=5,yshift=2cm,xshift=4cm,align=center,
        text height=0.3cm,
        text width=1.1cm]  {
    \nodepart{one} \vphantom{fruit}
    \nodepart{two} $s_n$
    \nodepart{three} $s_{n-1}$
    \nodepart{four} $\dotsb$
    \nodepart{five} $s_0$
  };
   \node [block] (s2) [yshift=3cm,xshift=1cm] {$\texttt{I}_{s_n}$};
 \draw [->,line width=1pt] (2.6,2.5) -- node[below]{\tiny{TOS}} (3.3, 2.5);
 \end{scope} 

 \draw [->,line width=1.7pt] (5,2) -- (7,2);

 \begin{scope}[xshift=7cm,yshift=0cm,very thick,
    node distance=2cm,on grid,>=stealth',
    block/.style={rectangle,draw,fill=cyan!20},
    comp/.style={circle,draw,fill=orange!40},
  stack/.style={rectangle split,rectangle split parts=#1,draw,anchor=center}]
  \node[stack=5,yshift=2cm,xshift=4cm,align=center,
        text height=0.3cm,
        text width=1.1cm]  {
    \nodepart{one} \vphantom{fruit}
    \nodepart{two} \vphantom{fruit}
    \nodepart{three} $s_{n-1}$
    \nodepart{four} $\dotsb$
    \nodepart{five} $s_0$
  };
   \node [comp]  (ca2) [xshift=1cm,yshift=1cm] {$\texttt{S}_{id}$};
   \node [block] (s2) [above=of ca2]   {$\texttt{I}_{s_n}$} edge [->] (ca2);
   \draw [->,line width=1pt] (2.6,1.9) -- node[below]{\tiny{TOS}} (3.3, 1.9);
   \end{scope} 
\selectlanguage{spanish}
\end{tikzpicture}



}
\item {
    \texttt{lift} \textit{id}, \textit{src} \textit{f}

      Crea una señal \textit{id} aplicando la función \textit{f}
    a la señal \textit{src}.
  
    \begin{center}
\begin{tikzpicture}
\selectlanguage{english}
  %\draw[step=1cm,gray,very thin,xshift=0cm,yshift=0cm] (0,0) grid (12,4);
  \begin{scope}[xshift=0cm,yshift=0cm,very thick,
    node distance=2cm,on grid,>=stealth',
    block/.style={rectangle,draw,fill=cyan!20},
    func/.style={rectangle,draw,fill=green!20},
    comp/.style={circle,draw,fill=orange!40}]
    \node [comp] (ca1) [xshift=2cm, yshift=3cm] {$\texttt{S}_{src}$};
   \end{scope} 
   \draw [->,line width=1.7pt](5,2) -- (7,2);
   \begin{scope}[xshift=10cm,yshift=0cm,very thick,
    node distance=2cm,on grid,>=stealth',
    block/.style={rectangle,draw,fill=cyan!20},
    func/.style={rectangle,draw,fill=green!20},
    comp/.style={circle,draw,fill=orange!40}]
   \node [func]  (f1) [xshift=1cm, yshift=1cm]  {f} ;
   \node [comp]  (cb)  [yshift=1cm]   {$\texttt{S}_{id}$} edge [-] (f1);
   \node [comp]  (ca1)  [above=of cb] {$\texttt{S}_{src}$} edge [->] (cb);
   \end{scope}
\selectlanguage{spanish}
\end{tikzpicture}
\end{center}

}
\item {
  \texttt{lift2} \textit{id}, $\textit{src}_1$ $\textit{src}_2$ \textit{f}

      Crea una señal \textit{id} aplicando el combinador \texttt{lift2}
    usando la función \textit{f}, y las señales $\textit{src}_1$ y
  $\textit{src}_2$.

    \begin{tikzpicture}
\selectlanguage{english}
  \begin{scope}[xshift=0cm,yshift=-4cm,very thick,
    node distance=2cm,on grid,>=stealth',
    block/.style={rectangle,draw,fill=cyan!20},
    func/.style={rectangle,draw,fill=green!20},
    comp/.style={circle,draw,fill=orange!40}]
   \node [comp] (ca1) [xshift=-1cm] {$S_{src_1}$};
   \node [comp]  (ca2) [right=of ca1] {$S_{src_2}$};
   \end{scope}
  \draw [dashed] (4.5,-3) -- (4.5,-5.5);
 \begin{scope}[xshift=8cm,yshift=-5cm,very thick,
    node distance=2cm,on grid,>=stealth',
    block/.style={rectangle,draw,fill=cyan!20},
    func/.style={rectangle,draw,fill=green!20},
    comp/.style={circle,draw,fill=orange!40}]
   \node [func] (f1) [xshift=1.3cm] {f};
   \node [comp] (cb) {$S_{id}$} edge [-] (f1);
   \node [comp] (ca1) [above=of cb,xshift=-1cm] {$S_{src_1}$} edge [->] (cb);
   \node [comp]  (ca2) [right=of ca1] {$S_{src_2}$} edge [->] (cb);
   \end{scope} 
\selectlanguage{spanish}
\end{tikzpicture}



}
\item {
    \texttt{folds} \textit{id}, \textit{src} \textit{f}

      Crea una señal \textit{id} aplicando el combinador \texttt{folds}.
    El valor inicial de la señal está dado por el tope del stack, luego
    el mismo se actualiza aplicando la función \textit{f} al valor actual
    y a los valores recibidos de la señal \textit{src}.

    \begin{center}
\begin{tikzpicture}
\selectlanguage{english}
  %%\draw[step=1cm,gray,very thin,xshift=0cm,yshift=0cm] (0,0) grid (12,4);
  \begin{scope}[xshift=0cm,yshift=0cm,very thick,
    node distance=2cm,on grid,>=stealth',
    block/.style={rectangle,draw,fill=cyan!20},
    func/.style={rectangle,draw,fill=green!20},
    comp/.style={circle,draw,fill=orange!40},
  stack/.style={rectangle split,rectangle split parts=#1,draw,anchor=center}]
  \node[stack=4,xshift=4cm,yshift=2cm,
        text width=1.1cm,align=center,text height=0.3cm] {
    \nodepart{one} $s_n$
    \nodepart{two} $s_{n-1}$
    \nodepart{three} $\dotsb$
    \nodepart{four} $s_o$
  };
  \node [comp] (ca1) [xshift=1cm, yshift=3cm] {$\texttt{S}_{src}$};
    \draw [->,line width=1pt] (2.7,2.8) -- node[below]{\tiny{TOS}} (3.3, 2.8);
   \end{scope} 
   \draw [->,line width=1.7pt](5,2) -- (7,2);
   \begin{scope}[xshift=7cm,yshift=0cm,very thick,
    node distance=2cm,on grid,>=stealth',
    block/.style={rectangle,draw,fill=cyan!20},
    value/.style={rectangle,draw,fill=red!20},
    func/.style={rectangle,draw,fill=green!20},
    comp/.style={circle,draw,fill=orange!40},
    stack/.style={rectangle split,rectangle split parts=#1,draw,anchor=center}]
  \node[stack=4,xshift=4cm,yshift=2cm,
        text width=1.1cm,align=center,text height=0.3cm] {
    \nodepart{one} \vphantom{$s_n$}
    \nodepart{two} $s_{n-1}$
    \nodepart{three} $\dotsb$
    \nodepart{four} $s_o$
  };
  \draw [->,line width=1pt] (2.7,2.2) -- node[below]{\tiny{TOS}} (3.3, 2.2);
   \node [func]  (f1) [xshift=2.5cm, yshift=1cm]  {f};
   \node [value] (v) [xshift=0cm, yshift=1cm]  {${s_n}$} ;
   \node [comp]  (cb)  [xshift=1.5cm, yshift=1cm]   {$\texttt{S}_{id}$} edge [-] (f1);
   \node [comp]  (ca1)  [above=of cb] {$\texttt{S}_{src}$} edge [->] (cb);
 \end{scope} 
   \draw[-,line width=1pt] (cb) -- node[below]{\tiny{state}} (v);
\selectlanguage{spanish}
\end{tikzpicture}
\end{center}

}
\item {
    \texttt{write} \textit{id}

    Envía los valores de la señal \textit{id} a la salida identificada
     con el valor $s_n$ ($\texttt{O}_{s_n}$) que se encuentra en el
     tope del stack (\texttt{TOS}).
    
    \begin{tikzpicture}
\selectlanguage{english}
  \begin{scope}[xshift=0cm,yshift=-5cm,very thick,
    node distance=2cm,on grid,>=stealth',
    block/.style={rectangle,draw,fill=cyan!20},
    comp/.style={circle,draw,fill=orange!40},
    stack/.style={rectangle split,rectangle split parts=#1,draw,anchor=center}]
  \node[stack=5]  {
    \nodepart{one}
    \nodepart{two}TOS
    \nodepart{three} $s_n$
    \nodepart{four} $\dotsb$
    \nodepart{five} $s_o$
  };
   \node [block] (s2) [yshift=1.5cm,xshift=2cm] {$\textit{Output}_{TOS}$};
 \end{scope} 
 \draw [dashed] (4.5,-3) -- (4.5,-5.5);
 \begin{scope}[xshift=8cm,yshift=-5cm,very thick,
    node distance=2cm,on grid,>=stealth',
    block/.style={rectangle,draw,fill=cyan!20},
    comp/.style={circle,draw,fill=orange!40},
  stack/.style={rectangle split,rectangle split parts=#1,draw,anchor=center}]
  \node[stack=5]  {
    \nodepart{one}
    \nodepart{two}
    \nodepart{three} $s_n$
    \nodepart{four} $\dotsb$
    \nodepart{five} $s_o$
  };

   \node [block] (s2) [xshift=2cm] {$\textit{Output}_{TOS}$};
   \node [comp] (ca2) [above=of s2] {$\textit{S}_{id}$} edge [->] (s2);
   \end{scope} 
\selectlanguage{spanish}
\end{tikzpicture}



}
\end{itemize}

  En el apéndice A.2 se encuentra el listado completo de operaciones y su
descripción.
