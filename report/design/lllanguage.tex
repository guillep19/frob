  Al compilar un programa \frob{}, se obtiene como salida un código
intermedio llamado lenguaje \alf{}.
  El mismo es independiente de la plataforma en la que va a ser ejecutado.
  Para lograr ésto, se define el lenguaje como un conjunto de instrucciones
con su semántica y una máquina abstracta que las ejecuta.

\subsection{Instrucciones}

  Utilizaré la siguiente notación para describir las instrucciones:

  $\texttt{codigo}\ \textit{inmediato}\ [{arg}_1 \dotsb {arg}_n]$

\begin{itemize}

\item \texttt{read id}

  Toma el tope del stack (\texttt{TOS}\footnote{Del inglés: Top of stack}) como identificador de una entrada.
  Crea una señal \texttt{id} que contendrá el valor de la entrada asociada.
  Como precondición, la señal \texttt{id} no debe existir.

\begin{tikzpicture}
\selectlanguage{english}
  \begin{scope}[xshift=0cm,yshift=-5cm,very thick,
    node distance=2cm,on grid,>=stealth',
    block/.style={rectangle,draw,fill=cyan!20},
    comp/.style={circle,draw,fill=orange!40},
    stack/.style={rectangle split,rectangle split parts=#1,draw,anchor=center}]
  \node[stack=5]  {
    \nodepart{one}
    \nodepart{two}TOS
    \nodepart{three} $s_n$
    \nodepart{four} $\dotsb$
    \nodepart{five} $s_o$
  };
   \node [block] (s2) [yshift=1.5cm,xshift=2cm] {$\textit{Input}_{TOS}$};
 \end{scope} 
 \draw [dashed] (4.5,-3) -- (4.5,-5.5);
 \begin{scope}[xshift=8cm,yshift=-5cm,very thick,
    node distance=2cm,on grid,>=stealth',
    block/.style={rectangle,draw,fill=cyan!20},
    comp/.style={circle,draw,fill=orange!40},
  stack/.style={rectangle split,rectangle split parts=#1,draw,anchor=center}]
  \node[stack=5]  {
    \nodepart{one}
    \nodepart{two}
    \nodepart{three} $s_n$
    \nodepart{four} $\dotsb$
    \nodepart{five} $s_o$
  };

  \node [comp]  (ca2) [xshift=2cm]     {$\textit{S}_{id}$};
   \node [block] (s2) [above=of ca2]   {$\textit{Input}_{TOS}$} edge [->] (ca2);
   \end{scope} 
\selectlanguage{spanish}
\end{tikzpicture}


\item \texttt{lift id src f}

 Crea una señal \texttt{id} aplicando la función \texttt{f}
a la señal \texttt{src}.

\begin{tikzpicture}
\selectlanguage{english}
  \begin{scope}[xshift=0cm,yshift=-5cm,very thick,
    node distance=2cm,on grid,>=stealth',
    block/.style={rectangle,draw,fill=cyan!20},
    func/.style={rectangle,draw,fill=green!20},
    comp/.style={circle,draw,fill=orange!40}]
    \node [comp]  (ca1) [yshift=1.5cm] {$S_{src}$};
 \end{scope} 
  \draw [dashed] (4.5,-3) -- (4.5,-5.5);
 \begin{scope}[xshift=8cm,yshift=-5cm,very thick,
    node distance=2cm,on grid,>=stealth',
    block/.style={rectangle,draw,fill=cyan!20},
    func/.style={rectangle,draw,fill=green!20},
    comp/.style={circle,draw,fill=orange!40}]
   \node [func]  (f1) [xshift=1.3cm]     {f} ;
   \node [comp]  (cb)  []   {$S_{id}$} edge [-] (f1);
   \node [comp]  (ca1)  [above=of cb] {$S_{src}$} edge [->] (cb);
   \end{scope} 
\selectlanguage{spanish}
\end{tikzpicture}


\item \texttt{lift2 id src\_1 src\_2 f}

  Crea una señal \texttt{id} aplicando el combinador \texttt{lift2} usando
la función \texttt{f}, y las señales \texttt{src\_1} y \texttt{src\_2}.

\begin{tikzpicture}
\selectlanguage{english}
  \begin{scope}[xshift=0cm,yshift=-4cm,very thick,
    node distance=2cm,on grid,>=stealth',
    block/.style={rectangle,draw,fill=cyan!20},
    func/.style={rectangle,draw,fill=green!20},
    comp/.style={circle,draw,fill=orange!40}]
   \node [comp] (ca1) [xshift=-1cm] {$S_{src_1}$};
   \node [comp]  (ca2) [right=of ca1] {$S_{src_2}$};
   \end{scope}
  \draw [dashed] (4.5,-3) -- (4.5,-5.5);
 \begin{scope}[xshift=8cm,yshift=-5cm,very thick,
    node distance=2cm,on grid,>=stealth',
    block/.style={rectangle,draw,fill=cyan!20},
    func/.style={rectangle,draw,fill=green!20},
    comp/.style={circle,draw,fill=orange!40}]
   \node [func] (f1) [xshift=1.3cm] {f};
   \node [comp] (cb) {$S_{id}$} edge [-] (f1);
   \node [comp] (ca1) [above=of cb,xshift=-1cm] {$S_{src_1}$} edge [->] (cb);
   \node [comp]  (ca2) [right=of ca1] {$S_{src_2}$} edge [->] (cb);
   \end{scope} 
\selectlanguage{spanish}
\end{tikzpicture}



\item \texttt{write index id}

  Envía los valores de la señal \texttt{id} a la salida identificada
con el tope del stack (TOS).

\begin{tikzpicture}
\selectlanguage{english}
  \begin{scope}[xshift=0cm,yshift=-5cm,very thick,
    node distance=2cm,on grid,>=stealth',
    block/.style={rectangle,draw,fill=cyan!20},
    comp/.style={circle,draw,fill=orange!40},
    stack/.style={rectangle split,rectangle split parts=#1,draw,anchor=center}]
  \node[stack=5]  {
    \nodepart{one}
    \nodepart{two}TOS
    \nodepart{three} $s_n$
    \nodepart{four} $\dotsb$
    \nodepart{five} $s_o$
  };
   \node [block] (s2) [yshift=1.5cm,xshift=2cm] {$\textit{Output}_{TOS}$};
 \end{scope} 
 \draw [dashed] (4.5,-3) -- (4.5,-5.5);
 \begin{scope}[xshift=8cm,yshift=-5cm,very thick,
    node distance=2cm,on grid,>=stealth',
    block/.style={rectangle,draw,fill=cyan!20},
    comp/.style={circle,draw,fill=orange!40},
  stack/.style={rectangle split,rectangle split parts=#1,draw,anchor=center}]
  \node[stack=5]  {
    \nodepart{one}
    \nodepart{two}
    \nodepart{three} $s_n$
    \nodepart{four} $\dotsb$
    \nodepart{five} $s_o$
  };

   \node [block] (s2) [xshift=2cm] {$\textit{Output}_{TOS}$};
   \node [comp] (ca2) [above=of s2] {$\textit{S}_{id}$} edge [->] (s2);
   \end{scope} 
\selectlanguage{spanish}
\end{tikzpicture}







\end{itemize}

TODO: Comentario de Marcos: Describir más instrucciones.
Mostrar que es una máquina de stack. Mostrar ejemplos. 
Hacer referencia al apéndice A.2.


\pagebreak

