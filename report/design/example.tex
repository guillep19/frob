  En el siguiente ejemplo se muestra un programa \frob{} que detiene un
robot cuando su sensor de distancia muestra un valor menor a un mínimo dado.

\begin{verbatim}
INPUT_DISTANCE = 1
OUTPUT_MOTOR = 1

MIN_DISTANCE = 30
FULL_SPEED = 100
STOP = 0

distanceToSpeed n = if (n < MIN_DISTANCE) then STOP else FULL_SPEED

do {
  distance <- read INPUT_DISTANCE,
  speed <- lift distanceToSpeed distance,
  output OUTPUT_MOTOR speed
}

\end{verbatim}

Primero se definió una señal llamada \emph{distance},
de tipo \emph{Event Number}.

\begin{verbatim}
  distance <- read INPUT_DISTANCE
\end{verbatim}

  La misma emitirá un evento con la distancia en \emph{centímetros} leída
por un sensor en el robot.

  Se quiere que cuando la distancia sea menor al valor
\emph{MINIMO} dado, el robot se detenga completamente, esto
quiere decir, que la velocidad sea $0$, en otro caso el mismo
se mueve a velocidad $100$, la cual asumimos es una velocidad
apropiada.
  Para esto se crea una nueva fuente de eventos, definida a partir
de \emph{distance}, aplicandole la función \texttt{distanceToSpeed} 
a cada valor para obtener la velocidad.

\begin{verbatim}
  speed <- lift distanceToSpeed distance
\end{verbatim}

  Esta expresión también es una fuente de eventos
de tipo \emph{Event Number}.

  Finalmente se aplica la función nativa \texttt{output}, cuyo
primer parámetro es el $id$ de la salida (en éste caso 
es \texttt{OUTPUT\_MOTOR}) y su segundo parámetro es una
fuente de eventos, en éste caso es la velocidad (\texttt{speed}).

TODO: Explicar si se pueden definir más eventos, combinarlos y cuántas veces
se ejecuta el bloque do.
