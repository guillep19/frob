
  Las funciones se definen con un nombre, una lista de argumentos y una 
expresión. Las variables libres de la expresión son sustituídas al evaluar
la función.

\begin{verbatim}
nombre argumento_1 .. argumento_n = expresion
\end{verbatim}

  Una expresión puede ser un valor primitivo,
una expresión aritmética (por ejemplo una suma o multiplicación),
la aplicación de una función, o una expresión condicional.
  Todas las expresiones retornan un valor al evaluarse.

  La sintaxis es muy similar a la del lenguaje \texttt{Haskell}, aunque
no se permiten funciones anónimas.

  Para declarar un valor constante simplemente se escribe una función sin
argumentos. Por convención se escriben con mayúsculas, pero no es una
restricción.

\begin{verbatim}
    NOMBRE_CONSTANTE = valor
\end{verbatim}

  Una expresión condicional, debe retornar un valor para cada posible
resultado de la condición.

  En la figura \ref{fig:fibo} se ve la implementación
de la función que retorna un número en la sucesión de Fibonacci,
utilizando una expresión condicional.

  Se puede ver que una función puede invocarse a si misma, está permitida
la recursión.

\begin{figure}[h!]
\begin{center}
  \caption{Función de Fibonacci}
  \begin{Verbatim}[frame=single]
# fibonacci
fibo n = if (n < 2) then 1 else fibo(n-1) + fibo(n-2)
  \end{Verbatim}
   \label{fig:fibo}
\end{center}
\end{figure}

  Un comentario es una línea que comienza con el símbolo \texttt{\#}.

  La gramática completa del lenguaje se puede ver en la
Figura \ref{fig:grammar}.

Acaaaaaa

\begin{figure}
  \begin{center}
    \caption{Gramática de \frob{}}
    \begin{Verbatim}[frame=single]
program := definitions "do {" frps "}";

definitions := definition | definition definitions;

definition := ident arg_list "=" expr;

frps = frp | frp frps;

frp := ident "<- read" expr
     | ident "<- lift" ident ident
     | ident "<- lift2" ident ident ident
     | ident "<- folds" ident value ident
     | "output" expr ident

expr := name
      | number
      | expr binop expr
      | "if" expr "then" expr "else" expr;

arg_list := ident | ident arg_list;

ident := [a-z_A-Z]+;

number := [-+]?[0-9]*\.?[0-9]+;

binop := '+' | '-' | '/' | '*' | 'or' | 'and'
       | '==' | '<=' | '>' | '<' | '<>' | '>=';
     \end{Verbatim}
   \label{fig:grammar}
   \end{center}
 \end{figure}

 Y??


