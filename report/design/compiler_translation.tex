
TODO: Describir el algoritmo de traducción.

El siguiente programa \frob{} muestra con un led si el robot está
frente a una casa:

\begin{Verbatim}
#Inputs
INPUT_DISTANCE = 1
#Outputs
OUTPUT_LED = 1

isHouse distance = if (distance < 100) then 1 else 0

do {
  signal_distance <- read INPUT_DISTANCE
  signal_house <- lift isHouse signal_distance
  output OUTPUT_LED signal_house
}
\end{Verbatim}

El mismo se traduce a \alf{} de la siguiente forma:

\begin{Verbatim}
0: t_call
1: 10
2: t_read 1
3: t_lift 0
4: 1
5: 16
6: t_call
7: 13
8: t_write 0
9: t_halt
10: t_push
11: 1
12: t_ret
13: t_push
14: 1
15: t_ret
16: t_load_param 0
17: t_push
18: 100
19: t_cmp_lt
20: t_jump_false
21: 26
22: t_push
23: 1
24: t_jump
25: 28
26: t_push
27: 0
28: t_ret
\end{Verbatim}

TODO: Poner comentarios. Comentar abajo.

