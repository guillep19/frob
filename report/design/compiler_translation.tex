
Dado un robot con un sensor de distancia y un led, el siguiente
programa \frob{} enciende el led cuando el robot detecta una
casa.

\begin{Verbatim}
#Inputs
INPUT_DISTANCE = 1
#Outputs
OUTPUT_LED = 1

isHouse distance = if (distance < 100) then 1 else 0

do {
  signal_distance <- read INPUT_DISTANCE
  signal_house <- lift isHouse signal_distance
  output OUTPUT_LED signal_house
}
\end{Verbatim}

El mismo se traduce a \alf{} de la siguiente forma:

\begin{Verbatim}
0: t_call
1: 10
2: t_read 1
3: t_lift 0
4: 1
5: 16
6: t_call
7: 13
8: t_write 0
9: t_halt
10: t_push
11: 1
12: t_ret
13: t_push
14: 1
15: t_ret
16: t_load_param 0
17: t_push
18: 100
19: t_cmp_lt
20: t_jump_false
21: 26
22: t_push
23: 1
24: t_jump
25: 28
26: t_push
27: 0
28: t_ret
\end{Verbatim}

Para entender el programa \alf{}, primero se divide en dos secciones.
Entre la línea $0$ y la línea $9$ está el código correspondiente a
la sección \texttt{do}.

A partir de la línea $10$ están las declaraciones de funciones.

La declaración:

\begin{Verbatim}
10: t_push
11: 1
12: t_ret
\end{Verbatim}

se corresponde con la definición de la constante \texttt{INPUT\_DISTANCE}, y la declaración

\begin{Verbatim}
13: t_push
14: 1
15: t_ret
\end{Verbatim}

es la definición de la constance \texttt{OUTPUT\_LED}.

La función \texttt{isHouse} se traduce a:

\begin{Verbatim}
16: t_load_param 0
17: t_push
18: 100
19: t_cmp_lt
20: t_jump_false
21: 26
22: t_push
23: 1
24: t_jump
25: 28
26: t_push
27: 0
28: t_ret
\end{Verbatim}

En el bloque \texttt{do} la señal \texttt{signal\_distance} se
crea cargando el valor de \texttt{INPUT\_DISTANCE} en la pila,
y luego usando la instrucción \texttt{read}.
El argumento $1$ de la instrucción \texttt{read} será el
identificador de la señal.

\begin{Verbatim}
0: t_call
1: 10
2: t_read 1
\end{Verbatim}


