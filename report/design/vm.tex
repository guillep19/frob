
  La máquina que interpreta el lenguaje \alf{} es una
\textit{máquina de stack}.\footnote{Stack machine en inglés}.

  En una máquina de stack las instrucciones están en notación
postfija.\footnote{RPN (\textit{Reverse polish notation}) del inglés}
  Para evaluar expresiones se colocan sus argumentos en una pila, y luego
se ejecuta la operación asociada.
  
  Por ejemplo la expresión ``$5 + 19 * 8$'' en RPN se
escribe ``$5\ 19\ 8 * +$''.
  
  A modo de ejemplo en \alf{} se representa con las
  siguientes 5 instrucciones.

  \begin{Verbatim}[frame=single]
  push 5
  push 19
  push 8
  mul
  add
  \end{Verbatim}

  El conjunto \textit{Inputs} representa las entradas de la máquina.
  Dadas $m$ entradas fijas, cada una se identifica con un entero único
  entre $1$ y $m = | \textit{Inputs} |$.

  Cada $\texttt{I}_i, i \in (1 \dotsb m)$ se corresponderá con un sensor
  definido en el robot.

\begin{definicion}
  \textit{Entradas de la máquina}\\
  \begin{center}
    $\textit{Inputs} \equiv \{\texttt{I}_1 \dotsb \texttt{I}_m\}$.
  \end{center}
\end{definicion}
  
  Graficamente las representaré con la notación:

  \begin{center}
\begin{tikzpicture}
\selectlanguage{english}
  %%\draw[step=1cm,gray,very thin,xshift=0cm,yshift=0cm] (0,0) grid (12,4);
  \begin{scope}[xshift=0cm,yshift=0cm,very thick,
    node distance=2cm,on grid,>=stealth',
    block/.style={rectangle,draw,fill=cyan!20},
    comp/.style={circle,draw,fill=orange!40},
    stack/.style={rectangle split,rectangle split parts=#1,draw,anchor=center}]
   \node [block] (s2) [yshift=0.5cm,xshift=1cm] {$\texttt{I}_{i}$};
\end{scope} 

\selectlanguage{spanish}
\end{tikzpicture}

\end{center}


  También se cuenta con un conjunto \textit{Outputs} de salidas,
  identificadas de $1$ a $k = | \textit{Outputs} |$.
  
  Cada $\texttt{O}_i, i \in (1 \dotsb k)$ se corresponderá con un actuador
  del robot.

\begin{definicion}
  \textit{Salidas de la máquina}\\
  \begin{center}
    $\textit{Outputs} \equiv \{\texttt{O}_1 \dotsb \texttt{I}_k\}$.
  \end{center}
\end{definicion}
  
  Graficamente las representaré con la notación:

  \begin{center}
\begin{tikzpicture}
\selectlanguage{english}
  %%\draw[step=1cm,gray,very thin,xshift=0cm,yshift=0cm] (0,0) grid (12,4);
  \begin{scope}[xshift=0cm,yshift=0cm,very thick,
    node distance=2cm,on grid,>=stealth',
    block/.style={rectangle,draw,fill=cyan!20},
    comp/.style={circle,draw,fill=orange!40},
    stack/.style={rectangle split,rectangle split parts=#1,draw,anchor=center}]
   \node [block] (s2) [yshift=0.5cm,xshift=1cm] {$\texttt{O}_{i}$};
 \end{scope}
\selectlanguage{spanish}
\end{tikzpicture}

\end{center}


  Las señales que se definan se denotarán $\texttt{S}_i$, siendo $i$ un
  índice único que las identifica. El conjunto de las señales se llama
  \textit{Signals}.

\begin{definicion}
  \textit{Señales}
  \begin{center}
    $\textit{Signals} \equiv \{\texttt{S}_1 \dotsb \texttt{S}_s\}$.
  \end{center}
\end{definicion}

  Graficamente las representaré con la notación:

  \begin{center}
\begin{tikzpicture}
\selectlanguage{english}
  %%\draw[step=1cm,gray,very thin,xshift=0cm,yshift=0cm] (0,0) grid (12,1);
  \begin{scope}[xshift=4cm,yshift=0cm,very thick,
    node distance=2cm,on grid,>=stealth',
    comp/.style={circle,draw,fill=orange!40}]
    \node [comp] (ca1) [xshift=2cm, yshift=0.5cm] {$\texttt{S}_{i}$};
   \end{scope} 
\selectlanguage{spanish}
\end{tikzpicture}
\end{center}


  La máquina tendrá una pila global, denotada \textit{Stack}. El mismo
  se representa con una secuencia de valores.

\begin{definicion}
  \textit{Pila global}
  \begin{center}
    $\textit{Stack} \equiv s_1, \dotsb, s_n$.
  \end{center}
\end{definicion}

  El \textit{Stack} lo representaré graficamente con la notación:

  \begin{center}
\begin{tikzpicture}
\selectlanguage{english}
  %%\draw[step=1cm,gray,very thin,xshift=0cm,yshift=0cm] (0,0) grid (12,3);
  \begin{scope}[xshift=4cm,yshift=0cm,very thick,
    node distance=2cm,on grid,>=stealth',
    block/.style={rectangle,draw,fill=cyan!20},
    comp/.style={circle,draw,fill=orange!40},
    stack/.style={rectangle split,rectangle split parts=#1,draw,anchor=center}]
  \node[stack=4,xshift=2cm,yshift=1.5cm,
        text width=1.1cm,align=center,text height=0.3cm] {
    \nodepart{one} $s_n$
    \nodepart{two} $s_{n-1}$
    \nodepart{three} $\dotsb$
    \nodepart{four} $s_o$
  };
 \draw [->,line width=1pt] (0.5,2.4) -- node[below]{\tiny{TOS}} (1.3, 2.4);
 \end{scope} 
  
\selectlanguage{spanish}
\end{tikzpicture}


\end{center}


  Donde \texttt{TOS}\footnote{Del inglés: Top
  of stack} indica el índice del tope del mismo.
  Se cumple que $\textit{Stack}_{TOS} = s_n$.



%% TODO: Esto es parte de la implementacion.
%%  A diferencia del compilador, es necesario implementar una máquina virtual
%%para cada arquitectura objetivo.\\
%%
%%  Por ejemplo, para ejecutar programas en un robot
%%  con un procesador \emph{arduino}, debe
%%  existir una implementación de la máquina para ese modelo
%%  de \emph{arduino}.
%%
%%  Al momento de implementar la máquina, se tomará en cuenta ésto para
%%  factorizar partes en común y sólo implementar por arquitectura, las
%%  partes que realmente sean diferentes como ser la comunicación con
%%  los periféricos de entrada/salida y las llamadas al sistema.

  El dispatcher, es quien implementa las acciones de la máquina.
  El mismo se encarga de recibir valores de los sensores y mapearlos
  a eventos en las entradas $\texttt{I}_i$.

  Éstos eventos, serán recibidos por los nodos $Nodes$.
  Cada $Node_i$ que espera por eventos entrará en estado activo cuando
todos los nodos por los que espera le envíen un evento.
  A su vez, el nodo en estado activo calcula un resultado y notifica a
todos sus nodos adyacentes.

  TODO: Forzar que Nodes sea un grafo acíclico en la descripción.

  El dispatcher, realizará implicitamente un orden topológico de los
nodos, como $Nodes$ es un grafo acíclico, éste proceso es posible y
termina, y cada salida cuenta con un valor, que será mapeado a los actuadores.

