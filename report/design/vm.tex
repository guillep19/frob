  
  La máquina virtual será la encargada de recibir el bytecode creado por
el compilador, e interpretarlo en la plataforma que esté ejecutando.
  A diferencia del compilador, es necesario implementar una máquina virtual
para cada arquitectura objetivo.\\

  Por ejemplo, para ejecutar programas en un robot
  con un procesador \emph{arduino}, debe
  existir una implementación de la máquina para ese modelo
  de \emph{arduino}.

  Al momento de implementar la máquina, se tomará en cuenta ésto para
  factorizar partes en común y sólo implementar por arquitectura, las
  partes que realmente sean diferentes como ser la comunicación con
  los periféricos de entrada/salida y las llamadas al sistema.


  TODO: Describir el diseño de la máquina virtual.

  La máquina virtual va a ser basada en stack. Ésto significa
que las operaciones van a tomar sus argumentos del stack y
colocar resultados en el mismo.


  TODO: Poner las partes de la máquina virtual acá:

\begin{enumerate}
\item \emph{Inputs}

  Los valores leídos en las entradas de hardware se mapean
en esta lista. Por ejemplo si el hardware cuenta con un botón,
y el identificador del botón es $i$,
su valor se representará con la notación:

  $Inputs_i$

  A cada entrada se le asocia un conjunto de
rutinas que deben invocarse cuando se tenga un
valor disponible en la entrada. El mismo puede ser vacío si
no hay rutinas esperando por su valor. Queda a criterio de quien
implementa la máquina si éstos valores deben ser guardados o
descartados.
  A éste conjunto de rutinas lo denotamos cómo:

  $Callbacks_i$

\item \emph{Nodes}

  Es utilizado para almacenar todas las transformaciones de señales.
  Cada nodo, tiene una lista de otros nodos que dependen de él y la
  posición del argumento por el que esperan.
  Además el nodo almacena el último valor calculado, y una lista
  de argumentos que le serán pasados, si son nuevos o no.

  Cada aplicación de \texttt{lift}, \texttt{lift2} o \texttt{folds}
será mapeada en un nodo.

  Cada nodo $Node_i$ tiene una lista de nodos adyacentes
que dependen de él.

\item \emph{Outputs}

  Mapea señales en salidas de hardware, los valores
son asignados con la operación \texttt{output}.

\item \emph{Stack}

  Es una pila global, se utiliza para ejecutar operaciones y
realizar cálculos. 
  Es única y global, y los hilos de ejecución no pueden guardar valores
persistentes en ella.

Usaremos el símbolo $TOS$ para referirnos al elemento en el tope
de la pila y $Stack$ para referirnos a la pila.

\item \emph{Ready}

Es una \emph{cola} que contiene punteros a los nodos
listos para ser ejecutados.

\item \emph{Dispatcher}

  El dispatcher, es quien implementa las acciones de la máquina.
  El mismo se encarga de recibir valores de los sensores y mapearlos
a eventos en las entradas $Input_i$.

  Éstos eventos, serán recibidos por los nodos $Nodes$.
  Cada $Node_i$ que espera por eventos entrará en estado activo cuando
todos los nodos por los que espera le envíen un evento.
  A su vez, el nodo en estado activo calcula un resultado y notifica a
todos sus nodos adyacentes.

  TODO: Forzar que Nodes sea un grafo acíclico en la descripción.

  El dispatcher, realizará implicitamente un orden topológico de los
nodos, como $Nodes$ es un grafo acíclico, éste proceso es posible y
termina, y cada salida cuenta con un valor, que será mapeado a los actuadores.

\end{enumerate}


