  La máquina virtual será la encargada de recibir el bytecode creado por
el compilador, e interpretarlo en la plataforma que esté ejecutando.
  A diferencia del compilador, es necesario implementar una máquina virtual
para cada arquitectura objetivo.\\

  Por ejemplo, para ejecutar programas en un robot
  con un procesador \emph{arduino}, debe
  existir una implementación de la máquina para ese modelo
  de \emph{arduino}.

  Al momento de implementar la máquina, se tomará en cuenta ésto para
  factorizar partes en común y sólo implementar por arquitectura, las
  partes que realmente sean diferentes como ser la comunicación con
  los periféricos de entrada/salida y las llamadas al sistema.


TODO: Describir el diseño de la máquina virtual.
