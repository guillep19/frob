En esta seccion se describe el diseño del lenguaje, su semántica,
así como el diseño del lenguaje de bajo nivel, y cómo es interpretado.

\section{Lenguaje de alto nivel}

\subsection{Sintaxis}

\subsection{Semántica}


\subsection{Ejemplo}

En el siguiente ejemplo se muestra como haría para detener un
robot cuando su sensor de distancia muestra un valor menor a un mínimo.

\begin{verbatim}
MINIMO = 30
INPUT_DISTANCE = 1
OUTPUT_MOTOR = 1

distanceToSpeed :: Number -> Number
distanceToSpeed n =
   if (n < MINIMO) 0 else 100

main :: IO()
main = let distance = read(INPUT_DISTANCE) in
   output OUTPUT_MOTOR (mapE (distanceToSpeed) distance)

\end{verbatim}

Primero se definió una fuente de eventos llamada \emph{distance},
de tipo \emph{Event Number}.

\begin{verbatim}
let distance = read(INPUT_DISTANCE)
\end{verbatim}

La misma emitirá un evento con la distancia en \emph{cm} leída
por un sensor en el robot.

  Luego, se quiere que cuando la distancia sea menor al valor
\emph{MINIMO} dado, el robot se detenga completamente, ésto
quiere decir, que la velocidad sea $0$, en otro caso el mismo
se mueve a velocidad $100$, la cuál asumimos es una velocidad
apropiada.
  Para ésto se crea una nueva fuente de eventos, definida a partir
de \emph{distance}, aplicandole la función \texttt{distanceToSpeed} 
a cada valor para obtener la velocidad.

\begin{verbatim}
(mapE (distanceToSpeed) distance)
\end{verbatim}

  Ésta expresión, también es una fuente de eventos
de tipo \emph{Event Number}.

  Finalmente, se aplica la función nativa \texttt{output} cuyo
primer parámetro es el $id$ de la salida, en éste caso es \texttt{OUTPUT\_MOTOR}
y su segundo parámetro es una fuente de eventos,
en éste caso son velocidades.



\section{Lenguaje de bajo nivel}

 Para presentar el lenguaje, primero defino las estructuras
necesarias para explicar la semántica de cada instrucción.

\begin{enumerate}

\item \emph{Inputs}

  
  Es utilizado para almacenar todos los eventos de una aplicación.
  Es una cola de eventos.

  

  Los valores leídos en las entradas de hardware se mapean
en esta lista. Por ejemplo si el hardware cuenta con un botón,
y el identificador del botón es $i$,
su valor se representará con la notación:

  $Inputs_i$

  A cada entrada se le asocia un conjunto de
rutinas que deben invocarse cuando se tenga un
valor disponible en la entrada. El mismo puede ser vacío si
no hay rutinas esperando por su valor. Queda a criterio de quien
implementa la máquina si éstos valores deben ser guardados o
descartados. 
  A éste conjunto de rutinas lo denotamos cómo:

  $Callbacks_i$

\item \emph{Events}


\item Stack

Es una pila global, se utiliza para ejecutar operaciones,
realizar cálculos, es único
y global, y los hilos de ejecución no pueden guardar valores
persistentes en él.

Usaremos el símbolo $TOS$ para referirnos al elemento en el tope
de la pila y $Stack$ para referirnos a la pila.

\item Ready

Es una \emph{cola} que contiene punteros a hilos de ejecución
listos para ser ejecutados.

\end{enumerate}

\subsection{Instrucciones}

Las siguiente tabla muestra las instrucciones de bajo nivel junto con su semántica.

\begin{itemize}
\item Recv inm

 $nextip = ip + 1$

 $CallBacks_{inm} := CallBacks_{inm} \cup nextip$
 
 $ip = 0$

\item Send inm value

 Length: 2

 $Events_{inm} := Events_{inm} : \{inm, value, now\}$

\item Output inm value
\end{itemize}
