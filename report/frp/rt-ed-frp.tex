%% Listo

  Si bien el paradigma clásico de FRP permite expresar naturalmente
programas reactivos, ésta expresividad no es gratuita, sino que
puede llevar a errores muy difíciles de encontrar en los programas.
  Un ejemplo de ésto es el llamado \emph{time leak}, al implementarse
en un lenguaje como Haskell, que cuenta con evaluación a demanda,
los cálculos a demanda sobre los \emph{Behaviours} puede que se
retrasen y se acumulen, y al momento de necesitarse, el cálculo
es tan largo que deja al programa sin memoria o afecta demasiado
el tiempo de respuesta.

  También pueden ocurrir \emph{space leaks} donde un cálculo se
retrasa indefinidamente, y la acumulación de los mismos consume
el total de la memoria.

  Como solución a ésto se propuso
Real-Time FRP \cite{wan2001:rtfrp}, una simplificación que
garantiza mayor eficiencia.
  Utilizando el tipo de datos paramétrico Maybe
  \footnote{
    El tipo \texttt{Maybe} $\alpha$ en Haskell tiene dos
    valores posibles,
    \texttt{Just} $\alpha$ y \texttt{Nothing}.
  },
  se realiza un isomorfismo entre Events y Behaviour.

\begin{definicion}
  Isomorfismo en RT-FRP entre Events y Behaviour.
  \center{
    $\textbf{Events}\ \alpha \approx \textbf{Behaviour}\ (\textbf{Maybe}\ \alpha)$
  }
\end{definicion}

  Utilizando ésta simplificación, se agrupan las dos definiciones en un
  nuevo tipo llamado \texttt{Signal}.

\begin{definicion}
  Señal (Signal).
  \center{
    $\textbf{Signal}\ \alpha = \textbf{Time} \rightarrow \alpha$
  }
\end{definicion}

  Para garantizar las restricciones de que el tiempo y el espacio requerido
por los programas es acotado, RT-FRP define un lenguaje base (lambda cálculo)
de alto órden, y luego sobre esa base define un lenguaje reactivo que obliga
a declarar las señales y sus conexiones.

  Sobre este lenguaje restringido, se proveen demostraciones de que
se cumplen las restricciones.

\paragraph{Event-Driven FRP}
  Poco después de ésto, se propuso \textit{Event-Driven FRP}
\cite{wan2002:edfrp}, otra simplificación que añade como restricción
que las señales sólo puedan ser modificadas mediante eventos discretos,
en lugar de ser contínuamente actualizadas con el paso del tiempo.

  Aunque parezca muy restrictiva, la justificación de la misma es que
los sistemas reactivos que se desean implementar están fuertemente
guiados por eventos.

  La propuesta de Event-Driven FRP es llevar la programación reactiva
a microcontroladores, para lo cuál define un lenguaje imperativo llamado
\textit{Simple C}, de tal forma que a partir del mismo sea muy simple
compilarlo al lenguaje \textit{C} o a las variantes que existen para 
microcontroladores.

  En particular se implementó un prototipo que es capaz de generar
código para el microcontrolador \textit{PIC16C66} \cite{microchip}.

