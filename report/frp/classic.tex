
%% FRP Clasico

El paradigma FRP comenzó a ser utilizado por Paul Hudak y Conal Elliot en
Fran: (Functional Reactive Animation \cite{ElliottHudak97:Fran})
para crear animaciones interactivas de forma declarativa.

Su implementación está embebida en el lenguaje Haskell.

Los programas funcionales puros, no permiten modificar valores,
sinó que una función siempre retorna el mismo valor dadas las mismas
entradas, sin causar efectos secundarios.

Ésta propiedad es deseable para fomentar la reutilización del código
pero no ayuda a mantener un estado.

En la programación reactiva, es necesario mantener un estado por
ejemplo para saber la posición del puntero del mouse en una interfaz,
o para saber la ubicación de un robot.

En FRP para representar estado, se modela como valores dependientes
del paso del tiempo.

Para ésto Fran define dos abstracciones principales:
 
\begin{definicion}
  Comportamiento (Behaviour).\\

  Un comportamiento es una función que dado un instante de tiempo
  retorna un valor.\\
  
  $\textbf{Behaviour} \alpha = \textbf{Time} \rightarrow \alpha$

\end{definicion}

  Los comportamientos son muy útiles al realizar animaciones,
para modelar propiedades físicas como velocidad, posición.
  Ésta abstracción permite que el desarrollador solo se ocupe de
definir como se calcula un valor, sin implementar la actualización
del mismo y dejando esos detalles al compilador.

\begin{definicion}
  Eventos. (Events)\\

  Los eventos representan una colección discreta finita o infinita de valores
  junto al instante de tiempo en el que cada uno ocurre.\\

  $\textbf{Event} \alpha = [(\textbf{Time}, \alpha)]$

\end{definicion}

  Los eventos se utilizan para representar entradas discretas como por
ejemplo cuando una tecla oprime en el teclado.\\

  
