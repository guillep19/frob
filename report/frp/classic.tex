
%% FRP Clasico

El paradigma FRP comenzó a ser utilizado por Paul Hudak y Conal Elliot en
Fran (Functional Reactive Animation \cite{ElliottHudak97:Fran})
para crear animaciones interactivas de forma declarativa.

Su implementación está embebida en el lenguaje Haskell.

Los programas funcionales puros, no permiten modificar valores,
sino que una función siempre retorna el mismo valor dadas las mismas
entradas, sin causar efectos secundarios.

Esta propiedad es deseable para fomentar la reutilización del código
pero no ayuda a mantener un estado.
En la programación reactiva, es necesario mantenerlo por
ejemplo para saber la posición del puntero del mouse en una interfaz,
o para saber la ubicación de un robot.

En FRP para representar estado, éste se modela como valores dependientes
del paso del tiempo.

Para esto, Fran define dos abstracciones principales,
que son \emph{Eventos} y \emph{Comportamientos}.

\begin{definicion}
  Comportamiento (Behaviour). \cite{petersonhudakelliot99:lambdainmotion} 
      \footnote{La definición de comportamientos en Fran no coincide con
      la definición de comportamiento normalmente utilizada en robótica.
      En bibliografía posterior, comportamientos fue cambiado por señales
      para evitar ésta ambiguedad.}\\

  Un comportamiento es una función que dado un instante de tiempo
  retorna un valor.

  \center{$\textit{type}\ \textbf{Behaviour}\ \alpha = \textbf{Time} \rightarrow \alpha$}

\end{definicion}

  Los comportamientos son muy útiles al realizar animaciones,
para modelar propiedades físicas como velocidad o posición.
  Esta abstracción permite que el desarrollador solo se ocupe de
definir cómo se calcula un valor, sin implementar la actualización
del mismo y dejando esos detalles al compilador.

  Ejemplos de comportamientos aplicados a robótica pueden ser:

\begin{itemize}
  \item \textit{entrada} sensor de distancia, temperatura, video.
  \item \textit{salida} velocidad, voltaje.
  \item \textit{estado} explícito como saber que tarea se está haciendo.
\end{itemize}

  Ejemplos de funciones que se pueden aplicar a los
comportamientos incluyen:

\begin{itemize}
\item \textit{Operaciones genéricas} Aritmética, integración, diferenciación
\item {
    \textit{Operaciones específicas de un dominio} como escalar video,
    aplicar filtros, detección de patrones.
}
\end{itemize}

%TODO(Marcos): No entiendo, se pueden por ejemplo sumar comportamientos?
% Quizas un ejemplo sea util.

\begin{definicion}
  Eventos. (Events)\cite{petersonhudakelliot99:lambdainmotion} \\

  Los eventos representan una colección discreta finita o infinita de valores
  junto al instante de tiempo en el que cada uno ocurre.

  \center{$\textit{type}\ \textbf{Events}\ \alpha = [(\textbf{Time}, \alpha)]$}

\end{definicion}

  Los eventos se utilizan para representar entradas discretas como por
ejemplo cuando una tecla es oprimida, o cuando se recibe un
mensaje o una interrupción.

  También pueden ser generados a partir de valores de un comportamiento,
como ser \emph{Temperatura alta}, \emph{Batería baja}, etc.

  Se puede generar nuevos eventos a partir de eventos existentes, por
ejemplo aplicando funciones a eventos, o seleccionando ciertos valores
usando las funciones \texttt{map} y \texttt{filter}:

\begin{itemize}
  \item {
      \textit{map} Obtiene un nuevo evento aplicando una función
                   a un evento existente.
  }
  \item {
      \textit{filter} Selecciona valores que son relevantes.
  }
\end{itemize}

%TODO(Marcos): Se podrian poner pequenos ejemplos de aplicaciones,
% como tambien antes se podrian poner ejemplos de comportamientos y eventos.

