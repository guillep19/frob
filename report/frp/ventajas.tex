
  La motivación para utilizar el paradigma presentado,
es que un programa reactivo escrito de forma iterativa
es susceptible a contener errores de concurrencia al modificar
valores en diferentes rutinas.
  A su vez, es difícil estructurar un programa iterativo para
que reaccione rápidamente a los cambios.

  Un patrón utilizado comunmente para estructurar un programa reactivo es el patrón \texttt{Observer}.
En dicho patrón, un \texttt{Sujeto} puede ser observado por un \texttt{Observador}, éste último se
subscribe al \texttt{Sujeto}, el cuál notifica a todos sus \texttt{Observadores} cuando su valor cambia.
  
  La desventaja de hacer un programa reactivo siguiendo ese esquema, es que es difícil ver en que
momento ocurren las actualizaciones de los \texttt{Observadores}.
  Si hay muchos \texttt{Observadores} suscritos a cambios de muchos \texttt{Sujetos},
se vuelve complejo mantener el código al tener tantas interacciones implícitas.

  Al usar un lenguaje funcional, las interacciones son especificadas declarativamente, y se puede
entender como un valor es formado a partir de otros.

  Además se cumple que una función invocada con las mismas entradas en diferentes
instantes de tiempo, siempre retorna el mismo valor, lo que ayuda a razonar sobre un programa.

  Para ver que es lo que está sucediendo en un programa se pueden obtener los valores de
  todas las entradas en un instante de tiempo, como si fuera una fotografía 
  y evaluar las señales para encontrar errores o corroborar si el mismo es correcto.

  De ésta manera si un programa tiene un error, se puede tomar una secuencia de instantes,
como si fuera una grabación y entender donde está el problema claramente,
sin necesidad de seguir varios hilos de ejecución ni razonar sobre la concurrencia.
