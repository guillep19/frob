
  La motivación para utilizar el paradigma presentado, es que un programa reactivo, si es escrito de
forma iterativa, es susceptible a cometer errores de concurrencia al modificar valores en diferentes
rutinas. A su vez, es difícil estructurar un programa iterativo para que reaccione rápidamente a
los cambios.

  Un patrón utilizado comunmente para estructurar un programa reactivo es el patrón \texttt{Observer}.
En dicho patrón, un \texttt{Sujeto} puede ser observado por un \texttt{Observador}, éste último se
subscribe al sujeto, y el sujeto notifica a todos sus observadores cuando su valor cambia.
  
  La desventaja de hacer un programa reactivo siguiendo ese esquema, es que es difícil ver en que
momento ocurren las actualizaciones de los observer. Si hay muchos observer suscritos a cambios
de varios sujetos, se vuelve complejo mantener el código al tener tantas interacciones implícitas.

  Al usar un lenguaje funcional, las interacciones son especificadas declarativamente, y se puede
entender como un valor es formado a partir de otros.
  Para ver que es lo que está sucediendo en un programa se pueden obtener los valores en un instante
de tiempo, como una fotografía y evaluar los errores o corroborar si el mismo es correcto.\\
  De ésta manera si un programa tiene un error, se puede tomar una secuencia de instantes,
como si fuera una grabación y entender donde está el problema,
sin necesidad de seguir varios hilos de ejecución ni razonar sobre la concurrencia.

  A su vez, no es necesario tener toda

