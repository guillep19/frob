
Tradicionalmente los programas reactivos
se escriben como una secuencia de acciones imperativas.
Existe un ciclo de control principal, donde en cualquier
momento se leen valores de las entradas, se procesan,
se mantiene un estado y se escriben valores en las salidas.

Los programas también suelen formarse por eventos
y código imperativo que se ejecuta cuando un evento ocurre.

Dicho código imperativo suele hacer referencia y manipular
un estado global desde diferentes rutinas.

Esta forma de programación tiene como consecuencia que
como un valor puede ser manipulado desde diferentes lugares,
puedan producirse problemas de concurrencia o llegar a un
estado global inconsistente.

En el paradigma FRP no hay un estado compartido explícito,
un programa se forma con valores dependientes del tiempo
cuya única forma de ser modificados
es a partir de su definición, preservando la consistencia.

  Un programa reactivo es aquel que interactúa con el ambiente,
intercalando entradas y salidas dependientes del tiempo.
  Por ejemplo un reproductor de música, video juegos o
controladores robóticos.
  Difiere de los programas \textbf{transformacionales} los cuáles
toman una entrada inicial y producen una salida
completa al finalizar su ejecución. Por ejemplo un compilador.

% TODO(Jorge): Capaz que poner un parrafo de intro a la prog funcional.
