
\section{Programación Funcional Reactiva}


% TODO(Marcos): Decir por que se decidio definir un lenguaje propio y no
% utilizar los que se estudiaron anteriormente.
% Tambien decir que comparte con esos lenguajes y en que difiere.

  En este capítulo se describe el diseño del lenguaje \frob{} junto con su semántica.
  Luego se explica de qué manera es traducido al lenguaje \alf{} de
bajo nivel, mas simple de interpretar, el cual podrá ser interpretado
por implementaciones de una misma máquina virtual en diferentes
plataformas de hardware.

  También se describirán las etapas de compilación, desde que se escribe
un programa en alto nivel hasta que es ejecutado en una
plataforma objetivo.

  El diagrama de la Figura \ref{fig:compilacion} resume todas las etapas y
componentes necesarios.

\begin{figure}[h]
\begin{center}
\caption{Etapas y componentes}
\includegraphics[width=0.9\textwidth]{graphs/compilacion.png}
\label{fig:compilacion}
\end{center}
\end{figure}

  En la parte de arriba de la Figura se ve un programa en el
lenguaje \frob{} de alto nivel.
  El desarrollador escribe dicho programa y
ejecuta el compilador \compilador{} y obtiene un archivo \alf{} binario.

  Debajo se muestran diversas plataformas, cada una
con su implementación de la máquina virtual \maquinavirtual{} instalada.

  El desarrollador podrá cargar el mismo código \alf{} en cualquier robot que
esté construido utilizando cualquiera de las plataformas, y la máquina
virtual se encargará de interpretarlo.


\subsection{FRP Clásico}

%% FRP Clasico

El paradigma FRP comenzó a ser utilizado por Paul Hudak y Conal Elliot en
Fran (Functional Reactive Animation \cite{ElliottHudak97:Fran})
para crear animaciones interactivas de forma declarativa.

Su implementación está embebida en el lenguaje Haskell.

Los programas funcionales puros, no permiten modificar valores,
sino que una función siempre retorna el mismo valor dadas las mismas
entradas, sin causar efectos secundarios.

Esta propiedad es deseable para fomentar la reutilización del código
pero no ayuda a mantener un estado.
En la programación reactiva, es necesario mantenerlo por
ejemplo para saber la posición del puntero del mouse en una interfaz,
o para saber la ubicación de un robot.

En FRP para representar estado, éste se modela como valores dependientes
del paso del tiempo.

Para esto, Fran define dos abstracciones principales,
que son \emph{Eventos} y \emph{Comportamientos}.

\begin{definicion}
  Comportamiento (Behaviour). \cite{petersonhudakelliot99:lambdainmotion} 
      \footnote{La definición de comportamientos en Fran no coincide con
      la definición de comportamiento normalmente utilizada en robótica.
      En bibliografía posterior, comportamientos fue cambiado por señales
      para evitar ésta ambiguedad.}\\

  Un comportamiento es una función que dado un instante de tiempo
  retorna un valor.

  \center{$\textit{type}\ \textbf{Behaviour}\ \alpha = \textbf{Time} \rightarrow \alpha$}

\end{definicion}

  Los comportamientos son muy útiles al realizar animaciones,
para modelar propiedades físicas como velocidad o posición.
  Esta abstracción permite que el desarrollador solo se ocupe de
definir cómo se calcula un valor, sin implementar la actualización
del mismo y dejando esos detalles al compilador.

  Ejemplos de comportamientos aplicados a robótica pueden ser:

\begin{itemize}
  \item \textit{entrada} sensor de distancia, temperatura, video.
  \item \textit{salida} velocidad, voltaje.
  \item \textit{estado} explícito como saber que tarea se está haciendo.
\end{itemize}

  Ejemplos de funciones que se pueden aplicar a los
comportamientos incluyen:

\begin{itemize}
\item \textit{Operaciones genéricas} Aritmética, integración, diferenciación
\item {
    \textit{Operaciones específicas de un dominio} como escalar video,
    aplicar filtros, detección de patrones.
}
\end{itemize}

  Por ejemplo dado el comportamiento \texttt{time} que representa el tiempo,
dada una aceleración constante \texttt{a} de un objeto, se puede definir la
velocidad en función del tiempo como:
    \begin{center}
    $\texttt{speed} = \texttt{a} * \texttt{time}$.
    \end{center}
  
  A su vez se puede definir la posición en función del tiempo como:
  \begin{center}    
  $\texttt{position} = \texttt{speed} * \texttt{time}$.
  \end{center}

\begin{definicion}
  Eventos. (Events)\cite{petersonhudakelliot99:lambdainmotion} \\

  Los eventos representan una colección discreta finita o infinita de valores
  junto al instante de tiempo en el que cada uno ocurre.

  \center{$\textit{type}\ \textbf{Events}\ \alpha = [(\textbf{Time}, \alpha)]$}

\end{definicion}

  Los eventos se utilizan para representar entradas discretas como por
ejemplo cuando una tecla es oprimida, o cuando se recibe un
mensaje o una interrupción.

  También pueden ser generados a partir de valores de un comportamiento,
como ser \emph{Temperatura alta}, \emph{Batería baja}, etc.

  Se puede generar nuevos eventos a partir de eventos existentes, por
ejemplo aplicando funciones a eventos, o seleccionando ciertos valores.
%usando las funciones \texttt{map} y \texttt{filter}:

%\begin{itemize}
%  \item {
%      \textit{map} Obtiene un nuevo evento aplicando una función
%                   a un evento existente.
%  }
%  \item {
%      \textit{filter} Selecciona valores que son relevantes.
%  }
%\end{itemize}

%TODO(Marcos): Se podrian poner pequenos ejemplos de aplicaciones,
% como tambien antes se podrian poner ejemplos de comportamientos y eventos.




\subsection{Para corregir}

  Luego fue generalizada la idea para poder representar otro tipo de valores que
varían en el tiempo, en particular un lenguaje específico de dominio llamado
Frob \cite{petersonhudakelliot99:lambdainmotion} fue creado para poder especificar
robots. El mismo utiliza Fran como base y crea abstracciones de los sensores
para poder abstraerse de las plataformas y de las dificultades de un hardware en particular.
  La parte más importante de Frob es la idea de crear abstracciones, y que no formen
parte de lo que debe resolver un programa.


\section{A-FRP}

  AFRP, significa \textit{Arrowized FRP}, es un paradima derivado del paradigma FRP.

  Otro framework más completo es Yampa \cite{yampa}, un lenguaje embebido en Haskell,
que busca 
  herramienta que expresa los programas como señales y
define operadores simples para combinarlas y funciones que procesan las señales.
  En Yampa, las señales dejan de ser objetos de primera clase, y las funciones que
las combinan pasan a tomar mayor importancia.


  
Para entender más la problemática se puede plantear el siguiente ejemplo:

\begin{center}
  Dada una cuenta bancaria, se quiere mantener el saldo disponible,
y realizar movimientos sobre la misma.\\

  Los movimientos posibles son extraer dinero, o depositar dinero,
o sea restar o sumar un valor al saldo.\\

  Al crearse una cuenta, el saldo inicial es el valor cero.\\

  El saldo de la cuenta, sólo puede ser modificado mediante un movimiento,
  y nunca debe ser menor a cero\\

  Un dato importante a ver, es que no es posible asignar un valor
sin que sea por medio de su propia definición, por lo que nadie
podría realizar la asignación $saldo = 1000000$.
  La misma operación sería posible creando un movimiento, el cuál
afectaría al saldo.
\end{center}

  Para modelar los movimientos, éstos pueden verse como una
secuencia de valores enteros, que ocurren en un instante de tiempo.

  Dada la definición se puede decir que contamos con una secuencia de
entradas \textbf{movimientos} de tipo \textbf{Event Number} denotado cómo:

\begin{verbatim}
  movimientos :: Event Number
\end{verbatim}

  A partir de \textbf{movimientos} puedo definir el saldo.
El mismo, es un valor en función del tiempo, pero no ocurre en
instantes discretos, sinó que es contínuo respecto al tiempo a lo
que llamaremos \textbf{Señal}.

  Las señales representan cualquier valor en función del tiempo,
por ejemplo:

\begin{itemize}
\item \textit{entrada} sensor de distancia, temperatura, video
\item \textit{salida} velocidad, voltaje
\item \textit{valores} intermedios calculados
\end{itemize}

Las operaciones que se pueden realizar sobre las señales incluyen:

\begin{itemize}
\item \textit{Operaciones genéricas} Aritmética, integración, diferenciación
\item \textit{Operaciones específicas de un dominio} como escalar video, aplicar filtros, detección de patrones.
\end{itemize}

Los eventos pueden ser ciertos sensores que aportan valores discretos,
específicos de un dominio, por ejemplo un
botón, un click, una interrupción o un mensaje asincrónico.\\
También puede ser generado a partir de valores de una señal,
como ser \emph{Temperatura alta}, \emph{Batería baja}.

Ejemplos de operaciones que se pueden realizar sobre los eventos pueden ser:

\begin{itemize}
\item \textit{map} Modifica un valor discreto a otro aplicando una función
\item \textit{filter} Selecciona valores que son relevantes.
\end{itemize}


  
  Por último existe un lenguaje llamado Elm \cite{evanczaplicki2012:Elm}, creado para
poder escribir aplicaciones web reactivas, de una forma funcional declarativa.
  En Elm, las entradas se asumen conocidas y son dadas, por ejemplo un click o el movimiento del mouse,
o una tecla presionada.\\
  Todos los valores son señales, el combinador $lift$
toma una señal y una función, y define otra señal resultado de la aplicación de la función
sobre cada valor de la señal.\\
  Para combinar más de una señal, se usa el combinador $lift_2$ o $lift_n$ con $n \in N$.\\
  En los casos que se desea tener memoria, por ejemplo llevar una cuenta de ocurrencias
  de una señal, o mantener un estado explícito, se utiliza el combinador $foldp$.\\

  El combinador $foldp$ opera sobre una señal como el operador $fold$ sobre una lista. Dado
un valor inicial y una función, aplica la función sobre cada valor, utilizando el último
valor conocido como primer argumento.


\section{Ejemplo}

  Usaremos la notación \texttt{<nombre> :: Event <tipo>} para denotar
fuentes de eventos.

  Con ella puedo expresar el saldo a partir de los mismos, y construir
un programa que muestre el saldo.

\begin{verbatim}
main = let saldos = (foldls (+) SALDO_INICIAL movimientos) in
     maps show saldos
\end{verbatim}

  Al especificarse $saldos$ declarativamente, es imposible asignar un valor al saldo,
éste se compone a partir del \texttt{SALDO\_INICIAL} y la suma de
los movimientos (positivos o negativos).
  El saldo siempre tiene un valor, por lo que es un \texttt{Comportamiento}, los movimientos
son discretos, por lo que es un \texttt{Evento}.

\section{Simplificación del paradigma}

  Al intentar implementar en un computador programas utilizando éste paradigma, nos encontramos
con varias limitaciones.
  Una de ellas es que no es posible tener valores que se modifiquen de forma contínua, la capacidad
de cómputo es finita, y no es posible ejecutar tareas al mismo tiempo, incluso en un entorno con
paralelismo, el mismo está acotado a la cantidad de procesadores con los que se cuente.\\

  Aunque el paradigma distinga Eventos de Comportamientos, se puede hacer una simplificación y
asumir que todos los valores son Eventos. Los comportamientos en realidad serán una secuencia de
cambios de valor, los eventos que dependan de un comportamiento serán notificados sólo en el momento
que éste cambie.\\
  Valores como el tiempo (reloj) y otros sensores como ser un sensor de distancia, entregarán valores
periódicamente. Sería imposible que un programa reaccione a un valor contínuo, sin embargo, es muy fácil
asumir que lo que nos importa del tiempo en realidad es dada una variación de tiempo cómo se debe
comportar nuestro programa.\\
  A su vez un sensor de distancia no nos puede entregar infinitos valores, y generalmente sólo importa
poder recibir un valor nuevo en un período relativamente corto de tiempo.

  Otra limitación en nuestro caso, al tratarse de robots con bajas capacidades de cómputo, es que
la implementación del lenguaje no siempre tendrá más de un procesador disponible.\\
  Para simular éste paralelismo se asumirá que el tiempo que se necesita para hacer cálculos es de
una magnitud mucho menor al tiempo que lleva recibir un dato de una entrada, o enviar un dato a una salida.
  Se implementará una máquina capaz de correr dichos programas, la cuál esperará por valores en las entradas,
y en base a los mismos, actualizará todos los eventos que dependan de ellas, y a su vez actualizará luego
las salidas que dependan de éstos.\\
  Cada actualización se hará secuencialmente hasta terminar, y en caso de contar con más de un procesador, se
asignarán en paralelo las actualizaciones utilizando varios hilos de ejecución.\\

\section{Ventajas}

  La motivación para utilizar el paradigma presentado, es que un programa reactivo, si es escrito de
forma iterativa, es susceptible a cometer errores de concurrencia al modificar valores en diferentes
rutinas. A su vez, es difícil estructurar un programa iterativo para que reaccione rápidamente a
los cambios.

  Un patrón utilizado comunmente para estructurar un programa reactivo es el patrón \texttt{Observer}.
En dicho patrón, un \texttt{Sujeto} puede ser observado por un \texttt{Observador}, éste último se
subscribe al sujeto, y el sujeto notifica a todos sus observadores cuando su valor cambia.
  
  La desventaja de hacer un programa reactivo siguiendo ese esquema, es que es difícil ver en que
momento ocurren las actualizaciones de los observer. Si hay muchos observer suscritos a cambios
de varios sujetos, se vuelve complejo mantener el código al tener tantas interacciones implícitas.

  Al usar un lenguaje funcional, las interacciones son especificadas declarativamente, y se puede
entender como un valor es formado a partir de otros.
  Para ver que es lo que está sucediendo en un programa se pueden obtener los valores en un instante
de tiempo, como una fotografía y evaluar los errores o corroborar si el mismo es correcto.\\
  De ésta manera si un programa tiene un error, se puede tomar una secuencia de instantes,
como si fuera una grabación y entender donde está el problema,
sin necesidad de seguir varios hilos de ejecución ni razonar sobre la concurrencia.

  A su vez, no es necesario tener toda

