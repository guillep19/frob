



\section{Programación Funcional Reactiva}

Tradicionalmente los programas son formados a partir de
una secuencia de acciones imperativas. Los programas
reactivos suelen formarse por eventos y código iterativo
que se corre cuando un evento ocurre.

Dicho código iterativo suele hacer referencia y manipular
variables compartidas con diferentes rutinas. Ésto lleva a que
como un valor puede ser manipulado desde diferentes lugares,
puedan producirse problemas de concurrencia y algunos valores
pueden quedar en un estado inconsistente.

En el paradigma FRP no existen valores compartidos, sinó que
dichos valores dependientes del tiempo, tienen una representación
llamada Comportamiento y la única forma de modificarlos, es
a partir de como fueron definidos.

\begin{definicion}
Programa reactivo.\\
Es aquel que interactúa con el ambiente, intercalando entradas
y salidas dependientes del tiempo. Por ejemplo un reproductor
de música, video juegos o controladores robóticos.

Difiere de los programas \emph{transformacionales} los cuáles
toman una entrada al inicio de la ejecución y producen una salida
completa al final. Por ejemplo un compilador.
\end{definicion}

\begin{definicion}
Comportamientos (Behaviours).\\
Un comportamiento es un valor contínuo que depende del paso del tiempo.
Los comportamientos se pueden definir, combinar, pasarlos como
argumentos a funciones, retornarlos.
Un comportamiento puede ser un valor constante, el tiempo mismo (un reloj),
o puede formarse combinando otros comportamientos, por ejemplo secuencialmente
o paralelamente.
\end{definicion}

\begin{definicion}
Eventos (Streams).\\
Son valores discretos dependientes del tiempo, que forman
una secuencia finita o infinita de ocurrencias. Cada ocurrencia
está formada por el valor y el instante de tiempo.
\end{definicion}

La principal diferencia entre Comportamientos y Eventos, es que los
comportamientos son valores contínuos y los eventos son discretos.

Los comportamientos representan cualquier valor en función del tiempo,
por ejemplo:
\begin{itemize}
\item \textit{entrada} sensor de distancia, temperatura, video
\item \textit{salida} velocidad, voltaje
\item \textit{valores} intermedios calculados
\end{itemize}

Las operaciones que se pueden realizar sobre los comportamientos incluyen:
\begin{itemize}
\item \textit{Operaciones genéricas} Aritmética, integración, diferenciación
\item \textit{Operaciones específicas de un dominio} como escalar video, aplicar filtros, detección de patrones.
\end{itemize}

Los eventos pueden ser sensores específicos de un dominio, por ejemplo un
botón, un click, una interrupción o mensaje asincrónico.
También puede ser generado a partir de valores de un comportamiento,
como ser \emph{Temperatura alta}, \emph{Batería baja}.

Las operaciones que se pueden realizar sobre los eventos incluyen:
\begin{itemize}
\item fmap, filter
\item Modificar un \emph{Comportamiento} reactivo
\end{itemize}


\section{Ejemplo}

Para entender un poco más las ideas de Comportamiento y Evento, se puede
plantear el siguiente ejemplo.

  En una cuenta de un banco, el saldo se puede definir
como un comportamiento, el cuál solo se modifica cuando ocurre
un movimiento.
  Un movimiento puede ser depositar dinero o extraer
dinero de la cuenta.
  Un dato importante a ver, es que no es posible asignar un valor
sin que sea por medio de su propia definición, por lo que nadie
podría realizar la asignación $saldo = 1000000$.
  La misma operación sería posible creando un movimiento, el cuál
afectaría al saldo.

  Usaremos la notación \texttt{<nombre> :: Event <tipo>} para definir
fuentes de eventos.
  Asumo conocida la fuente de eventos \emph{movimientos} que por
cada movimiento
emite su valor.

\begin{verbatim}
movimientos :: Event Number
\end{verbatim}

  Ahora puedo expresar el saldo a partir de los mismos, y construir
un programa que muestre el saldo.

\begin{verbatim}
main = let saldos = (foldlE (+) SALDO_INICIAL movimientos) in
     mapE show saldos
\end{verbatim}

  En el ejemplo se ve que es imposible asignar un valor al saldo,
éste se compone a partir del \texttt{SALDO\_INICIAL} y la suma de
los movimientos (positivos o negativos).

TODO: Mejorar esta sección explicando: Signal y SF (Signal Functions),
Arrows, Behaviours y Event, Event, time leaks.

Leer y citar \cite{peterson99:lambdainmotion} y también este otro
\cite{evanczaplicki2012:Elm} y \cite{yampa}.
