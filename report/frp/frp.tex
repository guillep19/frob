



\section{Programación Funcional Reactiva}

Tradicionalmente los programas son formados a partir de
una secuencia de acciones imperativas. Los programas
reactivos suelen formarse por eventos y código iterativo
que se ejecuta cuando un evento ocurre.

Dicho código iterativo suele hacer referencia y manipular
un estado global con diferentes rutinas.
Ésto tiene como consecuencia que
como un valor puede ser manipulado desde diferentes lugares,
puedan producirse problemas de concurrencia y algunos valores
pueden quedar en un estado inconsistente.

En el paradigma FRP no hay un estado compartido explícito,
un programa se forma con valores dependientes del tiempo,
llamados Comportamientos y la única forma de modificarlos,
es a partir de su definición.


\begin{definicion}
Programa reactivo.\\
Es aquel que interactúa con el ambiente, intercalando entradas
y salidas dependientes del tiempo. Por ejemplo un reproductor
de música, video juegos o controladores robóticos.

Difiere de los programas \emph{transformacionales} los cuáles
toman una entrada al inicio de la ejecución y producen una salida
completa al final. Por ejemplo un compilador.
\end{definicion}

\begin{definicion}
Comportamientos (Behaviours).\\
Un comportamiento es un valor contínuo que depende del paso del tiempo.
Los comportamientos se pueden definir, combinar, aplicarles una
función, retornarlos.\\
Un comportamiento puede ser una entrada, el tiempo mismo (un reloj),
o puede formarse combinando otros comportamientos, por ejemplo secuencialmente
o en paralelo.
\end{definicion}

\begin{definicion}
Eventos (Streams).\\
Son valores discretos dependientes del tiempo, que forman
una secuencia finita o infinita de ocurrencias.\\
Cada evento está formado por un valor y
el instante de tiempo en el que ocurre.
\end{definicion}

La principal diferencia entre Comportamientos y Eventos, es que los
comportamientos son valores contínuos y los eventos son discretos.

Los comportamientos representan cualquier valor en función del tiempo,
por ejemplo:
\begin{itemize}
\item \textit{entrada} sensor de distancia, temperatura, video
\item \textit{salida} velocidad, voltaje
\item \textit{valores} intermedios calculados
\end{itemize}

Las operaciones que se pueden realizar sobre los comportamientos incluyen:

\begin{itemize}
\item \textit{Operaciones genéricas} Aritmética, integración, diferenciación
\item \textit{Operaciones específicas de un dominio} como escalar video, aplicar filtros, detección de patrones.
\end{itemize}

Los eventos pueden ser ciertos sensores que aportan valores discretos,
específicos de un dominio, por ejemplo un
botón, un click, una interrupción o un mensaje asincrónico.\\
También puede ser generado a partir de valores de un comportamiento,
como ser \emph{Temperatura alta}, \emph{Batería baja}.

Ejemplos de operaciones que se pueden realizar sobre los eventos pueden ser:

\begin{itemize}
\item \textit{map} Modifica un valor discreto a otro aplicando una función
\item \textit{filter} Selecciona valores que son relevantes.
\end{itemize}



  Este paradigma comenzó siendo utilizado para realizar animaciones con programación
funcional (Fran \cite{ElliottHudak97:Fran}), para crear una animación se describía
cómo y dónde deberían ubicarse imágenes en función del tiempo, de ésta forma se simula
el movimiento.\\
  Luego fue generalizada la idea para poder representar otro tipo de valores que
varían en el tiempo, un lenguaje específico de dominio llamado
Frob \cite{petersonhudakelliot99:lambdainmotion} fue creado para poder especificar
robots. El mismo utiliza Fran como base y crea abstracciones de los sensores
para poder abstraerse de las plataformas y de las dificultades de un hardware en particular.
  La parte más importante de Frob es la idea de crear abstracciones, y que no formen
parte de lo que debe resolver un programa.

  Existen varios trabajos que tienen como objetivo crear abstracciones para escribir
programas reactivos funcionales.
  Yampa \cite{yampa} es una herramienta que expresa los programas como señales y
define operadores simples para combinarlas y funciones que procesan las señales.

  Por último existe un lenguaje llamado Elm \cite{evanczaplicki2012:Elm}, creado para
poder escribir aplicaciones web reactivas, de una forma funcional declarativa.
  En Elm, las entradas se asumen conocidas y son dadas, por ejemplo un click o el movimiento del mouse,
o una tecla presionada.\\
  Todos los valores son señales, el combinador $lift$
toma una señal y una función, y define otra señal resultado de la aplicación de la función
sobre cada valor de la señal.\\
  Para combinar más de una señal, se usa el combinador $lift_2$ o $lift_n$ con $n \in N$.\\
  En los casos que se desea tener memoria, por ejemplo llevar una cuenta de ocurrencias
  de una señal, o mantener un estado explícito, se utiliza el combinador $foldp$.\\

  El combinador $foldp$ opera sobre una señal como el operador $fold$ sobre una lista. Dado
un valor inicial y una función, aplica la función sobre cada valor, utilizando el último
valor conocido como primer argumento.


\section{Ejemplo}

Para entender un poco más las ideas de Comportamiento y Evento, se puede
plantear el siguiente ejemplo.\\

  En una cuenta de un banco, el saldo se puede definir
como un comportamiento, el cuál solo se modifica cuando ocurre
un movimiento.
  Un movimiento puede ser depositar dinero o extraer
dinero de la cuenta.
  Un dato importante a ver, es que no es posible asignar un valor
sin que sea por medio de su propia definición, por lo que nadie
podría realizar la asignación $saldo = 1000000$.
  La misma operación sería posible creando un movimiento, el cuál
afectaría al saldo.

  Usaremos la notación \texttt{<nombre> :: Event <tipo>} para definir
fuentes de eventos.
  Asumo conocida la fuente de eventos \emph{movimientos} que por
cada movimiento emite su valor.

\begin{verbatim}
movimientos :: Event Number
\end{verbatim}

  Con ella puedo expresar el saldo a partir de los mismos, y construir
un programa que muestre el saldo.

\begin{verbatim}
main = let saldos = (foldls (+) SALDO_INICIAL movimientos) in
     maps show saldos
\end{verbatim}

  Al especificarse $saldos$ declarativamente, es imposible asignar un valor al saldo,
éste se compone a partir del \texttt{SALDO\_INICIAL} y la suma de
los movimientos (positivos o negativos).
  El saldo siempre tiene un valor, por lo que es un \texttt{Comportamiento}, los movimientos
son discretos, por lo que es un \texttt{Evento}.

\section{Simplificación del paradigma}

  Al intentar implementar en un computador programas utilizando éste paradigma, nos encontramos
con varias limitaciones.
  Una de ellas es que no es posible tener valores que se modifiquen de forma contínua, la capacidad
de cómputo es finita, y no es posible ejecutar tareas al mismo tiempo, incluso en un entorno con
paralelismo, el mismo está acotado a la cantidad de procesadores con los que se cuente.\\

  Aunque el paradigma distinga Eventos de Comportamientos, se puede hacer una simplificación y
asumir que todos los valores son Eventos. Los comportamientos en realidad serán una secuencia de
cambios de valor, los eventos que dependan de un comportamiento serán notificados sólo en el momento
que éste cambie.\\
  Valores como el tiempo (reloj) y otros sensores como ser un sensor de distancia, entregarán valores
periódicamente. Sería imposible que un programa reaccione a un valor contínuo, sin embargo, es muy fácil
asumir que lo que nos importa del tiempo en realidad es dada una variación de tiempo cómo se debe
comportar nuestro programa.\\
  A su vez un sensor de distancia no nos puede entregar infinitos valores, y generalmente sólo importa
poder recibir un valor nuevo en un período relativamente corto de tiempo.

  Otra limitación en nuestro caso, al tratarse de robots con bajas capacidades de cómputo, es que
la implementación del lenguaje no siempre tendrá más de un procesador disponible.\\
  Para simular éste paralelismo se asumirá que el tiempo que se necesita para hacer cálculos es de
una magnitud mucho menor al tiempo que lleva recibir un dato de una entrada, o enviar un dato a una salida.
  Se implementará una máquina capaz de correr dichos programas, la cuál esperará por valores en las entradas,
y en base a los mismos, actualizará todos los eventos que dependan de ellas, y a su vez actualizará luego
las salidas que dependan de éstos.\\
  Cada actualización se hará secuencialmente hasta terminar, y en caso de contar con más de un procesador, se
asignarán en paralelo las actualizaciones utilizando varios hilos de ejecución.\\

\section{Ventajas}

  La motivación para utilizar el paradigma presentado, es que un programa reactivo, si es escrito de
forma iterativa, es susceptible a cometer errores de concurrencia al modificar valores en diferentes
rutinas. A su vez, es difícil estructurar un programa iterativo para que reaccione rápidamente a
los cambios.

  Un patrón utilizado comunmente para estructurar un programa reactivo es el patrón \texttt{Observer}.
En dicho patrón, un \texttt{Sujeto} puede ser observado por un \texttt{Observador}, éste último se
subscribe al sujeto, y el sujeto notifica a todos sus observadores cuando su valor cambia.
  
  La desventaja de hacer un programa reactivo siguiendo ese esquema, es que es difícil ver en que
momento ocurren las actualizaciones de los observer. Si hay muchos observer suscritos a cambios
de varios sujetos, se vuelve complejo mantener el código al tener tantas interacciones implícitas.

  Al usar un lenguaje funcional, las interacciones son especificadas declarativamente, y se puede
entender como un valor es formado a partir de otros.
  Para ver que es lo que está sucediendo en un programa se pueden obtener los valores en un instante
de tiempo, como una fotografía y evaluar los errores o corroborar si el mismo es correcto.\\
  De ésta manera si un programa tiene un error, se puede tomar una secuencia de instantes,
como si fuera una grabación y entender donde está el problema,
sin necesidad de seguir varios hilos de ejecución ni razonar sobre la concurrencia.

