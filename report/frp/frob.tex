\subsubsection{Frob}

  Utilizando como base el trabajo realizado en Fran, se construyó Frob
(Functional Robotics \cite{petersonhudakelliot99:lambdainmotion}
\cite{petersonhagerhudak99:frob})
un lenguaje funcional reactivo embebido en el lenguaje \haskell{},
aplicado al dominio de la robótica.
  
  En éste trabajo se introdujo el concepto de reactividad con el cuál
utilizando los conceptos de comportamientos y eventos, éstos se combinan
para realizar las tareas que un robot debe hacer.
  La estrategia que presenta, es de formalizar las tareas por medio de
comportamientos, y conseguir que los comportamientos se modifiquen
utilizando eventos y un conjunto de combinadores específicos.

  Un ejemplo es, dado un robot, este se tiene que mover a una velocidad
constante hasta que se supere un tiempo máximo o se detecte un objeto.
  En Frob se expresaría de esta manera:

\begin{verbatim}
goAhead :: Robot -> Time -> WheelControl
goAhead r t =
  (forward 30 'untilB'
    (predicate (time > t) .|.  predicate (frontSonar r < 20))
     -=> stop)
\end{verbatim}

  Lo que se leería cómo: ``Para el robot r, moverse hacia adelante a
velocidad 30, hasta que se exceda el tiempo t, o se detecte un objeto
a menos de distancia 20. En ese momento detenerse.''

\begin{itemize}
  \item \texttt{predicate} se utiliza para generar eventos a partir
        de comportamientos en base a una condición.
  \item \texttt{untilB} cambia de comportamiento en respuesta
        a un evento.
  \item \texttt{.|.} toma dos eventos y los intercala.
  \item \texttt{-=>} Asocia un nuevo valor, luego de que ocurre el evento.
\end{itemize}

  Otro punto importante de Frob, es que los periféricos del robot se
asumen implementados, permitiendo que el desarrollador se concentre
en la lógica específica de su problema, y no en resolver problemas
del hardware.
  Además la lógica de leer las entradas, procesarlas y escribir las salidas
es realizada por el flujo de control de Frob, y no de cada programa.
  En la implementación utilizaron un esquema simple, donde se leen todos
los valores de todas las entradas y se procesan lo más rápido posible.
  Está claro que no es la mejor estrategia, porque puede causar demoras
en el procesamiento y los datos leídos son válidos por un
período corto de tiempo.


