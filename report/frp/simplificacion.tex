
  Al intentar implementar en un computador programas utilizando éste
paradigma, nos encontramos con varias limitaciones.
  Una de ellas es que no es posible tener valores que se modifiquen de
forma contínua, la capacidad de cómputo es finita, y no es posible
ejecutar tareas al mismo tiempo, incluso en un entorno con
paralelismo, el mismo está acotado a la cantidad de procesadores
con los que se cuente.

  Aunque el paradigma distinga valores contínuos de valores
  discretos, se puede hacer una simplificación y asumir que todos
  los valores son señales.
  Las señales en realidad serán una secuencia de valores en el tiempo,
las señales que dependan de otras serán notificadas en el momento
que éstas cambien.

  Valores como el tiempo (reloj) y otros sensores como ser un sensor de
distancia, entregarán valores periódicamente.
  Sería imposible que un programa reaccione a un valor contínuo,
sin embargo, es muy fácil asumir que lo que importa del tiempo en realidad
es dada una variación de tiempo cómo se debe comportar nuestro programa.

  A su vez un sensor de distancia no nos puede entregar infinitos valores,
y generalmente sólo importa poder recibir un valor nuevo en un período
relativamente corto de tiempo.
  Al igual que una película que simula ser contínua, en realidad está
compuesta de una serie discreta de imágenes en el tiempo, un robot
reacciona de manera discreta a los cambios que detecte en sus actuadores,
simulando reacción en tiempo contínuo.

  Otra limitación en nuestro caso, al tratarse de robots con bajas
capacidades de cómputo, es que la implementación del lenguaje no
siempre tendrá más de un procesador disponible.
  Para simular paralelismo se asumirá que el tiempo que se necesita
para hacer cálculos es de una magnitud mucho menor al tiempo que
transcurre para recibir un nuevo dato de una entrada, o enviar un
dato a una salida.

  Se implementará una máquina capaz de correr dichos programas,
la cuál esperará por valores en todas las entradas, y en base a los
mismos, actualice todos los eventos que dependan de ellas,
y luego actualice las salidas que dependan de éstos.

  Cada actualización se hará secuencialmente hasta terminar, y en
caso de contar con más de un procesador, se asignarán en paralelo
las actualizaciones utilizando varios hilos de ejecución.
