En esta sección se muestra la utilización del verificador en dos casos de estudios distintos.
En estos casos de estudio se verifican propiedades expresadas en LTL y en CTL para distintos
sistemas modelados por sistemas de transiciones.

%\subsection{Luces de tránsito}
%A continuación se muestra el ejemplo de las luces de tránsito.
%Para este ejemplo se considera un sistema de transiciones con los estados \textit{red},
% \textit{yellow} y \textit{green} como se puede apreciar en el archivo \textit{GraphML}
% de la figura \ref{fig:luces.gml}.
%
%\begin{figure}[hbtp]
%\begin{center}
%\caption{Archivo \textit{GraphML} para sistema de transiciones de luces de tránsito}
%\fbox{\parbox{\textwidth}{\tiny\verbatiminput{mc/graphml/luces.graphml}}}
%\label{fig:luces.gml}
%\end{center}
%\end{figure}
%
%Se desea verificar que el sistema cumple con las siguientes propiedades expresadas en LTL:
%\begin{itemize}
%\item Cuando la luz está en rojo, la luz verde no debe ser la siguiente
%\[ \square (\text{red} \to \lnot \bigcirc \text{green}) \]
%El resultado de la verificación de esta propiedad es:
%
%\fbox{\parbox{\textwidth}{\scriptsize\verbatiminput{mc/salidas/salidaLTL_1.txt}}}
%
%\item Una vez que la luz está en verde, debe cambiar posteriormente a amarillo y quedarse en amarillo
% hasta cambiar a rojo
%\[ \square (\text{green} \to \bigcirc (\text{green} \cup (\text{yellow} \land \bigcirc (\text{yellow} \cup \text{red})))) \]
%El resultado de la verificación de esta propiedad es:
%
%\fbox{\parbox{\textwidth}{\scriptsize\verbatiminput{mc/salidas/salidaLTL_2.txt}}}
%
%\end{itemize}

\subsection{Mutua exclusión con árbitro}
A continuación se muestra el ejemplo de mutua exclusión con árbitro, visto en
 capítulos anteriores.
En la figura \ref{fig:arbitro_sincronizados} se muestra el sistemas de transiciones
 correspondiente.
El archivo en formato \textit{GraphML} para este caso de estudio se encuentra en la
 sección \ref{cap:graphml_arbitro} ubicada en los apéndices.


%\begin{figure}[hbtp]
%\begin{center}
%\caption{Archivo \textit{GraphML} para sistema de transiciones de mutua exclusión con árbitro}
%\fbox{\parbox{\textwidth}{\tiny\verbatiminput{mc/graphml/Handshaking.graphml}}}
%\label{fig:handshaking.gml}
%\end{center}
%\end{figure}

Se desea verificar que los sistemas en cuestión cumplen con las siguientes propiedades en CTL:
\begin{itemize}
\item Los procesos nunca acceden simultaneamente en la sección crítica
\[ \forall \square (\lnot \text{crit}_1 \lor \lnot \text{crit}_2) \]
El resultado de la verificación de esta propiedad es:

\fbox{\parbox{\textwidth}{\scriptsize\verbatiminput{mc/salidas/salidaCTL_1.txt}}}

\item Cada proceso accede infinitas veces a la sección crítica
\[ (\forall \square \forall \lozenge \text{crit}_1) \land (\forall \square \forall \lozenge \text{crit}_2) \]
El resultado de la verificación de esta propiedad es:

\fbox{\parbox{\textwidth}{\scriptsize\verbatiminput{mc/salidas/salidaCTL_2.txt}}}

\end{itemize}


\subsection{Ascensor}
A continuación se muestra el ejemplo del ascensor, también visto en
 capítulos anteriores.
En la figura \ref{fig:HS_ascensor} se muestra el sistemas de transiciones
 correspondiente.
El archivo en formato \textit{GraphML} para este caso de estudio se encuentra en la
 sección \ref{cap:graphml_ascensor} ubicada en los apéndices.

%\begin{figure}[hbtp]
%\begin{center}
%\caption{Archivo \textit{GraphML} para sistema de transiciones de ascensor con puertas}
%\fbox{\parbox{\textwidth}{\tiny\verbatiminput{mc/graphml/ascensor_1.graphml}}}
%\label{fig:handshaking.gml}
%\end{center}
%\end{figure}
%
%\begin{figure}[hbtp]
%\begin{center}
%%\caption{Archivo \textit{GraphML} para sistema de transiciones de ascensor con puertas (2)}
%\fbox{\parbox{\textwidth}{\tiny\verbatiminput{mc/graphml/ascensor_2.graphml}}}
%%\label{fig:handshaking.gml}
%\end{center}
%\end{figure}

Se desea verificar que los sistemas en cuestión cumplen con la siguiente propiedad en LTL:
\begin{itemize}
\item Si el ascensor está subiendo o bajando entonces ambas puertas están cerradas
\[ \square ((\text{subiendo} \lor \text{bajando}) \to (\text{arriba.cerrada} \land \text{abajo.cerrada})) \]
El resultado de la verificación de esta propiedad es:

\fbox{\parbox{\textwidth}{\scriptsize\verbatiminput{mc/salidas/salidaLTL_1.txt}}}

\end{itemize}
