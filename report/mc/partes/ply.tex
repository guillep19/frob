Para el análisis sintáctico de fórmulas se utilizó la herramienta \textit{Python Lex-Yacc}.

Para reconocer los elementos de un lenguaje, esta herramienta requiere la definición de
 sus elementos mediante expresiones regulares.
Luego se debe definir la sintaxis del lenguaje mediante una gramática libre de contexto.

A continuación se detallan los elementos y las gramáticas definidas para los lenguajes
 LTL y CTL.


\subsection{LTL}\label{cap:imp_ltl}
En la siguiente tabla se muestran los operadores de LTL con su correspondiente cadena de caracteres.

\begin{table}
\begin{center}
   \begin{tabular}{ | c | c | }
     \hline
     \textbf{Operador} & \textbf{Cadena de caracteres} \\ \hline
     $\lnot\bot$ & \texttt{TRUE} \\
     $\lnot$ & - \\
     $\land$ & \texttt{/$\backslash$} \\
     $\lor$ & \texttt{$\backslash$/} \\
     $\rightarrow$ & \texttt{-$>$} \\
     $\bigcirc$ & \texttt{O} \\
     $\square$ & \texttt{[]} \\
     $\lozenge$ & \texttt{$<>$} \\
     $\cup$ & \texttt{U} \\
     \hline
   \end{tabular}
\end{center}
\caption[Representación de operadores LTL]{Cadena de caracteres correspondiente a cada operador LTL.}
\end{table}

Además se definió la siguiente expresión regular para las proposiciones atómicas.
\[ [a-z\_ ][a-z0-9\_ ]* \]

\vbox{
Una vez definidos los elementos del lenguaje se definió la sintaxis mediante la siguiente gramática.

\begin{center}
   \begin{tabular}{ | r l | }
     \hline
     formula ::= & formula $\cup$ formula \\
     & $|$ formula $\land$ formula \\
	 & $|$ formula $\lor$ formula \\
	 & $|$ formula $\rightarrow$ formula \\
	 & $|$ $\bigcirc$ formula \\
	 & $|$ $\square$ formula \\
	 & $|$ $\lozenge$ formula \\
	 & $|$ $\lnot$ formula \\
	 & $|$ proposicion \\
	 & $|$ TRUE \\
     \hline
   \end{tabular}
\end{center}
}

Esta información se encuentra implementada en el archivo \texttt{parserLTL.py} del
 módulo \textit{LTL}.


\subsection{CTL}\label{cap:imp_ctl}

En la siguiente tabla se muestran los operadores de CTL con su correspondiente cadena de caracteres.

\begin{table}
\begin{center}
   \begin{tabular}{ | c | c | }
     \hline
     \textbf{Operador} & \textbf{Cadena de caracteres} \\ \hline
     $\lnot\bot$ & \texttt{TRUE} \\
     $\lnot$ & \texttt{-} \\
     $\land$ & \texttt{/$\backslash$} \\
     $\lor$ & \texttt{$\backslash$/} \\
     $\rightarrow$ & \texttt{-$>$} \\
     $\exists\bigcirc$ & \texttt{EO} \\
     $\forall\bigcirc$ & \texttt{AO} \\
     $\exists\square$ & \texttt{E[]} \\
     $\forall\square$ & \texttt{A[]} \\
     $\exists\lozenge$ & \texttt{E$<>$} \\
     $\forall\lozenge$ & \texttt{A$<>$} \\
     $\exists\cup$ & \texttt{EU} \\
     $\forall\cup$ & \texttt{AU} \\
     \hline
   \end{tabular}
\end{center}
\caption[Representación de operadores CTL]{Cadena de caracteres correspondiente a cada operador CTL.}
\end{table}

Al igual que en LTL se utilizó la siguiente expresión regular para las proposiciones
 atómicas.
\[ [a-z_][a-z0-9_]* \]

La gramática definida para CTL es la siguiente.

\begin{center}
   \begin{tabular}{ | r l | }
     \hline
     formula ::= & formula $\exists\cup$ formula \\
     & $|$ formula $\forall\cup$ formula \\
	 & $|$ formula $\land$ formula \\
	 & $|$ formula $\lor$ formula \\
	 & $|$ formula $\rightarrow$ formula \\
	 & $|$ $\exists\bigcirc$ formula \\
	 & $|$ $\forall\bigcirc$ formula \\
	 & $|$ $\exists\square$ formula \\
	 & $|$ $\forall\square$ formula \\
	 & $|$ $\exists\lozenge$ formula \\
	 & $|$ $\forall\lozenge$ formula \\
	 & $|$ $\lnot$ formula \\
	 & $|$ proposicion \\
	 & $|$ TRUE \\
     \hline
   \end{tabular}
\end{center}

Esta información se encuentra implementada en el archivo \texttt{parserCTL.py} del
 módulo \textit{CTL}.

