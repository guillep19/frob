Como se mencionó anteriormente se utilizó el formato de archivo \textit{GraphML} para
 representar los sistemas de transiciones.
Los estados son representados por el elemento \texttt{node} mientras que las transiciones
 son representadas por el elemento \texttt{edge}.

El formato \textit{GraphML} no establece como representar las etiquetas tanto en los nodos
 como en las aristas.
Para esto existe el atributo \texttt{data}, que proporciona flexibilidad
 para agregar atributos no contemplados por el formato.
Esto tiene un desventaja, y es que los atributos no contemplados por el formato no se
 representan de forma estándar, y por lo tanto su representación depende del editor
 utilizado.
En este caso el editor utilizado es \textit{yEd}.

Para los estados se guarda la siguiente información:
\begin{itemize}
\item Identificador

Este valor se guarda en el atributo \texttt{id} de cada nodo.
Es el identificador del estado, por lo que debe ser único.

Cuando se genera un sistema de transiciones mediante el verificador este genera los identificadores
 de cada estado automáticamente.

\item Proposiciones

Representan el conjunto de las proposiciones válidas en cada estado.

Se guardan en el atributo \texttt{y:NodeLabel}.
Este atributo es se encuentra dentro del atributo \texttt{data}, ya que no se encuentra
 especificado en el formato.

En caso de haber varias proposiciones, estas deben estar separadas por comas.

\end{itemize}

Además de esta información se debe indicar cuales son los estados iniciales.

Para las transiciones se debe guardar la siguiente información:
\begin{itemize}
\item Origen

Representa el estado de origen de la transición. Se guarda en el atributo \texttt{source}.

\item Destino

Representa el estado de destino de la transición. Se guarda en el atributo \texttt{target}.

\item Acción

Representa la acción que corresponde al cambio de estado.

Se guardan en el atributo \texttt{y:EdgeLabel}.

Este atributo es se encuentra dentro del atributo \texttt{data}, ya que no se encuentra
 especificado en el formato.

\end{itemize}

El parser de \textit{GraphML} a sistemas de transiciones se encuentra implementado en
 el objeto \texttt{ParserGraphML} del paquete \textit{Sistemas de transiciones}.

Como se mencionó anteriormente hay atributos que no están especificados en el formato y
 por lo tanto su interpretación depende del editor utilizado. Estos atributos son especificados
 en el parser mediante sub clases.
En este caso se utiliza el \textit{yEd Graph Editor}, para el cual fue implementado el objeto
 \texttt{ParserGraphML\_yEd}.
Este objeto es un parser de \textit{GraphML} que además interpreta la información dentro del
 atributo \texttt{data} como las etiquetas de los estados y transiciones.

A continuación se muestra un ejemplo de un sistema de transiciones y su correspondiente
 representación en \textit{GraphML}.

\paragraph{Ejemplo de sistema de transiciones en \textit{GraphML}} 


