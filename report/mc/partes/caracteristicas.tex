\section{Lenguaje de programación}
El lenguaje utilizado para desarrollar el verificador fue \textit{Python 2.7}. Esto se debe a que sus
 atributos se adaptaban a las necesidades del verificador.

\textit{Python} es un lenguaje de alto nivel con características de lenguaje funcional,
 y esto ayuda a que la implementación de
 algoritmos requiera menos instrucciones y que a su vez estas instrucciones sean de alto
 nivel, generando un código simple y fácil de entender por lo que es perfectamente
 mantenible por otras personas.
 
Además es un lenguaje extremadamente flexible, lo que permite reutilizar un algoritmo
 implementado permitiendo aplicarlo sobre distintas estructuras sin necesidad de adaptarlo o
 reescribirlo. Esto es de mucha utilidad ya que se manejan estructuras similares como ser
 fórmulas de distintas lógicas y sus estructuras quedan encapsuladas dentro de sus respectivos
 módulos, pudiendo aplicar el resto de los algoritmos sin necesidad de modificaciones.

Otro de los atributos importantes de este lenguaje es que soporta la programación orientada
 a objetos. Esto reafirma la idea de flexibilidad, modularidad y reutilización de código,
 por lo que fue otro de los atributos que se tuvieron en cuenta al momento de seleccionar
 el lenguaje de programación.



\section{Librerías auxiliares}
En la implementación del verificador se utilizaron dos librerías auxiliares.
La primera de ellas es \textit{Expat} que provee herramientas para
 parsear xml en objetos. La otra librería es \textit{Python Lex-Yacc}, una implementación
 de \textit{Yacc} para Python es utilizada para generar el parser correspondiente a cada lógica.

En esta sección se detallan las librerías mencionadas.

\subsection{Expat}
La librería \texttt{xml.parsers.expat} es una interfaz para el parser de XML \textit{Expat}.
Esta librería provee la clase \texttt{xmlparser}, el cual contiene entre otros los siguientes
 métodos

\begin{itemize}
\item \texttt{ParserCreate()}

Retorna una nueva instancia de la clase \texttt{xmlparser}.

\item \texttt{ParseFile(file)}

Parsea el contenido XML desde el archivo \textit{file}.

\item \texttt{StartElementHandler(name, attributes)}

Define la rutina que se invoca cada vez que comienza un nuevo elemento XML.

\textit{name} indica el nombre del elemento y \textit{attributes} es un hash
 conteniendo los atributos del elemento con sus respectivos valores.

\item \texttt{EndElementHandler(name)}

Define la rutina a invocar cuando se cierra un elemento XML.

\textit{name} indica el nombre del elemento a cerrar.

\end{itemize}

Esta librería fue utilizada para parsear los sistemas de transiciones desde \textit{GraphML}.
Este lenguaje se describe más adelante.

\subsection{Python Lex-Yacc}
\textit{Python Lex-Yacc} (PLY) es una implementación para \textit{Python} de las herramientas
 \textit{Lex} y \textit{Yacc}.

Es un paquete implementado puramente en \textit{Python} que permite generar parsers facilmente.
Este paquete está compuesto por los módulos \texttt{lex.py} y \texttt{yacc.py}.
El primero de ellos, \texttt{lex.py}, es utilizado para separar el texto a parsear en
 un conjunto de marcas llamados \textit{tokens} a partir de expresiones regulares definidas
 para cada uno de estos \textit{tokens}.

Una vez obtenido este conjunto de \textit{tokens} el módulo \texttt{yacc.py} reconoce los
 elementos del lenguaje previamente definidos mediante una gramática libre de contexto y
 construye el árbol sintáctico correspondiente.

Las expresiones regulares así como las gramáticas libres de contexto utilizadas para los
 lenguajes de LTL y CTL se encuentran definidas en la sección \ref{sec:ply}.

\section{GraphML}
\textit{GraphML} es un método para describir estructuras de grafos basado en el lenguaje
 de marcas XML.
No es un lenguaje en sí mismo, sino que define elementos y atributos en XML que permiten
 representar varios tipos de estructuras de grafos, como ser grafos dirigidos, no dirigidos,
 hipergrafos entre otros.

Los elementos básicos en \textit{GraphML} son
\begin{itemize}
\item \texttt{graph}

Este elemento representa el grafo en sí, conteniendo atributos como \texttt{edgedefault}
 que indica si sus aristas son dirigidas o no.

\item \texttt{node}

Contiene la información de cada nodo como su identificador.

\item \texttt{edge}

Representa cada arista, indicando su origen y destino en los atributos \texttt{source} y
 \texttt{target} respectivamente.

\end{itemize}

Al estar definido sobre un lenguaje tan simple y utilizado como XML se convierte en un método
 fácilmente parseable.

Además este método es altamente flexible para agregar nuevos elementos y atributos según
 la necesidad de cada aplicación específicamente.

En este proyecto \textit{GraphML} es utilizado para representar los sistemas de transiciones.