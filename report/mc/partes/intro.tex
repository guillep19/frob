En este capítulo se detalla el desarrollo del verificador.
Esto es, sus principales características como ser el el lenguaje de programación
 elegido, el lenguaje de marcas utilizado para representar grafos o librerías
 externas necesarias para el verificador.

También se ve la modularización del sistema implementado, que corresponde a los
 temas vistos en los capítulos anteriores.
Se implementaron módulos para la manipulación de sistemas de transiciones y de autómatas de Büchi,
 con operaciones sobre los mismos y las funcionalidades necesarias para el verificador.
Además se implementaron módulos para cada lógica (en este caso LTL y CTL) para poder interpretar
 cada una de ellas. En estos módulos se define su sintáxis, semántica y su algoritmo de verificación,
 de forma que para agregar otra lógica al verificador es necesario crear únicamente su módulo
 correspondiente.
