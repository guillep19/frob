En los capítulos anteriores se analizó la verificación de un sistema mediante propiedades expresadas
 en LTL.
LTL es una lógica lineal, esto se debe a que la noción de tiempo se basa en caminos.
Las distintas ejecuciones se tratan como secuencias de estados, en donde cada estado
 tiene un único sucesor posible.

En este capítulo se trata la lógica arborecente CTL, introducida en \cite{clarke}
 por Clarke y Emerson.
A diferencia de LTL, en CTL el conjunto de ejecuciones se representa
 con un único árbol en donde los nodos representan los estados y las aristas representan
 las distintas decisiones posibles en cada estado.

Debido a esto, LTL y CTL son lógicas incompatibles entre si.
Esto significa que existen fórmulas en LTL para las que no existe una fórmula equivalente
 en CTL y viceversa.
 
En este capítulo se introducirán los conceptos básicos de CTL, su sintaxis, semántica, y
 propiedades básicas que serán de utilidad en los capítulos posteriores.
Además se mostrará su incompatibilidad con LTL.

Finalmente se construirá un algoritmo que permite verificar si el modelo del sistema en cuestión
 satisface la propiedad planteada en términos de CTL mostrando ejemplos y contraejemplos según
 corresponda.