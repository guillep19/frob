Se implementó un verificador de modelos que utiliza herramientas y lenguajes conocidos
 para modelar sistemas reactivos y expresar propiedades sobre estos sistemas.
Estas propiedades son verificadas mediante distintos algoritmos dependiendo del lenguaje
 escogido.
Cada uno de estos lenguajes es implementado en un módulo separado en donde se encuentra
 representada tanto su sintaxis como su algoritmo de verificación, lo que permite agregar
 nuevos lenguajes sin afectar el resto de los módulos.

Además el algoritmo de verificación de los lenguajes utilizados se basan en dos enfoques
 distintos de la verificación de modelos.

Por otro lado esta implementación se basa en el paradigma de orientación a objetos.
Esto es un punto a favor ya que facilita la comprensión, la mantenibilidad
 y la posterior extensión del verificador.

Además se ejemplificó la utilización del verificador con ejemplos de sistemas y propiedades
 sobre los mismos.

Si bien existen herramientas similares como pueden ser \textit{Spin} o \textit{UPPAL}, estas
 no tienen las mismas características que la herramienta implementada en este proyecto.
Las herramientas existentes están enfocadas en el rendimiento y este aspecto es difícil
 de combinar con otros aspectos como la modularización y flexibilidad del código.
Además por la misma razón, para garantizar un buen rendimiento en la verificación de propiedades
 tienen ciertas limitaciones como pueden ser la profundidad de lás fórmulas que expresan las
 propiedades a verificar.