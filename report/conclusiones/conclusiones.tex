
% TODO(Marcos): La primera conclusion es que se disenaron los dos lenguajes.

  Se realizó una implementación modelo de cada parte del diseño
necesaria para poder escribir programas en alto nivel, usando
el paradigma de programación funcional reactiva, y se logró
seguir todos los pasos necesarios para ejecutar dichos programas
en plataformas con bajas capacidades de cómputo.

  A diferencia de otros trabajos, en éste proyecto los robots
resultantes son autónomos y no requieren de contacto con el
exterior ni directivas para realizar sus tareas.

  Es posible escribir programas cortos, en alto nivel, que permitan
pensar en los problemas y sus soluciones, sin necesidad de detenerse
a ver cada parte a bajo nivel.

  Se logró recortar el alcance de cada parte del proyecto para
poder ser implementado por una sola persona.
  Se diseñaron los lenguajes \frob{} y \alf{} que podrán ser
retomados en futuros trabajos para ser mejorados y extender
su funcionalidad, eliminando las restricciones impuestas
por el alcance actual.

  La elección del lenguaje \textit{Haskell} para implementar el
compilador fue correcta, ya que fácilmente se pueden realizar
modificaciones, algo que durante el proyecto agregó mucho valor.

  Se implementó la máquina virtual para la plataforma \textit{MBED}
y el tamaño de la misma, así como la capacidad de procesamiento no
fue un problema, al utilizar \texttt{C} como lenguaje y controlar
el uso de la memoria y las estructuras usadas.
  Lamentablemente dado el alcance, no se pudo hacer pruebas
con plataformas aún mas pequeñas como \texttt{Arduino} y \texttt{PIC}.

  Se construyó un robot físico, que permitió completar el flujo
de trabajo y ver como resultado final, la ejecución del caso de
estudio, utilizando todas las herramientas desarrolladas.

  A pesar de las limitaciones impuestas por el alcance, sería
posible luego de algunas mejoras utilizar el lenguaje para enseñar
conceptos de robótica.

\section{Trabajo futuro}

  Como principal trabajo futuro, se notó la necesidad de contar con
un simulador, para eliminar la necesidad de construir un robot físico
para resolver un problema. Permitiría también crear diversos escenarios
y tener una mayor diversidad de ambientes de prueba, para luego
crear el robot físico deseado e interactuar con el mismo.

  También sería muy útil contar con una funcionalidad de depuración, la cual
mostrara dependiendo del tiempo los valores de cada entrada, cada señal y cada
salida.

  Una posible opción de implementar depuración es comunicar
mediante el puerto serial el valor de cada señal al cambiar,
y mostrarlo en una interfaz web como la que provee
RXMarbles (ver \cite{rxmarbles}). 

  El lenguaje \textit{Elm} provee de una herramienta que permite ``viajar en el 
tiempo'', no solo permite ver los valores de las señales sino también
modificarlos y seguir la ejecución de un programa.
  En nuestro caso no sería posible modificar lo que el robot físico
realiza, pero si sería útil ver en la línea de tiempo que valores
tomaron sus señales. (ver \cite{elmdebug}) Además, si se contara
con un simulador sería posible.

%  También luego del desarrollo de todos los componentes de la solución,
%se concluyó que sería útil que el compilador pudiera generar código
%intermedio utilizando LLVM \ref{llvm}.
%  Muchos lenguajes en la industria lo toman como estándar y producen
%código LLVM desde sus compiladores, sería interesante ver si es posible
%utilizarlo. La ventaja de usar LLVM es que luego se pueden aprovechar las
%optimizaciones 

% TODO(Marcos): Ni en las conclusiones ni en el trabajo futuro se habla del
% sistemas de tipos de Willie.
% Solucion: Agregarlo a trabajo futuro y decir que esta fuera de alcance...

