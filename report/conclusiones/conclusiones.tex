

  Conclusiones..........................

\section{Trabajo futuro}

  Sería muy útil contar con una funcionalidad de depuración, la cuál
mostrara dependiendo del tiempo los valores de cada fuente de eventos.

  También durante el desarrollo, se notó la necesidad de un simulador,
que permita simular un robot, para no tener que construir un robot
físico para resolver el problema.

Una opción es comunicar mediante el puerto serial el valor de cada
señal al cambiar, y mostrarlo en una interfaz web como la que provee
RXMarbles (ver \cite{rxmarbles}). 

El lenguaje Elm provee de una herramienta que permite viajar en el 
tiempo, modificar y mostrar la ejecución de un programa, en nuestro
caso no sería posible modificar lo que el robot físico realiza, pero
si sería útil ver en la línea de tiempo que valores tomaron sus
señales. (ver \cite{elmdebug})

  También luego del desarrollo de todos los componentes de la solución,
se concluyó que la implementación de un lenguaje de bajo nivel específico
es compleja y no puede ser reutilizada.
  Sería interesante intentar utilizar LLVM \ref{llvm} como lenguaje de
bajo nivel, o lenguaje intermedio ya que muchos lenguajes en la 
industria lo toman como estándar y producen código LLVM desde sus
compiladores.
