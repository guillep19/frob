  La máquina, deberá ejecutar el código de bajo nivel en una plataforma
  objetivo.

  Existen dos limitaciones importantes a tener en cuenta, la primera es que
el espacio de memoria varía en diferentes plataformas, por lo que se desea
sea posible compilar la máquina aún con un espacio muy reducido.

  La segunda es que las plataformas varían en capacidades
de \textit{Entrada/Salida}, es importante que quien compila la máquina y
arma un entorno tenga conocimiento de cómo disponer las mismas y qué
limitaciones existen, por ejemplo: Cantidad de pines digitales o analógicos.

  La implementación modelo, se hizo utilizando
  la plataforma \textit{MBED LPC1768},
se puede encontrar documentación de la misma en \cite{mbed-LPC1768} 
y en \cite{mbed}.

  El lenguaje de programación elegido para el desarrollo de la máquina virtual
es \textit{C++} ya que es posible compilarlo para casi cualquier plataforma
objetivo.
  Además \textit{C++} permite acceder a muy bajo nivel, y manipular a
nivel de \emph{bytes} las estructuras.\\

  \textit{MBED} es una plataforma pensada para colaborar mediante
un entorno de desarrollo web, y compilador online, ese esquema de 
trabajo no es el más práctico para desarrollar la máquina virtual, por
lo que se descargaron de la página de mbed \cite{mbeddev}, las herramientas
de desarrollo para compilar offline.

  La máquina tendrá dos partes principales, una que interpreta el código e
implementa el despachador que actualiza las señales.
  Esta parte es común y puede ser portada a diferentes plataformas sin
necesidad de modificarla.

  Las instrucciones en memoria tendrán un ancho de palabra de 16 bit.
  La primera palabra contiene en los 8 bits más representativos, el código
  de la operación (\texttt{opcode}).
  Los 8 bits menos representativos, contienen un argumento inmediato
  opcional.
  Luego según el \texttt{opcode}, algunas instrucciones pueden tener
  argumentos adicionales, en las siguientes palabras de 16 bit.

  \begin{center}
      \begin{tabular}{|c|c|}
      \hline
      codigo de 8 bit & \cellcolor{gray!25}inmediato de 8 bit \\
      \hline
      \multicolumn{2}{|c|}{argumento $1$ opcional de 16 bit} \\
      \hline
      \multicolumn{2}{|c|}{...} \\
      \hline
      \multicolumn{2}{|c|}{argumento $n$ opcional de 16 bit} \\
      \hline
      \end{tabular}
  \end{center}

  La máquina mantiene un puntero a la siguiente instrucción a ejecutar llamado
  \texttt{ip} \footnote{IP: del inglés, Instruction Pointer significa puntero a instrucción}.
  Cuando \texttt{ip} no es nulo, la máquina ejecuta todas las instrucciones hasta
  que el mismo se haga nulo.
  El pseudocódigo de la máquina es:

\begin{Verbatim}
  1 - Crear grafo de señales vacío, inicializar pila.
  2 - Apuntar ip al inicio del código.
  3 - Ejecutar código hasta que ip se haga nulo.
  4 - Para siempre:
    4.1 - Leer entradas
    4.2 - Actualizar señales conectadas a las mismas, marcar como listas.
    4.2 - Mientras hay señales listas para procesar:
      4.2.1 - Cargar valores en stack.
      4.2.2 - Apuntar ip a inicio de la función asociada.
      4.2.3 - Ejecutar hasta que ip se haga nulo.
      4.2.4 - Actualiza señales conectadas a la misma.
    4.3 - Escribir salidas.
\end{Verbatim}

  Al principio se crea un grafo de señales vacío, y se reserva espacio para la pila, todo en memoria estática, tanto
  para los nodos del grafo como para la pila.
  En el punto 2, se interpretan las instrucciones del programa, hasta llegar a la instrucción \texttt{halt}. Las 
  instrucciones al inicio del código se corresponden con el bloque \texttt{do} del programa, por lo tanto
  al ejecutarlo se obtiene el grafo de las señales completo.

  Con el grafo armado, luego se obtienen los valores de las entradas
necesarias, y para cada señal conectada se calcula su valor.

  Luego, en el punto 4.2 el grafo de señales se recorre en orden
topológico, actualizando el valor de cada señal.

  Es sencillo notar, que la actualización de las señales listas para
procesar es un punto que con pocas modificaciones, puede ser realizado
en paralelo en un entorno multiprogramado.

  Las señales que estén conectadas a una salida, se usan para actualizar el valor de las mismas.

  Para cada instrucción hay una función definida que interpreta, el despachador
  tomará una a una las instrucciones e invocará la función que la maneja de acuerdo
  a que operación es.
  Las funciones son de tipo \texttt{void} y realizan cambios sobre el estado
  de la máquina. La referencia a las mismas es guardada en un vector \texttt{functions}.
  La posición de cada instrucción en el vector coincide con el código de operación.

\begin{Verbatim}
  void (*functions[])() = {
      f_halt,
      f_call, f_ret, f_load_param,
      f_lift, f_lift2, f_folds,
      f_read, f_write,
      f_jump, f_jump_false,
      f_cmp_eq, f_cmp_neq, f_cmp_gt, f_cmp_lt,
      f_add, f_sub, f_div,
      f_mul, f_op_and, f_op_or,
      f_op_not,
      f_push, f_pop, f_dup,
      f_store, f_load
  };
\end{Verbatim}

  Por ejemplo, al despachar el operador \texttt{push}, la siguiente palabra contiene
  el valor a colocar en el stack. El código en lenguaje \texttt{C} que maneja la instrucción es:

\begin{Verbatim}
  void f_push() {
      *++sp = *ip++;
  }
\end{Verbatim}

  Se creó un archivo \texttt{Makefile} para construir una imagen
  binaria de la máquina virtual.
  Ésto genera un archivo \emph{mbed\_alfvm.bin}.
  Para cargar la máquina en la placa \textit{MBED}, alcanza con conectarla
a un puerto USB y pegar el archivo en la carpeta /media/MBED.
