  La máquina, deberá ejecutar el código de bajo nivel en una plataforma
  objetivo.

  Existen dos limitaciones importantes a tener en cuenta, la primera es que
el espacio de memoria varía en diferentes plataformas, por lo que se desea
sea posible compilar la máquina aún con un espacio muy reducido.

  La segunda es que las plataformas varían en capacidades
de \textit{Entrada/Salida}, es importante que quien compila la máquina y
arma un entorno tenga conocimiento de cómo disponer las mismas y qué
limitaciones existen, por ejemplo: Cantidad de pines digitales o analógicos.

  La implementación modelo, se hizo utilizando
  la plataforma \textit{MBED LPC1768},
se puede encontrar documentación de la misma en \cite{mbed-LPC1768} 
y en \cite{mbed}.

  El lenguaje de programación elegido para el desarrollo de la máquina virtual
es \textit{C++} ya que es posible compilarlo para casi cualquier plataforma
objetivo.
  Además \textit{C++} permite acceder a muy bajo nivel, y manipular a
nivel de \emph{bytes} las estructuras.\\

  \textit{MBED} es una plataforma pensada para colaborar mediante
un entorno de desarrollo web, y compilador online, ese esquema de 
trabajo no es el más práctico para desarrollar la máquina virtual, por
lo que se descargaron de la página de mbed \cite{mbeddev}, las herramientas
de desarrollo para compilar offline.

TODO: Faltan muchos detalles.
TODO: Esto iría en el manual de usuario:
  El código de la máquina virtual está en el directorio
  /src/alfvm, para compilarlo se ejecuta:

\begin{verbatim}
  > cd src/alfvm
  > make
\end{verbatim}

  Ésto genera un archivo \emph{mbed\_alfvm.bin}.
  Para cargar la máquina en la placa \textit{MBED}, alcanza con conectarla
a un puerto USB y pegar el archivo en la carpeta /media/MBED.


 
  TODO: Colocar donde vaya:

  Las instrucciones van a tener la siguiente representación en memoria:

  \begin{center}
      \begin{tabular}{|c|c|}
      \hline
      codigo de 8 bit & \cellcolor{gray!25}inmediato de 8 bit \\
      \hline
      \multicolumn{2}{|c|}{argumento $1$ opcional de 16 bit} \\
      \hline
      \multicolumn{2}{|c|}{...} \\
      \hline
      \multicolumn{2}{|c|}{argumento $n$ opcional de 16 bit} \\
      \hline
      \end{tabular}
  \end{center}


