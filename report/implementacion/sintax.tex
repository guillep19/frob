  La segunda fase del compilador, recibe la lista de lexemas (\texttt{[Token]}) y
reconoce el lenguaje, generando un árbol de
sintaxis abstracta (\emph{AST}\footnote{Del inglés Abstract Syntax Tree}).

  Para reconocer la gramática se implementó un parser recursivo descendente.
  Utilizando la herramienta \textit{UU.Parser} \cite{uuparser}, se definió un tipo de datos
  \texttt{TokenParser a} que representa un parser que recibe una secuencia de lexemas de tipo \texttt{Token}
  y retorna un \emph{AST} de tipo \texttt{a}.

  \begin{Verbatim}
  type TokenParser a = Parser Token a
  \end{Verbatim}

  \textit{UU.Parser} define un conjunto de combinadores de parsers y utilizándolos se construyen parsers
  complejos a partir de parsers simples.

  Para representar el \emph{AST} se utiliza una gramática de atributos.
  Una gramática de atributos es como una gramática libre de contexto, pero agrega semántica a la misma.
  Para el análisis sintáctico, la semántica no es utilizada, pero será usada en la próxima etapa.

  El sistema de gramáticas de atributos
  \textit{UUAG}\cite{uuag} fue usado para la implementación.


  Se define un tipo de datos \texttt{Root} que representa la raíz del árbol.

  \begin{Verbatim}
  data Root
    | Root
      decls :: Decls
      dodecls :: Dodecls
  \end{Verbatim}

  El mismo tiene un único constructor \texttt{Root\_Root} que recibe un árbol de tipo
  \texttt{Decls} que representa las declaraciones, y un árbol de tipo \texttt{Dodecls} que
  representa el bloque \texttt{do}.

  Para crear el \emph{AST} usando \textit{UU.Parser} se define el parser \texttt{pRoot}:

  \begin{Verbatim}
  pRoot :: TokenParser Root
  pRoot
    = (\x y -> Root_Root x y) <$> pDecls <*> pDodecls
  \end{Verbatim}

  El cuál asume definido un parser de declaraciones \texttt{pDecls} y un parser
  del bloque \texttt{do} (\texttt{pDodecls}).
  
  \begin{Verbatim}
  pDecls :: TokenParser Decls

  pDodecls :: TokenParser Dodecls
  \end{Verbatim}

  El combinador ``$\langle * \rangle$'' \cite{uuparsing:piriapolis} se utiliza para
  combinar dos parser y resolver producciones de largo 2 en una gramática,
  en este caso reconocer primero la lista de declaraciones de funciones, y
  luego el bloque \texttt{do}.
  El tipo del combinador es:

  \begin{center}
    $(\langle*\rangle) :: \texttt{Parser}\ s\ a \rightarrow \texttt{Parser}\ s\ b \rightarrow \texttt{Parser}\ s\ (a, b)$
  \end{center}

  El combinador ``$\langle \$ \rangle$'' se utiliza para aplicar una
función al resultado de un parser, en éste caso la función es aplicar
el constructor \texttt{Root\_Root}.

  Para construir el parser completo, se va refinando sucesivamente
en parsers mas específicos, hasta construir completamente el \emph{AST}.
  En el apéndice \ref{appendix:parser} se encuentra código del parser
implementado.

