
En este capítulo se detalla la implementación de el compilador
y la máquina virtual diseñadas para utilizar el lenguaje
\frob en la plataforma elegida.
También se explica cuál sería el mecanismo para portar la
implementación a otra plataforma.

%TODO: Completar la intro luego de tener los caps.

\section{Compilador}

  El lenguaje utilizado para desarrollar el compilador fue \textit{Haskell}.
Las razones que llevaron a su elección son la portabilidad y la expresividad
del mismo.
Por un lado, el compilador es portable, ya que se puede compilar y ejecutar
en diversos sistemas operativos utilizando el compilador \textit{ghc}.\\

  Por otro lado, la arquitectura del compilador es de tubos y filtros, algo
  que es natural expresar en un lenguaje funcional.
  La gramática se corresponde directamente con un tipo de datos así
  como el código generado.\\

  Es usual realizar tareas de compilación en \textit{Haskell}, y existen
herramientas estándar para cada etapa.
  Para el análisis léxico, se utiliza \textit{Alex} \cite{alex} y
  para parsear y generar
  la gramática se utiliza \textit{Happy}. \cite{happy} \\

  La primera fase del compilador utiliza las dos herramientas mencionadas,
recibe el programa en lenguaje \frob (.willie) y genera un árbol de
sintaxis abstracta (\emph{AST}).\\

  La segunda etapa, utiliza gramáticas de atributos (\emph{Attribute Grammars})
definiendo atributos en el \emph{AST}.
  Uno de dichos atributos será el código en bajo nivel, que será la salida
de ésta etapa.
  Ésta salida se escribe en un archivo (.alf) terminando el proceso
  de compilación.

\section{Máquina virtual}

  La máquina, deberá ejecutar el código de bajo nivel en una plataforma
  objetivo.
  El lenguaje de programación elegido para el desarrollo de la máquina virtual
es \textit{C++} ya que es posible compilarlo para casi cualquier plataforma
objetivo.
  Además \textit{C++} permite acceder a muy bajo nivel, y manipular a
  nivel de \emph{bytes} las estructuras.\\

  Existen dos limitaciones importantes a tener en cuenta, la primera es que
el espacio de memoria varía en diferentes plataformas, por lo que se desea
sea posible compilar la máquina aún con un espacio muy reducido.
  La segunda es que las plataformas varían en capacidades
de \textit{Entrada/Salida}, es importante que quien compila la máquina y
arma un entorno tenga conocimiento de cómo disponer las mismas y qué
limitaciones existen, por ejemplo: Cantidad de pines digitales o analógicos.

  La implementación modelo, se hizo utilizando
  la plataforma \textit{MBED LPC1768},
se puede encontrar documentación de la misma en \cite{mbed-LPC1768} 
y en \cite{mbed}.

