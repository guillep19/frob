
En este capítulo se detalla la implementación del compilador
y la máquina virtual diseñadas para utilizar el lenguaje
\frob{} en la plataforma elegida.
También se explica cuál sería el mecanismo para portar la
implementación a otra plataforma.

%TODO: Completar la intro luego de tener los caps.

\section{Compilador}

  El compilador será el encargado de leer el programa \frob{} y traducirlo
a \alf{}.

  El lenguaje utilizado para desarrollar el compilador fue \textit{Haskell}.
  Las razones que llevaron a su elección son la portabilidad y la
expresividad del mismo.
  El compilador \compilador{} es portable, ya que se puede compilar y ejecutar
en diversos sistemas operativos utilizando el compilador \textit{ghc}.

  Es usual realizar tareas de compilación en \textit{Haskell} por lo que existen
herramientas estándar para cada etapa.

  El compilador constará de una secuencia de etapas: Análisis Léxico,
  Análisis Sintáctico, Análisis Semántico y Generación de Código.

  En la Figura \ref{fig:compiler} se puede ver la estructura más detallada
del compilador.

\begin{figure}[h!]
\begin{center}
\caption{Diagrama del compilador}
\includegraphics[width=0.9\textwidth]{graphs/compiler.png}
\label{fig:compiler}
\end{center}
\end{figure}


\subsubsection{Análisis Léxico}
  La primera etapa se llama análisis léxico, en esta se lee el código
  fuente en lenguaje \frob{} (.willie) y lo transforma en una
  lista de lexemas.

  Un lexema puede ser una palabra reservada (ej: \texttt{do}),
  un valor (ej: $19$), un identificador (eg: \texttt{distance}) o
  un símbolo reservado (eg: \texttt{+}).

  Para representar los lexemas, se utiliza la
  herramienta \textit{UU.Scanner}
  \cite{uuparser} que estandariza los mismos en el tipo de
  datos \texttt{Token}.

  Usando \textit{Alex}\cite{alex} se procesa el código fuente,
  se reconocen los lexemas y se retorna una lista de
  tipo \texttt{[Token]}.

  La etapa se puede resumir en la implementación de la
  función \texttt{tokenize}.

\begin{Verbatim}
  tokenize :: String -> String -> [Token]
\end{Verbatim}



\subsubsection{Análisis Sintáctico}
  La segunda fase del compilador, recibe la lista de lexemas (\texttt{[Token]}) y
reconoce el lenguaje, generando un árbol de
sintaxis abstracta (\emph{AST}\footnote{Del inglés Abstract Syntax Tree}).

  Para reconocer la gramática se implementó un parser recursivo descendente.
  Utilizando la herramienta \textit{UU.Parser} \cite{uuparser}, se definió un tipo de datos
  \texttt{TokenParser a} que representa un parser que recibe una secuencia de lexemas de tipo \texttt{Token}
  y retorna un \emph{AST} de tipo \texttt{a}.

  \begin{Verbatim}
  type TokenParser a = Parser Token a
  \end{Verbatim}

  \textit{UU.Parser} define un conjunto de combinadores de parsers y utilizándolos se construyen parsers
  complejos a partir de parsers simples.

  Para representar el \emph{AST} se utiliza una gramática de atributos.
  Una gramática de atributos es como una gramática libre de contexto, pero agrega semántica a la misma.
  Para el análisis sintáctico, la semántica no es utilizada, pero será usada en la próxima etapa.

  El sistema de gramáticas de atributos
  \textit{UUAG}\cite{uuag} fue usado para la implementación.

  Se define un tipo de datos \texttt{Root} que representa la raíz del árbol.
  El mismo tiene un único constructor \texttt{Root\_Root} que recibe un árbol de tipo
  \texttt{Decls} que representa las declaraciones, y un árbol de tipo \texttt{Dodecls} que
  representa el bloque \texttt{do}.

  Para crear el \emph{AST} usando \textit{UU.Parser} se define el parser \texttt{pRoot}:

  \begin{Verbatim}
  pRoot :: TokenParser Root
  pRoot
    = (\x y -> Root_Root x y) <$> pDecls <*> pDodecls
  \end{Verbatim}

  El cuál asume definido un parser de declaraciones \texttt{pDecls} y un parser
  del bloque \texttt{do} (\texttt{pDodecls}).

  \begin{Verbatim}
  pDecls :: TokenParser Decls

  pDodecls :: TokenParser Dodecls
  \end{Verbatim}

  Se va refinando sucesivamente en parsers mas específicos, hasta construir completamente el \emph{AST}.




\subsubsection{Análisis Semántico}
  Para la última etapa se utiliza la gramática de atributos para definir
semántica sobre el \emph{AST}.

  Las gramáticas de atributos (\emph{Attribute Grammars}) simplifican
la tarea de escribir catamorfismos.
Un catamorfismo es una función análoga a la función de alto orden
\texttt{foldr} pero aplicada sobre cualquier tipo de datos.

  De ésta manera se pueden definir atributos en el \emph{AST}, sintéticos
o heredados.

  Uno de dichos atributos será el código en bajo nivel, que será la salida
de esta etapa.
  Esta salida se escribe en un archivo (.alf) terminando el proceso
de compilación.




\section{Máquina virtual}

  La máquina que interpreta el lenguaje \alf{} es una
\textit{máquina de stack}.\footnote{Stack machine en inglés}.

  En una máquina de stack las instrucciones están en notación
postfija.\footnote{RPN (\textit{Reverse polish notation}) del inglés}
  Para evaluar expresiones se colocan sus argumentos en una pila, y luego
se ejecuta la operación asociada.
  
  Por ejemplo la expresión ``$5 + 19 * 8$'' en RPN se
escribe ``$5\ 19\ 8 * +$''.
  
  A modo de ejemplo en \alf{} se representa con las
  siguientes 5 instrucciones.

  \begin{Verbatim}[frame=single]
  push 5
  push 19
  push 8
  mul
  add
  \end{Verbatim}

  El conjunto \textit{Inputs} representa las entradas de la máquina.
  Dadas $m$ entradas fijas, cada una se identifica con un entero único
  entre $1$ y $m = | \textit{Inputs} |$.

  Cada $\texttt{I}_i, i \in (1 \dotsb m)$ se corresponderá con un sensor
  definido en el robot.

\begin{definicion}
  \textit{Entradas de la máquina}\\
  \begin{center}
    $\textit{Inputs} \equiv \{\texttt{I}_1 \dotsb \texttt{I}_m\}$.
  \end{center}
\end{definicion}
  
  Graficamente las representaré con la notación:

  \begin{center}
\begin{tikzpicture}
\selectlanguage{english}
  %%\draw[step=1cm,gray,very thin,xshift=0cm,yshift=0cm] (0,0) grid (12,4);
  \begin{scope}[xshift=0cm,yshift=0cm,very thick,
    node distance=2cm,on grid,>=stealth',
    block/.style={rectangle,draw,fill=cyan!20},
    comp/.style={circle,draw,fill=orange!40},
    stack/.style={rectangle split,rectangle split parts=#1,draw,anchor=center}]
   \node [block] (s2) [yshift=0.5cm,xshift=1cm] {$\texttt{I}_{i}$};
\end{scope} 

\selectlanguage{spanish}
\end{tikzpicture}

\end{center}


  También se cuenta con un conjunto \textit{Outputs} de salidas,
  identificadas de $1$ a $k = | \textit{Outputs} |$.
  
  Cada $\texttt{O}_i, i \in (1 \dotsb k)$ se corresponderá con un actuador
  del robot.

\begin{definicion}
  \textit{Salidas de la máquina}\\
  \begin{center}
    $\textit{Outputs} \equiv \{\texttt{O}_1 \dotsb \texttt{I}_k\}$.
  \end{center}
\end{definicion}
  
  Graficamente las representaré con la notación:

  \input{design/ll_diagram_output.tex}

  Las señales que se definan se denotarán $\texttt{S}_i$, siendo $i$ un
  índice único que las identifica. El conjunto de las señales se llama
  \textit{Signals}.

\begin{definicion}
  \textit{Señales}
  \begin{center}
    $\textit{Signals} \equiv \{\texttt{S}_1 \dotsb \texttt{S}_s\}$.
  \end{center}
\end{definicion}

  Graficamente las representaré con la notación:

  \begin{center}
\begin{tikzpicture}
\selectlanguage{english}
  %%\draw[step=1cm,gray,very thin,xshift=0cm,yshift=0cm] (0,0) grid (12,1);
  \begin{scope}[xshift=4cm,yshift=0cm,very thick,
    node distance=2cm,on grid,>=stealth',
    comp/.style={circle,draw,fill=orange!40}]
    \node [comp] (ca1) [xshift=2cm, yshift=0.5cm] {$\texttt{S}_{i}$};
   \end{scope} 
\selectlanguage{spanish}
\end{tikzpicture}
\end{center}


  La máquina tendrá una pila global, denotada \textit{Stack}. El mismo
  se representa con una secuencia de valores.

\begin{definicion}
  \textit{Pila global}
  \begin{center}
    $\textit{Stack} \equiv s_1, \dotsb, s_n$.
  \end{center}
\end{definicion}

  El \textit{Stack} lo representaré graficamente con la notación:

  \begin{center}
\begin{tikzpicture}
\selectlanguage{english}
  %%\draw[step=1cm,gray,very thin,xshift=0cm,yshift=0cm] (0,0) grid (12,3);
  \begin{scope}[xshift=4cm,yshift=0cm,very thick,
    node distance=2cm,on grid,>=stealth',
    block/.style={rectangle,draw,fill=cyan!20},
    comp/.style={circle,draw,fill=orange!40},
    stack/.style={rectangle split,rectangle split parts=#1,draw,anchor=center}]
  \node[stack=4,xshift=2cm,yshift=1.5cm,
        text width=1.1cm,align=center,text height=0.3cm] {
    \nodepart{one} $s_n$
    \nodepart{two} $s_{n-1}$
    \nodepart{three} $\dotsb$
    \nodepart{four} $s_o$
  };
 \draw [->,line width=1pt] (0.5,2.4) -- node[below]{\tiny{TOS}} (1.3, 2.4);
 \end{scope} 
  
\selectlanguage{spanish}
\end{tikzpicture}


\end{center}


  Donde \texttt{TOS}\footnote{Del inglés: Top
  of stack} indica el índice del tope del mismo.
  Se cumple que $\textit{Stack}_{TOS} = s_n$.



%% TODO: Esto es parte de la implementacion.
%%  A diferencia del compilador, es necesario implementar una máquina virtual
%%para cada arquitectura objetivo.\\
%%
%%  Por ejemplo, para ejecutar programas en un robot
%%  con un procesador \emph{arduino}, debe
%%  existir una implementación de la máquina para ese modelo
%%  de \emph{arduino}.
%%
%%  Al momento de implementar la máquina, se tomará en cuenta ésto para
%%  factorizar partes en común y sólo implementar por arquitectura, las
%%  partes que realmente sean diferentes como ser la comunicación con
%%  los periféricos de entrada/salida y las llamadas al sistema.

  El dispatcher, es quien implementa las acciones de la máquina.
  El mismo se encarga de recibir valores de los sensores y mapearlos
  a eventos en las entradas $\texttt{I}_i$.

  Éstos eventos, serán recibidos por los nodos $Nodes$.
  Cada $Node_i$ que espera por eventos entrará en estado activo cuando
todos los nodos por los que espera le envíen un evento.
  A su vez, el nodo en estado activo calcula un resultado y notifica a
todos sus nodos adyacentes.

  TODO: Forzar que Nodes sea un grafo acíclico en la descripción.

  El dispatcher, realizará implicitamente un orden topológico de los
nodos, como $Nodes$ es un grafo acíclico, éste proceso es posible y
termina, y cada salida cuenta con un valor, que será mapeado a los actuadores.


