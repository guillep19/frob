
En este capitulo se detalla la implementación de el compilador
y la máquina virtual diseñadas para utilizar el lenguaje
\emph{grob} en la plataforma elegida.
También se explica cuál sería el mecanismo para portar la
implementación a otra plataforma.


%TODO: Completar la intro luego de tener los caps.

\section{Compilador}

  El lenguaje utilizado para desarrollar el compilador fue \textit{Haskell}.
El compilador se divide en dos fases, la primera implementa un
parser que recibe como entrada el programa en lenguaje {FROB} y genera un árbol de sintaxis abstracta (\emph{AST}). En Haskell
se define el \emph{AST} como un tipo de datos.
  Luego se recorre el \emph{AST} y se genera como salida un archivo
con las instrucciones en bajo nivel.

\section{Máquina virtual}

  El lenguaje de programación para el desarrollo de la máquina virtual
es \textit{C++} ya que es posible compilarlo para casi cualquier plataforma
objetivo.
  Existen dos limitaciones importantes a tener en cuenta, la primera es que
el espacio de memoria varía en diferentes plataformas, por lo que se desea
sea posible compilar la máquina aún con un espacio muy reducido.
  La segunda es que las plataformas varían en capacidades
de \textit{Entrada/Salida}, es importante que quien compila la máquina y
arma un entorno tenga conocimiento de cómo disponer las mismas y qué
limitaciones existen, por ejemplo: Cantidad de pines digitales o analógicos.

  La implementación modelo, se hizo utilizando la plataforma \textit{MBED LPC1768},
se puede encontrar documentación de la misma en \cite{mbed-LPC1768} 
y en \cite{mbed}.

