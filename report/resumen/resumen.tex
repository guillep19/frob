
  El proyecto consiste en la creación de un lenguaje de programación
para robots, cuyas capacidades de cómputo son limitadas.
  Para esto se escogió el paradigma de Programación Funcional Reactiva
(\emph{FRP}) el cual permite expresar naturalmente reacciones a
valores que varían en función del tiempo.

  El objetivo es utilizarlo con fines educativos,
por lo tanto debe ser simple y fácil de usar por usuarios
inexpertos, no familiarizados con la electrónica ni la informática.

  Para resolver el problema, el mismo se dividió en tres etapas.  
  La primera consistió en la definición del lenguaje \frob{} de alto
nivel funcional y reactivo.

  Luego se definió el lenguaje \alf{} de bajo nivel (\emph{Bytecode}) más
simple de interpretar y se implementó usando le lenguaje \textit{Haskell} un
compilador que traduce un programa \frob{} al lenguaje \alf{}.

  La última etapa consiste en crear una máquina virtual, que
sea capaz de interpretar el lenguaje \alf{}.
  Por cada plataforma objetivo, es posible realizar una implementación de
la máquina, lo que permite ejecutar un mismo programa en alto nivel en
diferentes plataformas.

  El diseño de la máquina consiste de un núcleo común capaz de interpretar
instrucciones, y módulos bien definidos de
entrada/salida los cuáles varían de una plataforma a otra.
  Ésto permite mayor portabilidad y extensibilidad.

  Debe ser posible ejecutar programas en dicho lenguaje dentro de plataformas
de hardware reducido.
  Considerando ésto, el lenguaje de programación elegido para la
implementación de la máquina virtual es C/C++.

  De esta forma se creó un lenguaje reactivo con las
características deseadas y se codificó una máquina
virtual que permite su ejecución en una arquitectura objetivo deseada.

  Las implementaciones tanto de la máquina virtual como del compilador
son fáciles de mantener, portables y al ser modulares
cuentan con la flexibilidad necesaria para garantizar su extensibilidad.
