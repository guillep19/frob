

El proyecto consiste en la creación de un lenguaje de programación
para robots, cuyas capacidades de cómputo son limitadas.
Para esto se escogió el paradigma de Programación Funcional Reactiva
(\emph{FRP}) el cual permite expresar naturalmente reacciones a
valores que varían en función del tiempo.

El objetivo es utilizarlo con fines educativos,
por lo tanto debe ser simple y fácil de usar por usuarios
inexpertos, no familiarizados con la electrónica ni la informática.

% La implementación se dividió..... 
Para resolver el problema, se dividió en dos etapas.

La primera consiste en la implementación de
un compilador que traduce el lenguaje cuyo nivel es alto a
un lenguaje de bajo nivel (\emph{Bytecode}) más simple e interpretable.

La segunda etapa consiste en implementar una máquina virtual, que
sea capaz de interpretar el lenguaje de bajo nivel.
Por cada plataforma objetivo, es posible realizar una implementación de
la máquina, lo que permite ejecutar los mismos
programas en alto nivel, en diferentes plataformas.

%% makam: en futuro?
El lenguaje se implementará como un DSL embebido en Haskell.

El diseño de la máquina consiste de un núcleo común capaz de interpretar
instrucciones, y módulos bien definidos de
entrada/salida los cuáles varían de una plataforma a otra.
Ésto permite mayor portabilidad y extensibilidad.

Debe ser posible ejecutar programas en dicho lenguaje dentro de plataformas
de hardware reducido.
Considerando ésto, el lenguaje de programación elegido para la
implementación de la máquina virtual es C/C++.

De esta forma se implementó un lenguaje reactivo con las
características deseadas, y una implementación modelo de una máquina
virtual que permite
su ejecución en una arquitectura objetivo deseada.
La misma es fácil de mantener, portable y cuenta con suficiente
flexibilidad para ser extendida.
