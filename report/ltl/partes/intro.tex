En este capítulo se realiza una introducción a la Lógica Temporal Lineal, como una extensión de Lógica 
 Proposicional para poder expresar propiedades en sistemas reactivos.
Con esta finalidad, este lenguaje fue introducido por Pnueli en \cite{pnueli}.

La Lógica Lineal Temporal permite expresar propiedades sobre los sistemas en cuestón,
 ya que se agregan operadores que hacen referencia al tiempo y permite representar los distintos estados
 en distintos momentos durante la ejecución del sistema.
