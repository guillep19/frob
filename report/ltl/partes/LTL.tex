\subsection{Sintaxis de LTL}

Las fórmulas de LTL son construídas según la siguiente gramática
\[ \varphi ::= true | a | \varphi \wedge \varphi | \lnot \varphi | \bigcirc \varphi | \square \varphi | \lozenge \varphi | \varphi \cup \varphi \]

Los operadores temporales permiten establecer relaciones entre las etiquetas
 de los estados del sistema en una ejecución. Estos nuevos operadores son:

\begin{itemize}

\item $\lozenge p \longrightarrow $ eventually $p$ : 
inevitablemente en el futuro se cumplirá $p$

\item $\square p \longrightarrow $ always $p$ : 
ahora y en el futuro se cumple $p$

\item $\bigcirc p \longrightarrow $ next $p$ : 
en el siguiente paso se cumple p

\item $p \cup q \longrightarrow $ $p$ until $q$ : 
$q$ se cumplirá inevitablemente en el futuro, y mientras tanto se cumple $p$

\end{itemize}

%ejemplo 5.2 (mutua exclusión):

\paragraph{Ejemplo.}
Tomando como ejemplo el problema de mutua exclusión entre dos proceso $P_1$ y $P_2$,
 donde los procesos son modelados por tres estados:
\begin{itemize}
\item[(1)] sección no crítica
\item[(2)] espera para entrar a la sección crítica
\item[(3)] sección crítica
\end{itemize}

Sean las proposiciones $wait_i$ y $crit_i$ que representan los estados de espera y sección crítica
 respectivamente para el proceso $P_i$, se pueden representar las siguientes propiedades con LTL.
\begin{itemize}
\item $P_1$ y $P_2$ nunca acceden simultaneamente a la sección crítica
\[ \square (\neg crit_1 \vee \neg crit_2) \]
\item $P_1$ y $P_2$ acceden infinitas veces a la sección crítica
\[ (\square \lozenge crit_1) \wedge (\square \lozenge crit_2) \]
\end{itemize}


\subsection{Semántica de LTL}
La satisfacibilidad en LTL se basa en la satisfacibilidad de trazas, o más precisamente de fragmentos
de traza.
A continuación se definen los conceptos de \textit{traza} y \textit{satisfacibilidad} para
 una fórmula LTL.

\begin{definicion}
Trazas.\\
Sea un sistema de transiciones
 $\text{TS} = (\text{S}, \text{Act}, {\to}, \text{I}, \text{AP}, \text{L})$ y un fragmento de
 camino $\pi = s_0 s_1 s_2 ...$. La traza correspondiente a $\pi$ es
 $\text{Traza}(\pi) = L(s_0) L(s_1) L(s_2) ...$

Además se definen las trazas de un sistema de transiciones como:
\[ \text{Trazas}(\text{TS}) = \{ \text{Traza}(\pi) | \pi \text{ es un camino de TS} \} \]
\end{definicion}


\begin{definicion}
Satisfacibilidad en LTL para trazas.\\
Dada una traza $\sigma = A_0 A_1 A_2 ...$ y dos fórmulas LTL $\varphi$, $\psi$.
\begin{itemize}
\item $\sigma \models \textit{true}$
\item $\sigma \models a$ si y solo si $a \in A_0$
\item $\sigma \models \neg \varphi $ si y solo si $\sigma \not\models \varphi $
\item $\sigma \models \varphi \wedge \psi $ si y solo si $\sigma \models \varphi $ y $\sigma \models \psi $
\item $\sigma \models \bigcirc \varphi $ si y solo si $\sigma [1...] \models \varphi$
\item $\sigma \models \lozenge \varphi $ si y solo si $\exists j \ge 0, \sigma [j...] \models \varphi$
\item $\sigma \models \square \varphi $ si y solo si $\forall j \ge 0, \sigma [j...] \models \varphi$
\item $\sigma \models \varphi \cup \psi $ si y solo si $\exists j \ge 0, \sigma [j...] \models \psi $ y $\forall 0 \le i < j, \sigma [i...] \models \varphi $
\end{itemize}
\end{definicion}

Una vez definida la satisfacibilidad para trazas, se define la satisfacibilidad
 para un sistema de transiciones.
 
\begin{definicion}
Satisfacibilidad en LTL para un sistema de trasiciones.\\
Dado un sistema se transiciones $TS = (S, Act, \rightarrow, I, AP, L)$ y una fórmula LTL $\varphi$.

La relación de satisfacibilidad para un sistema de transiciones se define como 
\[ \text{TS} \models \varphi \text{ si y sólo si } \sigma \models \varphi, \forall \sigma \in \text{Trazas} (\text{TS}) \]
\end{definicion}







\subsection{Propiedades}
Las siguientes propiedades son necesarias para poder expresar cualquier fórmula de LTL utilizando
 un conjunto reducido de operadores ($true, \lnot, \land, \bigcirc, \cup$).

\begin{itemize}
\item $p \lor q = \lnot (\lnot p \land \lnot q)$
\item $p \rightarrow q = \lnot (p \land \lnot q)$
\item $p \leftrightarrow q = \lnot (p \land \lnot q) \land \lnot (\lnot p \land q)$
\item $\lozenge p = true \cup p$
\item $\square p = \lnot (true \cup \lnot p)$
\end{itemize}


\begin{definicion}
Conjunto funcionalmente completo de conectivos.\\
Sea un conjunto de conectivos $C$. Decimos que $C$ es funcionalmente completo si todas las fórmulas
tienen una fórmula equivalente que utiliza unicamente conectivos de $C$.
\end{definicion}


A partir de las propiedades anteriores se puede afirmar que el conjunto {$true, \lnot, \land, \bigcirc, \cup$}
 es un conjunto funcionalmente completo.


De esta forma se normalizará cada fórmula LTL a una equivalente pero con un conjunto reducido de operadores
 a fin de facilitar los prosesos posteriores.
