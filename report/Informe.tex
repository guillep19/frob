% Definimos el estilo del documento
\documentclass[12pt,a4paper,spanish]{book}

\usepackage[utf8]{inputenc}
\usepackage[spanish]{babel}
\usepackage{ amssymb }
\usepackage{ amsmath }
\usepackage{ amsthm }
\usepackage{ wasysym }
\usepackage{ verbatim }

\usepackage{ appendix }
\renewcommand{\appendixname}{Apéndices}
\renewcommand{\appendixtocname}{Apéndices}
\renewcommand{\appendixpagename}{Apéndices}
 
%\usepackage[dvips]{graphicx}
%\DeclareGraphicsExtensions{.pdf,.png,.jpg}

\usepackage{graphicx}

\usepackage{hyperref}
\hypersetup{colorlinks=false}

\usepackage{proof}

\newtheorem{definicion}{Definición}

%Empieza el documento
\begin{document}

% Definimos titulo, autor, fecha.
\title{FROB: Un lenguaje de programación funcional reactivo para robots con bajas capacidades de cómputo}
\author{Guillermo Pacheco}
\date{\today}

% Generamos titulo e indice de contenidos
\maketitle


% Resumen
\chapter*{Resumen}
\addcontentsline{toc}{chapter}{Resumen}
\markboth{RESUMEN}{RESUMEN}

\section{Resumen}

Aca va un resumen breve e introducción al documento.



% Indice general
\cleardoublepage
\addcontentsline{toc}{chapter}{Índice general} % para que aparezca en el indice de contenidos
\tableofcontents

% Indice de figuras
\cleardoublepage
\addcontentsline{toc}{chapter}{Índice de figuras} % para que aparezca en el indice de contenidos
\listoffigures

% indice de tablas
\cleardoublepage
\addcontentsline{toc}{chapter}{Índice de tablas} % para que aparezca en el indice de contenidos
\listoftables


\chapter{Introducción}

% TODO(Marcos): Decir por que se decidio definir un lenguaje propio y no
% utilizar los que se estudiaron anteriormente.
% Tambien decir que comparte con esos lenguajes y en que difiere.

  En este capítulo se describe el diseño del lenguaje \frob{} junto con su semántica.
  Luego se explica de qué manera es traducido al lenguaje \alf{} de
bajo nivel, mas simple de interpretar, el cual podrá ser interpretado
por implementaciones de una misma máquina virtual en diferentes
plataformas de hardware.

  También se describirán las etapas de compilación, desde que se escribe
un programa en alto nivel hasta que es ejecutado en una
plataforma objetivo.

  El diagrama de la Figura \ref{fig:compilacion} resume todas las etapas y
componentes necesarios.

\begin{figure}[h]
\begin{center}
\caption{Etapas y componentes}
\includegraphics[width=0.9\textwidth]{graphs/compilacion.png}
\label{fig:compilacion}
\end{center}
\end{figure}

  En la parte de arriba de la Figura se ve un programa en el
lenguaje \frob{} de alto nivel.
  El desarrollador escribe dicho programa y
ejecuta el compilador \compilador{} y obtiene un archivo \alf{} binario.

  Debajo se muestran diversas plataformas, cada una
con su implementación de la máquina virtual \maquinavirtual{} instalada.

  El desarrollador podrá cargar el mismo código \alf{} en cualquier robot que
esté construido utilizando cualquiera de las plataformas, y la máquina
virtual se encargará de interpretarlo.


\chapter{Programación Funcional Reactiva}




\section{Programación Funcional Reactiva}

Acá voy a hablar de programación funcional reactiva.

Tradicionalmente los programas son formados a partir de
una secuencia de acciones imperativas. Los programas
reactivos suelen formarse por eventos y código iterativo
que se corre cuando un evento ocurre.

Dicho código iterativo suele hacer referencia y manipular
variables compartidas con diferentes rutinas. Ésto lleva a que
como un valor puede ser manipulado desde diferentes lugares,
puedan producirse problemas de concurrencia y algunos valores
pueden quedar en un estado inconsistente.

En el paradigma FRP no existen valores compartidos, sinó que
dichos valores dependientes del tiempo, tienen una representación
llamada Comportamiento y la única forma de modificarlos, es
a partir de como fueron definidos.

 

\begin{definicion}
Comportamientos (Behaviours).\\
Un comportamiento es un valor contínuo que depende del paso del tiempo.
Los comportamientos se pueden definir, combinar, pasarlos como
argumentos a funciones, retornarlos.
Un comportamiento puede ser un valor constante, el tiempo mismo (un reloj),
o puede formarse combinando otros comportamientos, por ejemplo secuencialmente
o paralelamente.
\end{definicion}

\begin{definicion}
Eventos (Streams).\\
Son valores discretos dependientes del tiempo, que forman
una secuencia finita o infinita de ocurrencias. Cada ocurrencia
está formada por el valor y el instante de tiempo.
\end{definicion}

La principal diferencia entre Comportamientos y Eventos, es que los
comportamientos son valores contínuos y los eventos son discretos.

Los comportamientos representan cualquier valor en función del tiempo,
por ejemplo:
\begin{itemize}
\item \textit{entrada} sensor de distancia, temperatura, video
\item \textit{salida} velocidad, voltaje
\item \textit{valores} intermedios calculados
\end{itemize}

Las operaciones que se pueden realizar sobre los comportamientos incluyen:
\begin{itemize}
\item \textit{Operaciones genéricas} Aritmética, integración, diferenciación
\item \textit{Operaciones específicas de un dominio} como escalar video, aplicar filtros, detección de patrones.
\end{itemize}

Los eventos pueden ser sensores específicos de un dominio, por ejemplo un
botón, un click, una interrupción o mensaje asincrónico.
También puede ser generado a partir de valores de un comportamiento,
como ser \emph{Temperatura alta}, \emph{Batería baja}.

Las operaciones que se pueden realizar sobre los eventos incluyen:
\begin{itemize}
\item fmap, filter
\item Modificar un \emph{Comportamiento} reactivo
\end{itemize}


\section{Ejemplo}

Para entender un poco más las ideas de Comportamiento y Evento, se puede
plantear el siguiente ejemplo.

  En una cuenta de un banco, el saldo se puede definir
como un comportamiento, el cuál solo se modifica cuando ocurre
un movimiento.
  Un movimiento puede ser depositar dinero o extraer
dinero de la cuenta.
  Un dato importante a ver, es que no es posible asignar un valor
sin que sea por medio de su propia definición, por lo que nadie
podría realizar la asignación $saldo = 1000000$.
  La misma operación sería posible creando un movimiento, el cuál
afectaría al saldo.

  Usaremos la notación $<nombre> :: Event <tipo>$ para definir
eventos y la notación $<nombre> :: Behaviour <tipo>$ para
definir comportamientos.

\begin{verbatim}
movimiento :: Event Number
movimiento = read(0)

saldo :: Behaviour Number
saldo = movimiento

alert :: Event Bool
alert = saldo > 1000

on alert:
  write(0, 'El saldo es mayor a 1000')

\end{verbatim}

Un comportamiento, solo se modifica cuando cambian sus componentes,





%state = State(state, [input])
%Output(state)


%\chapter{Autómatas de Büchi}
%Los autómatas regulares fueron introducidos por Huffman \cite{huffman} como
 herramientas para reconocer palabras de largo finito.
Pero para representar las ejecuciones de un sistema se necesita trabajar con palabras de largo infinito.
Para esto Büchi introdujo el concepto de Autómatas de Büchi (NBA) \cite{buchi}.
Estos son una variante de los autómatas
 regulares utilizados para reconocer palabras de largo infinito.

%Para verificar si un sistema $TS$ dado satisface una determinada propiedad se trabaja con el autómata de
% Büchi $\mathcal{A}$, que representa los "bad traces" de la propiedad ($\mathcal{A}$ reconoce el complemento
% de la propiedad a verificar). Se realiza el producto entre $\mathcal{A}$ y $TS$ y mediante un análisis del
% grafo se determina si $TS \models P$.


\section{Lenguajes \textit{omega}-regulares}
Llamamos lenguajes regulares a los lenguajes que se pueden representar mediante expresiones regulares.
Las palabras de estos lenguajes tienen largo finito.
 %, es decir secuencias de símbolos de largo finito.
 En cambio si queremos tratar con palabras de largo infinito necesitaremos una clase distinta de
 lenguajes, asociados con una clase distinta de expresiones. Esta nueva clase de lenguajes se denomina
 lenguajes \textit{omega}-regulares.
 
%Una palabra infinita sobre el alfabeto $\Sigma$ es una secuencia de símbolos pertenecientes a $\Sigma$.
Utilizaremos la letra griega $\omega$ (\textit{omega}) para denotar la repetición infinita,
 por ejemplo $a^\omega$ es la palabra
 infinita que contiene sólo símbolos $a$.\footnote{Notar que $a^*$ representa un lenguaje, es el conjunto de todas las palabras finitas que contienen únicamente el símbolo $a$, mientras que $a^\omega$ representa
 solamente una palabra.}
 
Llamamos $\Sigma^\omega$ al conjunto de todas las palabras infinitas sobre $\Sigma$.
 Cualquier subconjunto de $\Sigma^\omega$ es un lenguaje de palabras infinitas.

\begin{definicion}
Expresión $\omega$-regular.\\
Una expresión $\omega$-regular $G$ sobre el alfabeto $\Sigma$ tiene la siguiente forma
\[ G = E_1.F_1^\omega + ... + E_n.F_n^\omega \]
donde $n \geq 1$ y $E_1, ..., E_n, F_1, ..., F_n$ son expresiones regulares sobre $\Sigma$ tales que
 $\varepsilon \not\in L(F_i)$, siendo $L(F_i)$ el lenguaje representado por la expresión regular $F_i$.
\end{definicion}

Un lenguaje $\omega$-regular, análogamente a los regulares, es un lenguaje que se puede representar
mediante una expresión $\omega$-regular.


 
 
\paragraph{Ejemplo.}
Considerando el NBA de la figura \ref{fig:nba} sobre el alfabeto $\{ A, B, C \}$.\\

\begin{figure}[hbtp]
\begin{center}
\caption{Ejemplo de Autómata de Büchi}
\includegraphics[width=0.7\textwidth]{buchi/imagenes/ejemplo4_28.png}
\label{fig:nba}
\end{center}
\end{figure}

El lenguaje aceptado por el NBA está dado por la expresión regular
\[ C^*AB(B^+ + BC^*AB)^\omega \]
 
\section{Autómata de Büchi Generalizado}
En este trabajo se utiliza una variante de los NBA, el Autómata de Buchi Generalizado (GNBA).
La estructura de los GNBAs es muy similar a la de los NBAs, varía sólo en el conjunto de aceptación.
En lugar de este contiene un conjunto F cuyos elementos son los conjuntos de aceptación $F_i$.
El criterio de aceptación en estos autómatas es visitar infinitas veces todos los conjuntos de
 aceptación $F_i$.

\begin{definicion}
Autómatas de Büchi Generalizado.\\
Los Autómatas de Büchi Generalizado (GNBA) consisten en una tupla $\mathcal{G} = (\text{Q}, \Sigma, \delta, \text{Q}_0, \mathcal{F})$, donde: $\text{Q}$, $\Sigma$, $\delta$ y $\text{Q}_0$ se definen
igual que en los NBA mientras que $\mathcal{F} \subseteq 2^Q$ es el conjunto de aceptación.
\end{definicion}

El lenguaje aceptado por un GNBA $\mathcal{G}$ consiste en todas las palabras que pasan infinitas
 veces por todos los conjuntos de aceptación.

Más formalmente, dado un GNBA $\mathcal{G} = (\text{Q}, \Sigma, \delta, \text{Q}_0, \mathcal{F})$ y
 una secuencia infinita de estados $q_0 q_1 q_2 q_3 ...$, esta es aceptada por $\mathcal{G}$ si
\[ \forall \text{F} \in \mathcal{F}, \exists \text{ infinitas } j \in \mathbb{N} \text{ tal que } q_j \in F \]

 
\paragraph{Ejemplo.}
En la figura \ref{fig:seccion_critica} se muestra un GNBA sobre el alfabeto $2^{AP}$
 donde $AP = \{ crit_1, crit_2 \}$ y los conjuntos de aceptación
 $F_1 = \{ q_1 \}$ y $F_2 = \{ q_2 \}$.

\begin{figure}[hbtp]
\begin{center}
\caption{GNBA para sección crítica}
\includegraphics[width=0.7\textwidth]{buchi/imagenes/ejemplo4_53.png}
\label{fig:seccion_critica}
\end{center}
\end{figure}

El lenguaje aceptado es la propiedad que contiene todas las palabras infinitas tal que $crit_1$ y 
$crit_2$ ocurren infinitas veces.
Este GNBA representa la propiedad de que ninguno de los procesos quede infinitamente
 esperando para ingresar a la sección crítica.


%\chapter{Lógica Temporal Lineal}
%\subsection{Sintaxis de LTL}

Las fórmulas de LTL son construídas según la siguiente gramática
\[ \varphi ::= true | a | \varphi \wedge \varphi | \lnot \varphi | \bigcirc \varphi | \square \varphi | \lozenge \varphi | \varphi \cup \varphi \]

Los operadores temporales permiten establecer relaciones entre las etiquetas
 de los estados del sistema en una ejecución. Estos nuevos operadores son:

\begin{itemize}

\item $\lozenge p \longrightarrow $ eventually $p$ : 
inevitablemente en el futuro se cumplirá $p$

\item $\square p \longrightarrow $ always $p$ : 
ahora y en el futuro se cumple $p$

\item $\bigcirc p \longrightarrow $ next $p$ : 
en el siguiente paso se cumple p

\item $p \cup q \longrightarrow $ $p$ until $q$ : 
$q$ se cumplirá inevitablemente en el futuro, y mientras tanto se cumple $p$

\end{itemize}

%ejemplo 5.2 (mutua exclusión):

\paragraph{Ejemplo.}
Tomando como ejemplo el problema de mutua exclusión entre dos proceso $P_1$ y $P_2$,
 donde los procesos son modelados por tres estados:
\begin{itemize}
\item[(1)] sección no crítica
\item[(2)] espera para entrar a la sección crítica
\item[(3)] sección crítica
\end{itemize}

Sean las proposiciones $wait_i$ y $crit_i$ que representan los estados de espera y sección crítica
 respectivamente para el proceso $P_i$, se pueden representar las siguientes propiedades con LTL.
\begin{itemize}
\item $P_1$ y $P_2$ nunca acceden simultaneamente a la sección crítica
\[ \square (\neg crit_1 \vee \neg crit_2) \]
\item $P_1$ y $P_2$ acceden infinitas veces a la sección crítica
\[ (\square \lozenge crit_1) \wedge (\square \lozenge crit_2) \]
\end{itemize}


\subsection{Semántica de LTL}
La satisfacibilidad en LTL se basa en la satisfacibilidad de trazas, o más precisamente de fragmentos
de traza.
A continuación se definen los conceptos de \textit{traza} y \textit{satisfacibilidad} para
 una fórmula LTL.

\begin{definicion}
Trazas.\\
Sea un sistema de transiciones
 $\text{TS} = (\text{S}, \text{Act}, {\to}, \text{I}, \text{AP}, \text{L})$ y un fragmento de
 camino $\pi = s_0 s_1 s_2 ...$. La traza correspondiente a $\pi$ es
 $\text{Traza}(\pi) = L(s_0) L(s_1) L(s_2) ...$

Además se definen las trazas de un sistema de transiciones como:
\[ \text{Trazas}(\text{TS}) = \{ \text{Traza}(\pi) | \pi \text{ es un camino de TS} \} \]
\end{definicion}


\begin{definicion}
Satisfacibilidad en LTL para trazas.\\
Dada una traza $\sigma = A_0 A_1 A_2 ...$ y dos fórmulas LTL $\varphi$, $\psi$.
\begin{itemize}
\item $\sigma \models \textit{true}$
\item $\sigma \models a$ si y solo si $a \in A_0$
\item $\sigma \models \neg \varphi $ si y solo si $\sigma \not\models \varphi $
\item $\sigma \models \varphi \wedge \psi $ si y solo si $\sigma \models \varphi $ y $\sigma \models \psi $
\item $\sigma \models \bigcirc \varphi $ si y solo si $\sigma [1...] \models \varphi$
\item $\sigma \models \lozenge \varphi $ si y solo si $\exists j \ge 0, \sigma [j...] \models \varphi$
\item $\sigma \models \square \varphi $ si y solo si $\forall j \ge 0, \sigma [j...] \models \varphi$
\item $\sigma \models \varphi \cup \psi $ si y solo si $\exists j \ge 0, \sigma [j...] \models \psi $ y $\forall 0 \le i < j, \sigma [i...] \models \varphi $
\end{itemize}
\end{definicion}

Una vez definida la satisfacibilidad para trazas, se define la satisfacibilidad
 para un sistema de transiciones.
 
\begin{definicion}
Satisfacibilidad en LTL para un sistema de trasiciones.\\
Dado un sistema se transiciones $TS = (S, Act, \rightarrow, I, AP, L)$ y una fórmula LTL $\varphi$.

La relación de satisfacibilidad para un sistema de transiciones se define como 
\[ \text{TS} \models \varphi \text{ si y sólo si } \sigma \models \varphi, \forall \sigma \in \text{Trazas} (\text{TS}) \]
\end{definicion}







\subsection{Propiedades}
Las siguientes propiedades son necesarias para poder expresar cualquier fórmula de LTL utilizando
 un conjunto reducido de operadores ($true, \lnot, \land, \bigcirc, \cup$).

\begin{itemize}
\item $p \lor q = \lnot (\lnot p \land \lnot q)$
\item $p \rightarrow q = \lnot (p \land \lnot q)$
\item $p \leftrightarrow q = \lnot (p \land \lnot q) \land \lnot (\lnot p \land q)$
\item $\lozenge p = true \cup p$
\item $\square p = \lnot (true \cup \lnot p)$
\end{itemize}


\begin{definicion}
Conjunto funcionalmente completo de conectivos.\\
Sea un conjunto de conectivos $C$. Decimos que $C$ es funcionalmente completo si todas las fórmulas
tienen una fórmula equivalente que utiliza unicamente conectivos de $C$.
\end{definicion}


A partir de las propiedades anteriores se puede afirmar que el conjunto {$true, \lnot, \land, \bigcirc, \cup$}
 es un conjunto funcionalmente completo.


De esta forma se normalizará cada fórmula LTL a una equivalente pero con un conjunto reducido de operadores
 a fin de facilitar los prosesos posteriores.


%\chapter{Lógica de Árbol Computacional}
%En Computation Tree Logic (CTL) se mantienen los conectivos de LTL. Sus significados son
 los mismos, pero a diferencia de LTL, los conectivos temporales están sujetos a cuantificadores
que hacen referencia a las posibles ejecuciones desde el estado actual. Esto es, mirando
 el árbol de ejecuciones, todos los posibles caminos a partir del estado actual.

\subsection{Sintaxis de CTL}

Las fórmulas de CTL son construídas según la siguiente gramática
\[ \varphi ::= true | a | \varphi \wedge \varphi | \lnot \varphi | \exists \bigcirc \varphi | \exists \square \varphi | \exists \lozenge \varphi | \exists \varphi \cup \varphi | \forall \bigcirc \varphi | \forall \square \varphi | \forall \lozenge \varphi | \forall \varphi \cup \varphi \]

Donde $a$ es una proposición atómica.	

\paragraph{Ejemplos de fórmulas correctas.} 

Dado el conjunto de proposiciones $\{p, q\}$, las siguientes son fórmulas CTL correctas:
\begin{itemize}
\item $\exists \bigcirc p $
\item $\forall p \cup q $
\item $\exists \lozenge (\forall ((\exists \square p) \cup q)) $
\end{itemize}


% Example 6.3

\paragraph{Ejemplo de la mutua exculsión.}
Considerando el ejemplo visto en el capítulo anterior para verificar la mutua exclusión:
La siguiente fórmula indica que los dos procesos no pueden acceder a la sección crítica simultáneamente
\[ \forall \square (\neg crit_1 \vee \neg crit_2) \]

Y la siguiente fórmula expresa que ambos procesos accederán infinitas veces a
 la sección crítica
\[ (\forall \square \forall \lozenge crit_1) \wedge (\forall \square \forall \lozenge crit_2) \]



\subsection{Semántica de CTL}
La satisfacibilidad en CTL para un sistema de transiciones se basa en la satisfacibilidad
 de estados del mismo, a diferencia de LTL que se basa en la satisfacibilidad de trazas.

A continuanción se presenta el concepto de satisfacibilidad de una fórmula CTL por un estado
 del sistema.
Este es el concepto de partida para analizar la satisfacibilidad para los sistemas de transiciones.

% Definition 6.4
\begin{definicion}
Satisfacibilidad en CTL.\\
Dado un sistema de transiciones con conjunto de estados $S$. Dados $s \in S$ y $\varphi$, $\psi$
 fórmulas CTL.
\begin{itemize}
\item $s \models a$ si y solo si $a \in L(s)$
\item $s \models \neg \varphi $ si y solo si $s \not\models \varphi $
\item $s \models \varphi \wedge \psi $ si y solo si $s \models \varphi $ y $s \models \psi $
\item $s \models \exists \bigcirc \varphi $ si y solo si $\pi \models \bigcirc \varphi $ para algún $\pi \in Paths(s)$
\item $s \models \exists \varphi \cup \psi $ si y solo si $\pi \models \varphi \cup \psi $ para algún $\pi \in Paths(s)$
\item $s \models \forall \bigcirc \varphi $ si y solo si $\pi \models \bigcirc \varphi $ para todo $\pi \in Paths(s)$
\item $s \models \forall \varphi \cup \psi $ si y solo si $\pi \models \varphi \cup \psi $ para todo $\pi \in Paths(s)$

\end{itemize}

Donde:
\begin{itemize}
\item $\pi \models \bigcirc \varphi $ si y solo si $\pi [1] \models \varphi $
\item $\pi \models \varphi \cup \psi $ si y solo si $\exists j >= 0::(\pi [j] \models \psi $ y $(\forall k:0 \le k < j:\pi [k] \models \varphi))$
\end{itemize}
\end{definicion}

Ahora que está definida la satisfacibilidad para un estado cualquiera de un sistema de transiciones
 se tratará la satisfacibilidad para un sistema.
 
\begin{definicion}
Satisfacibilidad en CTL para un sistema de trasiciones.\\
Dado un sistema se transiciones $TS = (S, Act, \rightarrow, I, AP, L)$ y una fórmula CTL $\varphi$. 

Sea el conjunto de satisfacibilidad $Sat(\varphi) = \{ s \in S | s \models \varphi \}$.

La relación de satisfacibilidad para un sistema de transiciones se define como 
\[ TS \models \varphi \text{ si y sólo si } s_0 \models \varphi, \forall s_0 \in I \]
\end{definicion}

En la figura \ref{fig:semantica_CTL} se ilustran ejemplos de fórmulas CTL y sus
 ejecuciones a partir de un estado representadas en el árbol de ejecución del sistema.

% Definition 6.5

\begin{figure}[hbtp]
\begin{minipage}{\textwidth}
\begin{center}
\caption[Semántica de CTL]%
{Semántica de CTL \footnote[1]{Imagen tomada de \cite{katoen}}}
\includegraphics[width=0.8\textwidth]{ctl/imagenes/semanticaCTL.png}
\label{fig:semantica_CTL}
\end{center}
\end{minipage}
\end{figure}


\subsection{Propiedades}
En esta sección se muestran propiedades útiles para poder expresar todas las fórmulas
 de CTL utilizando un conjunto reducido y funcionalmente completo de conectivos.

\begin{itemize}
\item $\exists \lozenge \varphi = \exists (true \cup \varphi) $
\item $\forall \lozenge \varphi = \forall (true \cup \varphi) $
\item $\exists \square \varphi = \lnot \forall \lozenge \lnot \varphi $
\item $\forall \square \varphi = \lnot \exists \lozenge \lnot \varphi $
\item $\forall \bigcirc \varphi = \lnot \exists \bigcirc \lnot \varphi $
\item $\forall \lozenge \varphi = \lnot \exists \square \lnot \varphi $
\end{itemize}

Con estas propiedades es posible expresar cualquier propiedad en CTL mediante
 una fórmula CTL equivalente que sólo contenga el cuantificador existencial.


\subsection{CTL vs LTL}
Tanto CTL como LTL son lenguajes que permiten expresar muchas propiedades para un sistema.
 Pero son lógicas incompatibles entre sí. Esto quiere decir que existen fórmulas de CTL
 para las cuales no hay una equivalente en LTL y viceversa.

\begin{definicion}
Equivalencia entre fórmulas CTL y fórmulas LTL.\\
Una fórmula CTL $\varphi$ y una fórmula LTL $\psi$ son equivalentes si para todo
 sistema de transiciones $TS$:
\[ TS \models \varphi \text{ si y solo si } TS \models \psi \]
\end{definicion}

Un estado satisface una fórmula LTL $\varphi$ cuando todos los caminos a partir de
 este estado satisfacen $\varphi$.
De esto se puede observar que para obtener una fórmula LTL equivalente a una
 fórmula CTL dada basta con eliminar los cuantificacdores universales de la misma.
Más precisamente, dada una fórmula CTL, en caso de que exista un fórmula LTL
 equivalente, esta se obtiene eliminando los cuantificadores tanto universales
 como existenciales de la misma.
 

%\chapter{Implementación del verificador}
%
% TODO(Marcos): Decir por que se decidio definir un lenguaje propio y no
% utilizar los que se estudiaron anteriormente.
% Tambien decir que comparte con esos lenguajes y en que difiere.

  En este capítulo se describe el diseño del lenguaje \frob{} junto con su semántica.
  Luego se explica de qué manera es traducido al lenguaje \alf{} de
bajo nivel, mas simple de interpretar, el cual podrá ser interpretado
por implementaciones de una misma máquina virtual en diferentes
plataformas de hardware.

  También se describirán las etapas de compilación, desde que se escribe
un programa en alto nivel hasta que es ejecutado en una
plataforma objetivo.

  El diagrama de la Figura \ref{fig:compilacion} resume todas las etapas y
componentes necesarios.

\begin{figure}[h]
\begin{center}
\caption{Etapas y componentes}
\includegraphics[width=0.9\textwidth]{graphs/compilacion.png}
\label{fig:compilacion}
\end{center}
\end{figure}

  En la parte de arriba de la Figura se ve un programa en el
lenguaje \frob{} de alto nivel.
  El desarrollador escribe dicho programa y
ejecuta el compilador \compilador{} y obtiene un archivo \alf{} binario.

  Debajo se muestran diversas plataformas, cada una
con su implementación de la máquina virtual \maquinavirtual{} instalada.

  El desarrollador podrá cargar el mismo código \alf{} en cualquier robot que
esté construido utilizando cualquiera de las plataformas, y la máquina
virtual se encargará de interpretarlo.


\section{Lenguaje de programación}
El lenguaje utilizado para desarrollar el verificador fue \textit{Python 2.7}. Esto se debe a que sus
 atributos se adaptaban a las necesidades del verificador.

\textit{Python} es un lenguaje de alto nivel con características de lenguaje funcional,
 y esto ayuda a que la implementación de
 algoritmos requiera menos instrucciones y que a su vez estas instrucciones sean de alto
 nivel, generando un código simple y fácil de entender por lo que es perfectamente
 mantenible por otras personas.
 
Además es un lenguaje extremadamente flexible, lo que permite reutilizar un algoritmo
 implementado permitiendo aplicarlo sobre distintas estructuras sin necesidad de adaptarlo o
 reescribirlo. Esto es de mucha utilidad ya que se manejan estructuras similares como ser
 fórmulas de distintas lógicas y sus estructuras quedan encapsuladas dentro de sus respectivos
 módulos, pudiendo aplicar el resto de los algoritmos sin necesidad de modificaciones.

Otro de los atributos importantes de este lenguaje es que soporta la programación orientada
 a objetos. Esto reafirma la idea de flexibilidad, modularidad y reutilización de código,
 por lo que fue otro de los atributos que se tuvieron en cuenta al momento de seleccionar
 el lenguaje de programación.



\section{Librerías auxiliares}
En la implementación del verificador se utilizaron dos librerías auxiliares.
La primera de ellas es \textit{Expat} que provee herramientas para
 parsear xml en objetos. La otra librería es \textit{Python Lex-Yacc}, una implementación
 de \textit{Yacc} para Python es utilizada para generar el parser correspondiente a cada lógica.

En esta sección se detallan las librerías mencionadas.

\subsection{Expat}
La librería \texttt{xml.parsers.expat} es una interfaz para el parser de XML \textit{Expat}.
Esta librería provee la clase \texttt{xmlparser}, el cual contiene entre otros los siguientes
 métodos

\begin{itemize}
\item \texttt{ParserCreate()}

Retorna una nueva instancia de la clase \texttt{xmlparser}.

\item \texttt{ParseFile(file)}

Parsea el contenido XML desde el archivo \textit{file}.

\item \texttt{StartElementHandler(name, attributes)}

Define la rutina que se invoca cada vez que comienza un nuevo elemento XML.

\textit{name} indica el nombre del elemento y \textit{attributes} es un hash
 conteniendo los atributos del elemento con sus respectivos valores.

\item \texttt{EndElementHandler(name)}

Define la rutina a invocar cuando se cierra un elemento XML.

\textit{name} indica el nombre del elemento a cerrar.

\end{itemize}

Esta librería fue utilizada para parsear los sistemas de transiciones desde \textit{GraphML}.
Este lenguaje se describe más adelante.

\subsection{Python Lex-Yacc}
\textit{Python Lex-Yacc} (PLY) es una implementación para \textit{Python} de las herramientas
 \textit{Lex} y \textit{Yacc}.

Es un paquete implementado puramente en \textit{Python} que permite generar parsers facilmente.
Este paquete está compuesto por los módulos \texttt{lex.py} y \texttt{yacc.py}.
El primero de ellos, \texttt{lex.py}, es utilizado para separar el texto a parsear en
 un conjunto de marcas llamados \textit{tokens} a partir de expresiones regulares definidas
 para cada uno de estos \textit{tokens}.

Una vez obtenido este conjunto de \textit{tokens} el módulo \texttt{yacc.py} reconoce los
 elementos del lenguaje previamente definidos mediante una gramática libre de contexto y
 construye el árbol sintáctico correspondiente.

Las expresiones regulares así como las gramáticas libres de contexto utilizadas para los
 lenguajes de LTL y CTL se encuentran definidas en la sección \ref{sec:ply}.

\section{GraphML}
\textit{GraphML} es un método para describir estructuras de grafos basado en el lenguaje
 de marcas XML.
No es un lenguaje en sí mismo, sino que define elementos y atributos en XML que permiten
 representar varios tipos de estructuras de grafos, como ser grafos dirigidos, no dirigidos,
 hipergrafos entre otros.

Los elementos básicos en \textit{GraphML} son
\begin{itemize}
\item \texttt{graph}

Este elemento representa el grafo en sí, conteniendo atributos como \texttt{edgedefault}
 que indica si sus aristas son dirigidas o no.

\item \texttt{node}

Contiene la información de cada nodo como su identificador.

\item \texttt{edge}

Representa cada arista, indicando su origen y destino en los atributos \texttt{source} y
 \texttt{target} respectivamente.

\end{itemize}

Al estar definido sobre un lenguaje tan simple y utilizado como XML se convierte en un método
 fácilmente parseable.

Además este método es altamente flexible para agregar nuevos elementos y atributos según
 la necesidad de cada aplicación específicamente.

En este proyecto \textit{GraphML} es utilizado para representar los sistemas de transiciones.

\section{Módulos desarrollados}
El verificador está compuesto por cuatro módulos, destacando principalmente los módulos
 \textit{LTL} y \textit{CTL} que implementan las interpretaciones  de las respectivas
 lógicas.
Esto permite agregar nuevos lenguajes para expresar propiedades, como pueden ser otros tipos
 de lógicas, mediante la implementación de su respectivo módulo.

En el caso de LTL se tuvo que implementar el módulo auxiliar \textit{Autómatas de Büchi}
 para manipular dichos autómatas, ya que son necesarios para el algoritmo de verificación
 de esta lógica.
Si bien este es un módulo auxiliar para la verificación de propiedades en LTL también
 es posible expresar propiedades directamente mediante estos autómatas para luego ser
 verificadas.

También se tiene un módulo destinado a los sistemas de transiciones para representar los
 modelos de los sistemas reactivos.

En la figura \ref{fig:paquetes} se muestra la integración entre los distintos módulos.

\begin{figure}[hbtp]
\begin{center}
\caption{Módulos del sistema}
\includegraphics[width=0.7\textwidth]{mc/imagenes/paquetes.png}
\label{fig:paquetes}
\end{center}
\end{figure}


\subsection{Sistemas de transiciones}
El módulo \textit{Sistemas de transiciones} contiene la implementación de la estructura
 de un sistema de transiciones.
También se encuentran implementadas las operaciones básicas entre sistemas. De esta forma
 es posible crear sistemas de transiciones complejos a partir de sistemas más simples.
Además estas operaciones permiten modelar sistemas reactivos que interactúan entre sí.

Este módulo se compone de tres clases.
La clase principal del módulo es \texttt{TS}, que contiene la estructura y las operaciones
 necesarias para representar los sistemas de transiciones.
Además se tienen las clases \texttt{TSEstado} y \texttt{TSTransicion} para representar
 los estados y las transiciones del sistema respectivamente.

\begin{figure}[hbtp]
\begin{center}
\caption{Módulo de Sistemas de transiciones}
\includegraphics[width=0.8\textwidth]{mc/imagenes/ts.png}
\label{fig:modulo_TS}
\end{center}
\end{figure}

\subsubsection{Operaciones entre sistemas de transiciones}
Este módulo implementa dos de las operaciones principales entre sistemas de
 transiciones.
Las operaciones implementadas fueron vistas anteriormente y son el Intercalado y el
 \textit{Handshaking}.
Estas operaciones se encuentran implementadas en las funciones \texttt{intercaladoTS}
 y \texttt{productoSynTS} de la clase \texttt{TS}.

El autómata devuelto como resultado de cada una de estas operaciones corresponde únicamente
 a la parte alcanzable del mismo.

\subsection{Autómatas de Büchi}	
Este módulo fue implementado debido a la necesidad de tratar estos autómatas para la
 verificación de propiedades expresadas en lenguajes lineales, como LTL.

Como ya se vio en los capítulos anteriores el algoritmo de verificación para propiedades en
 LTL se basa en la construcción de un \textit{Autómata de Büchi}.
El primer paso de dicho algoritmo es traducir la propiedad a un autómata equivalente.

También es posible expresar propiedades directamente sobre estos autómatas.
Se pueden expresar las ejecuciones no deseadas con un autómata de Büchi y verificar
 si el sistema permite alguna de estas ejecuciones.
En definitiva es esto lo que realiza el algoritmo de verificación para LTL, por lo que
 esto es equivalente a saltear el primer paso del mismo.

Este módulo consta de dos clases principales.
\begin{itemize}
\item \texttt{GNBA}

Esta clase contiene la estructura los autómatas de Büchi.
Con este fin también fue necesario implementar la clase \texttt{Nodo} para representar los
 nodos de estos autómatas.

\item \texttt{ProductoTSNBA}

Esta clase contiene la información del producto entre un sistema de transiciones y un
 autómata de Büchi.

Este producto se implementó en una calse específica, ya que para generarse necesita un
 algoritmo específico. Además es la clase que implementa la parte principal del algoritmo
 de verificación.

Esta clase también puede ser parte del módulo \textit{LTL} pero finalmente se optó por
 mantenerla en este módulo ya que como se mencionó anteriormente las propiedades pueden
 ser expresadas directamente mediante autómatas y el algoritmo de verificación es el mismo.

También se implementaron las clases \texttt{ProductoTSNBANodo} y\\
 \texttt{ProductoTSNBATransicion}
 para representar los nodos y transiciones del producto respectivamente.

\end{itemize}

\begin{figure}[hbtp]
\begin{center}
\caption{Módulo de Autómatas de Büchi}
\includegraphics[width=\textwidth]{mc/imagenes/buchi.png}
\label{fig:modulo_buchi}
\end{center}
\end{figure}

\subsubsection{Construcción del producto entre un sistema de transiciones y un autómata de Büchi}
Este producto genera como resultado un nuevo autómata, el cuál está representado
 por la clase \texttt{ProductoTSNBA}.

Esta clase implementa el algoritmo de contrucción de este producto mediante
 la función \texttt{ProductoTSNBA(ts, nba)}, donde:
\begin{itemize}
\item \texttt{ts} es un sistema de transiciones.
\item \texttt{nba} es un autómata de Büchi.
\end{itemize}

Esta función devuelve la parte alcanzable del producto.

\subsubsection{Verificación de propiedadades}
Una vez obtenido el producto mencionado en la sección anterior el siguiente paso
 es verificar si existen palabras reconocidas por este autómata.

Como se vio en capítulos anteriores, como el autómata reconoce las palabras que no cumplen
 la propiedad deseada, este proceso es equivalente encontrar las ejecuciones del sistema
 que no cumplen la propiedad.
En caso de existir al menos una, esta es un contrajemplo que muestra que el sistema no
 cumple dicha propiedad.

Este proceso se encuentra implementado en la función \texttt{verificar()} de la clase
 \texttt{ProductoTSNBA}.
Esta función se basa en el recorrido DFS para buscar un ciclo en el autómata.
Este ciclo debe pasar por al menos un estado de cada conjunto de aceptación del
 autómata, como se vio en el capítulo correspondiente a Autómatas de Büchi.

\subsection{LTL}
El módulo \textit{LTL} está comprendido por las clases necesarias para implementar dicha
 lógica.
En la figura \ref{fig:modulo_LTL} se muestran las relaciones entre la clase \texttt{FormulaLTL}
 y todas sus especializaciones que implementan la sintaxis y semántica de LTL.

\begin{figure}[hbtp]
\begin{center}
\caption{Módulo de LTL}
\includegraphics[width=\textwidth]{mc/imagenes/ltl.png}
\label{fig:modulo_LTL}
\end{center}
\end{figure}

Es importante destacar que este módulo no implementa el algoritmo de verificación para LTL.
El algoritmo es implementado en su mayor parte por el módulo \textit{Autómatas de Büchi}.
El módulo \textit{LTL} simplemente traduce la fórmula a verificar en un autómata equivalente
 y traslada la responsabilidad de la verificación al módulo correspondiente.

La clase que inicia el proceso de verificación es \texttt{LTLMC}.
El único objetivo de esta clase es implementar la función \texttt{verificar(ts, formula)},
 donde:
\begin{itemize}
\item \texttt{ts} es un sistema de transiciones.
\item \texttt{formula} es una fórmula LTL.
\end{itemize}

Esta función es la encargada de iniciar el proceso de verificación traduciendo la fórmula LTL
 al autómata correspondiente, delegar la verificación al módulo de \textit{Autómatas de Büchi}
 y luego desplegar el resultado de la misma.

\subsection{CTL}
Este módulo al igual que el módulo anterior implementa la sintaxis y semántica de CTL.
Pero a diferencia del módulo \textit{LTL} este implementa el algoritmo de verificación.
Esto se debe a que el algoritmo para este tipo de lógica no requiere la implementación
 de módulos ni estructuras auxiliares.
Este algoritmo se basa en el cálculo del conjunto de satisfacibilidad de una fórmula
 y sus subfórmulas. Se encuentra implementado en la función \texttt{verificar(ts, formula)},
 donde:
\begin{itemize}
\item \texttt{ts} es un sistema de transiciones.
\item \texttt{formula} es una fórmula CTL.
\end{itemize}

Esta función verifica que el conjunto de los estados iniciales de \texttt{ts} esté
 incluído en el conjunto de satisfacibilidad de \texttt{formula}, tal como se vio
 en el capítulo de CTL.
Además esta función se encuentra en la clase \texttt{CTLMC}, cuyo único objetivo es
 alojar la misma.

Por otro lado el cálculo del conjunto de satisfacibilidad se encuentra definido en la
 función \texttt{getSat(ts)} de la clase \texttt{FormulaCTL}, donde \texttt{ts} es un
 sistema de transiciones.

\begin{figure}[hbtp]
\begin{center}
\caption{Módulo de CTL}
\includegraphics[width=\textwidth]{mc/imagenes/ctl.png}
\label{fig:modulo_CTL}
\end{center}
\end{figure}


\section{GraphML para sistemas de transiciones}
Como se mencionó anteriormente se utilizó el formato de archivo \textit{GraphML} para
 representar los sistemas de transiciones.
Los estados son representados por el elemento \texttt{node} mientras que las transiciones
 son representadas por el elemento \texttt{edge}.

El formato \textit{GraphML} no establece como representar las etiquetas tanto en los nodos
 como en las aristas.
Para esto existe el atributo \texttt{data}, que proporciona flexibilidad
 para agregar atributos no contemplados por el formato.
Esto tiene un desventaja, y es que los atributos no contemplados por el formato no se
 representan de forma estándar, y por lo tanto su representación depende del editor
 utilizado.
En este caso el editor utilizado es \textit{yEd}.

Para los estados se guarda la siguiente información:
\begin{itemize}
\item Identificador

Este valor se guarda en el atributo \texttt{id} de cada nodo.
Es el identificador del estado, por lo que debe ser único.

Cuando se genera un sistema de transiciones mediante el verificador este genera los identificadores
 de cada estado automáticamente.

\item Proposiciones

Representan el conjunto de las proposiciones válidas en cada estado.

Se guardan en el atributo \texttt{y:NodeLabel}.
Este atributo es se encuentra dentro del atributo \texttt{data}, ya que no se encuentra
 especificado en el formato.

En caso de haber varias proposiciones, estas deben estar separadas por comas.

\end{itemize}

Además de esta información se debe indicar cuales son los estados iniciales.

Para las transiciones se debe guardar la siguiente información:
\begin{itemize}
\item Origen

Representa el estado de origen de la transición. Se guarda en el atributo \texttt{source}.

\item Destino

Representa el estado de destino de la transición. Se guarda en el atributo \texttt{target}.

\item Acción

Representa la acción que corresponde al cambio de estado.

Se guardan en el atributo \texttt{y:EdgeLabel}.

Este atributo es se encuentra dentro del atributo \texttt{data}, ya que no se encuentra
 especificado en el formato.

\end{itemize}

El parser de \textit{GraphML} a sistemas de transiciones se encuentra implementado en
 el objeto \texttt{ParserGraphML} del paquete \textit{Sistemas de transiciones}.

Como se mencionó anteriormente hay atributos que no están especificados en el formato y
 por lo tanto su interpretación depende del editor utilizado. Estos atributos son especificados
 en el parser mediante sub clases.
En este caso se utiliza el \textit{yEd Graph Editor}, para el cual fue implementado el objeto
 \texttt{ParserGraphML\_yEd}.
Este objeto es un parser de \textit{GraphML} que además interpreta la información dentro del
 atributo \texttt{data} como las etiquetas de los estados y transiciones.

A continuación se muestra un ejemplo de un sistema de transiciones y su correspondiente
 representación en \textit{GraphML}.

\paragraph{Ejemplo de sistema de transiciones en \textit{GraphML}} 




\section{Analizador sintáctico de fórmulas}
\label{sec:ply}
Para el análisis sintáctico de fórmulas se utilizó la herramienta \textit{Python Lex-Yacc}.

Para reconocer los elementos de un lenguaje, esta herramienta requiere la definición de
 sus elementos mediante expresiones regulares.
Luego se debe definir la sintaxis del lenguaje mediante una gramática libre de contexto.

A continuación se detallan los elementos y las gramáticas definidas para los lenguajes
 LTL y CTL.


\subsection{LTL}\label{cap:imp_ltl}
En la siguiente tabla se muestran los operadores de LTL con su correspondiente cadena de caracteres.

\begin{table}
\begin{center}
   \begin{tabular}{ | c | c | }
     \hline
     \textbf{Operador} & \textbf{Cadena de caracteres} \\ \hline
     $\lnot\bot$ & \texttt{TRUE} \\
     $\lnot$ & - \\
     $\land$ & \texttt{/$\backslash$} \\
     $\lor$ & \texttt{$\backslash$/} \\
     $\rightarrow$ & \texttt{-$>$} \\
     $\bigcirc$ & \texttt{O} \\
     $\square$ & \texttt{[]} \\
     $\lozenge$ & \texttt{$<>$} \\
     $\cup$ & \texttt{U} \\
     \hline
   \end{tabular}
\end{center}
\caption[Representación de operadores LTL]{Cadena de caracteres correspondiente a cada operador LTL.}
\end{table}

Además se definió la siguiente expresión regular para las proposiciones atómicas.
\[ [a-z\_ ][a-z0-9\_ ]* \]

\vbox{
Una vez definidos los elementos del lenguaje se definió la sintaxis mediante la siguiente gramática.

\begin{center}
   \begin{tabular}{ | r l | }
     \hline
     formula ::= & formula $\cup$ formula \\
     & $|$ formula $\land$ formula \\
	 & $|$ formula $\lor$ formula \\
	 & $|$ formula $\rightarrow$ formula \\
	 & $|$ $\bigcirc$ formula \\
	 & $|$ $\square$ formula \\
	 & $|$ $\lozenge$ formula \\
	 & $|$ $\lnot$ formula \\
	 & $|$ proposicion \\
	 & $|$ TRUE \\
     \hline
   \end{tabular}
\end{center}
}

Esta información se encuentra implementada en el archivo \texttt{parserLTL.py} del
 módulo \textit{LTL}.


\subsection{CTL}\label{cap:imp_ctl}

En la siguiente tabla se muestran los operadores de CTL con su correspondiente cadena de caracteres.

\begin{table}
\begin{center}
   \begin{tabular}{ | c | c | }
     \hline
     \textbf{Operador} & \textbf{Cadena de caracteres} \\ \hline
     $\lnot\bot$ & \texttt{TRUE} \\
     $\lnot$ & \texttt{-} \\
     $\land$ & \texttt{/$\backslash$} \\
     $\lor$ & \texttt{$\backslash$/} \\
     $\rightarrow$ & \texttt{-$>$} \\
     $\exists\bigcirc$ & \texttt{EO} \\
     $\forall\bigcirc$ & \texttt{AO} \\
     $\exists\square$ & \texttt{E[]} \\
     $\forall\square$ & \texttt{A[]} \\
     $\exists\lozenge$ & \texttt{E$<>$} \\
     $\forall\lozenge$ & \texttt{A$<>$} \\
     $\exists\cup$ & \texttt{EU} \\
     $\forall\cup$ & \texttt{AU} \\
     \hline
   \end{tabular}
\end{center}
\caption[Representación de operadores CTL]{Cadena de caracteres correspondiente a cada operador CTL.}
\end{table}

Al igual que en LTL se utilizó la siguiente expresión regular para las proposiciones
 atómicas.
\[ [a-z_][a-z0-9_]* \]

La gramática definida para CTL es la siguiente.

\begin{center}
   \begin{tabular}{ | r l | }
     \hline
     formula ::= & formula $\exists\cup$ formula \\
     & $|$ formula $\forall\cup$ formula \\
	 & $|$ formula $\land$ formula \\
	 & $|$ formula $\lor$ formula \\
	 & $|$ formula $\rightarrow$ formula \\
	 & $|$ $\exists\bigcirc$ formula \\
	 & $|$ $\forall\bigcirc$ formula \\
	 & $|$ $\exists\square$ formula \\
	 & $|$ $\forall\square$ formula \\
	 & $|$ $\exists\lozenge$ formula \\
	 & $|$ $\forall\lozenge$ formula \\
	 & $|$ $\lnot$ formula \\
	 & $|$ proposicion \\
	 & $|$ TRUE \\
     \hline
   \end{tabular}
\end{center}

Esta información se encuentra implementada en el archivo \texttt{parserCTL.py} del
 módulo \textit{CTL}.



\section{Casos de estudio}
En esta sección se muestra la utilización del verificador en dos casos de estudios distintos.
En estos casos de estudio se verifican propiedades expresadas en LTL y en CTL para distintos
sistemas modelados por sistemas de transiciones.

%\subsection{Luces de tránsito}
%A continuación se muestra el ejemplo de las luces de tránsito.
%Para este ejemplo se considera un sistema de transiciones con los estados \textit{red},
% \textit{yellow} y \textit{green} como se puede apreciar en el archivo \textit{GraphML}
% de la figura \ref{fig:luces.gml}.
%
%\begin{figure}[hbtp]
%\begin{center}
%\caption{Archivo \textit{GraphML} para sistema de transiciones de luces de tránsito}
%\fbox{\parbox{\textwidth}{\tiny\verbatiminput{mc/graphml/luces.graphml}}}
%\label{fig:luces.gml}
%\end{center}
%\end{figure}
%
%Se desea verificar que el sistema cumple con las siguientes propiedades expresadas en LTL:
%\begin{itemize}
%\item Cuando la luz está en rojo, la luz verde no debe ser la siguiente
%\[ \square (\text{red} \to \lnot \bigcirc \text{green}) \]
%El resultado de la verificación de esta propiedad es:
%
%\fbox{\parbox{\textwidth}{\scriptsize\verbatiminput{mc/salidas/salidaLTL_1.txt}}}
%
%\item Una vez que la luz está en verde, debe cambiar posteriormente a amarillo y quedarse en amarillo
% hasta cambiar a rojo
%\[ \square (\text{green} \to \bigcirc (\text{green} \cup (\text{yellow} \land \bigcirc (\text{yellow} \cup \text{red})))) \]
%El resultado de la verificación de esta propiedad es:
%
%\fbox{\parbox{\textwidth}{\scriptsize\verbatiminput{mc/salidas/salidaLTL_2.txt}}}
%
%\end{itemize}

\subsection{Mutua exclusión con árbitro}
A continuación se muestra el ejemplo de mutua exclusión con árbitro, visto en
 capítulos anteriores.
En la figura \ref{fig:arbitro_sincronizados} se muestra el sistemas de transiciones
 correspondiente.
El archivo en formato \textit{GraphML} para este caso de estudio se encuentra en la
 sección \ref{cap:graphml_arbitro} ubicada en los apéndices.


%\begin{figure}[hbtp]
%\begin{center}
%\caption{Archivo \textit{GraphML} para sistema de transiciones de mutua exclusión con árbitro}
%\fbox{\parbox{\textwidth}{\tiny\verbatiminput{mc/graphml/Handshaking.graphml}}}
%\label{fig:handshaking.gml}
%\end{center}
%\end{figure}

Se desea verificar que los sistemas en cuestión cumplen con las siguientes propiedades en CTL:
\begin{itemize}
\item Los procesos nunca acceden simultaneamente en la sección crítica
\[ \forall \square (\lnot \text{crit}_1 \lor \lnot \text{crit}_2) \]
El resultado de la verificación de esta propiedad es:

\fbox{\parbox{\textwidth}{\scriptsize\verbatiminput{mc/salidas/salidaCTL_1.txt}}}

\item Cada proceso accede infinitas veces a la sección crítica
\[ (\forall \square \forall \lozenge \text{crit}_1) \land (\forall \square \forall \lozenge \text{crit}_2) \]
El resultado de la verificación de esta propiedad es:

\fbox{\parbox{\textwidth}{\scriptsize\verbatiminput{mc/salidas/salidaCTL_2.txt}}}

\end{itemize}


\subsection{Ascensor}
A continuación se muestra el ejemplo del ascensor, también visto en
 capítulos anteriores.
En la figura \ref{fig:HS_ascensor} se muestra el sistemas de transiciones
 correspondiente.
El archivo en formato \textit{GraphML} para este caso de estudio se encuentra en la
 sección \ref{cap:graphml_ascensor} ubicada en los apéndices.

%\begin{figure}[hbtp]
%\begin{center}
%\caption{Archivo \textit{GraphML} para sistema de transiciones de ascensor con puertas}
%\fbox{\parbox{\textwidth}{\tiny\verbatiminput{mc/graphml/ascensor_1.graphml}}}
%\label{fig:handshaking.gml}
%\end{center}
%\end{figure}
%
%\begin{figure}[hbtp]
%\begin{center}
%%\caption{Archivo \textit{GraphML} para sistema de transiciones de ascensor con puertas (2)}
%\fbox{\parbox{\textwidth}{\tiny\verbatiminput{mc/graphml/ascensor_2.graphml}}}
%%\label{fig:handshaking.gml}
%\end{center}
%\end{figure}

Se desea verificar que los sistemas en cuestión cumplen con la siguiente propiedad en LTL:
\begin{itemize}
\item Si el ascensor está subiendo o bajando entonces ambas puertas están cerradas
\[ \square ((\text{subiendo} \lor \text{bajando}) \to (\text{arriba.cerrada} \land \text{abajo.cerrada})) \]
El resultado de la verificación de esta propiedad es:

\fbox{\parbox{\textwidth}{\scriptsize\verbatiminput{mc/salidas/salidaLTL_1.txt}}}

\end{itemize}


\section{Ejemplo de extensión}\label{cap:extension}
Como ya se mencionó anteriormente, uno de los principales objetivos de este verificador
 es la posibilidad de extensión para permitir verificar otro tipo de propiedades sobre
 sistemas reactivos.

En esta sección se muestra un ejemplo de extensión para verificar propiedades de
 tiempo real.
Para esto se introducen básicamente los principales conceptos y se describen las
 modificaciones necesarias para verificar este tipo de propiedades.
Esto incluye implementar un lenguaje que
 permita expresar las propiedades deseadas, pero además se necesita modificar la
 estructura de los sistemas de transiciones para modelar el tiempo y así poder expresar
 propiedades teniendo en cuenta el mismo.

\subsection{Sistemas de transiciones con tiempo}
En capítulos anteriores se trabajó con Sistemas de transiciones, pero al momento de verificar
 propiedades de tiempo real estos no permiten modelar el tiempo.
Para poder hacerlo se puede utilizar otro tipo de sistemas de transiciones.
Los sistemas de transiciones con tiempo \cite{henzinger} agregan un conjunto de relojes junto con un
 conjunto de operaciones de comparación y una operación de reinicio.

\begin{definicion}
Sistema de transiciones.\\
Un sistema de trancisiones consiste en una tupla $\text{TS} = (\text{S}, \text{I}, \text{Act}, \text{X}, \text{C}, {\to})$. Donde:
\begin{itemize}
\item $\text{S}$ es el conjunto de estados
\item $\text{I} \subseteq \text{S}$ es el conjunto de estados iniciales
\item $\text{Act}$ es el conjunto de acciones
\item $\text{X}$ es el conjunto de relojes
\item $\text{C}$ es el conjunto comparaciones sobre los relojes
\item ${\to} \subseteq \text{S} \times 2^X \times \text{Act} \times \text{S}$ es la relación de transición
\end{itemize}
\end{definicion}

La definición anterior introduce el conjunto $X$ como el conjunto de las variables
 que modelan los relojes. También introduce el conjunto ${C : S \to {\Phi (X)}}$ de
 comparaciones sobre los relojes. Donde
\[ \Phi (X) = x \leq c | x \geq c | x < c | x > c | \Phi_1 (X) \land \Phi_2 (X) \]
con $c \in \mathbb{R}$.

\paragraph{Ejemplo.} Sistema de transiciones con tiempo para una lámpara\\
Se considera el sistema de transiciones de la figura \ref{fig:lampara} para modelar una
 lámpara. Donde:
\begin{itemize}
\item $\text{S} = \{ \text{apagada}, \text{encendida}_1, \text{encendida}_2 \}$
\item $\text{I} = \{ \text{apagada} \}$
\item $\text{Act} = \{ \text{presionar} \}$
\item $\text{X} = \{ x \}$ 
\item $\text{C} = \{ x < 1 \}$
\item ${\to} = \{ (\text{apagada},\text{presionar},\text{encendida}_1), (\text{encendida}_1,\text{presionar},\text{encendida}_2), (\text{encendida}_1,\text{presionar},\text{apagada}), (\text{encendida}_2,\text{presionar},\text{apagada}) \}$
\end{itemize}

Esta lámpara tiene dos niveles de iluminación, 1 y 2. Para encender la lámpara en el
 modo 2 se debe presionar el botón de encendido dos veces en menos de un segundo.
 
\begin{figure}[hbtp]
\begin{center}
\caption{Sistema de transiciones con tiempo para una lámpara}
\includegraphics[width=0.5\textwidth]{mc/imagenes/tts.png}
\label{fig:lampara}
\end{center}
\end{figure}


\subsection{Lenguaje de tiempo}
Los lenguajes vistos en capítulos anteriores no permiten expresar propiedades de tiempo
 real, para esto se necesita un lenguaje que incluya variables de tiempo.
Una posibilidad es Lógica Temporal Proposicional de Tiempo (TPTL) \cite{alur}. Este lenguaje es una
 extensión de LTL que permite agregar comparaciones sobre los relojes, las mismas que
 utilizan los sistemas de transiciones con tiempo.
Las fórmulas de TPTL son construídas según la siguiente gramática
\[ \varphi ::= true | a | \varphi \wedge \varphi | \lnot \varphi | \bigcirc \varphi | \square \varphi | \lozenge \varphi | \varphi \cup \varphi | \Phi (X) \]
Donde
\[ \Phi (X) = x \leq c | x \geq c | x < c | x > c | \Phi_1 (X) \land \Phi_2 (X) \]
con $c \in \mathbb{R}$.

\paragraph{Ejemplo.} Propiedades de tiempo para una lámpara\\
A continuación se muestran posibles propiedades en TPTL sobre una lámpara.
\begin{itemize}
\item Si la lámpara está apagada y se presiona dos veces el botón
 de encendido en un tiempo mayor a un segundo, la lámpara se enciende apaga
\[ \lnot ((\text{apagada} \wedge \bigcirc \bigcirc (x \geq 1)) \wedge (\lnot \bigcirc \bigcirc \text{apagada})) \]
\item Si la lámpara está apagada y se presiona dos veces el botón
 de encendido en menos de un segundo, la lámpara se enciende en modo de iluminación 2
\[ \lnot ((\text{apagada} \wedge \bigcirc \bigcirc (x < 1)) \wedge (\lnot \bigcirc \bigcirc \text{encendida}_2)) \]
\end{itemize}


\subsection{Modificaciones en la implementación}
Para implementar los sistemas de transiciones con tiempo se debe agregar una subclase
 de \texttt{TS} con las operaciones necesarias para manipular los relojes.
Ademas se debe implementar una subclase de \texttt{TSTransicion} que incluya las
 guardas y los reinicios de relojes.

Por otro lado para agregar un lenguaje de propiedades es necesario implementar un nuevo
 módulo, incluyendo el parser con la definición de la gramática para este lenguaje.
Este módulo además debe tener una función \texttt{verificar()} que implemente el algoritmo
 de verificación para este lenguaje\footnote{El algoritmo de verificación no se introcuce
 en este documento.}.

\chapter{Conclusiones}
%
Intro .... que se hizo, puntos a favor, etc..

\section{Trabajo futuro}

El principal trabajo futuro sería ...

Sería muy útil contar con una funcionalidad de depuración, la cuál
mostrara dependiendo del tiempo los valores de cada fuente de eventos.

Una opción es comunicar mediante el puerto serial el valor de cada
señal al cambiar, y mostrarlo en una interfaz web como la que provee
RXMarbles (ver \cite{rxmarbles}). 
El lenguaje Elm provee de una herramienta que permite viajar en el 
tiempo, modificar y mostrar la ejecución de un programa, en nuestro
caso no sería posible modificar lo que el robot físico realiza, pero
si sería útil ver en la línea de tiempo que valores tomaron sus
señales. (ver \cite{elmdebug})



% Bibliografia
\cleardoublepage
\addcontentsline{toc}{chapter}{Bibliografía}
\begin{thebibliography}{99}
\bibitem{katoen} \emph{Principles of Model Checking}.\\
 Christel Baier y Joost-Pieter Katoen.\\
 The MIT Press, 2008.

% http://conal.net/papers/frp.html
%@InProceedings{ElliottHudak97:Fran,
%  title        = {Functional Reactive Animation},
%  url          = {http://conal.net/papers/icfp97/},
%  author       = "Conal Elliott and Paul Hudak",
%  booktitle    = "International Conference on Functional Programming",
%  year         = 1997
%}

% Yampa, Arrows and Robots.

%@InProceedings{Peterson99:LambdaInMotion,
%  author       = {John Peterson and Paul Hudak and Conal Elliott},
%  title        = {Lambda in Motion: Controlling Robots with {Haskell}},
%  url          = {http://haskell.org/frob/padl99/padl99.ps},
%  booktitle    = {Practical Aspects of Declarative Languages},
%  year         = 1999
%}
% http://cs.brown.edu/research/pubs/techreports/reports/CS-03-20.html

% Informal pero ayudo:
% https://gist.github.com/staltz/868e7e9bc2a7b8c1f754
%
% Rx, Bacon.js, 
%
% Para debug muy bueno: (con RX es este)
% https://github.com/jaredly/rxvision
% http://lambdor.net/?p=44

\bibitem{python} Sitio web de \emph{Python 2.7}:\\
 \url{http://docs.python.org/2/}\\
 Último acceso: 31/10/2013.

\bibitem{gml} Sitio web de \emph{GraphML File Format}:\\
 \url{http://graphml.graphdrawing.org/}\\
 Último acceso: 31/10/2013.

\end{thebibliography}


% Apendices
\appendix
\cleardoublepage
\addappheadtotoc
\appendixpage

%\chapter{Manual de usuario}
%

  Para utilizar el compilador, dado un archivo \textit{Ejemplo.willie}, se
ejecuta:

\begin{Verbatim}
> williec < Ejemplo.willie > Ejemplo.alf
\end{Verbatim}



El código de la máquina virtual está en el directorio
  /src/alfvm, para compilarlo se ejecuta:
\begin{verbatim}
  > cd src/alfvm
  > make
\end{verbatim}



%
%\chapter{Archivos \textit{GraphML}}
%Como se mencionó anteriormente se utilizó el formato de archivo \textit{GraphML} para
 representar los sistemas de transiciones.
Los estados son representados por el elemento \texttt{node} mientras que las transiciones
 son representadas por el elemento \texttt{edge}.

El formato \textit{GraphML} no establece como representar las etiquetas tanto en los nodos
 como en las aristas.
Para esto existe el atributo \texttt{data}, que proporciona flexibilidad
 para agregar atributos no contemplados por el formato.
Esto tiene un desventaja, y es que los atributos no contemplados por el formato no se
 representan de forma estándar, y por lo tanto su representación depende del editor
 utilizado.
En este caso el editor utilizado es \textit{yEd}.

Para los estados se guarda la siguiente información:
\begin{itemize}
\item Identificador

Este valor se guarda en el atributo \texttt{id} de cada nodo.
Es el identificador del estado, por lo que debe ser único.

Cuando se genera un sistema de transiciones mediante el verificador este genera los identificadores
 de cada estado automáticamente.

\item Proposiciones

Representan el conjunto de las proposiciones válidas en cada estado.

Se guardan en el atributo \texttt{y:NodeLabel}.
Este atributo es se encuentra dentro del atributo \texttt{data}, ya que no se encuentra
 especificado en el formato.

En caso de haber varias proposiciones, estas deben estar separadas por comas.

\end{itemize}

Además de esta información se debe indicar cuales son los estados iniciales.

Para las transiciones se debe guardar la siguiente información:
\begin{itemize}
\item Origen

Representa el estado de origen de la transición. Se guarda en el atributo \texttt{source}.

\item Destino

Representa el estado de destino de la transición. Se guarda en el atributo \texttt{target}.

\item Acción

Representa la acción que corresponde al cambio de estado.

Se guardan en el atributo \texttt{y:EdgeLabel}.

Este atributo es se encuentra dentro del atributo \texttt{data}, ya que no se encuentra
 especificado en el formato.

\end{itemize}

El parser de \textit{GraphML} a sistemas de transiciones se encuentra implementado en
 el objeto \texttt{ParserGraphML} del paquete \textit{Sistemas de transiciones}.

Como se mencionó anteriormente hay atributos que no están especificados en el formato y
 por lo tanto su interpretación depende del editor utilizado. Estos atributos son especificados
 en el parser mediante sub clases.
En este caso se utiliza el \textit{yEd Graph Editor}, para el cual fue implementado el objeto
 \texttt{ParserGraphML\_yEd}.
Este objeto es un parser de \textit{GraphML} que además interpreta la información dentro del
 atributo \texttt{data} como las etiquetas de los estados y transiciones.

A continuación se muestra un ejemplo de un sistema de transiciones y su correspondiente
 representación en \textit{GraphML}.

\paragraph{Ejemplo de sistema de transiciones en \textit{GraphML}} 





% Termina el documento
\end{document}
