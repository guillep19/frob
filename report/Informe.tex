% Definimos el estilo del documento
\documentclass[12pt,a4paper,spanish]{book}

\usepackage[utf8]{inputenc}
\usepackage[english,spanish]{babel}
\usepackage{ amssymb }
\usepackage{ amsmath }
\usepackage{ amsthm }
\usepackage{ wasysym }
\usepackage{ verbatim }
\usepackage{ fancyvrb }
\usepackage{ url }
\usepackage[table]{xcolor}% http://ctan.org/pkg/xcolor
\usepackage{ tikz }
\usepackage{ csvsimple }
\usetikzlibrary{shapes.gates.logic.US,trees,positioning,arrows,shapes.multipart,snakes}


\usepackage{ appendix }
\renewcommand{\appendixname}{Apéndices}
\renewcommand{\appendixtocname}{Apéndices}
\renewcommand{\appendixpagename}{Apéndices}
\newcommand{\frob}{Willie}
\newcommand{\alf}{Alf}
\newcommand{\compilador}{\texttt{williec}}
\newcommand{\maquinavirtual}{Alfvm}
 
%\usepackage[dvips]{graphicx}
%\DeclareGraphicsExtensions{.pdf,.png,.jpg}

\usepackage{graphicx}

\usepackage{hyperref}
\hypersetup{colorlinks=false}
\usepackage{array}
\setlength\extrarowheight{2pt}
\usepackage{proof}

\newtheorem{definicion}{Definición}

%Empieza el documento
\begin{document}
\selectlanguage{spanish}

% Definimos titulo, autor, fecha.
\title{
  \frob: Programación funcional reactiva para robots con bajas capacidades de cómputo\\
    {\small Proyecto de grado, Facultad de Ingeniería, Universidad de la República}\\
  \begin{center}
    {\includegraphics[width=40mm]{fing.jpg}}
    {\includegraphics[width=40mm]{udelar.jpg}}
  \end{center}
}
\author{Guillermo Pacheco\\
{\small Tutores: Marcos Viera, Jorge Visca, Andrés Aguirre}}
\date{\today}

% Generamos titulo e indice de contenidos
\maketitle


% Resumen
\chapter*{Resumen}
\addcontentsline{toc}{chapter}{Resumen}
\markboth{RESUMEN}{RESUMEN}

\section{Resumen}

Aca va un resumen breve e introducción al documento.



% Indice general
\cleardoublepage
\addcontentsline{toc}{chapter}{Índice general} % para que aparezca en el indice de contenidos
\tableofcontents

% Indice de figuras
\cleardoublepage
\addcontentsline{toc}{chapter}{Índice de figuras} % para que aparezca en el indice de contenidos
\listoffigures

% indice de tablas
\cleardoublepage
\addcontentsline{toc}{chapter}{Índice de tablas} % para que aparezca en el indice de contenidos
\listoftables

\chapter{Introducción}

% TODO(Marcos): Decir por que se decidio definir un lenguaje propio y no
% utilizar los que se estudiaron anteriormente.
% Tambien decir que comparte con esos lenguajes y en que difiere.

  En este capítulo se describe el diseño del lenguaje \frob{} junto con su semántica.
  Luego se explica de qué manera es traducido al lenguaje \alf{} de
bajo nivel, mas simple de interpretar, el cual podrá ser interpretado
por implementaciones de una misma máquina virtual en diferentes
plataformas de hardware.

  También se describirán las etapas de compilación, desde que se escribe
un programa en alto nivel hasta que es ejecutado en una
plataforma objetivo.

  El diagrama de la Figura \ref{fig:compilacion} resume todas las etapas y
componentes necesarios.

\begin{figure}[h]
\begin{center}
\caption{Etapas y componentes}
\includegraphics[width=0.9\textwidth]{graphs/compilacion.png}
\label{fig:compilacion}
\end{center}
\end{figure}

  En la parte de arriba de la Figura se ve un programa en el
lenguaje \frob{} de alto nivel.
  El desarrollador escribe dicho programa y
ejecuta el compilador \compilador{} y obtiene un archivo \alf{} binario.

  Debajo se muestran diversas plataformas, cada una
con su implementación de la máquina virtual \maquinavirtual{} instalada.

  El desarrollador podrá cargar el mismo código \alf{} en cualquier robot que
esté construido utilizando cualquiera de las plataformas, y la máquina
virtual se encargará de interpretarlo.


\chapter{Programación Funcional Reactiva}




\section{Programación Funcional Reactiva}

Acá voy a hablar de programación funcional reactiva.

Tradicionalmente los programas son formados a partir de
una secuencia de acciones imperativas. Los programas
reactivos suelen formarse por eventos y código iterativo
que se corre cuando un evento ocurre.

Dicho código iterativo suele hacer referencia y manipular
variables compartidas con diferentes rutinas. Ésto lleva a que
como un valor puede ser manipulado desde diferentes lugares,
puedan producirse problemas de concurrencia y algunos valores
pueden quedar en un estado inconsistente.

En el paradigma FRP no existen valores compartidos, sinó que
dichos valores dependientes del tiempo, tienen una representación
llamada Comportamiento y la única forma de modificarlos, es
a partir de como fueron definidos.

 

\begin{definicion}
Comportamientos (Behaviours).\\
Un comportamiento es un valor contínuo que depende del paso del tiempo.
Los comportamientos se pueden definir, combinar, pasarlos como
argumentos a funciones, retornarlos.
Un comportamiento puede ser un valor constante, el tiempo mismo (un reloj),
o puede formarse combinando otros comportamientos, por ejemplo secuencialmente
o paralelamente.
\end{definicion}

\begin{definicion}
Eventos (Streams).\\
Son valores discretos dependientes del tiempo, que forman
una secuencia finita o infinita de ocurrencias. Cada ocurrencia
está formada por el valor y el instante de tiempo.
\end{definicion}

La principal diferencia entre Comportamientos y Eventos, es que los
comportamientos son valores contínuos y los eventos son discretos.

Los comportamientos representan cualquier valor en función del tiempo,
por ejemplo:
\begin{itemize}
\item \textit{entrada} sensor de distancia, temperatura, video
\item \textit{salida} velocidad, voltaje
\item \textit{valores} intermedios calculados
\end{itemize}

Las operaciones que se pueden realizar sobre los comportamientos incluyen:
\begin{itemize}
\item \textit{Operaciones genéricas} Aritmética, integración, diferenciación
\item \textit{Operaciones específicas de un dominio} como escalar video, aplicar filtros, detección de patrones.
\end{itemize}

Los eventos pueden ser sensores específicos de un dominio, por ejemplo un
botón, un click, una interrupción o mensaje asincrónico.
También puede ser generado a partir de valores de un comportamiento,
como ser \emph{Temperatura alta}, \emph{Batería baja}.

Las operaciones que se pueden realizar sobre los eventos incluyen:
\begin{itemize}
\item fmap, filter
\item Modificar un \emph{Comportamiento} reactivo
\end{itemize}


\section{Ejemplo}

Para entender un poco más las ideas de Comportamiento y Evento, se puede
plantear el siguiente ejemplo.

  En una cuenta de un banco, el saldo se puede definir
como un comportamiento, el cuál solo se modifica cuando ocurre
un movimiento.
  Un movimiento puede ser depositar dinero o extraer
dinero de la cuenta.
  Un dato importante a ver, es que no es posible asignar un valor
sin que sea por medio de su propia definición, por lo que nadie
podría realizar la asignación $saldo = 1000000$.
  La misma operación sería posible creando un movimiento, el cuál
afectaría al saldo.

  Usaremos la notación $<nombre> :: Event <tipo>$ para definir
eventos y la notación $<nombre> :: Behaviour <tipo>$ para
definir comportamientos.

\begin{verbatim}
movimiento :: Event Number
movimiento = read(0)

saldo :: Behaviour Number
saldo = movimiento

alert :: Event Bool
alert = saldo > 1000

on alert:
  write(0, 'El saldo es mayor a 1000')

\end{verbatim}

Un comportamiento, solo se modifica cuando cambian sus componentes,





%state = State(state, [input])
%Output(state)


\chapter{Plataformas de Hardware}

En esta sección se describen las plataformas de hardware relevadas
durante el estado del arte, junto con sus características.

\section{Arduino}

  Arduino \cite{arduino} es una plataforma abierta de prototipado, basada en
software y hardware flexible fácil de usar.
  Está pensada para ser usada por diseñadores, artistas, como
hobby, para crear objetos y ambientes interactivos.
  Entre sus productos, hay placas y kits de componentes.
  Los kits de arduino generalmente tienen interfaz usb con soporte
para programarlo usando la propia placa sin necesidad de un
programador por hardware.

  También los pines de entrada/salida del microprocesador
están diseñados para poder colocar fácilmente cables y
conectar periféricos sin necesidad de soldar.

  También incluyen leds y botones para resetear la placa o
utilizarlos como sensor.

  Existe un entorno de desarrollo integrado (IDE) que utiliza
una implementación del compilador \texttt{gcc} \cite{gcc} para la arquitectura
\texttt{avr} \cite{avr} de \textit{Atmel} \cite{atmel} y puede ser utilizado
para programar sobre los kits.

  Variedad de bibliotecas y abstracciones de sensores, actuadores y
protocolos de comunicación, ya están implementados y pueden ser
usados en los kits.
  Al ser un proyecto libre las bibliotecas son publicadas y mantenidas
por una comunidad abierta.

  La arquitectura usada por casi todos los kits
es \texttt{avr} \texttt{Atmel} pero existen algunos con
arquitectura \texttt{ARM} \cite{arm}.

  El lenguaje estándar para desarrollar programas se llama \texttt{Arduino},
sin embargo el lenguaje es \texttt{C/C++}, cambiando la forma en que
se invoca el programa principal y con algunas funciones y
formato predefinido.

  La Tabla \ref{table-arduino} muestra un listado de los modelos
de Arduino, cuánta memoria persistente tienen (Flash) en kilobytes, con
cuánta memoria RAM cuentan, cuánta memoria EEPROM tienen en kilobytes,
que procesador tienen y a que frecuencia funcionan.

  Salvo el modelo \texttt{Due}, el resto utilizan la
arquitectura \texttt{avr} de 8 bit. La memoria ram varía entre
16 y 512 kilobytes.
  Los modelos más populares y representativos, son el Arduino \texttt{Uno}
y el Arduino \texttt{Nano 328}, ambos con 32 KB de memoria Flash, 2 KB de memoria
RAM (SRAM) y procesador \texttt{ATmega328} a 16 MHz.

\begin{table}[htbp]
\centering
\scriptsize
\setlength\tabcolsep{2pt}
\caption{Modelos arduino}
\label{table-arduino}
\begin{tabular}{|c|c|c|c|c|c|c|}
  \hline
  Modelo & Flash & SRAM (kb) & EEPROM (kb) & Procesador & Arquitectura & Frecuencia \\
  \hline
  Uno & 32 KB & 2 & 1 & ATmega328 & 8 bit AVR & 16 MHz \\
  \hline
  Leonardo & 32 KB & 2.5 & 1 & ATmega32u4 & 8 bit AVR & 16 MHz \\
  \hline
  Due & 512 KB & 96 & - & AT91SAM3X8E & ARM Cortex-M3 & 84 Mhz \\
  \hline
  Yun & 32 KB & 2.5 & 1 & ATmega32u4 & 8 bit AVR & 16 MHz \\
  \hline
  Tre & 32 KB & 2.5 & 1 & ATmega32u4 & 8 bit AVR & 16 MHz \\
  \hline
  Micro & 32 KB & 2.5 & - & ATmega32u4 & 8 bit AVR & 16 MHz \\
  \hline
  Robot & 32 KB & 2.5 & 1 & ATmega32u4 & 8 bit AVR & 16 MHz \\
  \hline
  Esplora & 32 KB & 2.5 & 1 & ATmega32u4 & 8 bit AVR & 16 MHz \\
  \hline
  Mega ADK & 256 KB & 8 & 4 & ATmega2560 & 8 bit AVR & 16 MHz \\
  \hline
  Ethernet & 32 KB & 2 & 1 & ATmega328 & 8 bit AVR & 16 MHz \\
  \hline
Mega 2560 & 256 KB & 8 & 4 & ATmega2560 & 8 bit AVR & 16 MHz \\
  \hline
  Mini & 32 KB & 2 & 1 & ATmega328 & 8 bit AVR & 16 MHz \\
  \hline
  LilyPad USB & 32 KB & 2.5 & 1 & ATmega32u4 & 8 bit AVR & 8 Mhz \\
  \hline
  LilyPad Simple & 32 KB & 2 & 1 & ATmega328 & 8 bit AVR & 8 Mhz \\
  \hline
  LilyPad (168V) & 16 KB & 1 & 512 Bytes & ATmega168V & 8 bit AVR & 8 Mhz \\
  \hline
  LilyPad (328V) & 16 KB & 1 & 512 Bytes & ATmega328V & 8 bit AVR & 8 Mhz \\
  \hline
  Nano (168) & 16 KB & 1 & 512 Bytes & ATmega168 & 8 bit AVR & 16 MHz \\
  \hline
  Nano (328) & 32 KB & 2 & 1 & ATmega328 & 8 bit AVR & 16 MHz \\
  \hline
  Pro mini (3.3v) & 16 KB & 1 & 512 Bytes & ATmega168 & 8 bit AVR & 8 Mhz \\
  \hline
  Pro mini (5v) & 16 KB & 1 & 512 Bytes & ATmega168 & 8 bit AVR & 16 MHz \\
  \hline
  Pro (168) & 16 KB & 1 & 512 Bytes & ATmega168 & 8 bit AVR & 8 Mhz \\
  \hline
  Pro (328) & 32 KB & 2 & 1 & ATmega328 & 8 bit AVR & 16 MHz \\
  \hline
  Fio & 32 KB & 2 & 1 & ATmega328P & 8 bit AVR & 8 Mhz \\
  \hline 
\end{tabular}
\end{table}

%%\subsection{Arduino Uno}
%%Web: http://arduino.cc/en/Main/ArduinoBoardUno
%%Microcontrolador: ATmega328
%%Características generales:
%%Es una placa basada en el microcontrolador ATmega328.
%%0.5 kb de la memoria flash son utilizados por bootloader.
%%Existe una placa construida en uruguay llamada Urduino328,
%%la cuál es compatible con la Arduino Uno y tiene un costo aproximado de 50 dólares.
%%
%%\subsection{Arduino Leonardo}
%%Web:
%%http://arduino.cc/en/Main/ArduinoBoardLeonardo
%%Características generales:
%%Es una placa basada en el microcontrolador ATmega32u4. Tiene 20 pins de entrada/salida digitales, frecuencia de 16 MHz y conección micro USB.
%%La diferencia principal con otras placas es que el microcontrolador permite la comunicación usb sin necesidad de un microcontrolador secundario que la implemente.
%%Un bootloader es incluído, el cuál se puede utilizar para programar la placa sin un programador por hardware. Éste bootloader ocupa 4 kb de la memoria Flash del microcontrolador, puede ser eliminado pero teniendo en cuenta que luego no se cuenta con su funcionalidad.
%%Microcontrolador:
%%ATmega32u4
%%
%%\subsection{Arduino Due}
%%Web:
%%http://arduino.cc/en/Main/ArduinoBoardDue
%%Características:
%%Es la primer placa arduino basada en la arquitectura ARM de 32 bits.
%%Tiene 54 pins de entrada/salida digital, 12 de los cuáles pueden ser usados como salidas PWM.
%%12 entradas analógicas, un reloj de 84 MHz integrado, conección USB, 2 convertidores digital-analógico.
%%
%%Un botón de reset y un botón de borrado.
%%A diferencia de otras placas, ésta placa corre con un voltaje de 3.3V.
%%La mejora sustancial con respecto a otras placas, puede ejecutar operaciones sobre 4 bytes en un sólo ciclo de reloj,
%%tiene una frecuencia alta de reloj, 96 kbytes de SRAM, 512 kb de memoria flash para código y
%%un controlador DMA para liberar el CPU de tareas basadas en muchos accesos a memoria.
%%
%%El bootloader que incluye viene de fábrica y está en una ROM dedicada, por lo que no ocupa espacio de la memoria Flash.
%%
%%Microcontrolador: Atmel SAM3X8E ARM Cortex-M3
%%
%%\subsection{Arduino Robot}
%%Web:
%%http://arduino.cc/en/Main/Robot
%%Características:
%%Es el primer Arduino sobre ruedas oficial.
%%Cuenta con dos procesadores, cada uno sobre una placa.
%%Hay una placa utilizada para controlar los motores, y otra placa de control que maneja los
%%sensores y decide como operar.
%%Cada placa se puede programar por separado usando el IDE Arduino.
%%Algunos de los pines de la placa ya están mapeados a sensores y actuadores.
%%  El chasis cuenta con una brújula, un parlante, un panel de control de 5 botones, leds, conexiones I2C,
%%dos ruedas y sensores infrarrojos.
%%  También tiene zonas de prototipado.
%%
%%Microcontrolador: 2 microcontroladores ATmega32u4


\section{Mbed}

  Mbed, al igual que Arduino es una plataforma abierta de prototipado,
su objetivo es que se puedan desarrollar prototipos en un tiempo corto.

  Cuenta con herramientas colaborativas como ser un entorno de
  desarrollo integrado (IDE) web, interfaz web de control de versiones,
donde se pueden publicar proyectos, extender y colaborar con
proyectos de otros usuarios.

  Existe una gran variedad de bibliotecas desarrolladas para los
kits Mbed, que al igual que en Arduino, implementan funcionalidades
básicas como ser protocolos de comunicación e interacción con
componentes externos.

  La arquitectura usada por Mbed es \texttt{ARM}, principalmente
\texttt{ARM Cortex-M3} y \texttt{ARM Cortex-M0}.

  El compilador web es práctico para colaborar con otros
usuarios y no tener que armar un entorno local.
  Las aplicaciones pueden ser cargadas en las placas usando el
entorno web sin necesitar instalación del compilador.

  El lenguaje utilizado es \texttt{C/C++} con bibliotecas especializadas
de Mbed.
  Mbed también cuenta con un HDK (Hardware development kit) para diseño de
hardware especializado, luego de prototipar.

  En la Tabla \ref{table-mbed} se muestra un listado con los modelos
más relevantes y sus características.
  La memoria RAM varía entre 16 kilobytes y 1 megabyte, bastante similar
a Arduino. Sin embargo, el modelo más popular es el \texttt{LPC1768}
con 512 KB de Flash y 64 KB de memoria RAM.
  Éste modelo será utilizado para la implementación, al no ser tan
reducido, se puede crear una implementación modelo para el mismo, y luego
evaluar si es posible reducir el tamaño para trabajar con modelos
con menor capacidad de cómputo.

\begin{table}[htbp]
  \centering
  \scriptsize
  \caption{Modelos Mbed}
  \label{table-mbed}
  \begin{tabular}{|c|c|c|c|c|}
  \hline
    Modelo & Flash & RAM (KB) & Procesador & Frecuencia \\
  \hline
    NXP LPC1768 & 512 KB & 64 (sram) & ARM Cortex-M3 & 96 MHz \\
  \hline
    NXP LPC11U24 & 32 KB & 8 & ARM Cortex-M0 & 48 MHz \\
  \hline
    Freescale FRDM-KL25Z & 128 & 16 KB & ARM Cortex-M0+ & 48 MHz \\
  \hline
    NXP LPC800-MAX & 16 KB & 4 & ARM Cortex-M0+ & 30 MHz \\
  \hline
    NXP EA LPC4088 & 512 KB & 96 (sram) & ARM Cortex-M4 & 120 MHz \\
  \hline
    NXP DipCortex M0 & 32 KB & 8 & ARM Cortex-M0 & 50 MHz \\
  \hline
    NXP DipCortex M3 & 64 KB & 12 & ARM Cortex-M3 & 72 MHz \\
  \hline
    NXP BlueBoard-LPC11U24 & 32 KB & 8 & ARM Cortex-M0 & 48 MHz \\
  \hline
    NXP WiFi DipCortex & 64 KB & 12 & ARM Cortex-M3 & 72 MHz \\
  \hline
    NXP Seeeduino-Arch & 32 KB & 8 & ARM Cortex-M0 & 48 MHz \\
  \hline
    NXP mbed LPC1114FN28 & 32 KB & 4 & ARM Cortex-M0 & 50 MHz \\
  \hline
    Ublox U-blox C027 & 512 KB & 32 & ARM Cortex-M3 & 96 MHz \\
  \hline
    NXP EA LPC11U35 & 64 KB & 10 & ARM Cortex-M0 & 48 MHz \\
  \hline
    ST Nucleo F103RB & 128 KB & 20 (sram) & ARM Cortex-M3 & 72 MHz \\
  \hline
    Freescale FRDM-KL46Z & 256 KB & 32 & ARM Cortex-M0+ & 48 MHz \\
  \hline
    NXP Seeeduino-Arch-Pro & 512 KB & 32 & ARM Cortex-M3 & 96 MHz \\
  \hline
    ST Nucleo F302R8 & 64 KB & 16 (sram) & ARM Cortex-M4 & 72 MHz \\
  \hline
    ST Nucleo L152RE & 512 KB & 80 (sram) & ARM Cortex-M3 & 32 MHz \\
  \hline
    ST Nucleo F401RE & 512 KB & 96 (sram) & ARM Cortex-M4 & 84 MHz \\
  \hline
    ST Nucleo F030R8 & 64 KB & 8 (sram) & ARM Cortex-M0 & 48 MHz \\
  \hline
    Freescale FRDM-K64F & 1 MB & 256 & ARM Cortex-M4 & 120 MHz \\
  \hline
    Nordic nRF51822 & 128 KB & 16 & ARM Cortex-M0 & 16 MHz \\
  \hline
    FRDM-KL05Z & 32 KB & 4 & ARM Cortex-M0+ & 48 MHz \\
  \hline
    LPCXpresso1549 & 256 KB & 36 & ARM Cortex-M3 & 72 MHz \\
  \hline
    LPCXpresso11U68 & 256 KB & 36 & ARM Cortex-M0+ & 50 MHz \\
  \hline
  \end{tabular}
\end{table}


%\subsection{NXP LPC11U24}
%
%Homepage:
%https://mbed.org/handbook/mbed-NXP-LPC11U24
%Características:
%Diseñada para prototipado rápido, programador USB integrado, aplicaciones de bajo consumo eléctrico.
%8KB RAM, 32KB FLASH
%USB, 2xSPI, I2C, UART, 6xADC, GPIO 
%Procesador:
%32-bit ARM Cortex-M0, 48MHz

%\subsection{NXP LPC1768}
%
%Homepage:
%http://mbed.org/platforms/mbed-LPC1768/
%Características generales:
%Diseñada para prototipado rápido de aplicaciones de microcontroladores en general, Ethernet, USB.
%Tiene flexibilidad para varios periféricos y memoria FLASH.
%Ethernet, USB Host y Device, 2xSPI, 2xI2C, 3xUART, CAN, 6xPWM, 6xADC, GPIO.
%40 pines de entrada/salida.
%Interfaz de programación por USB integrada (drag and drop programmer)
%Procesador:
%32-bit ARM Cortex-M3, 96MHz
%Memoria:
%512kb FLASH, 32kb RAM



\section{Robotis}

  La empresa \texttt{Robotis} desarrolla robots para uso
educativo, así como una gama de robots para uso competitivo.
  Los kits de Robotis están diseñados para uso final, es decir,
proveen los controladores, así como los componentes para armar
la estructura, sensores y actuadores.

  Para el uso de los kits se deben utilizar las herramientas de desarrollo
de Robotis.

  No es un proyecto abierto, por lo que no puede ser fácilmente
extendido, ni modificado.
  Es posible programar utilizando \texttt{C} embebido, descargando
los archivos fuente para sus plataformas, aunque es un proceso
bastante complejo y no cuenta con buena documentación.

  Robotis cuenta con un lenguaje llamado \texttt{Task} y se necesita
  un entorno de desarrollo integrado propietario llamado 
  \texttt{IDE Roboplus} para utilizarlo.
  En el \texttt{IDE} se pueden generar tareas y programar
  movimientos del robot en base a movimiento de motores y un
  diseño tridimensional gráfico.

  Los microcontroladores utilizados se llaman \texttt{CM-510},
  \texttt{CM-530} y \texttt{CM-100A}. Algunos con
  arquitectura AVR y otros con ARM internamente.

  El \texttt{CM-510} contiene un microprocesador \texttt{ATMega128},
  el \texttt{CM-530} un \texttt{ARM Cortex-M3} y el
  \texttt{CM-100A} un controlador {ATMega8}.

  En la Tabla \ref{table-robotis} se pueden ver las especificaciones
  técnicas de los diferentes kits de \textit{Robotis}.

\begin{table}[htbp]
    \centering
    \scriptsize
    \caption{Modelos Robotis}
    \label{table-robotis}
  \begin{tabular}{|c|c|c|c|c|c|c|}
    \hline
      Modelo & Flash & RAM (kb) & EEPROM (kb) & Procesador & Arquitectura & Frecuencia \\
    \hline
      CM-100A & 8 KB & 1 (sram) & 512 & ATmega8 & 8 bit AVR & 16 MHz \\
    \hline
      CM-5 & 128 KB & 4 (sram) & 4 & ATmega128 & 8 bit AVR & 16 MHz \\
    \hline
      CM-510 & 256 KB & 8 (sram) & 8 & ATmega2561 & 8 bit AVR & 16 MHz \\
    \hline
      CM-530 & 512 KB & 64 & - & STM32F103RE & ARM Cortex-M3 & 72 MHz \\
    \hline
      CM-700 & 256 KB & 8 (sram) & 8 & ATMega2561 & 8 bit AVR & 16 MHz \\
    \hline
      CM-730 & 512 KB & 64 & - & STM32F103RE & ARM Cortex-M3 & 72 MHz \\
    \hline
      CM-900 & 64 KB & 20 (sram) & - & STM32F103C8 & ARM Cortex-M3 & 72 MHz \\
    \hline
    \end{tabular}
  \end{table}

\subsection{Bioloid STEM}
  Creado para uso educativo y competencias robóticas.
  El kit provee el hardware y clases enseñando a construir distintos
  robots para distintos usos, involucrando conceptos de ciencias,
  tecnología, ingeniería y matemáticas.

  Utiliza el controlador \texttt{CM-530} internamente.

  Cuenta con un conjunto de componentes que son:

  \begin{itemize}
  \item Sensor Infrarrojo
  \item Array de 7 sensores infrarrojos (detectan objetos)
  \item Control remoto y receptor
  \item 6 motores dinamixel
  \item Piezas para crear estructura de un robot
  \end{itemize}

\subsection{Bioloid Premium}
  Diseñado para educación, competiciones y entretenimiento.
  Se pueden construir variedad de robots como humanoide y animales.
  El kit contiene 29 ejemplos de robot y programas de ejemplo.
  Utiliza el controlador \texttt{CM-530}.

  Incluye los siguientes componentes:

  \begin{itemize}
  \item 18 motores dinamixel (AX-12A)
  \item Sensor giroscópico
  \item Receptor infrarrojo
  \item Control remoto y receptor
  \item Sensor de distancia
  \item Sensor infrarrojo para detección de objetos.
  \item Piezas para crear estructura de robot.
  \end{itemize}

\subsection{Bioloid GP}
  Humanoide optimizado para competencias robóticas.
  Esqueleto liviano y resistente.
  Instrucciones para jugar al fútbol y hacer tareas de recolección
  pre-programadas.
  Ajuste automático de postura con sensor giroscópico.

  Utiliza el controlador \texttt{CM-530}.

  Cuenta con los siguientes componentes:

  \begin{itemize}
    \item 18 motores dinamixel
    \item Sensor giroscópico
    \item Control remoto y receptor
    \item Sensor de distancia
    \item Piezas de aluminio.
  \end{itemize}



\section{Lego Mindstorms}

\section{Lego}


Kits Lego Mindstorms
Kit:
NXT Intelligent brick
Webpage:
http://shop.lego.com/en-US/NXT-Intelligent-Brick-9841
Procesador:
32-bit ARM7 microprocessor
Interfaces:
Support for Bluetooth, 1 USB 2.0 port, 4 input ports, 3 output ports.
Precio estimado:
US$ 149.99



\section{Fischertechnik}

\subsection{ROBO TX Controller}
Webpage:
http://www.fischertechnik.de/en/Home/info/computing/ROBO-TX-Controller.aspx/usetemplate-1\_column\_no\_pano/
Procesador:
32-bit processor, 200 MHz.


\section{Butiá}

  Arduino \cite{arduino} es una plataforma abierta de prototipado, basada en
software y hardware flexible fácil de usar.
  Está pensada para ser usada por diseñadores, artistas, como
hobby, para crear objetos y ambientes interactivos.
  Entre sus productos, hay placas y kits de componentes.
  Los kits de arduino generalmente tienen interfaz usb con soporte
para programarlo usando la propia placa sin necesidad de un
programador por hardware.

  También los pines de entrada/salida del microprocesador
están diseñados para poder colocar fácilmente cables y
conectar periféricos sin necesidad de soldar.

  También incluyen leds y botones para resetear la placa o
utilizarlos como sensor.

  Existe un entorno de desarrollo integrado (IDE) que utiliza
una implementación del compilador \texttt{gcc} \cite{gcc} para la arquitectura
\texttt{avr} \cite{avr} de \textit{Atmel} \cite{atmel} y puede ser utilizado
para programar sobre los kits.

  Variedad de bibliotecas y abstracciones de sensores, actuadores y
protocolos de comunicación, ya están implementados y pueden ser
usados en los kits.
  Al ser un proyecto libre las bibliotecas son publicadas y mantenidas
por una comunidad abierta.

  La arquitectura usada por casi todos los kits
es \texttt{avr} \texttt{Atmel} pero existen algunos con
arquitectura \texttt{ARM} \cite{arm}.

  El lenguaje estándar para desarrollar programas se llama \texttt{Arduino},
sin embargo el lenguaje es \texttt{C/C++}, cambiando la forma en que
se invoca el programa principal y con algunas funciones y
formato predefinido.

  La Tabla \ref{table-arduino} muestra un listado de los modelos
de Arduino, cuánta memoria persistente tienen (Flash) en kilobytes, con
cuánta memoria RAM cuentan, cuánta memoria EEPROM tienen en kilobytes,
que procesador tienen y a que frecuencia funcionan.

  Salvo el modelo \texttt{Due}, el resto utilizan la
arquitectura \texttt{avr} de 8 bit. La memoria ram varía entre
16 y 512 kilobytes.
  Los modelos más populares y representativos, son el Arduino \texttt{Uno}
y el Arduino \texttt{Nano 328}, ambos con 32 KB de memoria Flash, 2 KB de memoria
RAM (SRAM) y procesador \texttt{ATmega328} a 16 MHz.

\begin{table}[htbp]
\centering
\scriptsize
\setlength\tabcolsep{2pt}
\caption{Modelos arduino}
\label{table-arduino}
\begin{tabular}{|c|c|c|c|c|c|c|}
  \hline
  Modelo & Flash & SRAM (kb) & EEPROM (kb) & Procesador & Arquitectura & Frecuencia \\
  \hline
  Uno & 32 KB & 2 & 1 & ATmega328 & 8 bit AVR & 16 MHz \\
  \hline
  Leonardo & 32 KB & 2.5 & 1 & ATmega32u4 & 8 bit AVR & 16 MHz \\
  \hline
  Due & 512 KB & 96 & - & AT91SAM3X8E & ARM Cortex-M3 & 84 Mhz \\
  \hline
  Yun & 32 KB & 2.5 & 1 & ATmega32u4 & 8 bit AVR & 16 MHz \\
  \hline
  Tre & 32 KB & 2.5 & 1 & ATmega32u4 & 8 bit AVR & 16 MHz \\
  \hline
  Micro & 32 KB & 2.5 & - & ATmega32u4 & 8 bit AVR & 16 MHz \\
  \hline
  Robot & 32 KB & 2.5 & 1 & ATmega32u4 & 8 bit AVR & 16 MHz \\
  \hline
  Esplora & 32 KB & 2.5 & 1 & ATmega32u4 & 8 bit AVR & 16 MHz \\
  \hline
  Mega ADK & 256 KB & 8 & 4 & ATmega2560 & 8 bit AVR & 16 MHz \\
  \hline
  Ethernet & 32 KB & 2 & 1 & ATmega328 & 8 bit AVR & 16 MHz \\
  \hline
Mega 2560 & 256 KB & 8 & 4 & ATmega2560 & 8 bit AVR & 16 MHz \\
  \hline
  Mini & 32 KB & 2 & 1 & ATmega328 & 8 bit AVR & 16 MHz \\
  \hline
  LilyPad USB & 32 KB & 2.5 & 1 & ATmega32u4 & 8 bit AVR & 8 Mhz \\
  \hline
  LilyPad Simple & 32 KB & 2 & 1 & ATmega328 & 8 bit AVR & 8 Mhz \\
  \hline
  LilyPad (168V) & 16 KB & 1 & 512 Bytes & ATmega168V & 8 bit AVR & 8 Mhz \\
  \hline
  LilyPad (328V) & 16 KB & 1 & 512 Bytes & ATmega328V & 8 bit AVR & 8 Mhz \\
  \hline
  Nano (168) & 16 KB & 1 & 512 Bytes & ATmega168 & 8 bit AVR & 16 MHz \\
  \hline
  Nano (328) & 32 KB & 2 & 1 & ATmega328 & 8 bit AVR & 16 MHz \\
  \hline
  Pro mini (3.3v) & 16 KB & 1 & 512 Bytes & ATmega168 & 8 bit AVR & 8 Mhz \\
  \hline
  Pro mini (5v) & 16 KB & 1 & 512 Bytes & ATmega168 & 8 bit AVR & 16 MHz \\
  \hline
  Pro (168) & 16 KB & 1 & 512 Bytes & ATmega168 & 8 bit AVR & 8 Mhz \\
  \hline
  Pro (328) & 32 KB & 2 & 1 & ATmega328 & 8 bit AVR & 16 MHz \\
  \hline
  Fio & 32 KB & 2 & 1 & ATmega328P & 8 bit AVR & 8 Mhz \\
  \hline 
\end{tabular}
\end{table}

%%\subsection{Arduino Uno}
%%Web: http://arduino.cc/en/Main/ArduinoBoardUno
%%Microcontrolador: ATmega328
%%Características generales:
%%Es una placa basada en el microcontrolador ATmega328.
%%0.5 kb de la memoria flash son utilizados por bootloader.
%%Existe una placa construida en uruguay llamada Urduino328,
%%la cuál es compatible con la Arduino Uno y tiene un costo aproximado de 50 dólares.
%%
%%\subsection{Arduino Leonardo}
%%Web:
%%http://arduino.cc/en/Main/ArduinoBoardLeonardo
%%Características generales:
%%Es una placa basada en el microcontrolador ATmega32u4. Tiene 20 pins de entrada/salida digitales, frecuencia de 16 MHz y conección micro USB.
%%La diferencia principal con otras placas es que el microcontrolador permite la comunicación usb sin necesidad de un microcontrolador secundario que la implemente.
%%Un bootloader es incluído, el cuál se puede utilizar para programar la placa sin un programador por hardware. Éste bootloader ocupa 4 kb de la memoria Flash del microcontrolador, puede ser eliminado pero teniendo en cuenta que luego no se cuenta con su funcionalidad.
%%Microcontrolador:
%%ATmega32u4
%%
%%\subsection{Arduino Due}
%%Web:
%%http://arduino.cc/en/Main/ArduinoBoardDue
%%Características:
%%Es la primer placa arduino basada en la arquitectura ARM de 32 bits.
%%Tiene 54 pins de entrada/salida digital, 12 de los cuáles pueden ser usados como salidas PWM.
%%12 entradas analógicas, un reloj de 84 MHz integrado, conección USB, 2 convertidores digital-analógico.
%%
%%Un botón de reset y un botón de borrado.
%%A diferencia de otras placas, ésta placa corre con un voltaje de 3.3V.
%%La mejora sustancial con respecto a otras placas, puede ejecutar operaciones sobre 4 bytes en un sólo ciclo de reloj,
%%tiene una frecuencia alta de reloj, 96 kbytes de SRAM, 512 kb de memoria flash para código y
%%un controlador DMA para liberar el CPU de tareas basadas en muchos accesos a memoria.
%%
%%El bootloader que incluye viene de fábrica y está en una ROM dedicada, por lo que no ocupa espacio de la memoria Flash.
%%
%%Microcontrolador: Atmel SAM3X8E ARM Cortex-M3
%%
%%\subsection{Arduino Robot}
%%Web:
%%http://arduino.cc/en/Main/Robot
%%Características:
%%Es el primer Arduino sobre ruedas oficial.
%%Cuenta con dos procesadores, cada uno sobre una placa.
%%Hay una placa utilizada para controlar los motores, y otra placa de control que maneja los
%%sensores y decide como operar.
%%Cada placa se puede programar por separado usando el IDE Arduino.
%%Algunos de los pines de la placa ya están mapeados a sensores y actuadores.
%%  El chasis cuenta con una brújula, un parlante, un panel de control de 5 botones, leds, conexiones I2C,
%%dos ruedas y sensores infrarrojos.
%%  También tiene zonas de prototipado.
%%
%%Microcontrolador: 2 microcontroladores ATmega32u4


\section{Microcontroladores}

Aquí se presentan las diferentes arquitecturas de los
microcontroladores utilizadas en los ejemplos.

\subsection{ARM}
\subsubsection{ARM Cortex-M0}
Es el procesador más pequeño existente de ARM.
\subsubsection{ARM Cortex-M3}
32 bit
Diseñado para aplicaciónes de bajo costo y alto desempeño.

\subsection{AVR}

Son microcontroladores basados en RISC, la empresa Atmel
produce controladores de éstas características.
Intentan mantener la ejecución a un ritmo de una
instrucción por ciclo de reloj, para lograr velocidades
relativas de 1MIPS/MHz.
Es una línea de bajo consumo y alto desempeño.

\subsubsection{ATmega8}
8 kb de Flash de memoria programable, 1kb de SRAM, 512 k de EEPROM, conversión analógica/digital de 10 bits.
Soporta una frecuencia máxima de 16 Mhz a 1 MIPS/MHz.
Opera entre 2v y 3.5 volts. Se recomiendan 3 volts. Equivale a 2 baterías AA o LR.
Web: http://www.atmel.com/devices/atmega8.aspx

\subsubsection{ATmega128}
Combina 128 kb de Flash, 4kb de SRAM, 4kb de EEPROM, conversión analógica/digital de 10 bits. Soporta hasta 16 MIPS a un máximo de 16 Mhz. (1 MIPS/MHz)
Es un microcontrolador AVR basado en RISC de 8 bit, bajo consumo eléctrico y alto desempeño.
Opera entre 4.5 y 5.5 volts.
Variantes: Atmega128 y ATmega128L.
Web: http://www.atmel.com/devices/atmega128.aspx
ATmega328
Web: http://www.atmel.com/devices/atmega328.aspx
ATmega32u4
Web: www.atmel.com/devices/atmega32u4.aspx
ATmega2560
Web: www.atmel.com/devices/atmega2560.aspx
ATmega168
http://www.atmel.com/devices/atmega168.aspx
Variantes: ATmega168, ATmega168V
ATmega328
http://www.atmel.com/devices/atmega328.aspx
Variantes: ATmega328P, ATmega328V

\subsection{Microchip}

\subsubsection{PIC18F4550}

Producido también por la empresa Digikey.
Frecuencia de 48 MHz (oscilador interno), controlador de
8bit, 32 kbytes de memoria Flash de programa,
2bkytes de memoria RAM, 256 bytes de EEPROM. 




\chapter{Diseño}


% TODO(Marcos): Decir por que se decidio definir un lenguaje propio y no
% utilizar los que se estudiaron anteriormente.
% Tambien decir que comparte con esos lenguajes y en que difiere.

  En este capítulo se describe el diseño del lenguaje \frob{} junto con su semántica.
  Luego se explica de qué manera es traducido al lenguaje \alf{} de
bajo nivel, mas simple de interpretar, el cual podrá ser interpretado
por implementaciones de una misma máquina virtual en diferentes
plataformas de hardware.

  También se describirán las etapas de compilación, desde que se escribe
un programa en alto nivel hasta que es ejecutado en una
plataforma objetivo.

  El diagrama de la Figura \ref{fig:compilacion} resume todas las etapas y
componentes necesarios.

\begin{figure}[h]
\begin{center}
\caption{Etapas y componentes}
\includegraphics[width=0.9\textwidth]{graphs/compilacion.png}
\label{fig:compilacion}
\end{center}
\end{figure}

  En la parte de arriba de la Figura se ve un programa en el
lenguaje \frob{} de alto nivel.
  El desarrollador escribe dicho programa y
ejecuta el compilador \compilador{} y obtiene un archivo \alf{} binario.

  Debajo se muestran diversas plataformas, cada una
con su implementación de la máquina virtual \maquinavirtual{} instalada.

  El desarrollador podrá cargar el mismo código \alf{} en cualquier robot que
esté construido utilizando cualquiera de las plataformas, y la máquina
virtual se encargará de interpretarlo.


\section{Lenguaje \frob{}}
  El lenguaje \frob{} es funcional, tipado y con inferencia de tipos. No
cuenta con evaluación a demanda (\emph{lazy evaluation}).
  Para simplificar la implementación, solo se permiten valores
naturales ($\mathcal{N}$), y funciones de naturales en naturales. 
($\mathcal{N} \rightarrow \mathcal{N}$).

  Un programa constará de dos secciones, una sección de
declaraciones, donde se declararán funciones y una sección donde se
aplicarán combinadores de programación funcional reactiva, especificando
cómo son transformadas las señales para especificar el comportamiento de
los robots.

  En \frob{} todos los valores son inmutables, una vez que son declarados
no se pueden modificar.
  Las funciones declaradas son puras, por lo tanto no pueden tener efectos
secundarios.
  
TODO: Explicar que es funcional, \ref{fig:grammar} sin currying, es tipado, se infieren
los tipos.

\input{design/grammar.tex}

\subsection{Declaraciones}

  Las funciones se declaran de la siguiente manera.

\begin{verbatim}
nombre argumento_1 .. argumento_n = expresion
\end{verbatim}

  Donde una expresión puede ser un valor,
una expresión aritmética (por ejemplo una suma o multiplicación),
una aplicación de una función, o una expresión condicional.\\

%  La sintaxis es muy similar a la del lenguaje \texttt{Haskell}, aunque
% no se permiten funciones anónimas,

  Para declarar un valor constante simplemente se escribe:

\begin{verbatim}
    NOMBRE_CONSTANTE = valor
\end{verbatim}

  Ejemplo de declaración:

\begin{verbatim}
    # fibonacci
    fibo n = if (n < 2) then 1 else fibo(n-1) + fibo(n-2)
\end{verbatim}




\subsection{Combinadores de FRP}




\section{Programación Funcional Reactiva}

Acá voy a hablar de programación funcional reactiva.

Tradicionalmente los programas son formados a partir de
una secuencia de acciones imperativas. Los programas
reactivos suelen formarse por eventos y código iterativo
que se corre cuando un evento ocurre.

Dicho código iterativo suele hacer referencia y manipular
variables compartidas con diferentes rutinas. Ésto lleva a que
como un valor puede ser manipulado desde diferentes lugares,
puedan producirse problemas de concurrencia y algunos valores
pueden quedar en un estado inconsistente.

En el paradigma FRP no existen valores compartidos, sinó que
dichos valores dependientes del tiempo, tienen una representación
llamada Comportamiento y la única forma de modificarlos, es
a partir de como fueron definidos.

 

\begin{definicion}
Comportamientos (Behaviours).\\
Un comportamiento es un valor contínuo que depende del paso del tiempo.
Los comportamientos se pueden definir, combinar, pasarlos como
argumentos a funciones, retornarlos.
Un comportamiento puede ser un valor constante, el tiempo mismo (un reloj),
o puede formarse combinando otros comportamientos, por ejemplo secuencialmente
o paralelamente.
\end{definicion}

\begin{definicion}
Eventos (Streams).\\
Son valores discretos dependientes del tiempo, que forman
una secuencia finita o infinita de ocurrencias. Cada ocurrencia
está formada por el valor y el instante de tiempo.
\end{definicion}

La principal diferencia entre Comportamientos y Eventos, es que los
comportamientos son valores contínuos y los eventos son discretos.

Los comportamientos representan cualquier valor en función del tiempo,
por ejemplo:
\begin{itemize}
\item \textit{entrada} sensor de distancia, temperatura, video
\item \textit{salida} velocidad, voltaje
\item \textit{valores} intermedios calculados
\end{itemize}

Las operaciones que se pueden realizar sobre los comportamientos incluyen:
\begin{itemize}
\item \textit{Operaciones genéricas} Aritmética, integración, diferenciación
\item \textit{Operaciones específicas de un dominio} como escalar video, aplicar filtros, detección de patrones.
\end{itemize}

Los eventos pueden ser sensores específicos de un dominio, por ejemplo un
botón, un click, una interrupción o mensaje asincrónico.
También puede ser generado a partir de valores de un comportamiento,
como ser \emph{Temperatura alta}, \emph{Batería baja}.

Las operaciones que se pueden realizar sobre los eventos incluyen:
\begin{itemize}
\item fmap, filter
\item Modificar un \emph{Comportamiento} reactivo
\end{itemize}


\section{Ejemplo}

Para entender un poco más las ideas de Comportamiento y Evento, se puede
plantear el siguiente ejemplo.

  En una cuenta de un banco, el saldo se puede definir
como un comportamiento, el cuál solo se modifica cuando ocurre
un movimiento.
  Un movimiento puede ser depositar dinero o extraer
dinero de la cuenta.
  Un dato importante a ver, es que no es posible asignar un valor
sin que sea por medio de su propia definición, por lo que nadie
podría realizar la asignación $saldo = 1000000$.
  La misma operación sería posible creando un movimiento, el cuál
afectaría al saldo.

  Usaremos la notación $<nombre> :: Event <tipo>$ para definir
eventos y la notación $<nombre> :: Behaviour <tipo>$ para
definir comportamientos.

\begin{verbatim}
movimiento :: Event Number
movimiento = read(0)

saldo :: Behaviour Number
saldo = movimiento

alert :: Event Bool
alert = saldo > 1000

on alert:
  write(0, 'El saldo es mayor a 1000')

\end{verbatim}

Un comportamiento, solo se modifica cuando cambian sus componentes,





%state = State(state, [input])
%Output(state)


\section{Ejemplo}

\begin{figure}[h!]
\begin{center}
    \caption{Ejemplo completo}
\begin{Verbatim}[frame=single]
INPUT_DISTANCE = 1
OUTPUT_ENGINE = 1

MIN_DIST = 30
MAX_SPEED = 100
STOP = 0

distanceToSpeed n = if (n < MIN_DIST) then STOP else MAX_SPEED

do {
  distance <- read INPUT_DISTANCE,
  speed <- lift distanceToSpeed distance,
  output OUTPUT_ENGINE speed
}
\end{Verbatim}
   \label{fig:example1}
\end{center}
\end{figure}


\section{Lenguaje de bajo nivel}
  Para presentar el lenguaje \alf{}, primero defino las estructuras
necesarias para explicar la semántica de cada instrucción.

  TODO: Modificar toda esta sección, parece más una descripción de la máquina
virtual que del lenguaje.

\begin{enumerate}

\item \emph{Inputs}

  Los valores leídos en las entradas de hardware se mapean
en esta lista. Por ejemplo si el hardware cuenta con un botón,
y el identificador del botón es $i$,
su valor se representará con la notación:

  $Inputs_i$

  A cada entrada se le asocia un conjunto de
rutinas que deben invocarse cuando se tenga un
valor disponible en la entrada. El mismo puede ser vacío si
no hay rutinas esperando por su valor. Queda a criterio de quien
implementa la máquina si éstos valores deben ser guardados o
descartados.
  A éste conjunto de rutinas lo denotamos cómo:

  $Callbacks_i$

\item \emph{Nodes}

  Es utilizado para almacenar todas las transformaciones de señales.
  Cada nodo, tiene una lista de otros nodos que dependen de él y la
  posición del argumento por el que esperan.
  Además el nodo almacena el último valor calculado, y una lista
  de argumentos que le serán pasados, si son nuevos o no.

  Cada aplicación de \texttt{lift}, \texttt{lift2} o \texttt{folds}
será mapeada en un nodo.

  Cada nodo $Node_i$ tiene una lista de nodos adyacentes
que dependen de él.

\item \emph{Outputs}

  Mapea señales en salidas de hardware, los valores
son asignados con la operación \texttt{output}.

\item \emph{Stack}

  Es una pila global, se utiliza para ejecutar operaciones y
realizar cálculos. 
  Es única y global, y los hilos de ejecución no pueden guardar valores
persistentes en ella.

Usaremos el símbolo $TOS$ para referirnos al elemento en el tope
de la pila y $Stack$ para referirnos a la pila.

\item \emph{Ready}

Es una \emph{cola} que contiene punteros a los nodos
listos para ser ejecutados.

\item \emph{Dispatcher}

  El dispatcher, es quien implementa las acciones de la máquina.
  El mismo se encarga de recibir valores de los sensores y mapearlos
a eventos en las entradas $Input_i$.

  Éstos eventos, serán recibidos por los nodos $Nodes$.
  Cada $Node_i$ que espera por eventos entrará en estado activo cuando
todos los nodos por los que espera le envíen un evento.
  A su vez, el nodo en estado activo calcula un resultado y notifica a
todos sus nodos adyacentes.

  TODO: Forzar que Nodes sea un grafo acíclico en la descripción.

  El dispatcher, realizará implicitamente un orden topológico de los
nodos, como $Nodes$ es un grafo acíclico, éste proceso es posible y
termina, y cada salida cuenta con un valor, que será mapeado a los actuadores.

\end{enumerate}

\subsection{Instrucciones}

  A continuación se listan las instrucciones de bajo nivel
  más relevantes junto con su semántica.

\begin{itemize}
\item \texttt{lift id source\_id function\_pointer}

 Crea un nodo que recibe valores de la fuente de
 eventos \texttt{source\_id},
 les aplica la función \texttt{function\_pointer} y emite el resultado
 en la fuente de eventos $id$.\\
 El nodo se agrega a $Nodes$ y el dispatcher se encargará de aplicar
 la función siempre que sea necesario.

\item \texttt{lift2 id source-id\_1 source-id\_2 function\_pointer}

  Crea un nodo que recibe valores de \texttt{source-id\_1} y
\texttt{source-id\_2}, les aplica la función \texttt{function\_pointer} 
y emite el resultado en la fuente de eventos \texttt{id}.

  Cuando reciba un valor nuevo por cada uno de los que espera, el
dispatcher aplica la función y emite el resultado.

\item \texttt{read id input-id}

 Agrega un nodo a $Nodes$ que recibe valores de una entrada
 y emite el resultado en la fuente de eventos $id$.
 Cuando el dispatcher reciba un cambio en la entrada, el nodo emitirá
 el resultado.

\item \texttt{write index id}

  Crea un nodo $Output$ que recibe valores de la señal \texttt{id}
y los emite en la salida número \texttt{index}.
  El nodo guarda dicho valor y el dispatcher decide en que momento
debe escribirlo en la salida asociada.
  Se intentará que sea lo más instantáneo posible
ya que el valor depende del tiempo y a mayor demora, menos correcto
será el comportamiento.

\end{itemize}

TODO: Comentario de Marcos: Sólo ésto? Describir más instrucciones.
Mostrar que es una máquina de stack. Mostrar ejemplos. 
Hacer referencia al apéndice A.2.


\pagebreak



\section{Compilador}
  El compilador será el encargado de leer el programa de alto nivel y
traducirlo al lenguaje de bajo nivel.

  El programa resultado tendrá al principio las traducción de las
instrucciones correspondientes al bloque \texttt{do}.
  Las mismas al ser ejecutadas armarán el grafo de señales del programa.

  Al final del bloque, habrá una instrucción \texttt{halt} que cederá el
control al $dispatcher$.

  Luego estará el código correspondiente a cada función definida.
  Cada invocación a función, tendrá la referencia directa hacia la
posición en el código de la misma.

  El compilador constará de dos etapas principales.

  En la primer etapa lee el programa de alto nivel y
genera un árbol de sintaxis abstracto 
(AST)\footnote{Del inglés Abstract Syntax Tree} del mismo.
  Dicha etapa es llamada análisis sintáctico.

  Luego el ast se recorre y la segunda etapa es la generación del
código de bajo nivel.

  En la Figura \ref{fig:compiler} se puede ver la estructura más detallada
  del compilador.

\begin{figure}[hp]
\begin{center}
\caption{Diagrama del compilador}
\includegraphics[width=0.9\textwidth]{graphs/compiler.png}
\label{fig:compiler}
\end{center}
\end{figure}

  TODO: Describir la Figura.

\newpage


\subsection{Ejemplo de traducción}

TODO: Describir el algoritmo de traducción.

El siguiente programa \frob{} muestra con un led si el robot está
frente a una casa:

\begin{Verbatim}
#Inputs
INPUT_DISTANCE = 1
#Outputs
OUTPUT_LED = 1

isHouse distance = if (distance < 100) then 1 else 0

do {
  signal_distance <- read INPUT_DISTANCE
  signal_house <- lift isHouse signal_distance
  output OUTPUT_LED signal_house
}
\end{Verbatim}

El mismo se traduce a \alf{} de la siguiente forma:

\begin{Verbatim}
0: t_call
1: 10
2: t_read 1
3: t_lift 0
4: 1
5: 16
6: t_call
7: 13
8: t_write 0
9: t_halt
10: t_push
11: 1
12: t_ret
13: t_push
14: 1
15: t_ret
16: t_load_param 0
17: t_push
18: 100
19: t_cmp_lt
20: t_jump_false
21: 26
22: t_push
23: 1
24: t_jump
25: 28
26: t_push
27: 0
28: t_ret
\end{Verbatim}

TODO: Poner comentarios. Comentar abajo.



\section{Máquina Virtual}

  La máquina que interpreta el lenguaje \alf{} es una
\textit{máquina de stack}.\footnote{Stack machine en inglés}.

  En una máquina de stack las instrucciones están en notación
postfija.\footnote{RPN (\textit{Reverse polish notation}) del inglés}
  Para evaluar expresiones se colocan sus argumentos en una pila, y luego
se ejecuta la operación asociada.
  
  Por ejemplo la expresión ``$5 + 19 * 8$'' en RPN se
escribe ``$5\ 19\ 8 * +$''.
  
  A modo de ejemplo en \alf{} se representa con las
  siguientes 5 instrucciones.

  \begin{Verbatim}[frame=single]
  push 5
  push 19
  push 8
  mul
  add
  \end{Verbatim}

  El conjunto \textit{Inputs} representa las entradas de la máquina.
  Dadas $m$ entradas fijas, cada una se identifica con un entero único
  entre $1$ y $m = | \textit{Inputs} |$.

  Cada $\texttt{I}_i, i \in (1 \dotsb m)$ se corresponderá con un sensor
  definido en el robot.

\begin{definicion}
  \textit{Entradas de la máquina}\\
  \begin{center}
    $\textit{Inputs} \equiv \{\texttt{I}_1 \dotsb \texttt{I}_m\}$.
  \end{center}
\end{definicion}
  
  Graficamente las representaré con la notación:

  \begin{center}
\begin{tikzpicture}
\selectlanguage{english}
  %%\draw[step=1cm,gray,very thin,xshift=0cm,yshift=0cm] (0,0) grid (12,4);
  \begin{scope}[xshift=0cm,yshift=0cm,very thick,
    node distance=2cm,on grid,>=stealth',
    block/.style={rectangle,draw,fill=cyan!20},
    comp/.style={circle,draw,fill=orange!40},
    stack/.style={rectangle split,rectangle split parts=#1,draw,anchor=center}]
   \node [block] (s2) [yshift=0.5cm,xshift=1cm] {$\texttt{I}_{i}$};
\end{scope} 

\selectlanguage{spanish}
\end{tikzpicture}

\end{center}


  También se cuenta con un conjunto \textit{Outputs} de salidas,
  identificadas de $1$ a $k = | \textit{Outputs} |$.
  
  Cada $\texttt{O}_i, i \in (1 \dotsb k)$ se corresponderá con un actuador
  del robot.

\begin{definicion}
  \textit{Salidas de la máquina}\\
  \begin{center}
    $\textit{Outputs} \equiv \{\texttt{O}_1 \dotsb \texttt{I}_k\}$.
  \end{center}
\end{definicion}
  
  Graficamente las representaré con la notación:

  \input{design/ll_diagram_output.tex}

  Las señales que se definan se denotarán $\texttt{S}_i$, siendo $i$ un
  índice único que las identifica. El conjunto de las señales se llama
  \textit{Signals}.

\begin{definicion}
  \textit{Señales}
  \begin{center}
    $\textit{Signals} \equiv \{\texttt{S}_1 \dotsb \texttt{S}_s\}$.
  \end{center}
\end{definicion}

  Graficamente las representaré con la notación:

  \begin{center}
\begin{tikzpicture}
\selectlanguage{english}
  %%\draw[step=1cm,gray,very thin,xshift=0cm,yshift=0cm] (0,0) grid (12,1);
  \begin{scope}[xshift=4cm,yshift=0cm,very thick,
    node distance=2cm,on grid,>=stealth',
    comp/.style={circle,draw,fill=orange!40}]
    \node [comp] (ca1) [xshift=2cm, yshift=0.5cm] {$\texttt{S}_{i}$};
   \end{scope} 
\selectlanguage{spanish}
\end{tikzpicture}
\end{center}


  La máquina tendrá una pila global, denotada \textit{Stack}. El mismo
  se representa con una secuencia de valores.

\begin{definicion}
  \textit{Pila global}
  \begin{center}
    $\textit{Stack} \equiv s_1, \dotsb, s_n$.
  \end{center}
\end{definicion}

  El \textit{Stack} lo representaré graficamente con la notación:

  \begin{center}
\begin{tikzpicture}
\selectlanguage{english}
  %%\draw[step=1cm,gray,very thin,xshift=0cm,yshift=0cm] (0,0) grid (12,3);
  \begin{scope}[xshift=4cm,yshift=0cm,very thick,
    node distance=2cm,on grid,>=stealth',
    block/.style={rectangle,draw,fill=cyan!20},
    comp/.style={circle,draw,fill=orange!40},
    stack/.style={rectangle split,rectangle split parts=#1,draw,anchor=center}]
  \node[stack=4,xshift=2cm,yshift=1.5cm,
        text width=1.1cm,align=center,text height=0.3cm] {
    \nodepart{one} $s_n$
    \nodepart{two} $s_{n-1}$
    \nodepart{three} $\dotsb$
    \nodepart{four} $s_o$
  };
 \draw [->,line width=1pt] (0.5,2.4) -- node[below]{\tiny{TOS}} (1.3, 2.4);
 \end{scope} 
  
\selectlanguage{spanish}
\end{tikzpicture}


\end{center}


  Donde \texttt{TOS}\footnote{Del inglés: Top
  of stack} indica el índice del tope del mismo.
  Se cumple que $\textit{Stack}_{TOS} = s_n$.



%% TODO: Esto es parte de la implementacion.
%%  A diferencia del compilador, es necesario implementar una máquina virtual
%%para cada arquitectura objetivo.\\
%%
%%  Por ejemplo, para ejecutar programas en un robot
%%  con un procesador \emph{arduino}, debe
%%  existir una implementación de la máquina para ese modelo
%%  de \emph{arduino}.
%%
%%  Al momento de implementar la máquina, se tomará en cuenta ésto para
%%  factorizar partes en común y sólo implementar por arquitectura, las
%%  partes que realmente sean diferentes como ser la comunicación con
%%  los periféricos de entrada/salida y las llamadas al sistema.

  El dispatcher, es quien implementa las acciones de la máquina.
  El mismo se encarga de recibir valores de los sensores y mapearlos
  a eventos en las entradas $\texttt{I}_i$.

  Éstos eventos, serán recibidos por los nodos $Nodes$.
  Cada $Node_i$ que espera por eventos entrará en estado activo cuando
todos los nodos por los que espera le envíen un evento.
  A su vez, el nodo en estado activo calcula un resultado y notifica a
todos sus nodos adyacentes.

  TODO: Forzar que Nodes sea un grafo acíclico en la descripción.

  El dispatcher, realizará implicitamente un orden topológico de los
nodos, como $Nodes$ es un grafo acíclico, éste proceso es posible y
termina, y cada salida cuenta con un valor, que será mapeado a los actuadores.





\chapter{Implementación}

En este capítulo se detalla la implementación del compilador
y la máquina virtual diseñadas para utilizar el lenguaje
\frob{} en la plataforma elegida.
También se explica cuál sería el mecanismo para portar la
implementación a otra plataforma.

%TODO: Completar la intro luego de tener los caps.

\section{Compilador}

  El compilador será el encargado de leer el programa \frob{} y traducirlo
a \alf{}.

  El lenguaje utilizado para desarrollar el compilador fue \textit{Haskell}.
  Las razones que llevaron a su elección son la portabilidad y la
expresividad del mismo.
  El compilador \compilador{} es portable, ya que se puede compilar y ejecutar
en diversos sistemas operativos utilizando el compilador \textit{ghc}.

  Es usual realizar tareas de compilación en \textit{Haskell} por lo que existen
herramientas estándar para cada etapa.

  El compilador constará de una secuencia de etapas: Análisis Léxico,
  Análisis Sintáctico, Análisis Semántico y Generación de Código.

  En la Figura \ref{fig:compiler} se puede ver la estructura más detallada
del compilador.

\begin{figure}[h!]
\begin{center}
\caption{Diagrama del compilador}
\includegraphics[width=0.9\textwidth]{graphs/compiler.png}
\label{fig:compiler}
\end{center}
\end{figure}


\subsubsection{Análisis Léxico}
  La primera etapa se llama análisis léxico, en esta se lee el código
  fuente en lenguaje \frob{} (.willie) y lo transforma en una
  lista de lexemas.

  Un lexema puede ser una palabra reservada (ej: \texttt{do}),
  un valor (ej: $19$), un identificador (eg: \texttt{distance}) o
  un símbolo reservado (eg: \texttt{+}).

  Para representar los lexemas, se utiliza la
  herramienta \textit{UU.Scanner}
  \cite{uuparser} que estandariza los mismos en el tipo de
  datos \texttt{Token}.

  Usando \textit{Alex}\cite{alex} se procesa el código fuente,
  se reconocen los lexemas y se retorna una lista de
  tipo \texttt{[Token]}.

  La etapa se puede resumir en la implementación de la
  función \texttt{tokenize}.

\begin{Verbatim}
  tokenize :: String -> String -> [Token]
\end{Verbatim}



\subsubsection{Análisis Sintáctico}
  La segunda fase del compilador, recibe la lista de lexemas (\texttt{[Token]}) y
reconoce el lenguaje, generando un árbol de
sintaxis abstracta (\emph{AST}\footnote{Del inglés Abstract Syntax Tree}).

  Para reconocer la gramática se implementó un parser recursivo descendente.
  Utilizando la herramienta \textit{UU.Parser} \cite{uuparser}, se definió un tipo de datos
  \texttt{TokenParser a} que representa un parser que recibe una secuencia de lexemas de tipo \texttt{Token}
  y retorna un \emph{AST} de tipo \texttt{a}.

  \begin{Verbatim}
  type TokenParser a = Parser Token a
  \end{Verbatim}

  \textit{UU.Parser} define un conjunto de combinadores de parsers y utilizándolos se construyen parsers
  complejos a partir de parsers simples.

  Para representar el \emph{AST} se utiliza una gramática de atributos.
  Una gramática de atributos es como una gramática libre de contexto, pero agrega semántica a la misma.
  Para el análisis sintáctico, la semántica no es utilizada, pero será usada en la próxima etapa.

  El sistema de gramáticas de atributos
  \textit{UUAG}\cite{uuag} fue usado para la implementación.

  Se define un tipo de datos \texttt{Root} que representa la raíz del árbol.
  El mismo tiene un único constructor \texttt{Root\_Root} que recibe un árbol de tipo
  \texttt{Decls} que representa las declaraciones, y un árbol de tipo \texttt{Dodecls} que
  representa el bloque \texttt{do}.

  Para crear el \emph{AST} usando \textit{UU.Parser} se define el parser \texttt{pRoot}:

  \begin{Verbatim}
  pRoot :: TokenParser Root
  pRoot
    = (\x y -> Root_Root x y) <$> pDecls <*> pDodecls
  \end{Verbatim}

  El cuál asume definido un parser de declaraciones \texttt{pDecls} y un parser
  del bloque \texttt{do} (\texttt{pDodecls}).

  \begin{Verbatim}
  pDecls :: TokenParser Decls

  pDodecls :: TokenParser Dodecls
  \end{Verbatim}

  Se va refinando sucesivamente en parsers mas específicos, hasta construir completamente el \emph{AST}.




\subsubsection{Análisis Semántico}
  Para la última etapa se utiliza la gramática de atributos para definir
semántica sobre el \emph{AST}.

  Las gramáticas de atributos (\emph{Attribute Grammars}) simplifican
la tarea de escribir catamorfismos.
Un catamorfismo es una función análoga a la función de alto orden
\texttt{foldr} pero aplicada sobre cualquier tipo de datos.

  De ésta manera se pueden definir atributos en el \emph{AST}, sintéticos
o heredados.

  Uno de dichos atributos será el código en bajo nivel, que será la salida
de esta etapa.
  Esta salida se escribe en un archivo (.alf) terminando el proceso
de compilación.




\section{Máquina virtual}

  La máquina que interpreta el lenguaje \alf{} es una
\textit{máquina de stack}.\footnote{Stack machine en inglés}.

  En una máquina de stack las instrucciones están en notación
postfija.\footnote{RPN (\textit{Reverse polish notation}) del inglés}
  Para evaluar expresiones se colocan sus argumentos en una pila, y luego
se ejecuta la operación asociada.
  
  Por ejemplo la expresión ``$5 + 19 * 8$'' en RPN se
escribe ``$5\ 19\ 8 * +$''.
  
  A modo de ejemplo en \alf{} se representa con las
  siguientes 5 instrucciones.

  \begin{Verbatim}[frame=single]
  push 5
  push 19
  push 8
  mul
  add
  \end{Verbatim}

  El conjunto \textit{Inputs} representa las entradas de la máquina.
  Dadas $m$ entradas fijas, cada una se identifica con un entero único
  entre $1$ y $m = | \textit{Inputs} |$.

  Cada $\texttt{I}_i, i \in (1 \dotsb m)$ se corresponderá con un sensor
  definido en el robot.

\begin{definicion}
  \textit{Entradas de la máquina}\\
  \begin{center}
    $\textit{Inputs} \equiv \{\texttt{I}_1 \dotsb \texttt{I}_m\}$.
  \end{center}
\end{definicion}
  
  Graficamente las representaré con la notación:

  \begin{center}
\begin{tikzpicture}
\selectlanguage{english}
  %%\draw[step=1cm,gray,very thin,xshift=0cm,yshift=0cm] (0,0) grid (12,4);
  \begin{scope}[xshift=0cm,yshift=0cm,very thick,
    node distance=2cm,on grid,>=stealth',
    block/.style={rectangle,draw,fill=cyan!20},
    comp/.style={circle,draw,fill=orange!40},
    stack/.style={rectangle split,rectangle split parts=#1,draw,anchor=center}]
   \node [block] (s2) [yshift=0.5cm,xshift=1cm] {$\texttt{I}_{i}$};
\end{scope} 

\selectlanguage{spanish}
\end{tikzpicture}

\end{center}


  También se cuenta con un conjunto \textit{Outputs} de salidas,
  identificadas de $1$ a $k = | \textit{Outputs} |$.
  
  Cada $\texttt{O}_i, i \in (1 \dotsb k)$ se corresponderá con un actuador
  del robot.

\begin{definicion}
  \textit{Salidas de la máquina}\\
  \begin{center}
    $\textit{Outputs} \equiv \{\texttt{O}_1 \dotsb \texttt{I}_k\}$.
  \end{center}
\end{definicion}
  
  Graficamente las representaré con la notación:

  \input{design/ll_diagram_output.tex}

  Las señales que se definan se denotarán $\texttt{S}_i$, siendo $i$ un
  índice único que las identifica. El conjunto de las señales se llama
  \textit{Signals}.

\begin{definicion}
  \textit{Señales}
  \begin{center}
    $\textit{Signals} \equiv \{\texttt{S}_1 \dotsb \texttt{S}_s\}$.
  \end{center}
\end{definicion}

  Graficamente las representaré con la notación:

  \begin{center}
\begin{tikzpicture}
\selectlanguage{english}
  %%\draw[step=1cm,gray,very thin,xshift=0cm,yshift=0cm] (0,0) grid (12,1);
  \begin{scope}[xshift=4cm,yshift=0cm,very thick,
    node distance=2cm,on grid,>=stealth',
    comp/.style={circle,draw,fill=orange!40}]
    \node [comp] (ca1) [xshift=2cm, yshift=0.5cm] {$\texttt{S}_{i}$};
   \end{scope} 
\selectlanguage{spanish}
\end{tikzpicture}
\end{center}


  La máquina tendrá una pila global, denotada \textit{Stack}. El mismo
  se representa con una secuencia de valores.

\begin{definicion}
  \textit{Pila global}
  \begin{center}
    $\textit{Stack} \equiv s_1, \dotsb, s_n$.
  \end{center}
\end{definicion}

  El \textit{Stack} lo representaré graficamente con la notación:

  \begin{center}
\begin{tikzpicture}
\selectlanguage{english}
  %%\draw[step=1cm,gray,very thin,xshift=0cm,yshift=0cm] (0,0) grid (12,3);
  \begin{scope}[xshift=4cm,yshift=0cm,very thick,
    node distance=2cm,on grid,>=stealth',
    block/.style={rectangle,draw,fill=cyan!20},
    comp/.style={circle,draw,fill=orange!40},
    stack/.style={rectangle split,rectangle split parts=#1,draw,anchor=center}]
  \node[stack=4,xshift=2cm,yshift=1.5cm,
        text width=1.1cm,align=center,text height=0.3cm] {
    \nodepart{one} $s_n$
    \nodepart{two} $s_{n-1}$
    \nodepart{three} $\dotsb$
    \nodepart{four} $s_o$
  };
 \draw [->,line width=1pt] (0.5,2.4) -- node[below]{\tiny{TOS}} (1.3, 2.4);
 \end{scope} 
  
\selectlanguage{spanish}
\end{tikzpicture}


\end{center}


  Donde \texttt{TOS}\footnote{Del inglés: Top
  of stack} indica el índice del tope del mismo.
  Se cumple que $\textit{Stack}_{TOS} = s_n$.



%% TODO: Esto es parte de la implementacion.
%%  A diferencia del compilador, es necesario implementar una máquina virtual
%%para cada arquitectura objetivo.\\
%%
%%  Por ejemplo, para ejecutar programas en un robot
%%  con un procesador \emph{arduino}, debe
%%  existir una implementación de la máquina para ese modelo
%%  de \emph{arduino}.
%%
%%  Al momento de implementar la máquina, se tomará en cuenta ésto para
%%  factorizar partes en común y sólo implementar por arquitectura, las
%%  partes que realmente sean diferentes como ser la comunicación con
%%  los periféricos de entrada/salida y las llamadas al sistema.

  El dispatcher, es quien implementa las acciones de la máquina.
  El mismo se encarga de recibir valores de los sensores y mapearlos
  a eventos en las entradas $\texttt{I}_i$.

  Éstos eventos, serán recibidos por los nodos $Nodes$.
  Cada $Node_i$ que espera por eventos entrará en estado activo cuando
todos los nodos por los que espera le envíen un evento.
  A su vez, el nodo en estado activo calcula un resultado y notifica a
todos sus nodos adyacentes.

  TODO: Forzar que Nodes sea un grafo acíclico en la descripción.

  El dispatcher, realizará implicitamente un orden topológico de los
nodos, como $Nodes$ es un grafo acíclico, éste proceso es posible y
termina, y cada salida cuenta con un valor, que será mapeado a los actuadores.




\chapter{Casos de estudio}

  En esta sección veremos un caso de estudio usado para verificar la
implementación.

\section {Problema}

\section {Solución}

\section {Conclusiones del caso}




\chapter{Conclusiones}

Intro .... que se hizo, puntos a favor, etc..

\section{Trabajo futuro}

El principal trabajo futuro sería ...

Sería muy útil contar con una funcionalidad de depuración, la cuál
mostrara dependiendo del tiempo los valores de cada fuente de eventos.

Una opción es comunicar mediante el puerto serial el valor de cada
señal al cambiar, y mostrarlo en una interfaz web como la que provee
RXMarbles (ver \cite{rxmarbles}). 
El lenguaje Elm provee de una herramienta que permite viajar en el 
tiempo, modificar y mostrar la ejecución de un programa, en nuestro
caso no sería posible modificar lo que el robot físico realiza, pero
si sería útil ver en la línea de tiempo que valores tomaron sus
señales. (ver \cite{elmdebug})



% Bibliografia
\cleardoublepage
\addcontentsline{toc}{chapter}{Bibliografía}

%\begin{thebibliography}{99}
%\bibitem{katoen} \emph{Principles of Model Checking}.\\
 Christel Baier y Joost-Pieter Katoen.\\
 The MIT Press, 2008.

% http://conal.net/papers/frp.html
%@InProceedings{ElliottHudak97:Fran,
%  title        = {Functional Reactive Animation},
%  url          = {http://conal.net/papers/icfp97/},
%  author       = "Conal Elliott and Paul Hudak",
%  booktitle    = "International Conference on Functional Programming",
%  year         = 1997
%}

% Yampa, Arrows and Robots.

%@InProceedings{Peterson99:LambdaInMotion,
%  author       = {John Peterson and Paul Hudak and Conal Elliott},
%  title        = {Lambda in Motion: Controlling Robots with {Haskell}},
%  url          = {http://haskell.org/frob/padl99/padl99.ps},
%  booktitle    = {Practical Aspects of Declarative Languages},
%  year         = 1999
%}
% http://cs.brown.edu/research/pubs/techreports/reports/CS-03-20.html

% Informal pero ayudo:
% https://gist.github.com/staltz/868e7e9bc2a7b8c1f754
%
% Rx, Bacon.js, 
%
% Para debug muy bueno: (con RX es este)
% https://github.com/jaredly/rxvision
% http://lambdor.net/?p=44

\bibitem{python} Sitio web de \emph{Python 2.7}:\\
 \url{http://docs.python.org/2/}\\
 Último acceso: 31/10/2013.

\bibitem{gml} Sitio web de \emph{GraphML File Format}:\\
 \url{http://graphml.graphdrawing.org/}\\
 Último acceso: 31/10/2013.


\bibliographystyle{unsrt}
\bibliography{Informe}

%\end{thebibliography}


% Apendices
\appendix
\cleardoublepage
\addappheadtotoc
\appendixpage

\chapter{Apéndice}

\section{Manual de usuario}


  Para utilizar el compilador, dado un archivo \textit{Ejemplo.willie}, se
ejecuta:

\begin{Verbatim}
> williec < Ejemplo.willie > Ejemplo.alf
\end{Verbatim}



El código de la máquina virtual está en el directorio
  /src/alfvm, para compilarlo se ejecuta:
\begin{verbatim}
  > cd src/alfvm
  > make
\end{verbatim}



\label{appendix:usermanual}

\section{Manual de Referencia}


\subsection{Instrucciones de bajo nivel}

  A continuación se presenta el resto de las instrucciones
de bajo nivel y pseudocódigo indicando su semántica.

\begin{itemize}

\item \texttt{halt}

  Detiene el hilo de ejecución actual.

  \texttt{ip = 0} 

\item \texttt{call function}

  Invoca la funcion $function$.
  Se asume que los parámetros están en el stack.

\item \texttt{ret}

  Toma el resultado de una función del tope del stack,
  limpia el espacio ocupado por la función,
  y deja el resultado en el nuevo tope del stack.

  \texttt{value = stack.pop()}

  \texttt{stack.pop\_frame();}

  \texttt{ip = code[stack.pop()]}

  \texttt{fp = stack.pop()}

  \texttt{stack.pop\_args()}

  \texttt{stack.push(value)}

\item \texttt{load\_param inm}

  \texttt{a = stack.get\_arg(inm)}

  \texttt{stack.push(a)}

\item \texttt{jump}

  \texttt{goto position}

\item \texttt{jump\_false position}

  \texttt{a = stack.pop()}

  \texttt{if not a: goto position}

\item \texttt{cmp\_eq}

  \texttt{a = stack.pop()}

  \texttt{b = stack.pop()}

  \texttt{stack.push(a == b)}

\item \texttt{cmp\_neq}

  \texttt{a = stack.pop()}

  \texttt{b = stack.pop()}

  \texttt{stack.push(a != b)}

\item \texttt{cmp\_gt}

  \texttt{a = stack.pop()}

  \texttt{b = stack.pop()}

  \texttt{stack.push(a > b)}

\item \texttt{cmp\_lt}

  \texttt{a = stack.pop()}

  \texttt{b = stack.pop()}

  \texttt{stack.push(a < b)}

\item \texttt{add}

  \texttt{a = stack.pop()}

  \texttt{b = stack.pop()}

  \texttt{stack.push(a + b)}

\item \texttt{sub}

  \texttt{a = stack.pop()}

  \texttt{b = stack.pop()}

  \texttt{stack.push(a - b)}

\item \texttt{div}
  
  \texttt{a = stack.pop()}

  \texttt{b = stack.pop()}

  \texttt{stack.push(a / b)}

\item \texttt{mul}

  \texttt{a = stack.pop()}

  \texttt{b = stack.pop()}

  \texttt{stack.push(a * b)}

\item \texttt{op\_and}

  \texttt{a = stack.pop()}

  \texttt{b = stack.pop()}

  \texttt{stack.push(a and b)}

\item \texttt{op\_or}

  \texttt{a = stack.pop()}

  \texttt{b = stack.pop()}

  \texttt{stack.push(a or b)}

\item \texttt{op\_not}

  Coloca un valor constante en el stack.

  \texttt{stack.push(word)}

\item \texttt{push word}

  Coloca un valor constante en el stack.

  \texttt{stack.push(word)}

\item \texttt{pop}

  Elimina el tope del stack.

  \texttt{stack.pop()}

\item \texttt{dup}

  Duplica el tope del stack.

  \texttt{stask.push(stack.tos())}

\item \texttt{store inm}

  Guarda el tope del stack en la variable $inm$.

  \texttt{var[inm] = stack.pop()}

\item \texttt{load inm}

  Carga la variable $inm \in {0..255}$ en el stack.

  \texttt{stack.push(var[inm])}

\end{itemize}


\label{appendix:vmref}

\section{Parser}
\label{appendix:parser}


\begin{Verbatim}
module Parser where

import UU.Parsing
import UU.Scanner
import Lexer
import AttributeGrammar

type TokenParser a = Parser Token a

-- Parser with starting nonterminal Root
-- Semantic functions generated by UUAG
pRoot :: TokenParser Root
pRoot
  = (\x y -> Root_Root x y) <$> pDecls <*> pDodecls

pDecls :: TokenParser Decls
pDecls
  = pList pDecl

pDecl :: TokenParser Decl
pDecl
  = (\x xs _ y -> Decl_Function x xs y) <$>
      pVarid <*> pArgs <*> pKey "=" <*> pExpr
    <|> (\x _ y -> Decl_Const x y) <$>
      pConid <*> pKey "=" <*> pExpr

pArgs :: TokenParser [String]
pArgs
  = pList pVarid

pExpr :: TokenParser Expr
pExpr
  = pAdd
  <|> pIfExpr

pIfExpr :: TokenParser Expr
pIfExpr
  = (\_ cond _ t _ e -> Expr_If cond t e) <$>
      pKey "if" <*> pExpr <*> pKey "then" <*> pExpr <*> pKey "else" <*> pExpr

-- Lowest precedence operators.
pBinOp :: TokenParser String
pBinOp
  = pKey "+"
  <|> pKey "-"
  <|> pKey "<"
  <|> pKey ">"
  <|> pKey "<="
  <|> pKey ">="
  <|> pKey "=="
  <|> pKey "/="
  <|> pKey "and"
  <|> pKey "or"

-- Highest precedence operators
pBinOpH :: TokenParser String
pBinOpH
  = pKey "*"
  <|> pKey "/"

-- Lowest precedence operators expressions.
pAdd :: TokenParser Expr
pAdd
  = pFactor
  <|> (\x op y -> Expr_BinExpr op x y) <$> pFactor <*> pBinOp <*> pExpr

-- Highest precedence operators expressions.
pFactor :: TokenParser Expr
pFactor
  = pTerm
  <|> (\x op y -> Expr_BinExpr op x y) <$> pTerm <*> pBinOpH <*> pFactor

-- (Atom) Simple terms: Numbers, Variables, Constants or parenthized expr.
pTerm :: TokenParser Expr
pTerm
  = (\x -> Expr_Var x) <$> pVarid
    <|> (\x -> Expr_Const x) <$> pConid
    <|> (\x -> Expr_Int $ read x) <$> pInteger16
    <|> (\_ x _ -> x) <$> pKey "(" <*> pExpr <*> pKey ")"

-- do declarations
--
pDodecls :: TokenParser Dodecls
pDodecls
  = (\_ _ x _ -> x) <$>
      pKey "do" <*> pKey "{" <*> pList pDodecl <*> pKey "}"

pDodecl :: TokenParser Dodecl
pDodecl
  = (\_ x y -> Dodecl_Output x y) <$> 
      pKey "output" <*> pExpr <*> pVarid
  <|> (\x _ _ y -> Dodecl_Read x y) <$>
      pVarid <*> pKey "<-" <*> pKey "read" <*> pExpr
  <|> (\x _ _ f s -> Dodecl_Lift x f s) <$>
      pVarid <*> pKey "<-" <*> pKey "lift" <*> pVarid <*> pVarid
  <|> (\x _ _ f s1 s2 -> Dodecl_Lift2 x f s1 s2) <$>
      pVarid <*> pKey "<-" <*> pKey "lift2" <*> pVarid <*> pVarid <*> pVarid
  <|> (\x _ _ f v s -> Dodecl_Folds x f v s) <$>
      pVarid <*> pKey "<-" <*> pKey "folds" <*> pVarid <*> pExpr <*> pVarid
\end{Verbatim}


%
%\chapter{Archivos \textit{GraphML}}
%Como se mencionó anteriormente se utilizó el formato de archivo \textit{GraphML} para
 representar los sistemas de transiciones.
Los estados son representados por el elemento \texttt{node} mientras que las transiciones
 son representadas por el elemento \texttt{edge}.

El formato \textit{GraphML} no establece como representar las etiquetas tanto en los nodos
 como en las aristas.
Para esto existe el atributo \texttt{data}, que proporciona flexibilidad
 para agregar atributos no contemplados por el formato.
Esto tiene un desventaja, y es que los atributos no contemplados por el formato no se
 representan de forma estándar, y por lo tanto su representación depende del editor
 utilizado.
En este caso el editor utilizado es \textit{yEd}.

Para los estados se guarda la siguiente información:
\begin{itemize}
\item Identificador

Este valor se guarda en el atributo \texttt{id} de cada nodo.
Es el identificador del estado, por lo que debe ser único.

Cuando se genera un sistema de transiciones mediante el verificador este genera los identificadores
 de cada estado automáticamente.

\item Proposiciones

Representan el conjunto de las proposiciones válidas en cada estado.

Se guardan en el atributo \texttt{y:NodeLabel}.
Este atributo es se encuentra dentro del atributo \texttt{data}, ya que no se encuentra
 especificado en el formato.

En caso de haber varias proposiciones, estas deben estar separadas por comas.

\end{itemize}

Además de esta información se debe indicar cuales son los estados iniciales.

Para las transiciones se debe guardar la siguiente información:
\begin{itemize}
\item Origen

Representa el estado de origen de la transición. Se guarda en el atributo \texttt{source}.

\item Destino

Representa el estado de destino de la transición. Se guarda en el atributo \texttt{target}.

\item Acción

Representa la acción que corresponde al cambio de estado.

Se guardan en el atributo \texttt{y:EdgeLabel}.

Este atributo es se encuentra dentro del atributo \texttt{data}, ya que no se encuentra
 especificado en el formato.

\end{itemize}

El parser de \textit{GraphML} a sistemas de transiciones se encuentra implementado en
 el objeto \texttt{ParserGraphML} del paquete \textit{Sistemas de transiciones}.

Como se mencionó anteriormente hay atributos que no están especificados en el formato y
 por lo tanto su interpretación depende del editor utilizado. Estos atributos son especificados
 en el parser mediante sub clases.
En este caso se utiliza el \textit{yEd Graph Editor}, para el cual fue implementado el objeto
 \texttt{ParserGraphML\_yEd}.
Este objeto es un parser de \textit{GraphML} que además interpreta la información dentro del
 atributo \texttt{data} como las etiquetas de los estados y transiciones.

A continuación se muestra un ejemplo de un sistema de transiciones y su correspondiente
 representación en \textit{GraphML}.

\paragraph{Ejemplo de sistema de transiciones en \textit{GraphML}} 





% Termina el documento
\end{document}
