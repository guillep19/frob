
Este documento desarrolla la construcción de herramientas
que permitan programar robots utilizando el paradigma de
programación funcional reactiva.

Para ello se define el lenguaje \emph{Frob} de alto nivel,
que permite expresar los comportamientos de un robot y sus
interacciones.
El lenguaje permite expresar los mismos en base a
\emph{Comportamientos} y \emph{Eventos}.

Los eventos son 





%Este documento desarrolla la construcción de una herramienta para
% verificar propiedades sobre distintos sistemas.
%Estas propiedades son expresadas en distintas lógicas basadas en proposiciones
%mientras que los sistemas son representados por grafos etiquetados a los que
% llamaremos sistemas de transiciones.
%Esta herramienta es llamada verificador de modelos.
% 
%Este verificador es construído con fines académicos.
%Su objetivo es que sea fácil de utilizar,
% utilizando herramientas conocidas para modelar sistemas y
% lenguajes intuitivos para expresar propiedades.
%A su vez debe ser fácil de mantener y flexible para agregar otros
% tipos de lógica para expresar propiedades.
%
%En los siguientes capítulos se introducen herramientas para modelar
% sistemas reactivos así como distintos lenguajes para expresar
% propiedades sobre estos sistemas.
%
%Una vez que se tienen tanto el modelo del sistema representado por un
% sistema de transiciones $TS$, como la propiedad $\varphi$
% especificada formalmente en alguno de los lenguajes vistos, se utiliza
% un verificador de modelos para resolver el problema
%
%\[ TS \models \varphi \]
%
%Esto significa que el sistema de transiciones $TS$ cumple con la
% propiedad~$\varphi$.
%
%Los lenguajes seleccionados en este documento para especificar las
% propiedades utilizan distintos paradigmas de verificación.
%El algoritmo de verificación para LTL, uno de los lenguajes utilizados,
% construye un autómata que representa la propiedad y se basa en el
% reconocimiento de palabras para los autómatas.
%En este caso se realizan operaciones tanto sobre los autómatas como
% sobre el sistema de transiciones que representa el sistema reactivo.
%Por otro lado el algoritmo utilizado para CTL, el otro lenguaje
% utilizado en este proyecto, se basa en el análisis de satisfacibilidad de la
% propiedad por cada estado del sistema de transiciones.
%En este caso no se realiza ninguna operación sobre el sistema de
% transiciones, por lo que no se modifica el modelo original del
% sistema.
% 
%El verificador de modelos implementado contiene la sintaxis de cada
% lenguaje así como la implementación de sus respectivos algoritmos
% de verificación de forma modular.
%De esta forma se mantiene cierta flexibilidad para agregar nuevos
% lengujes utilizando los algoritmos de verificación ya implementados.
% 
%Para finalizar se tratan algunos casos de estudio completos.
%Estos consisten en el modelado de un sistema, la especificación
% de propiedades y el análisis de su satisfacibilidad.
