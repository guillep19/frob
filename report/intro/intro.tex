  Este documento desarrolla el proceso de construcción de
herramientas que permitan programar robots utilizando
el paradigma de \emph{programación funcional reactiva}.
  El proyecto es motivado por la necesidad de contar con
un lenguaje que brinde soporte al uso de la robótica
como herramienta educativa.

  La característica más destacable es utilizar un enfoque
orientado a programación funcional, y lograr con ello resolver problemas
de robótica razonando en alto nivel.

  Para analizar el problema, se desarrolla un estado del
arte relevando un conjunto de lenguajes de programación funcional
reactiva y se introducen los conceptos principales.

  Luego se analiza un conjunto de plataformas de hardware con bajas
capacidades de cómputo, en lo posible de bajo costo, que puedan
ser utilizadas para construir robots autónomos.

  Se define el lenguaje \frob{} funcional reactivo de alto nivel,
que permite expresar los comportamientos de un robot y sus
interacciones.
  El lenguaje permite expresar los mismos en base a
\emph{Señales} y relaciones entre ellas, que permiten
capturar la naturaleza reactiva del dominio.
  Al ser construído para enseñar conceptos de robótica,
uno de los objetivos es que sea simple de utilizar e intuitivo.

  Luego se muestra el proceso desde que se escribe un programa \frob{}
hasta que es ejecutado dentro de una plataforma de hardware de las
anteriormente relevadas.

  Para cumplir con el objetivo de que la implementación sea
portable, se separó la misma en dos fases.
  Los programas \frob{} son compilados a código \alf{} de
menor nivel de abstracción, el cual es ejecutado por una
máquina virtual que definiremos.
  La misma puede ser portada a distintas plataformas.

  Para finalizar se trata un caso de estudio completo, en el
que se construye un robot físico y se implementa usando \frob{}
la solución a un desafío robótico tomado de la
competencia \textit{SumoUY} \cite{sumouy}.
