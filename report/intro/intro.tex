%TODO(marcos): La intro se parece mas a un resumen que a una intro.
% Dedicar mas espacio a presentar y motivar el problema.


  Este documento desarrolla el proceso de construcción de
herramientas que permitan programar robots utilizando
el paradigma de programación funcional reactiva.

  Para ello se define el lenguaje \frob{} de alto nivel,
que permite expresar los comportamientos de un robot y sus
interacciones.
  El lenguaje permite expresar los mismos en base a
\emph{Señales} y relaciones entre ellas, que permiten
capturar la naturaleza reactiva del dominio.

  Éste lenguaje fue construído para
enseñar conceptos de robótica, por lo tanto
uno de los objetivos es que sea simple de
utilizar e intuitivo.

  Otro objetivo planteado es que debe ser posible programar
sistemas robóticos embebidos en plataformas de
hardware de baja capacidad de cómputo, en lo posible de
bajo costo.
  Se hizo un estado del arte investigando una gran variedad
de plataformas, evaluando sus características,
especialmente la capacidad de cómputo, cantidad de memoria y
costo estimado.

  En los siguientes capítulos se introduce el concepto de
programación funcional reactiva, junto con sus variantes y
algunos ejemplos.
  Luego de decidir cuál de éstas variantes es más útil
en nuestro dominio, se muestra la definición del lenguaje
\frob{} y cuál es el proceso desde que se escribe un programa
hasta que es ejecutado dentro de una plataforma de hardware.

  Para cumplir con el objetivo de que la implementación sea
portable, se separó la misma en dos fases. Los programas son
compilados a código \alf{} de menor nivel de abstracción,
el cuál es ejecutado por una máquina virtual que definiremos.
  La misma puede ser portada a distintas plataformas.

  Para finalizar se trata un caso de estudio completo, en el
que se implementa usando \frob{} y construyendo un robot físico
la solución a un desafío robótico tomado de la
competencia \textit{SumoUY} \cite{sumouy}.
