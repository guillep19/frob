
  Este documento desarrolla el proceso de construcción de
herramientas
que permitan programar robots utilizando el paradigma de
programación funcional reactiva.

  Para ello se define el lenguaje \frob\ de alto nivel,
que permite expresar los comportamientos de un robot y sus
interacciones.
  El lenguaje permite expresar los mismos en base a
\emph{Eventos} y relaciones entre ellos, que permiten
capturar la naturaleza reactiva del dominio.

% @makam: ... se sabe que cualquier proy. de grado
% es con fines académicos. Esta en la tapa esto, no? 
  Este lenguaje fue construído con fines académicos y de
investigación, uno de los objetivos es que sea simple de
utilizar e intuitivo.

  Otro objetivo planteado es que debe ser posible programar
con el sistema robótico embebido en plataformas de
hardware de baja capacidad de cómputo, y en lo posible
bajo costo.

  En los siguientes capítulos se introduce el concepto de
programación funcional reactiva, junto con sus variantes y
algunos ejemplos.
  Luego de decidir cuál de éstas variantes es más útil
en nuestro dominio, se muestra la definición del lenguaje
\frob\ y cuál es el proceso desde que se escribe un programa
hasta que es ejecutado dentro de una plataforma.

  Para cumplir con el objetivo de que la implementación sea
portable, se separó la misma en dos fases. Los programas son
compilados a código de menor abstracción, el cuál es
ejecutado por una máquina virtual que definiremos.
  La misma puede ser portada a distintas plataformas.

  Para finalizar se trata un caso de estudio completo, en el
que se implementa un desafío robótico utilizado en la competencia
SumoUY.
